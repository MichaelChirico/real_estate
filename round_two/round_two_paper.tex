\documentclass[12pt]{article}
\usepackage{amssymb}
\usepackage{theapa}
\usepackage{titlepage}
\usepackage{amsmath}
\usepackage{setspace}
\usepackage[dvips]{graphicx}
\usepackage{rotating}
\usepackage[usenames,dvipsnames]{pstricks}
\usepackage{epsfig}
\usepackage{pst-grad}
\usepackage{pst-plot}
\usepackage{color}
\usepackage{pstricks-add}
\usepackage{rotating}
\usepackage{threeparttable}
\usepackage{array,multirow}
\usepackage{pdflscape}
\usepackage{float,lscape}
\usepackage{csquotes}
\usepackage{textcomp}

\renewcommand{\baselinestretch}{1.5}
\parindent=.2in
\evensidemargin=.05in 
\oddsidemargin=-.05in 
\topmargin=-0.05in
\textwidth=6.5in 
\textheight=8in

\newtheorem{fact}{Stylized Fact}
\newtheorem{theorem}{Theorem}
\newtheorem{corollary}{Corollary}
\newtheorem{definition}{Definition}
\newtheorem{lemma}{Lemma}
\newtheorem{prop}{Proposition}
\newtheorem{assumption}{Assumption}
\newtheorem{remark}[theorem]{Remark}
\newtheorem{solution}[theorem]{Solution}
\renewcommand{\thefootnote}{\fnsymbol{footnote}}


\begin{document}

\title{Procrastination and Property Tax Compliance: Evidence from a Field Experiment}

\author{Michael Chirico, Robert Inman, Charles Loeffler, \\ 
John MacDonald, and Holger Sieg\thanks{We would like to thank Rob Dubow,
    Clarena Tolson, and Marisa Waxman in the Department of Revenue of
    the City of Philadelphia for their help and support. We thank Kent
    Smettters and the Wharton Initiative for Public Policy for funding
    this field experiment. We would also like to thank Jeff Brown, Kai
    Konrad, Robert Moffitt, Jim Poterba, Chris Sanchirico, Wolfgang
    Sch\"on, Reed Shuldiner and participants of numerous seminars for
    comments and suggestions. The views expressed here are those of
    the authors and do not necessarily represent or reflect the views
    of the City of Philadelphia.}  \\ 
University of Pennsylvania}

\date{\today}

\maketitle

\begin{abstract}

Property taxes play a central role in the financing of municipal
government services. Yet, municipal governments commonly confront
problems with property tax collection despite the fact that the tax
base is known.  We model delinquent tax payers as
procrastinators. Late payments arise due to present bias and lack of
salience.  This paper develops and implements a randomized controlled
experiment conducted with the City of Philadelphia. The experiment is
based on different notification or nudge strategies.  Within this
structure, our experiment allow us to identify the relative importance
of the three key sets of parameters of our model.  We find that all
notification strategies increase property tax compliance which
indicates that lack of salience is a key component of non-compliance
behavior.  The most effective notifications are the those that
threaten to take out a lien on the property or to foreclose by
sheriff's sale for continued failure to pay taxes.

\bigskip

\noindent KEYWORDS: Tax Compliance, Property Taxation, Field
Experiment, Deterrence, Public Service Appeal, Appeal to Civic Duty.

\end{abstract}
\renewcommand{\thefootnote}{\arabic{footnote}}

\newpage

\section{Introduction}

Property taxation is the primary tax for most U.S. cities.  In 2013
over 72 percent of all local government revenue and over 50 percent of
own revenues came from the property tax.  Yet collection of the tax
has, in many cities, been problematic.  While some U.S. cities do an
excellent job in collecting the tax, receiving over 95 percent of
assessed revenues in the year the tax is due, other cities have over
the last ten years done significantly worse -- notably Flint (78\%),
Cleveland (84\%), Pittsburgh (86\%), Milwaukee (87\%), Philadelphia
(88\%), Detroit (89\%), and St. Louis (89\%).\footnote{For more
  details, see \citeA{CILMS-16}.}  While Flint, Detroit, Cleveland and
Milwaukee are relatively poor cities, Philadelphia and Pittsburgh are
not.  Among the list of cities with outstanding tax collection records
are Buffalo, Birmingham, Houston, and New Orleans.  While city poverty
is important, it cannot be the whole explanation for low rates of
collection.  Poor tax administration is likely to be an important
contributing factor as well.

This failure to collect the property tax on time creates
budget uncertainty at best and budget deficits at worst.  Yet
collecting the property tax should be straightforward.  In contrast to
collecting self-reported taxes such as those on income, profits, and
sales, property tax obligations equal to the city's assigned assessed
value of the taxed property times the city chosen tax rate are known
by both the city and the taxpayer.  There is no uncertainty as to what
is due, or when.\footnote{Much of the current literature on tax
  compliance has focused on taxpayers truthful reporting of income or
  sales under the threat of a tax audit; see \citeA{Slemrod-07} for a
  review and more recently the research of \citeA{Kleven-11} and
  \citeA{Pomeranz-15}.}

Payment is primarily  a matter of enforcement.  The most common
enforcement strategy is the economic stick: fines and penalties.
Failure to pay property taxes in time leads to interest penalties
sufficiently large that there is no arbitrage advantage to waiting,
and perhaps to a significant late fine as well.  

When a delinquent tax payer does not respond to penalties and fines,
the city can take out a tax lien on the property.  A lien does not
impose any immediate direct, tangible costs on a delinquent tax payer
since payments are typically only realized at the time of a
transaction.\footnote{A city can also sell tax liens to investors to
  speed up the revenue collection process. Liens often sell at above
  par prices because of the foreclosure option. But selling liens to
  "vulture investors" can be politically costly for a city
  administration.}  However, obtaining a tax lien enables the owner of
the lien to eventually start a foreclosure process. When the owner of
a property located in a city fails to make a payment arrangement on
municipal debt levied on his or her property, that property may be
sold at auction to allow the city to collect on that unpaid debt.
However, the foreclosure process is costly and time
intensive.\footnote{Auctions are administered in Philadelphia by the
  Office of the Sheriff.  This process of offloading a property at
  Sheriff's Sale can take nine months to a year.}  While there are
some problems with the effectiveness of the existing enforcement
mechanisms, it is only possible to avoid payment by abandoning the
property in the long run. Needless to say, this is a very costly
option for most properties.

Despite the fact that there are no obvious financial gains to not
paying property taxes, we observe that a significant fraction of tax
payers do not pay on time. To explain the behavior of these
procrastinators, researchers have started to explore the effectiveness
of softer, nudge approaches or notification strategies to reinforce
the different motivations of tax compliance.  This paper uses a field
experiment involving over 19,000 delinquent Philadelphia taxpayers to
examine the effectiveness of seven alternative strategies for
improving city property tax collection. Each involves a randomly
assigned tax ``nudge'' of a delinquent taxpayer.  The first is a
simple reminder that the payment is late.  The next two involve the
reminder plus a threat of a significant sanction if payment is not
received by the end of the calendar year: a lien on the home when sold
equal to taxes due plus accrued interest and penalties \textit{or} the
lien coupled with an immediate sheriff's sale of the home to collect
the lien.  The final four nudges include the reminder coupled with an
appeal to what the tax compliance literature has called a ``tax
morale'' motive from paying one's tax taxes.\footnote{See
  \citeA{Luttmer-14} for a review of the tax morale strategies for tax
  compliance.  They identify three tax morale motivations in the
  literature, each grounded in a positive gain in utility from the act
  of paying one's taxes.  These include: 1) a motive from reciprocity
  where the taxpayer recognizes they are part of a larger group
  playing a non-cooperative game with other taxpayers; 2) a motive
  from peer behavior where the taxpayer gains utility from knowledge
  that they are part of larger group of contributors; and 3) an
  intrinsic motivation that provides a direct utility benefit from the
  act of paying one's taxes.  \citeA{Luttmer-14} also mention taxpayer
  culture and taxpayer behavior other than utility maximization as
  additional explanations for the rate of taxpayer compliance. }  The
four morale motives included here are: first, a reminder that taxes
pay for neighborhood services such as street repairs, trash pick-up
and the local park; second, a reminder that taxes pay for important
city-wide services such as police protection and public schools;
third, a reminder that 9 out of 10 Philadelphians have paid their
taxes and you have not; and fourth, a reminder that paying one's taxes
is an important obligation of citizenship in a democracy.  Tax
compliance after receiving one of the seven ``nudges'' is then
compared to compliance for those who have not received a ``nudge''--
our ``holdout'' sample.

To understand the potential influence of each nudge, we model tax
delinquency as a problem of taxpayer procrastination following
\citeA{Akerlof-91} and \citeA{DR-99}.  Procrastination occurs 
because of present bias as in \citeA{DR-99} and declining saliency as in
\citeA{Akerlof-91}.  Present bias is always present.  Saliency can be
nudged by a reminder letter.  The reminder letters stressing liens or
liens and the sale of one's home add a future expected cost to
non-payment as in the tax compliance model of
\citeA{Allingham-Sandmo-72}.  The reminder letters stressing the tax
morale are modeled as utility gains to the procrastinator from paying
ones taxes.  Within this structure, our experiment is able to identify
the relative importance of each motive for tax compliance.

Our work here is closely related to the recent work of
\citeA{Hallsworth-14} studying the effect of taxpayer nudges on the
timeliness of income tax payments in the UK.  Like our study, the
amount owed to the tax authority is known by the authority and the
taxpayer with certainty; the only issue is payment.  Like our
analysis, their empirical work follows from a model of taxpayer
procrastination.  The primary focus of their field experiments is the
framing of the morale nudge, comparing the effectiveness of what they
call a \textit{descriptive} message (``a majority of citizens pay
their taxes'') to that of an \textit{injunctive} message (``you
\textit{should} pay your taxes because'').  Our analysis also includes
a descriptive message (``9 out 10 taxpayers have paid their tax'') and
an injunctive message (paying one's taxes is a duty of citizenship'').
We differ from the \citeA{Hallsworth-14}, by including a more strongly
worded message on the penalties for non-compliance and by allowing a
longer period of study for compliance behavior (3 weeks vs. 6 months
in our study).  The longer period allows a sharper identification of
the saliency of each nudge.  Finally, they study compliance for the
payment of an important national tax; we study compliance for an
important local tax.\footnote{We conducted an earlier pilot study of
  property tax compliance in Philadelphia.  The results are reported
  in \fullciteA{CILMS-16}.  We find evidence that motives driven by
  reciprocity, peer effects, and civic duty can positively impact
  property tax payment compliance. The sample in that pilot study
  was too small to draw reliable inference. Moreover, the sample
  focused on repeat instead of first time tax delinquents.}

Our experiment began on June 15, 2015 with the identification of
19,028 delinquent taxpayers who owed more than \$10 to the city in
taxes for the fiscal year, 2015.  By design, 2,088 of the original
sample were excluded from receiving a reminder letter and became our
holdout sample.  Between June 15 and August 15, the remaining 16,940
taxpayers received one of our seven reminder letters.  The reminder
letters were mailed only once.  Beginning on August 15, the City then
allowed its two outside collection agencies to begin their efforts at
collection from those who had not yet paid their taxes.  The agencies
sent letters or called the delinquent taxpayers on a weekly basis
reminding them that their taxes were due and that liens including
taxes and penalties will be assigned at the end of the collection
year, December 31, 2015.

By July 15, 2015, 30.5 percent of all taxpayers in the holdout (``do
nothing'') sample had made some contribution towards their tax
liabilities.  The compliance rate in the control sample who just
received a neutral reminder letter was 3.8 percent higher than in the
hold out sample. The estimate is significantly different form zero
which indicates that lack of salience is an important factor in
explaining non-compliance behavior.  The fact that messages that
stress tax moral did not outperform the reminder letter, indicates
that potential utility gains associated with the letters are
insignificant. In contrast 39.5 percent of taxpayers that received the
lien letter and 37.9 percent that received the sheriff's sale letters
had made payments. This finding suggests that these two letters not
only improved saliency, but also increased deterrence.

The findings are similar after three months.  By September 15, 2015,
51.4 percent of all households in the holdout sample had made some
contribution compared to 55.3 in the control sample. The compliance
rate was 60.6 percent for the lien and 60.2 percent for the sheriff's
sale letters. Using fairly conservative estimates for the revenue per
letter. The three-month impact on revenue for the lien and sheriff's
sale letters were approximately \$75 per letter and \$32 for the
control letter.  We also find that approximately 80 percent of the tax
payers that responded to our letters paid the amount due in full and
did not need a payment plan. This suggests that liquidity concerns are
only relevant for up to approximately 20 percent of the delinquent tax
payers.

By December 15, 2015, we observe some convergence in the effectiveness
of all nudge strategies. We think that this result is probably due to
the activities of the collection agencies that were employed by the
city after roughly three month into the experiment. Finally, there is
no evidence that our letters or the efforts of the collection agency
had a lasting effect on taxpayer compliance into 2016, either among
the entire sample or more likely among those who had responded
positively to 2015 enforcement strategies.

The rest of the paper is organized as follows.  Section 2 provides a
behavioral model of tax compliance that allows for delay and
non-compliance in equilibrium and shows how different nudge strategies
affect the decision to delay compliance. Section 3 discusses details
of our field experiment including a detailed description of the
treatments and the randomization procedure. Section 4 discusses our
randomization procedure.  Section 5 reports the main empirical
findings. Section 6 offers conclusions and briefly discusses the
policy impact of our experiment and discussions on enforcement
activities in the City of Philadelphia.

\section{Taxpayers As Procrastinators}

Most city residents are law abiding citizens.  If late in their city
tax payments it is unlikely it is part of a strategic plan to avoid
payment.  Property tax payments are computed by the city as assessed
home value times the city's property tax rate and are known both to
the city and the taxpayer.  While it is possible to avoid payment by
abandoning the property, this is very costly.  For the vast majority
of taxpayers the only issue is timely payment.  Taxpayers receive
their tax bill in January of the fiscal year with full payment or an
agreed to payment schedule required by the end of March.  Most
families have the payment withheld in an escrow account as part of
their monthly mortgage payments.  If payment, or enrollment in a
payment plan, has not been made by the end of April, the city starts
enforcement proceedings against the taxpayer.  Enforcement begins with
a reminder letter that all taxes and additional accrued interest and
penalties are now due.  In Philadelphia, those reminder letters are
mailed in early May.  A sample of the City's standard reminder letter
is shown in Figure 1.  We are studying the payment decisions of these
delinquent, or late, taxpayers. Following the analysis of O'Donoghue
and Rabin (1999), our late taxpayer are seen as procrastinators who
struggle with the problem of when to pay their property
taxes.\footnote{We are not the first to model taxpayer compliance as a
  problem of procrastination; see Hallsworth, et. al.  (2014). We
  differ from their analysis in two ways.  First, their focus is on
  late taxpayers as possibly credit-constrained households.  That is
  less of an issue for our work as all our taxpayers are homeowners
  with assets that can used as collateral for a loan to pay taxes.  It
  is true that homeowners, particularly the elderly, may not utilize
  such loans, but that is problem of financial literacy not tax
  compliance.  Second, while we both rely upon the fundamental work of
  O'Donoghue and Rabin (1999), we amend that analysis to include the
  insight of Akerlof (1991) on the importance of ``saliency'' to the
  problem for procrastinators. We also extend the model to allow for
  active tax enforcement by the city.}

Our taxpayer makes a decision every two weeks or perhaps every month
as they pay their family bills.  They can pay their taxes today, or
postpone the decision until ``tomorrow.''  If they pay their taxes
today, they bear the immediate cost equal to the payment made.
Taxpayers enjoy a benefit from having paid their taxes, but those
benefits are not realized until ``tomorrow,'' either as the simple
relief of knowing their taxes are paid or perhaps from the good
feelings -- that is, tax ``morales'' -- of knowing they have met their
obligations to their fellow residents.\footnote{ We make the realistic
  assumption that our taxpayers will receive their public services
  whether they pay their taxes or not.  If they do not pay their
  taxes, then they will be free-riding on the good will of their more
  responsible neighbors.  }  This is O'Donoghue and Rabin's problem of
the procrastinator facing immediate costs and delayed benefits. The
decision period is today at time $t$, where $t$ represents the number
of periods since first receiving a notice that taxes are due.  In
deciding today as to whether to pay or not pay taxes, the taxpayer's
inter-temporal utility function is specified over possible dates for
payment. If the tax payer makes a payment at time $t$, lifetime
utility at time $t$ is given by:
\begin{eqnarray}\label{eq1}
U_t^t  &=& (\varphi^{t+1} \beta \delta) \;  V - c_ t 
\end{eqnarray}
where $c_t$ is the cost of tax payment at time $t$, and $V$ is the
benefit of knowing one's taxes are paid but not enjoyed until the
period after payment.  We assume $V$ is constant for whenever taxes
are paid. Later on we extend the model and show that $V$ may depend on
notifications which may reinforce the benefits of tax compliance.

Benefits are evaluated in today (period $t$) dollars allowing for
declining saliency to future benefits and costs at rate $\varphi$ ($0
\le \varphi \le 1$), possible present bias to all discounting at rate
$\beta$ ($0 \le \beta \le 1$), and the usual discounting of money
values at rate $\delta$ ($0 \le \delta\le 1$).  If the tax payer plans
to make a payment at time $t + s$, then the anticipated lifetime
utility at time $t$ of that payment is given by:
\begin{eqnarray}\label{eq2}
U_t^{t+s} &=& (\varphi^{t+s+1} \beta \delta^{s+1}) \; V \; - \; (\varphi^{t+s}
\beta \delta^{s}) c_ {t+s} \; \; \; \; \; \; s=1,2, ...
\end{eqnarray}
where $c_{t+s}$ are the costs of tax payments at time $t+s$. The costs
of tax payment may rise over time with accruing interest and
penalties.

While tax payments made today are realized as a cost  ($c_{t}$) today,
tomorrow's tax payments and tomorrow's benefits are both realized in
the next period, and are, therefore, discounted for today's decisions.
Outcomes realized one period from today are discounted at the rate
$\varphi^{t+1} \beta \delta$ and if realized in the two periods from
today at the rate $\varphi^{t+2} \beta \delta^2$. 

In our analysis the length of each individual period is relatively
short, perhaps two weeks to a month between paying one's bills, and
the overall decision horizon of our delinquent taxpayer's is no longer
than several months.  We will, therefore, assume that $\delta=1$.  The
taxpayer may display a present bias, however, represented by a further
discounting of future costs and benefits at a rate $\beta <1$; time
consistent taxpayers do not display a present bias so $\beta=1$.
Finally, our delinquent taxpayer may be forgetful which we represent
as a declining rate of awareness or saliency, $\varphi^{t+s}$.
Constrained by bounded rationality, taxpayers may only be able to pay
attention to limited set of facts or tasks (Akerlof, 1991).  For the
forgetful taxpayer, $\varphi < 1$; for the fully aware taxpayer,
$\varphi = 1$. In the extreme future or for the very forgetful
taxpayer, $\varphi \simeq 0$ - that is, ``out of sight, out of mind.''
Introducing the concept of saliency is a relatively simply way to give
``reminders'' an explicit role in taxpayer compliance.\footnote{
  Saliency and reminders play a similar role in the behavioral
  economics of health policies; see \citeA{Kessler-Zhang-14} for a
  review.} As we discuss in detail below, saliency can explain
differences in the response rates of tax payers in the holdout sample
and tax payers that just received a neutral reminder letter.

Our analysis focuses on the type of taxpayer who O'Donoghue and Rabin
identify as the naive procrastinator.  Her payment behavior stands in
contrast to that of the fully aware ($\varphi = 1$) and time
consistent ($\beta = 1$) taxpayer who will always pay her taxes on
time (see below) and the sophisticated procrastinator who recognizes
she is forgetful and/or present biased but is able to commit to an
optimal payment schedule in advance.  Here, that commitment device
could be an escrow account with the mortgage bank or a city arranged
tax payment plan.  In contrast, the naive procrastinator assumes that
she will remember to pay her taxes next period and do so in an
optimal, time consistent way -- but she does not.  As a result, she
may keep postponing payment until the end of the tax year when some
drastic action -- for example, court seizure of the home or
garnishment of wages -- is taken to collect all taxes, interest, and
penalties due.  Since both time consistent and sophisticated
procrastinators will have paid, or have arranged to have paid, their
property tax, they will not be in our sample of late taxpayers.  Only
naive procrastinators will be in our sample.

How does the naive procrastinator decide to pay her taxes?  She will
pay her taxes if the benefits from paying today are greater than
benefits of paying at some later date.  Following O'Donoghue and
Rabin, we assume the naive taxpayer adopts what they call a
\textit{perception-perfect strategy} and pays her taxes today only if
doing so gives them more perceived utility today than by paying at
some future date.  In our problem with constant $V$ and rising costs
$c_{t+s}$ because of accumulating interests and penalties, the best
alternative date for paying taxes will always be in the immediate next
period $ t + 1$.  If so and assuming $\delta=1$, the naive
procrastinator pays today, if the lifetime utility of paying today are
greater or equal to the lifetime utility if she delays:
\begin{eqnarray}\label{eq3}
(\varphi^{t+1} \beta) \; V - c_ t &\ge& (\varphi^{t+2} \beta) \; V -
  (\varphi^{t+1} \beta) \; c_ {t+1}
\end{eqnarray}
or if: 
\begin{eqnarray}\label{eq4}
(\varphi^{t+1} \beta) \; (V (1-\varphi) + c_{t+1}) &\ge& c_ {t}
\end{eqnarray}
The RHS of equation (\ref{eq4}) is the perceived cost of paying one's
taxes today.  The LHS of equation (\ref{eq4}) is the perceived cost of
paying taxes one period later and is equal to the actual payment of
those taxes one period later ($c_{t+1}$) plus the benefits
``forgotten'' ($V(1 - \varphi)$) because of declining saliency.  If
the perceived costs of paying one's taxes one period later are greater
than or equal to the perceived costs of paying one's taxes today, the
taxpayer will pay today.

Current period costs of compliance will equal taxes owed ($T$) plus
accumulated interest and penalties at rate $\rho$ now due from not
paying taxes in prior periods:\footnote{Strictly speaking interest and
  penalties do not begin to accumulate until some number of periods
  after the tax bill was first received. Rather than interest and
  penalties accumulating from the first date of the receipt of the tax
  bill for t periods as specified here, penalties only begin to accrue
  after a grace period.  In the case of Philadelphia, the grace period
  between when the bill is received and taxes are due is three months.
  We adopt this simpler specification for the timing of payments to
  minimize the use of superscripts for dating all the periods. All
  that is required to ensure the same level of accumulated penalties
  is is to lower the rate of interest and penalties, $\rho$, in our
  specification to reflect the grace period. All comparative statics
  from the model will be the same. }
\begin{eqnarray}\label{eq5}
c_{t+s} &=& T \; (1 + \rho)^{t+s} \; \; \; s=0,1,2,...
\end{eqnarray}
Substituting this definition into equation (\ref{eq4}) gives:  
\begin{eqnarray}\label{eq6}
\varphi^{t+1} \beta \; (V (1-\varphi) + T \; (1 + \rho)^{t+1}) &\ge&
  T \; (1 + \rho)^{t}
\end{eqnarray}
as the requirement for current period tax compliance.  More simply,
rearrange and divide both sides by $T(1 + \rho)^{t}$ and the condition
for immediate tax payment becomes:
\begin{eqnarray}\label{eq7}
\varphi^{t+1} \beta \;  (v (1-\varphi) +  (1 + \rho))  &\ge&   1
\end{eqnarray}
where $v = V/[T(1 + \rho)^{t}$] are the benefits of paying one's taxes
per dollar of taxes (and penalties) paid.  The RHS of equation
(\ref{eq7}) is the cost of paying one dollar of taxes today; the LHS
of equation (\ref{eq7}) is the perceived costs of delaying and paying
one's taxes in the next period.  The perceived costs of delay are
equal to the future benefits ``forgotten'' per dollar of taxes paid
\textit{plus} the added tax penalties from waiting.  The taxpayer will
pay her taxes today if the cost of paying a tax dollar today is less
than or equal to the costs of waiting and paying that tax dollar in
the next period.

In contrast to the naive procrastinator who is forgetful ($\varphi
<1$) and/or present biased ($\beta < 1$) and may therefore delay
payment, the fully aware ($\varphi= 1$) and time consistent ($\beta =
1$) taxpayer always pays her taxes on time -- that is, $1 + \rho >
1$.

In addition to the usual \textit{passive} enforcement of late payments
that occurs through the payment of interest and penalties when taxes
are paid, the city may also use an \textit{activist} enforcement
strategy that audits some delinquent taxpayers at the beginning of the
current period. Alternatively, we can interpret the enforcement
strategy as taking out a lien and proceeding with a sheriff's sale.

If audited and determined to be a delinquent taxpayer, with
probability $\pi$, the taxpayer must then pay an additional fine $F$
in the next period.  $F$ might include ``booting'' the taxpayer's car,
removing the taxpayer's children from school until payment is
received, or additional fines equal to added administrative costs plus
penalties.  A city might target its activist strategy at those
taxpayers with very large tax bills or with a long history of late
payments.

We assume, for simplicity, that activist enforcement is only in period
$t$ and not later.\footnote{The extension to a model in which
  enforcement occurs in each period with probability $\pi$ is not
  difficult and all results summarized in Proposition 1 also apply in
  that model.}  If the tax payer does not pay in period $t$, then
under the activist enforcement strategy, the expected lifetime utility
in the next period if there is delay must allow for the possible
imposition of the penalty, $F$.  In that case, the expected lifetime
utility from a one period delay becomes:
\begin{eqnarray}\label{eq8}
U_t^{t+1} &=& \pi  \; [\varphi^{t+2} \beta \; V - \varphi^{t+1} \beta c_
  {t+1} - \varphi^{t+1} \beta F ) \; +  \; (1-\pi) \;  [\varphi^{t+2} \beta V -
    \varphi^{t+1} \beta c_ {t+1}], \; \; \; \mbox{or}, \nonumber  \\ 
&=& \varphi^{t+1} \beta \; [ \varphi V - c_ {t+1} - \pi F ]
 \end{eqnarray}
Now the tax payer's decision rule is to pay if the expected utility of
delay is less than the expected utility of paying today, or with the
normalization that $f = F/(T(1 + \rho)^{t})$ , if:
\begin{eqnarray}\label{eq9}
\varphi^{t+1} \beta \; (v \; (1-\varphi) + (1 + \rho) \; + \; \pi f)
&\ge& 1
\end{eqnarray}
Note that the likelihood of making tax payments increases in the
enforcement parameters $\pi$ and $f$. The following proposition then 
summarizes our analysis above.
\begin{prop}
Naive procrastinating taxpayers will pay their taxes today if their
perceived expected lifetime utility of delaying payment are greater
than or equal to the lifetime utility of paying their taxes today.
The likelihood of payment will increase as:
\begin{enumerate}
\item taxpayer saliency of future benefits and costs increases
  ($\varphi$ rises);
\item taxpayer present bias is reduced ($\beta$ rises); 
\item the benefits from the act of tax payment increases ($v$ rises);
\item the rate of interest and penalties upon late payment increase
  ($\rho$ rises); and
\item the enforcement probability ($\pi$) and the fines ($f$) increase.
\end{enumerate}
\end{prop}
The specification for taxpayer compliance summarized by the
Proposition 1 provides the conceptual framework for our field
experiment outlined in Section III as well as the basis for our
empirical analysis of the experiment reported in Sections IV and V.

\section{A Field Experiment }


The research setting for the experiment is the City of Philadelphia
for fiscal year, 2015.  Notices of property tax payments are sent on
January 1, and the full balance of taxes are due by March 31.  If
payment has not been received by that date, or the taxpayer has not
entered into a tax payment plan with the City, then taxes are
considered delinquent and interest and penalties begin to accrue.  On
April 1, the City's Department of Revenue (DoR) begins contracting all
taxpayers with unpaid accounts, informing them of taxes due and
accumulated interest and penalties for late payment.  At this time,
the City will normally send two-thirds of the delinquent accounts to
outside collection agencies acting as co-counsel for the City.  The
outside collection agencies are reimbursed at the rate of six percent
of all their delinquent revenues collected by December 31.  The
remaining one-third of the delinquent accounts remain with the DoR for
collection.  All accounts still delinquent on December 31 are then
assigned to new outside collection agencies. For the purposes of our
experiment the City of Philadelphia agreed to delay sending the
delinquent accounts to the collection agencies until August 15, 2015.

Our experiment was implemented with those taxpayers newly delinquent
on March 31, 2015.  Of the 579,828 properties in the city receiving
2015 tax bills, approximately 100,000 or 17 percent were late in
payment as of April 1.  Of these 100,000 accounts, 27,264 accounts
still owed more than \$10 as of May 15. None had been assigned to the
two outside collection agencies.  For these 27,264 accounts, we
identified the 19,333 tax account holders responsible for payment. Our
experiment therefore excludes all chronically delinquent taxpayers who
owed taxes from prior years.  Of the 19,333 newly delinquent
taxpayers, 2,393 taxpayers owned more than one property.  While all
19,333 taxpayers were included in the experiment, we focus our
empirical work on the 16,940 taxpayers who owned only one
property.\footnote{ As a robustness check we repeated our empirical
  analysis for the full sample of 19,333 taxpayers and the results are
  identical those we report in Sections IV and V below. }  Our
experiment began with the mailing of reminder letters in mid-June,
2015 and continued to December 31, 2015.  Of the unary taxpayers,
16,940 taxpayers received a standard or experimental reminder letter
and 2,088 delinquent taxpayers did not receive a reminder.  This
sample of 2,088 taxpayers became our ``holdout'' sample and the basis
for identifying the importance of saliency in taxpaying behavior. To
ensure that our experiment was not contaminated by other treatments
not under our control, the DoR agreed to postpone all other
enforcement activities until August 15.  In particular, the outside
collection agencies were not allowed to begin their collection efforts
until after that date.  The likely earliest date that those efforts
led to any contact with a taxpayer is September 1.

Each reminder letter in our experiment was drafted to identify the
possible impact on taxpayer compliance of the key variables in
equation (\ref{eq9}).  We could not, however, measure the effect of
either taxpayer present bias ($\beta$) because our sample was limited
to delinquent taxpayers only. We also cannot evaluate the direct
impact of a more activist enforcement strategy $(\pi, f)$ as the city
had not adopted such a strategy in our sample year, 2015.\footnote{It
  is possible and consistent with our findings reported below that
  lien and the sherif's sale notification strategies had an indirect
  effect by changing the subjective beliefs regarding $(\pi,f)$.} Our
main aim is to identify the potential importance of taxpayer saliency
($\varphi$), tax morales as they impact the benefits of tax payment
($v$), and interest and penalties ($\rho$).  Each reminder letter was
approved by City's DoR to ensure that it could be understood by a
taxpayer with at least a fourth or fifth grade level of English
reading comprehension.  Each letter also provided contact information
for assistance for non-English speaking taxpayers.  Translation were
available for a number of different languages.\footnote{The full
  template for each letter is available as an online appendix.}  For
brevity we present here the important distinguishing feature of each
letter.

\bigskip

\noindent \textit{Reminder-only}: \textbf{Our records indicate 
that you have a balance due of \textit{balance. }} If you have 
already paid, thank you.  If not, please pay now or contact us 
to arrange a payment plan.  The fastest and easiest way to pay is 
online at  www.phila.gov/pay. Paying by E-check only costs 35 cent 
-- less than the cost of a stamp.!

\bigskip

 The reminder-only letter allows us to identify the potential
 importance of tax saliency to taxpayer compliance.  From equation
 \eqref{eq9} our holdout sample has a rate of saliency of
 $\varphi^{t+1}$ when evaluating future benefits and costs.  But those
 receiving our reminder letter today have a rate of saliency when
 evaluating future benefits and costs of $\varphi$ only.  When
 saliency is important, future taxes and benefits will be more salient
 after the receipt of the reminder, thus increasing the likelihood of
 taxpayer compliance. A higher rate of compliance among taxpayers
 receiving the reminder-only letter compared to those in the hold-out
 cohort identifies a separate role for saliency in taxpayer
 compliance.\footnote{Our experimental design can identify the
   presence of saliency by an increase in compliance for those
   receiving a reminder letter, but time staggered reminder letters at
   a two-week or monthly interval would be needed to identify the
   actual rate of saliency -- that is, the value of $\varphi$.  This
   was not possible within the time constraints imposed by DoR on our
   experiment.  }
   
\bigskip

\noindent \textit{Reminder plus Tax Lien}: Failure to pay your Real
Estate Taxes may result in a tax lien on your property in an amount
equal to your back taxes plus all penalties and interest.  When your
property is sold, those delinquent tax payments will be deducted from
the sale price.  By paying your taxes now, you can avoid these
penalties and interest.  Properties near you in your neighborhood that
have liens placed on them include: $<$ List Three Properties and Sale
Dates $>$ \textbf{Pay your taxes now to avoid a lien being placed on
  your property.  Our records indicate that you have a balance due of
  \textit{balance}.  }

\bigskip

\noindent \textit{Reminder plus Lien and Sheriff's Sale}: Failure to
pay your Real Estate Taxes may result in the sale of your property by
the City in order to collect back taxes.  In the past year we have
sold \textit{N} properties in your neighborhood at a Sheriff's Sale.
Included in these \textit{N} properties are the following properties
near you: $<$List Three Properties and Sale Dates$>$ \textbf{Pay your
  taxes now to prevent the sale of your property.  Our records
  indicate that you have a balance due of \textit{balance}.}

\bigskip

The reminder letter coupled with the threat of a lien, or a lien plus
a sheriff's sale of the taxpayer's home, increase the expected
interest and penalties to the costs of delay -- that is, an increase
in penalties ($\rho$).  Both letters make clear that interest and
penalties will be collected by listing neighborhood properties where
these added enforcement measures have been implemented.  A taxpayer
lien for all interest and penalties will be collected at the future
date of home sale, which may be a very large obligation if the home is
sold significantly in the future.  A lien coupled with a sheriff's
sale may occur sooner and thus have lower accumulated interest and
penalties, but the forced sale of one's home is likely to have very
high psychic costs.  Which of the two added penalties is larger, and
therefore likely to have a stronger impact on compliance, will depend
upon the circumstances of the individual delinquent taxpayer.
However, both letters should increase compliance over the holdout
cohort from the reminder effect on saliency and from the added
expected penalty, and both letters should increase compliance over the
reminder-only letter from the added expected penalty.

Our final four reminder letters test for the potential role of ``tax
morale'' motives for compliance.  An appeal to a tax morale is meant
to cue a possible benefit from having paid one's taxes, apart from the
actual receipt of services those payments may make possible.  In
contrast to user fees, property tax payments are not tied to the
citizen's receipt of particular services during our experimental
period.  In effect, each delinquent taxpayer is a free rider, and the
appeal to a tax morale for payment is meant to overcome such
self-interest.  In our model of taxpayer compliance these higher
motives are captured by $v$ in equation (\ref{eq9}), the morale benefits
from paying per dollar of taxes, interest and penalties paid.

We test for the importance of four such motives: 1) the value of
knowing one is a contributor to the immediate services of one's
neighborhood, $v_{N}$; 2) the value of knowing one is a contributor to
the wider services that benefit the city as a whole, $v_{C}$; 3) the
value of knowing one is part of a collective effort with other
taxpayers or ``peers'' in paying for city services, $v_{P}$; and 4)
the value of knowing one has meet one's obligations as a citizen in a
democracy, $v_{D}$.  Each of these benefits may motivate taxpayer
compliance, and our reminder letters are meant to trigger a possible
recognition of the importance of each motive.  Some delinquent
taxpayers may respond to one motive, some to another, and perhaps
others to none at all if the free-rider motive is decisive.  The four
tax morale reminder letters are:

\bigskip

\noindent \textit{Reminder Plus Appeal to Neighborhood Services}: We
want to remind you that your taxes pay for essential public services
in \textit{neighborhood name}, such as $<$List Two Local Amenities$>$,
your local police officer, snow removal, street repairs, and trash
collection.  \textbf{Please pay your taxes to help the city provide
  these services in your neighborhood.} \textbf{Our records indicate
  that you have a balance due of \textit{balance}.}

\bigskip

\noindent \textit{Reminder Plus Appeal to City-Wide Services}: Your
taxes pay for important services that make a city great. Your tax
dollars are essential for ensuring all Philadelphia's children receive
a quality education and all Philadelphians feel safe in their
neighborhoods.  \textbf{Please pay your taxes as soon as you can to
  help us pay for these important services.  Our records indicate that
  you have a balance due of \textit{balance}.}

\bigskip

\noindent \textit{Reminder Plus Appeal to Peer Behavior}: You have not
paid your Real Estate Taxes.  Almost all of your neighbors pay their
fair share: 9 out of 10 Philadelphians do so.  \textbf{By failing to
  pay, you are abusing the good will of your Philadelphia neighbors.
  Our records indicate that you have a balance due of
  \textit{balance}.}

\bigskip

\noindent \textit{Reminder Plus Appeal to Civic Duty}: For democracy
to work, all citizens need to pay their fair share of taxes for
community services.  \textbf{By failing to do so, you are not meeting
  your duty as a citizen of Philadelphia.  Our records indicate that
  you have a balance due of \textit{balance}.}

\bigskip

The morale benefits from knowing one has paid one's taxes equals a
weighted average of these motivations ($v$) plus a possible additional
weight ($v_{i}$) when one of the reminder letters reinforces or
enhances the affected benefit from tax payment: $v + \sum_{i}
\omega_{i} v_{i}$, where $i =$ N, C, P, or D, and where $\omega_{i} =
1$ if a reminder letter is received targeting benefit $i$, and $v_{i}$
is the additional weight given to that motivation. We take as evidence
that an increase in tax morale increases the likelihood of tax
compliance when a tax morale reminder letter increases the rate of
compliance above that of those receiving a reminder-only letter.  If
none of the tax morale letters impact compliance above a reminder-only
letter then, at least on the margin for paying the property tax, the
free-rider motivation is decisive for delinquent Philadelphia
taxpayers.  In this case, increased enforcement will need to appeal to
reminders and penalties.

  
\section{Randomization Procedure}

Randomization took place in two stages.  As a baseline control, we
randomly removed 3,000 delinquent properties from the possibility of
receiving any reminder letter at all.  These taxpayers (N=2,088)
became our holdout sample and allowed us to estimate the efficacy of
simply communicating with the taxpayer after the date that taxes are
due. We next grouped all remaining properties by owner and block
randomized all owners to treatments based on the total amount of
property taxes owed on all of their properties. Since most property
owners and delinquent property owners own only one property, our main
interest in this study is on unary owners, i.e. households that only
own one property in the city. Once we restrict attention to this
sample,we have 16,940 observations in the treatment group and 2,088
observations in the holdout sample.  The total sample size is
19,028.\footnote{We also trimmed the sample and excluded the 28 owners
  with highest total assessed property value. None of the findings
  reported in the paper depend on this trimming.}  Table \ref{balance}
checks whether the treatment and control group are balanced based on
the two most important variables, amount due and assessed property
value.

Table \ref{balance} shows that randomization was successful in the
unary owner sample.  The average debt owed by each owner was \$1,287
in the treatment group and \$1,233 in the holdout sample. The average
assessed property value is \$144,145 in the treatment group and
\$142,630 in the control group. As a further test of our randomization
procedure, we also checked to see whether randomization achieved
spatial uniformity throughout the geographic expanse of the city. As
reported in Table \ref{balance} geographic balance was achieved.

Next we test whether randomization was successful among the seven
experimental treatment groups. Table \ref{balance} shows the results
for the unary owner sample. Overall, we find no evidence that would
suggest any problems with randomization.



\begin{sidewaystable}[htbp]
\centering
\caption{Balance on Observables (Unary Owners)}\label{balance}
\vspace{10mm}
\begin{tabular}{lrrrrrrrrc}
\hline \hline Variable & Holdout & Control & Neighborhood & Community
& Duty & Peer & Lien & Sheriff & $p$-value \\ \hline Amount Due (June)
& \$1,233 & \$1,256 & \$1,289 & \$1,290 & \$1,299 & \$1,280 & \$1,280
& \$1,315 & 0.92 \\ Property Value & \$142,630 & \$158,370 & \$159,079
& \$130,265 & \$165,617 & \$130,936 & \$130,642 & \$134,334 & 0.53
\\ \hline Center City & 109 & 111 & 118 & 105 & 129 & 114 & 109 & 115
& 0.67 \\ Northeast Philadelphia & 352 & 427 & 397 & 399 & 394 & 427 &
383 & 370 & \\ North Philadelphia & 449 & 520 & 491 & 498 & 527 & 533
& 525 & 526 & \\ Northwest Philadelphia & 537 & 601 & 620 & 654 & 611
& 615 & 645 & 666 & \\ South Philadelphia & 210 & 211 & 242 & 234 &
248 & 241 & 253 & 239 & \\ West Philadelphia & 431 & 549 & 519 & 551 &
523 & 486 & 514 & 500 & \\ \hline \# Owners & 2,088 & 2,419 & 2,387 &
2,441 & 2,432 & 2,416 & 2,429 & 2,416 & \\ \hline
\multicolumn{10}{l}{\scriptsize{$p$-values in rows 1-2 are $F$-test
    $p$-values from regressing each variable on treatment dummies. A
    $\chi^2$ test was used for the geographic distribution.}} \\
\end{tabular}
\end{sidewaystable}

While the vast majority of properties in the city of Philadelphia are
owned by unary owners, approximately 10 percent of the properties are
owned by individuals or firms that own multiple properties. Since we
are interested in taxpayer compliance and not property compliance, we
identified owners of multiple delinquent properties by their legal
name and randomly assigned each owner to a treatment
group.\footnote{We lacked an objective identifier such as a social
  security.  There is some possibility that two or more different
  owners have the same name, but inspection by the authors found this
  to be very rare.  To the extent that it occurs, we consider this
  random noise to the experiment.} Any delinquent taxpayer holding
multiple properties within each treatment group received the same
letter for each of those properties.  Given the high correlation
between the propensity to pay taxes and total debt owed, randomization
blocks were defined according to owner-level total debt to assure
uniformity of samples along the dimension of debt owed. Each property
assigned to receive a reminder letter was equally likely to receive
each of the seven treatments. Results for multiple property owners,
which do not differ from results for unary property owners, are
reported in Table \ref{balance2} in the appendix.  Including the
holdout sample and excluding multiple owners gives us a sample size of
19,028 observations.\footnote{Unfortunately, we were not able to
  include the holdout sample in the block-randomization procedure. As
  a consequence, we can only include the holdout sample into our
  analysis if we condition on unary ownership.} Table \ref{balance2}
displays the balance tests for pre-randomization characteristics
\ref{balance}. It confirms that randomization was also successful in
this larger sample that included multiple property owners.  There are
no statistically significant differences across reminder letters.

\section{Empirical Results}

\subsection{Short Term Impact}

In this section we focus on the short-term impact of our intervention,
which we define as the first three months after our intervention
letters were posted. During this time period, tardy taxpayers were
only exposed to our intervention. As a consequence, our estimates of
the treatment effects are not contaminated by other interventions by
the tax authority.

To gain a more complete insight into the nature of tax compliance in
Philadelphia, we consider two distinct measures of tax compliance. We
define partial compliance as a tardy taxpayer making any real estate
tax payment at all.  Partial Compliance is of interest because even
small additional payments help, but perhaps more importantly, a tax
contribution represents a willingness by the taxpayer to be engaged
with city governance.  Further, it is common for late taxpayers to pay
down their debt gradually instead of in lump sum. The ever-paid
outcome in particular does not differentiate between taxpayers that
made full repayment and those who made only a partial contribution.
Full compliance is defined as eliminating real estate tax debt.

\begin{table}[htb]
\centering
\caption{Short-Term Linear Probability Model Estimates} \label{sh_lin}
\bigskip
\begin{tabular}{l c c c c }
\hline
 & \multicolumn{2}{c}{Ever Paid} & \multicolumn{2}{c}{Paid in Full} \\
          & One Month & Three Months & One Month & Three Months \\
Holdout   & $30.5$ & $51.4$ & $23.5$ & $40.8$ \\
\hline
Control   & $3.8^{***}$  & $3.9^{***}$  & $2.2^{*}$    & $3.0^{**}$   \\
          & $(1.4)$      & $(1.5)$      & $(1.3)$      & $(1.5)$      \\
Neighborhood & $1.7$        & $2.7^{*}$    & $-0.2$       & $1.5$        \\
          & $(1.4)$      & $(1.5)$      & $(1.3)$      & $(1.5)$      \\
Community     & $3.8^{***}$  & $2.8^{*}$    & $1.3$        & $2.5^{*}$    \\
          & $(1.4)$      & $(1.5)$      & $(1.3)$      & $(1.5)$      \\
Duty      & $2.4^{*}$    & $3.6^{**}$   & $0.7$        & $2.3$        \\
          & $(1.4)$      & $(1.5)$      & $(1.3)$      & $(1.5)$      \\
Peer      & $3.9^{***}$  & $3.5^{**}$   & $1.8$        & $3.4^{**}$   \\
          & $(1.4)$      & $(1.5)$      & $(1.3)$      & $(1.5)$      \\
Lien      & $9.0^{***}$  & $9.2^{***}$  & $5.6^{***}$  & $7.2^{***}$  \\
          & $(1.4)$      & $(1.5)$      & $(1.3)$      & $(1.5)$      \\
Sheriff   & $7.4^{***}$  & $8.8^{***}$  & $4.5^{***}$  & $6.8^{***}$  \\
          & $(1.4)$      & $(1.5)$      & $(1.3)$      & $(1.5)$      \\
\hline
Num. obs. & 19028        & 19028        & 19028        & 19028        \\
\hline
\multicolumn{5}{l}{\scriptsize{$^{***}p<0.01$, $^{**}p<0.05$,
    $^*p<0.1$. Holdout values in levels; remaining figures relative to
    this}}
\end{tabular}
\end{table}

We start by considering the partial compliance results that pertain to
the sample in which we exclude owners of multiple properties.  Table
\ref{sh_lin} reports the estimated participation rates in the holdout
sample as well as the differences in participation among the seven
treatment samples. Standard errors clustered within randomization
blocks are reported in parentheses. We
find that all seven treatments increased partial compliance at the
one- and three-month snapshots. Almost all of these increases in
compliance behavior are statistically significant at standard levels
of significance.

After one month, approximately 30 percent of all taxpayers in the
holdout sample had made some contribution towards their tax
liabilities. In contrast, 40 percent of those that received the lien
letter and 37 percent of those that received the sheriff's sale
letters had made payments. The results are similar after three months
of the intervention.  The overall participation rate rose, with 51
percent of all owners in the holdout sample having made some
contribution after three months.  This is in turn dwarfed by the 61
percent of households that received the lien letter and 60 percent of
households that received the sheriff's sale letter that made some
payments in the same interval.

As shown in Table \ref{sh_lin} the results are qualitatively and
quantitatively the same if we use ``paid in full'' as our compliance
outcome. The main difference is that the neighborhood, community and
duty letters lead to a significant increase in compliance relative to
the holdout group only for partial compliance, though the sign of the
estimate is mostly the same. All other findings are similar. As a
robustness check we also estimated Logit models shown in the Appendix.
Not surprisingly, the main findings are qualitatively and quantitatively 
the same.

Next we conduct some simple back-of-the-envelope calculations to
assess the impact of these estimates on revenues. Here we focus on the
results after three months. We take the median nonzero eventual
payment (i.e., the median positive remission by year's end) in each
subsample and multiply the median payment with the increase in the
compliance probability reported in Table \ref{sh_lin}. This product
can be interpreted as the impact of each treatment on revenue per
letter. To obtain the total estimated impact we then multiply the
impact per letter with the total number of individuals in each
treatment. These results are reported in Table \ref{sh_rev}.  Overall,
we find that all seven treatments generated positive revenues for the
City.  The three-month impact of these letters ranged between \$21 for
the neighborhood letter and approximately \$76 per the lien letter. We
thus conclude that receiving a reminder letter improved taxpayer
compliance in the short run, and further that the lien and sheriff's
sale letters were the most effective in inducing repayment.

\begin{table}[htbp]
\caption{Estimated Three-Month Impact on Revenue}\label{sh_rev}
\bigskip
\centering
\begin{tabular}{lcc}
  \hline
Treatment & Impact Per Letter & Total Impact \\ 
  \hline
Control & \$32.51 & \$78,634 \\ 
  Neighborhood & \$22.32 & \$53,287 \\ 
  Community & \$23.61 & \$57,623 \\ 
  Duty & \$30.05 & \$73,084 \\ 
  Peer & \$28.95 & \$69,955 \\ 
  Lien & \$76.4 & \$185,580 \\ 
  Sheriff & \$73.27 & \$177,020 \\ 
   \hline
\end{tabular}
\end{table}

Another way to determine the revenue implications of our different
treatments is to regress the total amount of revenue raised on
indicator variables for each treatment.  These regressions confirm our
estimates reported in Table \ref{sh_rev}. We find that the average
payments in the holdout sample were \$323 after one month and \$636
after three months. All of our letters, including the control
treatment, increased payments at the one- and three-month
cross-sections. The two threat letters were the only two letters that
significantly increased revenue collection. After one month the lien
(sheriff) treatment increased payments by \$90 (\$69). After three
months, the increases are approximately \$97 per letter for both of
these treatments.  This supports our assertion that the estimates
reported in Table \ref{sh_rev} are conservative estimates of the
effectiveness of our treatments.


\begin{table}[htbp]
\caption{Robustness Analysis: Multiple Owners}\label{sh_lin_rob}
\begin{center}
\begin{tabular}{l c c c c }
\hline
 & \multicolumn{2}{c}{All Owners} & \multicolumn{2}{c}{Single-Property Owners} \\
          & One Month & Three Months & One Month & Three Months \\
\hline
Neighborhood & $-0.01$      & $-0.01$      & $-0.02$      & $-0.01$      \\
          & $(0.01)$     & $(0.01)$     & $(0.01)$     & $(0.01)$     \\
Community     & $-0.00$      & $-0.01$      & $0.00$       & $-0.01$      \\
          & $(0.01)$     & $(0.01)$     & $(0.01)$     & $(0.01)$     \\
Duty      & $-0.01$      & $-0.00$      & $-0.01$      & $-0.00$      \\
          & $(0.01)$     & $(0.01)$     & $(0.01)$     & $(0.01)$     \\
Peer      & $0.00$       & $-0.01$      & $0.00$       & $-0.00$      \\
          & $(0.01)$     & $(0.01)$     & $(0.01)$     & $(0.01)$     \\
Lien      & $0.05^{***}$ & $0.05^{***}$ & $0.05^{***}$ & $0.05^{***}$ \\
          & $(0.01)$     & $(0.01)$     & $(0.01)$     & $(0.01)$     \\
Sheriff   & $0.03^{**}$  & $0.05^{***}$ & $0.04^{**}$  & $0.05^{***}$ \\
          & $(0.01)$     & $(0.01)$     & $(0.01)$     & $(0.01)$     \\
\hline
Num. obs. & 19333        & 19333        & 16940        & 16940        \\
\hline
\multicolumn{5}{l}{\scriptsize{$^{***}p<0.001$, $^{**}p<0.05$, $^*p<0.1$}}
\end{tabular}
\end{center}
\end{table}


Finally, we conducted a number of robustness checks. Recall that we
randomized the seven treatments at the ownership level. In Table
\ref{sh_lin_rob} we replicate the analysis done above, excluding the
holdout sample and expressing estimates relative to the control
treatment. We then estimate the model using the larger sample that
also includes owners of multiple properties. Overall, we find the
results are similar to the ones reported in Table \ref{sh_lin}. If
anything, the treatment effects are stronger in the unary owner
sample. This evidence leads us to conclude that owners of multiple
properties are less likely to respond the kind of nudge strategies
explored in this paper.

Finally, we estimated all models using Logit models instead of linear
probability models. The results are reported in Tables \ref{sh_logit}
and \ref{sh_logit_rob} in the Appendix. Overall, the results are
qualitatively and quantitatively very similar.

\subsection{Long-Term Impact}

Recall that all tardy tax payers were assigned to a collection agency
in the middle of August 2015 and were thereby subjected to another
enforcement activity which largely consisted of another ``threat''
treatment -- phone calls that threatened them with penalties and fines
to coax compliance.  While our experimental design is still valid for
these outcomes, it is harder to cleanly interpret the findings due to
this mixed bag of interventions.  All taxpayers that had not paid by
mid-August were subject to this uniform second treatment by a
collection agency.  Our estimates of the longer-term treatment effects
thus reflect two treatments: our initial letter treatment plus the
phone calls performed by the collection agency. Since we were not able
to randomize on the treatment by the collection agency, we can only
identify the effect of the joint treatments.

\begin{table}[htb]
\caption{Long-Term Linear Probability Model Estimates}
\begin{center}
\begin{tabular}{l c c c c }
\hline
 & \multicolumn{2}{c}{Ever Paid} & \multicolumn{2}{c}{Paid in Full} \\
          & Six Months & Tax Year 2016 & Six Months & Tax Year 2016 \\
Holdout   & $73.3$ & $65.5$ & $63.2$ & $52.5$ \\
\hline
Control   & $1.3$        & $-1.4$       & $1.5$        & $-0.7$       \\
          & $(1.3)$      & $(1.4)$      & $(1.4)$      & $(1.5)$      \\
Neighborhood & $-0.2$       & $-3.1^{**}$  & $-0.0$       & $-2.2$       \\
          & $(1.3)$      & $(1.4)$      & $(1.4)$      & $(1.5)$      \\
Community     & $0.9$        & $-1.8$       & $1.1$        & $-2.0$       \\
          & $(1.3)$      & $(1.4)$      & $(1.4)$      & $(1.5)$      \\
Duty      & $2.1$        & $-1.6$       & $1.0$        & $-1.9$       \\
          & $(1.3)$      & $(1.4)$      & $(1.4)$      & $(1.5)$      \\
Peer      & $1.3$        & $-1.9$       & $2.3$        & $-1.1$       \\
          & $(1.3)$      & $(1.4)$      & $(1.4)$      & $(1.5)$      \\
Lien      & $3.8^{***}$  & $-0.9$       & $4.8^{***}$  & $-0.7$       \\
          & $(1.3)$      & $(1.4)$      & $(1.4)$      & $(1.5)$      \\
Sheriff   & $3.8^{***}$  & $-0.6$       & $3.0^{**}$   & $-1.1$       \\
          & $(1.3)$      & $(1.4)$      & $(1.4)$      & $(1.5)$      \\
\hline
Num. obs. & 19028        & 19025        & 19028        & 19025        \\
\hline
\multicolumn{5}{l}{\scriptsize{$^{***}p<0.01$, $^{**}p<0.05$, $^*p<0.1$.
 Holdout values in levels; remaining figures relative to this}}
\end{tabular}
\label{lg_pc_lin}
\end{center}
\end{table}


Given the similarity of results, we focus on the unary owner sample.
Table \ref{lg_pc_lin} reports the estimated participation rates in the
holdout sample as well as the differences in participation in the
seven treatment samples after six months of the intervention.  We find
that only two of the treatments increased partial compliance at the
six-month juncture.  We find that 73 percent of households in the
holdout sample made some payments to the City. The only letters that
significantly improve the compliance above or below that rate were the
lien and sheriff letters, which increased compliance by 3 to 4
percentage points. The findings are similar using full repayment as
the outcome measure.

To translate these participation rates into revenue, we again take the
median nonzero payment in each subsample and multiply it with the
increase in compliance probability reported in Table \ref{lg_pc_lin}.
The results are reported in Table \ref{lg_rev}. We find that the
six-month impact of these two letters was approximately \$31 per
letter relative to the holdout sample and \$21 relative to the control
letter.\footnote{Again, we also ran a revenue regression as a
  robustness check, but it was inconclusive. Only the duty treatment
  was marginally significant, which was due to the influence of a few
  outliers.}

\begin{table}[htbp]
\caption{Estimated Six Month Impact on Revenue} \label{lg_rev}
\bigskip
\centering
\begin{tabular}{lcc}
  \hline
Treatment & Impact Per Letter & Total Impact \\ 
  \hline
Control & \$10.74 & \$25,976 \\ 
  Neighborhood & -\$1.47 & -\$3,517 \\ 
  Community & \$7.2 & \$17,568 \\ 
  Duty & \$17.32 & \$42,134 \\ 
  Peer & \$11.16 & \$26,972 \\ 
  Lien & \$31.42 & \$76,328 \\ 
  Sheriff & \$31.43 & \$75,940 \\ 
   \hline
\end{tabular}
\end{table}

We find that the impact of our six letters relative to the holdout or
control group is much attenuated after six months. This finding is
consistent with the view that the enforcement activities of the
collection agency are probably more closely aligned with our
deterrence letters than with the other five letters that we
explored. As a consequence, it is not surprising that we find some
strong convergence in the effectiveness of all treatments after six
months -- as the threat-like approach of the agency comes to the fore,
the much earlier receipt of our treatment has a dwindling impact.
However, it is also possible that some convergence of the
effectiveness of treatment would have resulted even in the absence of
the introduction of the collection agency. Since we were not allowed
to randomize on that treatment, we cannot offer a definitive
conclusion.

For revenue collecting agencies, particularly those with revenue
collection problems, acceleration of payment can be understood to be a
useful result in and of itself. Bills must be paid and debts must be
serviced on regular schedules. Relatedly, the longer that tax bills
remain unpaid, the more expensive it becomes to collect. Whether
handled internally or externally through debt collection firms,
downstream collection practices leave diminished revenues. For both of
these reasons, early collection is, \textit{ceteris paribus}, better
collection. This is not to say, however, that early collection is
social-welfare-improving. If tardy-but-eventually-compliant taxpayers
forestall early payment to cover other expenses or invest
in other assets which they eventually use to repay their tax bill with
interest, late repayment may have been individually optimal, 
in which case the welfare of the tax agency may not be synonymous
with the welfare of society. This is especially true if tax payments
are made weeks rather than months late, such that monthly payments can
still be made based on expected monthly receipts.  In the case
of chronically-delayed but consistently-paid payments, it is less
obvious that accelerating eventual or inevitable payment constitutes
something of value.

We also obtained data characterizing tax compliance of our sample of
taxpayers in the tax year 2016 which followed our 2015 intervention.
The goal was to investigate the existence of any spill-over effects of
our treatments into the next tax year. Again, our analysis is subject
to the same mulled-treatment constraint discussed above.  Focusing
again on the unary owner sample, the second and fourth columns of
Table \ref{lg_pc_lin} summarize our main findings. Overall, we find
that none of our treatments had any continued differential impact on
tax compliance in 2016. However, it is worth noting that the overall
compliance rate in 2016 was significantly higher than that in 2015,
which may be a direct consequence of our increased enforcement
activities.


\section{Conclusions}


Collection of the property tax has been problematic in many
U.S. cities.  Noncompliance a significant concern since local
governments are denied the revenues needed to provide basic public
services essential for ensuring the safety, health, and minimal
well-being of all citizens. If there is significant non-compliance and
basic services are to be provided, then tax rates will need to rise on
those who pay taxes. Non-compliance also undermines the principle that
everyone has to pay their ``fair share" of taxes. Given the
administrative advantages of the property tax, payment is primarily a
matter of enforcement.  Despite the fact that there are no obvious
financial gains to not paying property taxes, we observe that a
significant fraction of tax payers do not pay on time. We follow
Akerlof (1991) and O'Donoghue and Rabin (1999) and model tarty or
delinquent tax payers as procrastinators. Payment delay occurs in our
model because of present bias and declining saliency.

We then designed and implemented a new, multi-arm field experiment in
the City of Philadelphia.  Within the structure of our experiment, we
can identify and estimate the relative importance of the key
parameters within our model.  Note that present bias is always present
and helps to distinguish regular tax ayers from delinquent tax payers.
Our experiment has, therefore, focused on first time delinquent tax
payers.  Our model suggests that saliency can be nudged by a reminder
letter.  Saliency can explain the large and significant differences in
payment rates between individuals in our hold-out sample and those
that received any type of reminder letter. The additional gains of
letters that stress tax moral over neutral reminder letters are
insignificant. We modeled these reminder letters as utility gains to
the procrastinator from paying ones taxes. Our empirical results are
consistent with the view that there are no additional utility gains
form stressing various aspects of tax moral. In contrast, the reminder
letters stressing liens or liens and the sale of one's home add a
future expected cost to non-payment as in the tax compliance model of
Allingham and Sandmo (1972).  We find that these reminder letters
consistently outperform simple reminder letters in the short and the
long run.  We find that these letters are also effective, but the
gains are smaller than the letters that reinforce deterrence. Finally,
we found that, conditional on any payment, most tardy tax payers
became near perfect compliers, making full payments in almost all
cases. This suggests that, at least for the margin affected, liquidity
constraints are not the primary reason for initial non-payment.

Our results clearly suggest that many tax delinquents are struggling
with making timely payments, which may be partially due to a lack of
salience. The provision of detailed information had a clear impact on
the likelihood of repayment. Our main result echoes other recent
findings that clear, consistent, and timely provision of information
on consequences, particularly in the context of compliance monitoring,
can lead to notable improvements in behavioral compliance
\cite{Bhargava-15}. The puzzle of high non-payment rates can also be
understood as a case of under-enforcement.

Our results suggest that a variety of treatments may be successful in
increasing tax compliance and raising revenue for any city that is
struggling with collecting property taxes. Hence, a revenue director
can choose from a menu of effective messages to increase tax
compliance and revenue collection. There are, however, some important
quantitative differences in the effectiveness of the different
messages, which imply some trade-offs faced by the revenue director.
If a revenue director wants to generate more property tax collection
in a setting such as Philadelphia, it appears that he or she should
choose a tougher and politically more costly message.  Compared to the
years before the experiment, in which roughly 24,000 accounts were
sent for action by outside revenue collections firms (23,187--FY'14
and 24,922--FY'15), in the year of the experiment (FY'16) only 18,004
accounts were sent for collection. In 2016 the City of Philadelphia
decided to send a version of our sheriff's sale letter to all late tax
payers in the summer of 2016.\footnote{The appendix shows the letter
  used by the city which is almost identical to the letter that we
  used in our experiment.} Our experiment, therefore, changed the
equilibrium and convinced the City of Philadelphia to adopt a more
promising enforcement strategy.  It is reasonable to conjecture that
other U.S. cities that have struggled with collecting property taxes,
could benefit from the insights of this paper.

\newpage

%{\footnotesize \NoTitleCaseChange\citepunct{(}{and}{, }{; }{, }{)}{}{.} 

\bibliographystyle{theapa}
\bibliography{references}

\bigskip

\bigskip

\bigskip

\newpage

\begin{appendix}

\section{Appendix: Additional Figures and Tables}

The appendix contains Tables \ref{sh_logit} and \ref{sh_logit_rob}
which report estimates based on Logit models.  It also contains Table
\ref{balance2} which summarizes additional balance tests. Finally, we
include the original reminder letter that was used by the city prior
to our experiment in Figure 1 and the letter that was adopted by the
city in 2016 in Figure 2.



\begin{table}[htb]
\caption{Short-Term Logistic Model Estimates}\label{sh_logit}
\begin{center}
\begin{tabular}{l c c c c }
\hline
 & \multicolumn{2}{c}{Ever Paid} & \multicolumn{2}{c}{Paid in Full} \\
 & One Month & Three Months & One Month & Three Months \\
Holdout        & $-0.8$ & $0.1$       & $-1.2$ & $-0.4$ \\
\hline
Control        & $0.2^{***}$  & $0.2^{***}$ & $0.1^{*}$    & $0.1^{**}$   \\
               & $(0.1)$      & $(0.1)$     & $(0.1)$      & $(0.1)$      \\
Neighborhood   & $0.1$        & $0.1^{*}$   & $-0.0$       & $0.1$        \\
               & $(0.1)$      & $(0.1)$     & $(0.1)$      & $(0.1)$      \\
Community      & $0.2^{***}$  & $0.1^{*}$   & $0.1$        & $0.1^{*}$    \\
               & $(0.1)$      & $(0.1)$     & $(0.1)$      & $(0.1)$      \\
Duty           & $0.1^{*}$    & $0.1^{**}$  & $0.0$        & $0.1$        \\
               & $(0.1)$      & $(0.1)$     & $(0.1)$      & $(0.1)$      \\
Peer           & $0.2^{***}$  & $0.1^{**}$  & $0.1$        & $0.1^{**}$   \\
               & $(0.1)$      & $(0.1)$     & $(0.1)$      & $(0.1)$      \\
Lien           & $0.4^{***}$  & $0.4^{***}$ & $0.3^{***}$  & $0.3^{***}$  \\
               & $(0.1)$      & $(0.1)$     & $(0.1)$      & $(0.1)$      \\
Sheriff        & $0.3^{***}$  & $0.4^{***}$ & $0.2^{***}$  & $0.3^{***}$  \\
               & $(0.1)$      & $(0.1)$     & $(0.1)$      & $(0.1)$      \\
\hline
AIC            & 24493.1      & 26068.9     & 21605.6      & 26093.5      \\
BIC            & 24556.0      & 26131.7     & 21668.4      & 26156.3      \\
Log Likelihood & -12238.6     & -13026.4    & -10794.8     & -13038.7     \\
Deviance       & 24477.1      & 26052.9     & 21589.6      & 26077.5      \\
Num. obs.      & 19028        & 19028       & 19028        & 19028        \\
\hline
\multicolumn{5}{l}{\scriptsize{$^{***}p<0.01$, $^{**}p<0.05$,
    $^*p<0.1$. Holdout values in levels; remaining figures relative to
    this}}
\end{tabular}
\end{center}
\end{table}

\newpage

\begin{table}
\caption{Logit Estimates Including Multiple Owners}\label{sh_logit_rob}
\begin{center}
\begin{tabular}{l c c c c }
\hline
 & \multicolumn{2}{c}{All Owners} & \multicolumn{2}{c}{Unary Owners} \\
 & One Month & Three Months & One Month & Three Months \\
\hline
Neighborhood   & $-0.05$      & $-0.03$      & $-0.09$      & $-0.05$      \\
               & $(0.06)$     & $(0.05)$     & $(0.06)$     & $(0.06)$     \\
Community      & $-0.02$      & $-0.06$      & $0.00$       & $-0.04$      \\
               & $(0.06)$     & $(0.05)$     & $(0.06)$     & $(0.06)$     \\
Duty           & $-0.06$      & $-0.01$      & $-0.06$      & $-0.01$      \\
               & $(0.06)$     & $(0.05)$     & $(0.06)$     & $(0.06)$     \\
Peer           & $0.01$       & $-0.03$      & $0.01$       & $-0.02$      \\
               & $(0.06)$     & $(0.05)$     & $(0.06)$     & $(0.06)$     \\
Lien           & $0.21^{***}$ & $0.20^{***}$ & $0.23^{***}$ & $0.22^{***}$ \\
               & $(0.06)$     & $(0.05)$     & $(0.06)$     & $(0.06)$     \\
Sheriff        & $0.15^{**}$  & $0.19^{***}$ & $0.16^{**}$  & $0.20^{***}$ \\
               & $(0.06)$     & $(0.05)$     & $(0.06)$     & $(0.06)$     \\
\hline
AIC            & 25179.24     & 26349.91     & 21922.44     & 23174.00     \\
BIC            & 25234.33     & 26405.00     & 21976.61     & 23228.16     \\
Log Likelihood & -12582.62    & -13167.95    & -10954.22    & -11580.00    \\
Deviance       & 25165.24     & 26335.91     & 21908.44     & 23160.00     \\
Num. obs.      & 19333        & 19333        & 16940        & 16940        \\
\hline
\multicolumn{5}{l}{\scriptsize{$^{***}p<0.001$, $^{**}p<0.05$, $^*p<0.1$}}
\end{tabular}
\end{center}
\end{table}


\begin{sidewaystable}[htbp]
\centering
\caption{Balance on Observables (cont.)}\label{balance2}
\bigskip
\begin{tabular}{lrrrrrrrc}
\hline
\multicolumn{9}{c}{Unary Owners} \\
\hline
Variable & Control & Neighborhood & Community & Duty & Peer & Lien & Sheriff & $p$-value \\ 
\hline
Amount Due (June) & \$1,256 & \$1,289 & \$1,290 & \$1,299 & \$1,280 & \$1,280 & \$1,315 & 0.98 \\ 
Assessed Property Value & \$158,370 & \$159,079 & \$130,265 & \$165,617 & \$130,936 & \$130,642 &
 \$134,334 & 0.46 \\ 
\# Owners & 2,419 & 2,387 & 2,441 & 2,432 & 2,416 & 2,429 & 2,416 & 0.99 \\ 
\hline
\multicolumn{9}{c}{Unary and Multiple Owners} \\
\hline
Variable & Control & Neighborhood & Community & Duty & Peer & Lien & Sheriff & $p$-value \\ 
\hline
Amount Due (June) & \$1,593 & \$1,589 & \$1,583 & \$1,583 & \$1,572 & \$1,593 & \$1,590 & 1 \\ 
Assessed Property Value & \$180,664 & \$180,172 & \$153,528 & \$183,991 & \$155,438 & \$155,499 & \$157,398 & 0.48 \\ 
\% with Unary Owner & 87.6 & 86.4 & 88.4 & 88.1 & 87.5 & 88.0 & 87.5 & 0.42 \\ 
\% Overlap with Holdout & 3.69 & 3.73 & 3.40 & 3.40 & 3.55 & 3.44 & 3.29 & 0.97 \\ 
\# Properties per Owner & 1.27 & 1.32 & 1.26 & 1.26 & 1.26 & 1.26 & 1.26 & 0.67 \\ 
\# Owners & 2,762 & 2,762 & 2,762 & 2,762 & 2,762 & 2,761 & 2,762 & 1 \\ 
\hline
\multicolumn{9}{l}{\scriptsize{$p$-values in rows 1-5 are $F$-test
    $p$-values from regressing each variable on treatment dummies. A
    $\chi^2$ test was used for the count of owners.}} \\
\end{tabular}
\end{sidewaystable}


\end{appendix}

\begin{figure}
\begin{center}
\includegraphics[width=6in, height=8.5in]{PastDueLetter.pdf}
\end{center}
\end{figure}

\begin{figure}
\begin{center}
\includegraphics[width=6in, height=8.0in]{2016_letter.pdf}
\end{center}
\end{figure}

\end{document}
