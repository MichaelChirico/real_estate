\documentclass[12pt]{article}
\usepackage{amssymb}
\usepackage{theapa}
\usepackage{titlepage}
\usepackage{amsmath}
\usepackage{setspace}
\usepackage[dvips]{graphicx}
\usepackage{rotating}
\usepackage[usenames,dvipsnames]{pstricks}
\usepackage{epsfig}
\usepackage{pst-grad}
\usepackage{pst-plot}
\usepackage{color}
\usepackage{pstricks-add}
\usepackage{rotating}
\usepackage{threeparttable}
 \usepackage{array,multirow}
\usepackage{pdflscape}
\usepackage{float,lscape}
\usepackage{csquotes}


\renewcommand{\baselinestretch}{1.5}
\parindent=.2in
\evensidemargin=.05in 
\oddsidemargin=-.05in 
\topmargin=-0.05in
\textwidth=6.5in 
\textheight=8.5in

\newtheorem{fact}{Stylized Fact}
\newtheorem{theorem}{Theorem}
\newtheorem{corollary}{Corollary}
\newtheorem{definition}{Definition}
\newtheorem{lemma}{Lemma}
\newtheorem{prop}{Proposition}
\newtheorem{assumption}{Assumption}
\newtheorem{remark}[theorem]{Remark}
\newtheorem{solution}[theorem]{Solution}
\renewcommand{\thefootnote}{\fnsymbol{footnote}}

\begin{document}

\title{Deterring Delinquency: A Field Experiment in Improving Tax Compliance Behavior}

\author{Michael Chirico, Robert Inman, Charles Loeffler, \\ John
  MacDonald, and Holger Sieg\thanks{We would like to thank Rob Dubow
    (Director of Finance), Clarena Tolson (Revenue Commissioner), and
    Marisa Waxman (Deputy Commissioner for Assessment of Properties)
    in the Department of Revenue of the City of Philadelphia for their
    help and support, and the Wharton Initiative for Public Policy for
    funding this experiment. We would also like to thank Jeff Brown,
    Chris Sanchirico, Wolfgang Sch\"on, Reed Shuldiner and
    participants of numerous seminars for comments and
    suggestions. The views expressed here are those of the authors and
    do not necessarily represent or reflect the views of the City of
    Philadelphia.}  \\ University of Pennsylvania \\
    \\
    PRELIMINARY DRAFT}

\date{\today}

\maketitle

\begin{abstract}

Property taxation plays a central role in the financing of municipal government services. Local taxing authorities commonly confront problems of property tax collection even when the tax base is known.  Using a multi-arm RCT conducted with the City of Philadelphia, we compare the responses of taxpayers to seven different notifications
reflecting a wide variety of theorized motivations for tax compliance. We find that  all our nudge strategies significantly outperform the ``do nothing'' alternative both in the rate of taxpayer compliance and in the level of payments.  Among the seven nudges, the most effective in encouraging tax payment are the two economic strategies that threaten large financial (``lien'') or financial and psychic (``sheriff's sale") penalties.  Overall, the rate of return on these nudge strategies are large. After six month, that letter induced \$31 more in payment per letter than doing nothing and \$21 more than the control letter.

\noindent KEYWORDS: Tax Compliance, Property Taxation, Field
Experiment, Deterrence, Public Service Appeal, Appeal to Civic Duty.

\end{abstract}
\renewcommand{\thefootnote}{\arabic{footnote}}

\newpage

\section{Introduction}

Property taxation plays a central role in the financing of municipal
government services in the United States.  The potential economic
advantages and disadvantages of a local property tax are well known.
With mobile households and local zoning the tax approximates a benefit
tax for the financing of local services.  In large cities, where these
assumptions are unlikely to hold, the tax will have adverse incentive
effects on new construction and improvements in growing cities but
approximates a land tax in its incentive effects in declining or
stable cities with modest new construction \cite{Aaron-75}.  As a tax
on value, the property tax is a proportional tax on wealth and thus
attractive from the perspective of lifetime equity
\cite{Mieszkowski-72}.  As a matter of practical local government
finance, it is the mainstay of city and school district budgets.  In
2013, over 72 percent of all local government tax revenues and nearly
50 percent of all own revenues came from property taxation.

For all these virtues to be realized, it is essential that the
property tax be collected, both efficiently and fairly.  Among large
US cities, this is not assured.  Because of the importance of the tax
to city budgets, even small differences in collection rates can
significantly affect the provision of local public services. While the
average rate of tax collection among a sample of large U.S. cities is
95 percent within the year of tax assessments, many cities collect
only 90 percent of taxes due, and several cities do far worse.  Among
the poorest performers are Cleveland (84\%), Detroit (68\%), Flint
(64\%), Milwaukee (86\%), and Pittsburgh (85\%). While the poorest
performers are all high poverty cities, there are many high poverty
cities that in fact collect most all of their property taxes; for
example, Baltimore (96\%), Birmingham (98\%), Dallas (98\%), Houston
(98\%), Minneapolis (98\%), and even New Orleans (95\%)
\cite{CILMS-16}.  It is not the city's underlying economy that fully
explains the rate of property tax compliance.  This paper suggests
that tax collection strategies will matter and examines the efficacy
of seven ``nudges" for improved property tax collection in one major
US city, Philadelphia.

The collection of property taxes has one very important administrative
advantage over the collection of other taxes: the legal tax obligation
is known to the taxpayer and the taxing authorities.  Self-reporting
of tax bases, as required for income, profits, sales, and VAT
taxation, is not needed for the property tax. Each taxpayer has an
assigned tax base, the value of property, against which a common tax
rate is assessed.  This avoids problems of misreporting tax bases or
working outside the formal (i.e., taxable) economy.\footnote{See
  \citeA{Blumenthal-01}, \citeA{Kleven-11}, and \citeA{Pomeranz-15}.}
The administrative issue for the property tax is simple: Did the
taxpayer pay the tax on time or not?  If not, then what can the tax
administrator do to enforce compliance?  Seven ``nudge" strategies are
evaluated here for their ability to improve compliance; each appeals
to a different motive for delinquent behaviors.

Our first nudge strategy is a simple reminder letter to the taxpayer
that their taxes are due; the letter is identical in content to the
initial tax bill listing the tax due and (now) any penalties for late
payments.  The reminder letter will be our ``control nudge" and is
meant to address non-payment due to forgetfulness or oversight.

The second set of strategies are meant to address an economic motive
for non-payment.  The delinquent taxpayer is assumed to be making an
economic calculation that by not paying there is a positive
probability that delinquency will go undetected or if detected,
ignored for administrative reasons, and that the expected economic
gains of not paying exceed the expected economic costs of being caught
and fined \cite{Allingham-Sandmo-72}.  In most real world tax
settings, however, the probability of being caught and the size of the
likely sanction are both too low to rationally account for most
observed levels of taxpayer noncompliance \cite{Alm-92}. \footnote{An
  alternative specification for taxpayer utility that allows for loss
  aversion has done a better job in explaining taxpayer compliance
  among Swedish taxpayers than did the classical expected utility
  specification with always declining marginal utilities in income;
  see \citeA{Engstrom-15}. } Here we test for the effect of two nudges
with potentially large economic consequences, one where delinquent
taxes plus a graduated fine growing over time are collected as a
``lien'' on the property at sale, and a second, where the property
is seized for a ``sheriff's sale'' with a portion of the
proceeds used to pay delinquent taxes and penalties. The lien imposes
a growing real dollar future loss on the delinquent taxpayer as the
interest rate for penalties exceeds the taxpayer's alternative rate
of return. The sheriff's sale imposes an immediate economic loss,
but further, requires the delinquent taxpayer to find a new residence.
Both of these nudges threaten large economic, and in the case of the
sheriff's sale large psychic, costs for continued noncompliance.

Two additional nudges appeal to what \citeA{Luttmer-14} have called
``tax morale."  First, we remind taxpayers that their payments do
provide valuable public services.  Here we seek to address
noncompliance due to a desire by taxpayers to free-ride on the
payments of their neighbors or of Philadelphians generally.  The first
free-rider strategy seeks to motivate payment by reminding the
taxpayer that his payments go to providing services for his family and
his immediate neighbors and lists specific neighborhood amenities
likely to be affected; we call this strategy the ``neighborhood"
nudge.  The second free-rider strategy reminds the taxpayer that his
taxes support important city-wide services such as education and city
safety. We call this strategy the ``community" nudge.
  
A final set of nudges appeals to a possibly deeper motive for tax
compliance, fulfilling one's obligations to a self-identified
community of peers \cite{Posner-00} or to an abstract community of
citizens \cite{Rawls-71}.  The first of these community strategies we
call the ``peer" nudge.  The second we call the ``civic duty" nudge.
  
 \noindent COMMENT: NEED TO UPDATE DISCUSSION OF RESULTS AND DIFFERENTIATE BETWEEN SHORT AND LONG TERM RESULTS
  
The seven nudge strategies for increased taxpayer compliance are first
compared to the alternative of doing nothing beyond sending the first
tax bill.  We then compare the seven nudge strategies among themselves
to see which are most effective in encouraging taxpayer compliance.
We find, first, that  most of our nudge strategies significantly
outperform the ``do nothing'' alternative both in the rate of
taxpayer compliance and in the level of payments, conditional on
compliance.  Second, among the seven nudges, the most effective in
encouraging tax payment are the two economic strategies that threaten
large financial (``lien'') or financial and psychic (sheriff's
sale) penalties.  Third, the level of taxpayer compliance improved
over time, with no further reminders.  After one month, approximately
36 percent of all taxpayers receiving a letter had made some
contribution towards their tax liabilities (39 percent for the lien
and sheriff sale letters); 55 percent had made some contribution after
three months (61 percent for the lien and sheriff sale letters); and
75 percent had made some contribution after six months (78 percent for
lien and sheriff sale letters) In contrast, the rate for some
compliance for those receiving no reminder letter at all was 30
percent after one month, 51 percent after three months, and 73 percent
after six months.  Receiving a reminder letter improved both the rate
and timing of taxpayer compliance, with the lien and sheriff sale
letters the most effective.
  
Reminder letters also improved the level of tax payments, given that
the taxpayer complied. After six months, the average payment from those
receiving a reminder, and paying, was higher than those without a
letter.  Among the seven reminders letters, the letter that induced
the greatest payment, given that a payment was made, was the lien
letter.  After six month, that letter induced \$31 more in payment per letter
than doing nothing and \$21 more than the control letter.
  
The total cost of the seven reminder
letters mailed to the owners of the 24,264 delinquent properties in
our experiment was \$16,000.  Our of the envelop calculations suggest that our experiment 
generated approximately \$257,496. We estimate that if extended to the whole of our sample of delinquent
taxpayers, the two most effective nudges -- the lien or sheriff
sales reminders -- have the potential to increase collected revenues
during each tax year by as much as 2 million dollars. 

The rest of the paper is organized as follows. Section 2 contains a brief literature review and discusses
how our paper is related to the existing literature. Section 3 discusses details of our field experiment including a detailed description of the treatments and the randomization procedure. Section 4 reports the main empirical findings. Section 5 offers some conclusions.
  

  

\section{Literature Review (NEEDS TO BE UPDATED)}

Our study is related to different branches of the empirical literature on 
tax compliance. Early
empirical studies focused the effectiveness of penalties and fines. They found little impact of such penalties on aggregate
tax compliance \citeA{Slemrod-07}.  But more recent, nuanced
studies, have found an impact of fines on both the level and speed of
tax payments.  \citeA{Fellner-13} find that a reminder letter for
payment of the Austrian TV license fee that explicitly threatens legal
action if the resident does not provide the required information for
assessment performed significantly better than the standard reminder
letter informing residents that they had not yet returned the required
forms.  \citeA{Wenzel-Taylor-04} find that including a letter
reminding taxpayers that their statement of rental income can be
audited and that faulty reporting may lead to fines significantly
reduced deductions when compared to forms submitted by taxpayers who
did not receive the threatening letter.  \citeA{Hallsworth-14}
find the speed with which taxpayers pay their liabilities can also be
improved with increased fines.

But fines only work if taxpayers believe they will be enforced.  Large
fines may be seen by taxpayers as a signal of a desperate and
ineffective tax collector, as politically inviable and thus as empty
threats, or in the extreme, as a breakdown of cooperative democratic
governance.  If so, an increase in fines may even reduce tax
compliance, as indeed happened in Israel with the payment of corporate
taxes \cite{Ariel-12}.  On balance, the estimated effects of fines
on tax payments have been positive, but modest in magnitude.

Both the instrumental motive and the motive born from civic duty have
been used to stimulate tax compliance.  The evidence is mixed.  The
most careful study of the two motives was done \citeA{Blumenthal-01}, 
where two different letters were sent to Minnesota state
taxpayers reminding the taxpayer when taxes were due and to report
their income accurately.  Hoping to elicit cooperative behavior from
an instrumental perspective, one letter stressed that taxes pay for
important state services.  Hoping to tap a personal sense of civic
duty, the second letter emphasized that most state taxpayers correctly
report their taxable income on time.  There were 15,000 taxpayers in
each group, and their reported taxable incomes were compared to a
control group of 15,000 taxpayers who received no letter.  We should
expect the largest effect on self-reported incomes as shown by 
\citeA{Kleven-11} and \citeA{Pomeranz-15}.  For both letters, there were
statistically significant positive and negative effects on the various
categories of self-reported incomes, with no statistically significant
change in aggregate taxable income over that reported by the control
group.  The one strong effect was a relatively large negative effect
on reported income by the richest taxpayers from having received the
civic duty letter.

Two more recent studies have been more encouraging as to the impact
of behavioral appeals. In an effort to improve the speed of tax
compliance for British income taxpayers, \citeA{Hallsworth-14} sent
either of two letters to taxpayers both encouraging them to pay their
taxes on time.  Again, appealing to the strategic advantages of
cooperative behavior in the decision to pay taxes, one letter stressed
that payment ensures important national services will be provided.  A
second letter appealed to a citizen's personal sense of community and
stressed that ``nine out of ten" taxpayers pay their taxes on time.
Both sets of letters had a statistically significant effect in
encouraging sooner tax payments, and the effects were greatest for the
appeal to ``civic duty" when the letter explicitly mentioned the 
taxpayer's most likely reference group of fellow citizens.

\citeA{Perez-Truglia-Toiano-15} explored the impact of what they call
a ``shaming penalty" administered through a letter to a subset of
delinquent state taxpayers reminding them that the state has placed
their name on a publicly available list of tax delinquents and that
only payment in full or acceptance of a payment schedule can remove
their name from the delinquent list.  The reminder letter made a
significant positive difference to eventual tax compliance, with the
greatest effects observed for taxpayers with the lowest level of taxes
owed.  In addition, reminding tax delinquents that there is a growing
financial penalty to late payments also had a positive impact on
compliance and particularly so for wage-only taxpayers whose income
could be most easily attached for payment and penalties.

  
\section{ A Tax Reminder Experiment}
  
\subsection{Treatments}

The research setting for this experiment is the City of Philadelphia.
Notices of property tax payments are sent each year on January 1, and
the full balance of taxes are due by March 31.  If payment has not
been received by that date, or the taxpayer has not entered into a
taxpaying plan with the City, fines and interest penalties begin to
accrue.  On April 1, the Department of Revenue begins contacting
unpaid accounts informing taxpayers of taxes due and the accumulation
of fines and penalties for late payment.  Normally, two-thirds of the
delinquent accounts are sent to outside collection agencies acting as
co-counsel for the City; one-third of the delinquent accounts remain
within the Revenue Department for collection.  The outside collecting
agents are reimbursed at the rate of 6 percent of all delinquent
revenues collected by December 31st.  All accounts still delinquent
after that time are then assigned to new collection agents.  Our
experiment was implemented using the City's share of delinquent
taxpayers for the tax year, 2015.
  
Of the 579,828 properties in the city in 2015, approximately 100,000
properties, or 83 percent of all properties, were delinquent as of
April 1st.  The sample included in our experiment were the 27,264
properties remaining with the Revenue Department and still owing at
least \$10 in property taxes as on May 15, 2015.  Our sample includes
only new delinquent taxpayers; it excludes all chronically delinquent
taxpayers who owe taxes from prior years.  Our experiment began in
mid-June, 2015 and continued until December 31, 2015. 
  
Our seven reminder letters were designed in coordination with
officials of the Department of Revenue.  Each letter was vetted by the
Department to ensure that it could be understood by a taxpayer with at
least a fourth or fifth grade level of reading comprehension.  Each
letter also provided contact information for assistance for
non-English speaking taxpayers.  The full letter templates are
included in an Appendix.  Here we present the important distinguishing
feature of each letter.  Our control letter provides a generic
reminder to the taxpayer. Specifically:
 
{\it Treatment Letter 1: Control } \\ {\bf Our records indicate that you have a balance due of $balance$.
	If you have already paid, thank you. If not, please pay now or contact us
	to arrange a payment plan. The fastest and easiest way to pay is online at
	\underline{www.phila.gov/pay}. Paying by E-check only costs 35c - 
	less than the cost of a stamp!"}

Two letters were mailed to test the efficacy of either of our
  two economic penalties.  The first imposes an economic penalty
        only and is called the lien letter.  The lien letter notes
        that the City will impose a lien on the delinquent property
        which entitles the City to deduct the amount of the lien from
        any future arms-length market sale of the property.  
        
{\it Treatment Letter 2: Lien } \\  {\bf  Failure to pay your Real Estate Taxes may result in the sale of 
	your property by the City in order to collect back taxes. In the 
	past year, we have sold $N$ properties in your neighborhood at 
	Sheriff's Sale.	Included in these $N$ are the following properties 
	near you: $<$three properties and their sale dates$>$ 

	Pay your taxes now to prevent the sale of your property. 
	Our records indicate that you have a balance due of $balance$.}

$N$ is the number of properties sold in neighborhood between June 2014
and May 2015.

The three listed properties in the taxpayer's neighborhood were
randomly selected from a list of properties that had been recently
sold and included tax liens on the sale.  All delinquent taxpayers
receiving the lien letter and in the same neighborhood saw the same
list of three properties.\footnote{An initial plan to select the three
  lien sale properties nearest each delinquent property met with
  privacy concerns and was therefore not pursued.}

The second letter including an explicit mention of an economic penalty
was the sheriff's sale letter.  We view this treatment letter as the
most onerous economically.  It not only imposes the full economic
penalty of taxes plus fines plus interest at the time of sale, but it
forces the sale of the taxpayer's property.  The inconvenience and,
perhaps more importantly, the psychic costs of moving may be
significant.  
	
{\it Treatment Letter 3: Sheriff's Sale} \\ {\bf Failure to pay your Real Estate Taxes will result in a tax
  lien on your property in an amount equal to your back taxes plus all
  penalties and interest. When your property is sold, those delinquent
  tax payments will be deducted from the sale price. By paying your
  taxes now, you can avoid these penalties and interest. Properties
  near you in $neighborhood$ that have had liens placed on them
  include: $<$three properties and their sale dates$>$

  Pay your taxes now to avoid a lien being placed on your property.
  Our records indicate that you have a balance due of $balance$.}
	
N is the number of properties sold by sheriff's sale in each
neighborhood between June, 2014 and May, 2015.  The three listed
properties in the taxpayer's neighborhood were randomly selected from
a list of properties that had been recently sold through a sheriff's
sale.  Again, all delinquent taxpayers receiving the sheriff's sale
letter and in the same neighborhood saw the same list of three
properties.

The next two reminder letters address the free rider motive for
non-payment.  The first letter appeals for payment from those who
might see their gain from non-payment largely in terms of their
private benefits from neighborhood services, what we call the
neighborhood letter.  

{\it Treatment Letter 4: Amenity } \\ {\bf We want to remind you that your taxes pay for essential
  public services in $neighborhood$, such as $<$two local
  amenities$>$, your local police officer, snow removal, street
  repairs, and trash collection. Please pay your taxes to help the
  city provide these services in your neighborhood.}

The neighborhood amenities were chosen at random for each property
from a list of City provided parks, recreation centers, and libraries
in the neighborhood of the delinquent property.  The second free rider
letter appeals for payment from those who see their gain from
non-payment in terms of their public benefits from Philadelphia-wide
services, what we call the community letter.  This letter reads:

{\it Treatment Letter 5: Community} {\bf Your taxes pay for important services that make a city
  great. Your tax dollars are essential for ensuring all Philadelphia
  children receive a quality education and all Philadelphians feel
  safe in their neighborhoods. Please pay your taxes as soon as you
  can to help us pay for these important services.}

The final two reminder letters appeal to a taxpayer's sense of
community more generally.  The first asks the delinquent taxpayer to
recognize that he is not a contributing member of his (personally
defined) community of peer taxpayers, a letter we call the peer
letter.  

{\it Treatment Letter 6: Peer } {\bf  You have not paid your Real Estate Taxes. Almost all of
  your neighbors pay their fair share: 9 out of 10 Philadelphians do
  so. By failing to pay, you are abusing the good will of your
  Philadelphia neighbors.}

The second letter stresses that non-payment will violate a wider
community norm of honest and responsible tax compliance needed for a
functioning democracy, a letter we call the civic duty letter. 

{\it Treatment Letter 7: Duty } {\bf For democracy to work, all citizens need to pay their fair
  share of taxes for community services. You have not yet paid your
  taxes. By failing to do so, you are not meeting your duty as a
  citizen of Philadelphia.}


As a baseline control, we randomly removed 3,000 delinquent taxpayers
from the possibility of receiving any reminder letter at all.  These
taxpayers became our holdout sample and allowed us to estimate the
efficacy of simply communicating with the taxpayer after the date that
taxes are due.\footnote{We tested one more intervention that has been
  successfully used by private firms in collecting overdue credit card
  payments.  This is to send the payment reminder in an envelope
  larger than the usually sized envelopes used for the first mailing
  of tax bills.  Credit card firms have found that reminders mailed in
  usual envelopes (4 1/8" by 9 ?") were often ignored, while reminders
  mailed in larger envelopes (9" by 12") resulted in greater payments.
  The total number of properties in this additional treatment was
  12,193 randomized over the seven treatment letters.  We found no
  statistically significant effect of letter size on compliance
  behavior or size of payment.  These results are available upon
  request.}
	
	
	
\subsection{Randomization Procedure}

Randomization took place in two stages.  First, 3,000 of the 27,264
eligible properties were randomly assigned to the Holdout Sample.  Of
the remaining 24,264 properties, 89 percent were owned by unique
owners, owning only one property in Philadelphia, The remaining 11
percent were held by owners with multiple holdings in our sample.
Since we are interested in taxpayer compliance and not property
compliance, we identified owners of multiple delinquent properties by
their legal name and sent those owners one treatment letter.  We
identified multiple property owners by matching the legal name
associated with each property.\footnote{We lacked an objective
  identifier such as a social security number so identification was by
  the owner's legal name.  There is possibility that two or more
  different owners might have the same name, but inspection by the
  authors found this to be very rare.  We consider this random noise
  to the experiment.}
	
The average debt owed by each owner was \$1,810 and the median debt
owed was \$907, suggesting a significant upper tail to the
distribution of delinquent taxes owed.
	
Second, having identified owners, we then randomly assigned each owner
to a treatment group.  Any delinquent taxpayer holding multiple
properties within each treatment group received the same letter for
each of those properties.  Given the high correlation between the
propensity to pay taxes and total debt-owed , the treatment groups
were defined according to owner-level total debt to assure uniformity
of samples along the dimension of debt owed.  Standard errors in our
analysis are clustered by the treatment group as randomized.  Each
property assigned to receive a reminder letter was equally likely to
receive one of the seven treatments.  Balance tests for
pre-randomization characteristics are reported in Table \ref{balance} .  Results
confirm that randomization was successful.  There are no significant
differences across reminder letters.\footnote{While multiple property
  owners introduced small variations between treatment groups, the
  randomization procedures worked as designed; see Table 1.  Excluding
  multiple property owners from the sample produced near perfect
  balance on all measured characteristics of the properties.  Result
  available upon request.}
	      
\noindent COMMENT: We need to report randomization checks for the unique owner sample that includes the Hold-out Sample	        

\begin{sidewaystable}[htbp]
\caption{Balance on Observables} \label{balance}
\bigskip
\centering
\begin{tabular}{l c c c c c c c c c c}
\hline
\multicolumn{9}{c}{Full Sample} \\
   \hline
Variable & Control & Amenities & Community & Duty & Peer & Lien & Sheriff & $p$-value \\ 
   \hline
Amount Due (June) & \$1,847 & \$2,209 & \$1,954 & \$1,700 & \$1,772 & \$1,735 & \$1,887 & 0.78 \\ 
  Assessed Property Value & \$195,029 & \$224,412 & \$220,963 & \$191,199 & \$165,957 & \$173,690 & \$178,556 & 0.76 \\ 
  \% with Unique Owner & 87.5 & 86.4 & 88.3 & 88.0 & 87.4 & 88.0 & 87.5 & 0.45 \\ 
  \% Overlap with Holdout & 3.72 & 3.80 & 3.47 & 3.47 & 3.58 & 3.47 & 3.29 & 0.96 \\ 
  \# Properties per Owner & 1.33 & 1.36 & 1.29 & 1.27 & 1.26 & 1.32 & 1.26 & 0.55 \\ 
  \# Owners & 2,766 & 2,766 & 2,766 & 2,766 & 2,766 & 2,765 & 2,766 & 1 \\ 
   \hline 
\multicolumn{9}{c}{Unique Owners} \\
   \hline
Variable & Control & Amenities & Community & Duty & Peer & Lien & Sheriff & $p$-value \\ 
   \hline
Amount Due (June) & \$1,383 & \$1,950 & \$1,290 & \$1,316 & \$1,338 & \$1,389 & \$1,613 & 0.38 \\ 
  Assessed Property Value & \$163,084 & \$206,214 & \$130,265 & \$166,791 & \$130,936 & \$147,573 & \$155,597 & 0.28 \\ 
  \# Owners & 2,420 & 2,389 & 2,441 & 2,433 & 2,417 & 2,432 & 2,419 & 0.99 \\ 
   \hline 
 \multicolumn{9}{l}{\scriptsize{$p$-values in rows 1-5 are $F$-test $p$-values from regressing each variable on treatment dummies. A $\chi^2$ test was used for the count of owners.}} \\ 
\end{tabular}

\end{sidewaystable}

\begin{table}[ht]
\centering
\caption{Balance between Holdout and Treated Samples} 
\label{tbl:bal_hold}
\begin{tabular}{|r|rrr|rrr|}
   \hline
 & \multicolumn{3}{c|}{Full Sample} & \multicolumn{3}{c|}{Unique Owners} \\
\hline
Variable & Treated & Holdout & $p$-value & Treated & Holdout & $p$-value \\ 
  Amount Due (June) & \$1,872 & \$1,241 & 0.02 & \$1,467 & \$1,233 & 0.34 \\ 
  Assessed Property Value & \$192,832 & \$142,973 & 0.15 & \$157,094 & \$142,630 & 0.57 \\ 
  \% with Unique Owner & 87.6 & 99.1 & 0 &  &  &  \\ 
  \# Properties per Owner & 1.30 & 1.01 & 0 &  &  &  \\ 
  \# Owners & 19,361 & 2,107 &  & 16,951 & 2,088 &  \\ 
   \hline
\end{tabular}
\end{table}

\newpage

\section{Empirical Results}

\subsection{Short Term Impact}

In this section we focus on the short term impact of our intervention. We define the
short term as the first three months after our intervention. During this time period
tardy tax payers were only exposed to our intervention. As a consequence, our estimates
of the treatment effects are not contaminated by other interventions.

To gain insight into the nature of tax compliance in Philadelphia, 
we consider two discrete measures of tax compliance. We define partial compliance if the
 the tardy taxpayer makes any payment at all  and zero.  
Partial Compliance is of interest because even small 
additional payments help, but perhaps more importantly, a tax 
contribution represents a willingness by the taxpayer to be engaged 
with city governance.  The ever-paid outcome does not differentiate between
taxpayers that made full payment and those who made only a partial
contribution.  Full compliance is defined as making a full payment. 

\begin{table}[ht]
\centering
\caption{Linear Probability Model Estimates} 
\label{tbl:marg_3mo}
\begin{tabular}{lllll}
  \hline
   & \multicolumn{2}{c}{Ever Paid} & 
             \multicolumn{2}{c}{Paid in Full} \\
 & One Month & Three Months & One Month & Three Months \\
Holdout & 30.5 & 51.4 & 23.5 & 40.8 \\ 
   \hline
Control & 3.7*** & 3.9** & 2.2* & 3** \\ 
   & (1.4) & (1.5) & (1.3) & (1.5) \\ 
  Amenities & 1.7 & 2.6* & -0.2 & 1.6 \\ 
   & (1.4) & (1.5) & (1.3) & (1.5) \\ 
  Moral & 3.8*** & 2.8* & 1.3 & 2.5* \\ 
   & (1.4) & (1.5) & (1.3) & (1.4) \\ 
  Duty & 2.4* & 3.6** & 0.7 & 2.3 \\ 
   & (1.4) & (1.5) & (1.3) & (1.5) \\ 
  Peer & 3.9*** & 3.5** & 1.8 & 3.4** \\ 
   & (1.4) & (1.5) & (1.3) & (1.5) \\ 
  Lien & 9*** & 9.2*** & 5.7*** & 7.3*** \\ 
   & (1.4) & (1.5) & (1.3) & (1.5) \\ 
  Sheriff & 7.3*** & 8.8*** & 4.5*** & 6.7*** \\ 
   & (1.4) & (1.5) & (1.3) & (1.5) \\ 
   
 \hline 
 \multicolumn{5}{l}{\scriptsize{Holdout values are in levels; remaining figures are relative to Holdout}} \\ 
\end{tabular}
\end{table}

COMMENT: ADD ``Paid in Full" AND REPORT ESTIMATES BASED ON LINEAR REGRESSIONS. SEE COMMENT BELOW.


We start and consider the partial compliance results that pertain to the sample in which we exclude owners of multiple properties. This exclusion is done for the following two reasons. First, the number of multiple owners is small. Second the group of multiple owners is rather heterogenous and divers.

Table \ref{pc_rates_s} lists the participation rates
in the hold out sample as the difference in the participation in the seven treatment samples. Robust 
standard errors are reported in parentheses. We find that all seven treatment increased partial compliance
at the one month and three month date. Almost all of these increases in compliance behavior are statistically 
significant at standard levels of significance. We also find that only the lien and the sheriff's sale letter have positive and significant impact after six months.

\begin{table}[htbp]
\caption{Logistic Model Estimates}
\begin{center}
\begin{tabular}{l c c c c }
\hline
 & \multicolumn{2}{c}{Ever Paid} & \multicolumn{2}{c}{Paid Full} \\
\hline
 & One Month & Three Months & One Month & Three Months \\
\hline
Control        & $0.17^{***}$ & $0.16^{***}$ & $0.12^{*}$   & $0.12^{**}$  \\
               & $(0.06)$     & $(0.06)$     & $(0.07)$     & $(0.06)$     \\
Amenities      & $0.08$       & $0.11^{*}$   & $-0.01$      & $0.06$       \\
               & $(0.06)$     & $(0.06)$     & $(0.07)$     & $(0.06)$     \\
Moral          & $0.17^{***}$ & $0.11^{*}$   & $0.07$       & $0.10^{*}$   \\
               & $(0.06)$     & $(0.06)$     & $(0.07)$     & $(0.06)$     \\
Duty           & $0.11^{*}$   & $0.14^{**}$  & $0.04$       & $0.09$       \\
               & $(0.06)$     & $(0.06)$     & $(0.07)$     & $(0.06)$     \\
Peer           & $0.18^{***}$ & $0.14^{**}$  & $0.10$       & $0.14^{**}$  \\
               & $(0.06)$     & $(0.06)$     & $(0.07)$     & $(0.06)$     \\
Lien           & $0.40^{***}$ & $0.37^{***}$ & $0.29^{***}$ & $0.30^{***}$ \\
               & $(0.06)$     & $(0.06)$     & $(0.07)$     & $(0.06)$     \\
Sheriff        & $0.33^{***}$ & $0.36^{***}$ & $0.24^{***}$ & $0.27^{***}$ \\
               & $(0.06)$     & $(0.06)$     & $(0.07)$     & $(0.06)$     \\
\hline
Log Likelihood & -12243.94    & -13035.32    & -10801.06    & -13046.81    \\
Num. obs.      & 19039        & 19039        & 19039        & 19039        \\
\hline
\multicolumn{5}{l}{\scriptsize{$^{***}p<0.01$, $^{**}p<0.05$, $^*p<0.1$}}
\end{tabular}
\label{tbl:reg8_3mo}
\end{center}
\end{table}

COMMENT: Add Paid in Full to Table


To formalize this procedure we also estimated logit models. Table 
\ref{pc_logit} summarizes the estimates and the estimated
standard errors for the three samples that we considered above. We
report robust standard errors that are clustered at the treatment level. The main difference
here is that our approach to computing standard errors  is more conservative than in the previous table.
However the main findings are qualitatively and quantitatively the same.

\noindent COMMENT: THE ONE AND THREE MONTHS IN TABLES 2 AND 3 SEEMS TO BE INCONSISTENT. WE NEED TO RESOLVE THIS ISSUE. I SUGGEST WE RUN LINEAR PROBABILITY MODELS AND SEE WHAT WE GET WHEN WE COMPUTE ROBUST STD ERRORS. IF WE SWITCH TO LIN PROB, THAT WOULD SAVE US A TABLE AND MAKES THE COEFFICIENTS MUCH EASIER TO INTERPRET!


\begin{table}[ht]
\centering
\caption{Estimated Impact on Revenue: 3 Months} 
\label{rev_ep}
\begin{tabular}{rrr}
  \hline
Treatment & Impact Per Letter & Total Impact \\ 
  \hline
Control & \$32.35 & \$78,298 \\ 
  Amenities & \$21.97 & \$52,494 \\ 
  Moral & \$23.63 & \$57,690 \\ 
  Duty & \$29.9 & \$72,742 \\ 
  Peer & \$29.14 & \$70,440 \\ 
  Lien & \$76.21 & \$185,345 \\ 
  Sheriff & \$72.73 & \$175,945 \\ 
   \hline
\end{tabular}
\end{table}


Next we conduct some simple of the envelop calculations to assess the impact of these estimates on revenues. Here focus on the results after six month. We take the median payment in each subsample and multiply the median payment with the increase in the compliance probability reported in Table \ref{pc_rates_s}. This product can be interpreted as the impact of each treatment on revenue per letter. To obtain the total estimated impact we then multiply the impact per letter with the total number of individuals in the sample. The results are reported in Table \ref{pc_rev_s}.
Overall, we find that six of the seven treatments generated positive revenues for the city. Our estimates per letter range between \$ X and \$ XX dollars.

\noindent COMMENT: WE SHOULD REPORT THE REVENUE REGRESSION FOR THIS SAMPLE HERE. 

Regressions of total amount paid and treatment variables produce somewhat noisy estimates of the treatment effect
The estimate of the treatment effect of the lien relative to the control letter after three (one) months was XX (XX). Similar  revenue estimates are obtained for the sheriff's sale letter.  We thus conclude that  the estimates reported in Table \ref{pc_rev_s} are, if anything, conservative estimates of the impact of the lien treatment on revenues.

We conducted a number of robustness checks. Recall that we randomized the seven treatments at the ownership level. In Table \ref{pc_rob_s} we replicate the analysis done above excluding the hold-out sample. We then estimate the model using the larger sample that also includes owners of multiple properties. Overall, we find the results
are similar to the ones reported in Table \ref{pc_logit_s}. If anything, the treatment effects are stronger in the single owner sample. We thus conclude that owners of multiple properties are less likely to respond the kind of nudge strategies explored in this paper.

\begin{sidewaystable}[htbp]
\caption{Robustness Analysis: Multiple Owners}\label{pc_rob_s}
\begin{center}
\begin{tabular}{|l| c c | c c | }
\hline
               & One Month & Three Months  & One Month & Three Months  \\
\hline
 & \multicolumn{2}{c}{All Owners} & \multicolumn{2}{|c|}{Single-Property Owners} \\
Amenities      & $-0.04$      & $-0.03$         & $-0.09$      & $-0.05$          \\
               & $(0.06)$     & $(0.05)$         & $(0.06)$     & $(0.06)$        \\
Community          & $-0.02$      & $-0.06$          & $0.00$       & $-0.04$        \\
               & $(0.06)$     & $(0.05)$        & $(0.06)$     & $(0.06)$       \\
Duty           & $-0.05$      & $-0.01$         & $-0.06$      & $-0.01$       \\
               & $(0.05)$     & $(0.05)$        & $(0.06)$     & $(0.06)$      \\
Peer           & $0.02$       & $-0.03$         & $0.01$       & $-0.02$        \\
               & $(0.06)$     & $(0.05)$       & $(0.06)$     & $(0.06)$     \\
Lien           & $0.21^{***}$ & $0.20^{***}$  & $0.23^{***}$ & $0.22^{***}$  \\
               & $(0.05)$     & $(0.06)$        & $(0.06)$     & $(0.06)$       \\
Sheriff        & $0.15^{***}$ & $0.19^{***}$  & $0.15^{***}$ & $0.20^{***}$  \\
               & $(0.06)$     & $(0.05)$        & $(0.06)$     & $(0.06)$     \\
\hline
Log Likelihood & -12599.09    & -13189.11     & -10959.60    & -11588.89       \\
Num. obs.      & 19361        & 19361              & 16951        & 16951            \\
\hline
\multicolumn{5}{l}{\scriptsize{$^{***}p<0.01$, $^{**}p<0.05$, $^*p<0.1$}}
\end{tabular}
\end{center}
\end{sidewaystable}

COMMENT: Add Paid in Full to Table

\newpage

\subsection{Long Term Returns}

COMMENT: Here we want to focus on the six month results. We need to explain that
all tardy tax payers were assigned to a collection agency at the end of August 2015 and
thus received another "treatment." This may explain why we see a convergence in the 
estimates of the treatment effects.

\begin{table}[ht]
\centering
\caption{Linear Probability Model Estimates: 6 Months} 
\label{tbl:marg_6mo}
\begin{tabular}{lll}
  \hline
   & Ever Paid & Paid in Full \\
Holdout & 73.3 & 63.2 \\ 
   \hline
Control & 1.3 & 1.5 \\ 
   & (1.3) & (1.5) \\ 
  Amenities & -0.2 & 0 \\ 
   & (1.3) & (1.5) \\ 
  Moral & 0.9 & 1.1 \\ 
   & (1.3) & (1.4) \\ 
  Duty & 2.1 & 1 \\ 
   & (1.3) & (1.5) \\ 
  Peer & 1.4 & 2.3 \\ 
   & (1.3) & (1.5) \\ 
  Lien & 3.7*** & 4.8*** \\ 
   & (1.3) & (1.4) \\ 
  Sheriff & 3.7*** & 2.9** \\ 
   & (1.3) & (1.4) \\ 
   
 \hline 
 \multicolumn{3}{l}{\scriptsize{Holdout values are in levels; remaining figures are relative to Holdout}} \\ 
\end{tabular}
\end{table}


\begin{table}[htbp]
\centering
\caption{Estimated Impact on Revenue: 6 Months} \label{pc_rev_6}
\begin{tabular}{rrr}
  \hline
Treatment & Impact Per Letter & Total Impact \\ 
  \hline
Control & \$10.49 & \$25,397 \\ 
  Amenities & -\$1.98 & -\$4,740 \\ 
  Community & \$7.21 & \$17,588 \\ 
  Duty & \$17.43 & \$42,405 \\ 
  Peer & \$11.26 & \$27,225 \\ 
  Lien & \$31.01 & \$75,420 \\ 
  Sheriff & \$30.67 & \$74,201 \\ 
   \hline
\end{tabular}
\end{table}


Next we conduct some simple of the envelop calculations to assess the impact of these estimates on revenues. Here focus on the results after six month. We take the median payment in each subsample and multiply the median payment with the increase in the compliance probability reported in Table \ref{pc_rates_6}. This product can be interpreted as the impact of each treatment on revenue per letter. To obtain the total estimated impact we then multiply the impact per letter with the total number of individuals in the sample. The results are reported in Table \ref{pc_rev_6}.
Overall, we find that six of the seven treatments generated positive revenues for the city. Our estimates per letter range between \$7 and \$31 dollars.

\noindent COMMENT: WE SHOULD REPORT THE REVENUE REGRESSION FOR THIS SAMPLE HERE. 

Regressions of total amount paid and treatment variables produce somewhat noisy estimates of the treatment effect
The estimate of the treatment effect of the lien relative to the control letter after the first month was 80 (42), after three month 107 (46) and 66 (44) after six month. Similar 
revenue estimates are obtained for the sheriff's sale letter.  We thus conclude that  the estimates reported in Table \ref{pc_rev} are, if anything, conservative estimates of the impact of the lien treatment on revenues.

\subsection{Spill-overs to the Next Year}

\noindent COMMENT: WE SHOULD DISCUSS THAT THERE ARE NO EFFECTS IN THE FOLLOWING YEAR. PRODUCE A TABLE.

\noindent COMMENT: NEED TO DISCUSS FULL IMPLEMENTATION BY DoR in 2016


\section{Conclusions}

In this paper, we report the results of a multi-arm field experiment 
designed to test which low-cost notification strategies increase tax 
payment rates. In the context of this field experiment in municipal 
tax collection, our results suggest that simple notification strategies, 
much discussed in recent studies, can accelerate participation but are 
ineffective at reducing tax delinquency. This finding covers a 
wide swath of theorized  ``nudges" including social norming, moral suasion, 
and tax morale. In addition, simply notifying delinquents more often was 
found to be an accelerator of payment while leaving long-term levels 
unchanged. Finally, we observe no effect of increased envelope size. 
Apparently, taxpayers open tax bills even if they do not pay them.
These non-findings are contrasted with the single persistent effect 
that delinquency levels can be lowered quickly and permanently using 
credible threats regarding the foreseeable consequences of nonpayment. 

For revenue collecting agencies, particularly those with revenue collection 
problems, acceleration of payment can be understood to be a useful result 
in and of itself. Bills must be paid and debts must be serviced on regular 
schedules. Relatedly, the longer that tax bills remain unpaid, the more 
expensive it becomes to collect. Whether handled internally or externally 
through debt collection firms, downstream collection practices leave 
diminished revenues. For both of these reasons, early collection is, 
\textit{ceteris paribus}, better collection. This is not to say, 
however, that early collection is social welfare improving. If tardy 
but eventually-compliant tax delinquents forestall early payment to 
cover other expenses or invest in other assets which they eventually 
use to repay their tax bill with interest, then the welfare of the tax 
agency may not be synonymous with the welfare of society. This is 
especially true if tax payments are made weeks rather than months late, 
such that monthly payments can still be made based on expected monthly 
receipts. However, in the case of persistently-delayed but consistently-paid 
payments, it is less obvious that accelerating eventual or inevitable 
payment constitutes something of value. 

For this reason, the fact that credible threat notifications increase 
repayment rates and convert, in the language of Rubin causal framework, 
defiers into compliers is particularly notable. It suggests that 
increasing repayment at low cost is possible, and that more research is 
needed on the reasons for different behavioral responses to 
consequentialist and non-consequentialist messages. Perhaps, the difference 
between treatments that merely accelerate payment and those that increase 
payment lie in alternative theories of why tax delinquents exist in the 
first place. Non-consequentialist theories of non-payment implicitly rest 
on the assumption that non-compliers are not liquidity-constrained and are
merely unaware of the collective consequences of non-payment. Under this 
analysis, tax delinquency is due to discouragement, indifference, 
lack of appreciation, or unawareness. Delinquents merely need to be 
encouraged or reminded to participate. Our results, however, suggest 
that this is an incomplete explanation for tax delinquency. 

Tax delinquents are not simply discouraged. Providing them with 
information about peer behavior, amenities, Community arguments, or civic 
duties does nothing to increase the overall repayment rates. This may 
reflect the fact that delinquents are indifferent to or already aware 
of their peers' positive and negative behavior. Likewise, the provision 
of information on public goods assumes that recipients have not 
incorporated consideration of public goods funded through tax dollars 
into their payment behavior. If services are considered paid for by 
other taxes or the quality of the public services are considered to be 
sub-par, then the rationale for funding these initiatives may not be 
particularly compelling.  Similarly, social contract theories of citizen- 
and state-shared responsibilities offer an idealized vision of civic 
responsibilities that ignores the reality that both sides are chronically 
dissatisfied with the performance of other parties.

At the same time, our results suggest that if tax delinquents are not 
simply discouraged, they are also seemingly unaware of the consequences 
of nonpayment. Our provision of simple information about the collection 
process had a clear impact on the likelihood of repayment. This result 
echoes other recent findings that clear, consistent, and timely provision 
of information on consequences, particularly in the context of compliance 
monitoring, can lead to notable improvements in behavioral compliance 
\citeA{hawken}. Perhaps, then, the puzzle of high non-payment 
rates despite perfect public information on noncompliance can be understood 
as a case of under-enforcement. This possibility is reinforced by the fact 
that conditional on any payment, converted defiers became near perfect 
compliers, making full payments in almost all cases. This suggests that at 
least for the margin affected, liquidity-constraints are not the primary 
reason for initial non-payment.

\newpage
{\footnotesize \NoTitleCaseChange\citepunct{(}{and}{, }{; }{, }{)}{}{.} 
\bibliographystyle{theapa}
\bibliography{references}

\end{document}

