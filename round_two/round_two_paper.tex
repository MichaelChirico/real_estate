\documentclass[12pt]{article}
\usepackage{amssymb}
\usepackage{theapa}
\usepackage{titlepage}
\usepackage{amsmath}
\usepackage{setspace}
\usepackage[dvips]{graphicx}
\usepackage{rotating}
\usepackage[usenames,dvipsnames]{pstricks}
\usepackage{epsfig}
\usepackage{pst-grad}
\usepackage{pst-plot}
\usepackage{color}
\usepackage{pstricks-add}
\usepackage{rotating}
\usepackage{threeparttable}
\usepackage{array,multirow}
\usepackage{pdflscape}
\usepackage{float,lscape}
\usepackage{csquotes}
\usepackage{textcomp}

\renewcommand{\baselinestretch}{1.5}
\parindent=.2in
\evensidemargin=.05in 
\oddsidemargin=-.05in 
\topmargin=-0.05in
\textwidth=6.5in 
\textheight=8in

\newtheorem{fact}{Stylized Fact}
\newtheorem{theorem}{Theorem}
\newtheorem{corollary}{Corollary}
\newtheorem{definition}{Definition}
\newtheorem{lemma}{Lemma}
\newtheorem{prop}{Proposition}
\newtheorem{assumption}{Assumption}
\newtheorem{remark}[theorem]{Remark}
\newtheorem{solution}[theorem]{Solution}
\renewcommand{\thefootnote}{\fnsymbol{footnote}}


\begin{document}

\title{Deterring Delinquency: A Field Experiment in Improving Tax Compliance Behavior}

\author{Michael Chirico, Robert Inman, Charles Loeffler, \\ 
John MacDonald, and Holger Sieg\thanks{We would like to thank Rob Dubow,
    Clarena Tolson, and Marisa Waxman in the Department of Revenue of
    the City of Philadelphia for their help and support. We thank Kent
    Smettters and the Wharton Initiative for Public Policy for funding
    this field experiment. We would also like to thank Jeff Brown, Kai
    Konrad, Robert Moffitt, Jim Poterba, Chris Sanchirico, Wolfgang
    Sch\"on, Reed Shuldiner and participants of numerous seminars for
    comments and suggestions. The views expressed here are those of
    the authors and do not necessarily represent or reflect the views
    of the City of Philadelphia.}  \\ 
University of Pennsylvania}

\date{\today}

\maketitle

\begin{abstract}

Property taxes play a central role in the financing of municipal
government services. Yet, municipal governments commonly confront
problems with property tax collection even when the tax base is known.
There is surprisingly little evidence on what authorities can do to
increase property tax compliance.  This paper analyzes seven different
property tax notification strategies through a randomized controlled
experiment conducted with the City of Philadelphia.  All seven
notification strategies increase property tax compliance over the
usual approach of simply sending a bill.  The most effective
notifications are the those that threaten to take out a lien on the
property or to foreclose by sheriff's sale for continued failure to
pay taxes.  The results suggest that economic motives to pay property
taxes are more effective than those that appeal to social norms.

\bigskip

\noindent KEYWORDS: Tax Compliance, Property Taxation, Field
Experiment, Deterrence, Public Service Appeal, Appeal to Civic Duty.

\end{abstract}
\renewcommand{\thefootnote}{\arabic{footnote}}

\newpage

\section{Introduction}

Property taxation plays a central role in the financing of municipal
government services in the United States. As a matter of practical
local government finance, it is the mainstay of city and school
district budgets.  In 2013, over 72 percent of all local government
tax revenues and nearly 50 percent of all own revenues came from
property taxation. The potential economic advantages and disadvantages
of a local property tax are well known.  With mobile households and
local zoning the property tax approximates a benefit for the financing
of local services \cite{Hamilton-75}.  In cities with stagnant growth
property taxes will have adverse effects on new construction and home
improvements.  In growing cities property taxes approximates a land
tax that is proportional to a wealth tax and attractive from an equity
standpoint.\footnote{See, for example, \citeA{Aaron-75} and
  \citeA{Mieszkowski-72}.}

The collection of property taxes has one very important administrative
advantage over the collection of other taxes: the legal tax obligation
is known to the taxpayer and the taxing authorities.  Self-reporting
of tax bases, as required for income, profits, sales, and VAT
taxes, is not needed for the property tax.  Each taxpayer has an
assigned tax base, the value of their property, against which a common tax
rate is assessed.  This avoids problems of misreporting tax bases or
working outside the formal or  taxable economy.\footnote{See
  \citeA{Blumenthal-01}, \citeA{Kleven-11}, and \citeA{Pomeranz-15}.}

Yet, local taxing authorities commonly confront challenges in
collecting property taxes.  Standard enforcement strategies-- such as
liens and foreclosures -- are time intensive and costly to implement.
The inherent problems associated with standard enforcement activities
raise the question whether there are lower-cost alternatives that can
enhance revenue collection. Recent scholarship has suggested that
simple reminders as well as information provision on social norms
could improve tax compliance\cite{delcarpio}. However, whether
these approaches increase compliance or merely accelerate it remains 
an open question.   In this paper we conducted a randomized
field experiment to assess the efficacy of different property tax
notification strategies in City of Philadelphia.

For all these virtues to be realized, it is essential that the
property tax be collected, both efficiently and fairly.  Among large
U.S. cities, this is not assured.  Because of the importance of the
tax to city budgets, even small differences in collection rates can
significantly affect the provision of local public services. While the
average yearly rate of tax collection is 95 percent among a sample of
large cities in the U.S., many cities collect only 90 percent of taxes
due, and several cities do far worse.  Among the poorest performers
are Cleveland (84\%), Detroit (68\%), Flint (64\%), Milwaukee (86\%),
and Pittsburgh (85\%). While the poorest performers are all
high-poverty cities, poverty alone is an inadequate explanation for
their performance. There are many high-poverty cities that in fact
collect almost all of their property taxes; for example, Baltimore
(96\%), Birmingham (98\%), Dallas (98\%), Houston (98\%), Minneapolis
(98\%), and even New Orleans (95\%).\footnote{For a more detailed
  analysis see \citeA{CILMS-16}.}  These high-poverty,
high-tax-compliance cities suggest that other factors are likely at
work.  This paper explores one of these factors: tax collection
strategies. The administrative issue for the property tax is simple:
did the taxpayer pay the tax on time or not?  If not, what can the tax
administrator do to enforce compliance?  Answering these questions is
important, as there is considerable disagreement on how to ensure
property tax compliance.

The most common strategy to enforce compliance are fines and
penalties.  Failure to pay property taxes on time leads to interest
charges sufficiently large as to preclude any arbitrage advantage to
waiting, and perhaps a significant late fine.  Fines, however, only
work if taxpayers believe they will be enforced.  Large fines may be
seen by taxpayers as a signal of a desperate and ineffective tax
collector; as politically unviable and as empty threats; or, in the
extreme, as a breakdown of cooperative democratic governance.

When a delinquent tax payer does not respond to penalties and fines,
the city can take out a tax lien on the property.  After a tax lien
has been obtained, it can also start the foreclosure process.  A tax
lien is a lien imposed by law upon a property to secure the payment of
taxes.  Before the owner can sell the property and give clear title to
the buyer, he must pay off the lien. However, a lien does not impose a
direct, tangible cost on a delinquent taxpayer and thus may only
provide weak incentives to comply with the tax laws. Moreover, liens
are ineffective tools to collect tax debt if the property will not be
transacted in the market (i.e., at arms' length), or is believed to be
such by the current owner. If, for example, the transaction occurs
within a family, it may not require an official transfer of the deed
-- and even official deed transfers may occur at nominal prices and be
exempted from rules governing arms-length sales.

A city can sell tax liens to investors.\footnote{In July 2015 the City
  of Philadelphia tried to auction off 865 liens online. The results
  were disappointing. The city sold only sold 28 percent of the liens
  for a total of \$2.1 million, according to data from the city's
  revenue department. (Philadelphia Magazine, 2015). One potential
  problem of this sale was that the minimum bid for the liens might
  have been priced too high.} Properties often sell for a premium at
lien auctions. This means that the lien holder gets zero percent
interest and actually pays a premium to acquire the lien with the hope
of foreclosing and obtaining clear title to the property. However,
selling liens to ``vulture investors'' can be costly from a political
perspective.

Lien holders can start the foreclosure process if they do not recover
the taxes in due time. When the owner of a property located in a city
fails to make a payment arrangement on municipal debt levied on
his/her property, that property may be sold at auction (administered
in Philadelphia by the Office of the Sheriff) to allow the city to
collect on that unpaid debt. However, the foreclosure process is
costly and time-intensive. In Philadelphia, the process of offloading
a property at sheriff's sale can take nine months to a year. Given
that the median outstanding tax debt in Philadelphia in a single year
is typically less than \$1,000, sheriff sales may not be cost
effective for a variety of properties with low-to-median market
values.

Given the inherent problems associated with standard enforcement
strategies, attention has turned to low-cost alternatives. The main
option for cities is to develop more effective notification
strategies.  In this paper we explore the efficacy of seven ``nudges''
for improved property tax collection Philadelphia.\footnote{For a
  survey on field experiments see \citeA{Harrison-List-04}.}  Our
first nudge strategy is a simple reminder letter to the taxpayer that
their taxes are due; the letter is identical in content to the initial
tax bill listing the tax due and any penalties for late payments.  The
reminder letter will be our ``control nudge'' and is meant to address
non-payment due to forgetfulness or oversight.

The second set of strategies are meant to address an economic motive
for non-payment.  The delinquent taxpayer is assumed to be making an
economic calculation that by not paying there is a positive
probability that delinquency will go undetected or if detected,
ignored for administrative reasons, and that the expected economic
gains of not paying exceed the expected economic costs of being caught
and fined \cite{Allingham-Sandmo-72}.  In most real world tax
settings, the probability of being caught and the size of the
likely sanction are both too low to rationally account for most
observed levels of taxpayer noncompliance \cite{Alm-92}.\footnote{An
  alternative specification for taxpayer utility that allows for loss
  aversion has done a better job in explaining taxpayer compliance
  among Swedish taxpayers than did the classical expected utility
  specification with always-declining marginal utilities in income;
  see \citeA{Engstrom-15}. } Here, we test for the effect of two nudges
addressing potentially large economic consequences -- one where delinquent
taxes plus a graduated fine growing over time are collected as a
``lien'' on the property at sale, and a second where the property
is seized for auction (at sheriff's sale), with a portion of the
proceeds used to pay delinquent taxes and penalties. The lien imposes
a growing real dollar future loss on the delinquent taxpayer as the
interest rate for penalties exceeds the taxpayer's alternative rate
of return. The sheriff's sale imposes an immediate economic loss,
and further requires the delinquent taxpayer to find a new residence.
Both of these nudges threaten large economic and, in the case of the
sheriff's sale, large psychic costs for continued noncompliance. Both
are also designed to increase the salience of the economic consequences
of non-compliance\cite{Bhargava-15} rather than the probability of 
discovery\cite{delcarpio}, which is known and invariant for the property tax.

Two additional nudges appeal to what \citeA{Luttmer-14} have called
``tax morale.''  First, we remind taxpayers that their payments do
provide valuable public services.  Here we seek to address
noncompliance due to a desire by taxpayers to free-ride on the
payments of their neighbors or of Philadelphians generally.  The first
free-rider strategy seeks to motivate payment by reminding the
taxpayer that his payments go to providing services for his family and
his immediate neighbors and lists specific amenities in their vicinity
likely to be affected; we call this strategy the ``neighborhood''
nudge.  The second free-rider strategy reminds the taxpayer that his
taxes support important city-wide services such as education and public
safety. We call this strategy the ``community'' nudge.
  
A final set of nudges appeals to a possibly deeper motive for tax
compliance -- fulfilling one's obligations to a self-identified
community of peers \cite{Posner-00} or to an abstract community of
citizens \cite{Rawls-71}.  The former of these community strategies we
call the ``peer'' nudge; the latter we call the ``civic duty'' nudge.  

The seven nudge strategies for increased taxpayer compliance are first
compared to the alternative of doing nothing beyond sending the first
tax bill.  We then compare the seven nudge strategies among themselves
to see which are most effective in encouraging taxpayer compliance.

Our experiment started in the beginning of June 2015. No other
enforcement activities were undertaken by the City until the middle of
August 2015. It is, therefore, useful to distinguish between the
short- and long-term impacts of our intervention. Short term outcomes
are those that measure compliance up to three months into the
experiment. These outcomes are clean measures of the impact of the
intervention since no other enforcement activities took place during
that time period.  In the short term, we find that most of our nudge
strategies significantly outperform the ``do nothing'' alternative
both in the rate of taxpayer compliance and in the level of payments,
conditional on compliance.  Second, among the seven nudges, the most
effective in encouraging tax payment are the two economic strategies
that threaten large financial (``lien'') or financial and psychic
(sheriff's sale) penalties.

After one month, approximately 30 percent of all taxpayers in the
holdout (``do nothing'') sample had made some contribution towards
their tax liabilities. In contrast 40 percent of taxpayers that
received the lien letter and 37 percent that received the sheriff's
sale letters had made payments. The results are similar after three
months of the intervention.  Fifty-one percent of all households in
the holdout sample had made some contribution after three months,
versus 61 percent for the lien and 60 percent for the sheriff's sale
letters. Reminder letters also improved the level of tax payments,
given that the taxpayer complied. The three-month impact of these two
letters was approximately \$75 per letter.  Receiving a reminder
letter improved taxpayer compliance in the short run, with the lien
and sheriff's sale letters the most effective.

We also consider long-term outcomes measured six months after our
experiment. While our experimental design is still valid for these
outcomes, it is harder to interpret the findings since other
interventions occurred during that time period, such as enforcement
activities by collection agencies. In particular, the city contracted
with two private collection agencies. These collectors primarily use
phone calls to contact tardy tax payers and threaten them with
penalties and fines to obtain compliance. All taxpayers that have not
paid the taxes are subject to this uniform second treatment by the
collection agencies.  Our estimates of the longer term treatment effects
thus reflect two treatments: our initial letter treatment plus the
phone calls performed by the collection agencies. Since we were not able
to randomize on the treatment by the collection agency, we can only
identify the effect of the joint treatments.

Our findings with respect to longer-run outcomes indicate that there
was at least some convergence in the effectiveness of the nudge
strategies. After 6 months, 73 percent of households in the holdout
sample had made some payments to the City. The only letters that
significantly improve the compliance above that rate were the lien and
sheriff letters, which increased compliance by 3 to 4 percentage
points. The six-month impact of these two letters was approximately
\$31 per letter relative to the holdout sample and \$21 relative to
the control letter. As such it is clear that there has been a fair
amount of convergence in the effectiveness of the different treatments
at the six month stage. We do not know whether this convergence in the
effectiveness is due to additional treatment by the collection agencies
or whether it would have also occurred in the absence of the second
treatment since all persistently delinquent property owners were assigned
to collection agencies after three months.

We find that the impact of all of our six letters relative to the
holdout or control group is much attenuated after six months. This
finding is consistent with the view that the enforcement methodologies
of the collection agency are probably more closely aligned with our
deterrence letters than with the other five letters that we
explored. As a consequence it is not surprising that we find some
strong convergence in the effectiveness of all treatments after six
months. Finally, we investigated whether our experiment had any
spill-over effects into the following tax year. Looking at compliance
rates through July 11, 2016, we find no statistically detectable
impact of any of our letters on subsequent-year compliance.

Ignoring the opportunity costs of time, the cost of our experiment was
a mere \$23,000.  Our back-of-the-envelope calculations suggest that
our experiment generated approximately \$690,000 after three
months. We estimate that if extended to the whole of our sample of
delinquent taxpayers, the two most effective nudges -- the lien or
sheriff's sales reminders -- have the potential to increase collected
revenues during each tax year by as much as 3 to 4 million dollars.

The rest of the paper is organized as follows. Section 2 contains a
brief literature review. Section 3 discusses details of our field
experiment including a detailed description of the treatments and the
randomization procedure. Section 4 discusses our randomization
procedure.  Section 5 reports the main empirical findings. Section 6
offers conclusions and briefly discusses the policy impact of our
experiment and discussions on enforcement activities in the City of
Philadelphia.
    

\section{Literature Review}

Our study is related to different branches of the empirical literature
on tax compliance. Early empirical studies focused the effectiveness
of penalties and fines and found little impact of such penalties on
aggregate tax compliance \cite{Slemrod-07}.  More recently studies
have found an impact of threats of fines on increasing tax payments
among individuals for an Austrian TV license fee \cite{Fellner-13} and
for rental incomes \cite{Wenzel-Taylor-04}.  \citeA{Hallsworth-14}
also find the speed with which taxpayers pay their liabilities can
also be improved with increased fines.  In contrast, \citeA{Ariel-12}
find that threats of fines reduced corporate tax compliance in Israel.

A range of alternative theories have been suggested to explain
taxpayer compliance behaviors.\footnote{For a survey, see,
  \citeA{Andreoni-Erard-Feinstein-98}.}  First is incomplete
information.  Perhaps taxpayers are honest, but simply do not
understand what their true obligations are. Tax forms and regulations
can be complicated.  A recent study of Finnish small business owners
showed that explicitly mentioning the rate change in VAT as part of a
general questionnaire regarding tax administration significantly
increased tax compliance to the higher rate \citeA{kosonen}.

Second is dishonesty -- true obligations may be known to the taxpayer,
but they may choose to cheat.  They can do so in two ways.  When taxes
are self-assessed (e.g., income tax, VAT, profits), taxpayers can
under-report incomes or sales and over-report costs and purchases, or
they can simply not pay by working outside the formal economy.
Research on Danish and Chilean taxpayers found that taxpayers'
reported incomes and value-added sales increased as the ability of the
tax administration to independently assess those income and sales
improved via outside reporting \citeA{Kleven-11} and
\citeA{Pomeranz-15}.  Reported taxable incomes by Minnesota residents
were also found to increase when the probability of an official tax
audit was increased \cite{Slemrod-01}.

When tax obligations are known to both the taxpayer and the tax
authorities, citizens may still choose to cheat if the likelihood that
they will be detected, prosecuted, and/or fined are low.  From the
economic model of tax compliance, taxpayers make their decision to
comply by balancing the economic savings from non-payment against the
uncertain costs they bear from being caught and fined as first
specified by \citeA{Allingham-Sandmo-72}.  In most studied instances
of tax compliance, however, the probability of being caught and the
associated fines are too low to rationally account for the observed
high rates of tax compliance.  Nor can the answer be found in any
plausible estimate of taxpayer risk aversion \cite{Alm-92}. Efforts to
understand taxpayer compliance need to consider explanations beyond
the narrow framework of individual utility maximization under
uncertainty.

There are two extensions of the usual economic framework to consider.
The first re-specifies the taxpayer's utility from income to allow for
non-convex reactions to equal gains and losses.  A recent study of
Swedish taxpayers finds loss aversion as defined by prospect theory
can account for taxpayer compliance in a way that classical utility
maximizing behavior with risk aversion cannot \cite{Engstrom-15}.
Taxpayers facing a loss from a \$1000 tax payment were significantly
more likely to overstate allowed deductions than taxpayers facing a
\$1000 refund for the same deductions.

The second extension retains the classic specification for taxpayer
welfare from income, but adds "tax morale" as one or more additional
motives for payment \cite{Luttmer-14}.  Tax morale includes
reciprocity or payment for public goods received; norm behavior or
peer effects; and civic duty.  Reciprocity argues that citizens
understand that to not pay their taxes will mean less public services.
In this case, government services along with after-tax income
determine taxpayer welfare.  One would expect this motive to be
strongest when tax payments are directly linked by the taxpayer to
services received like local street repairs.  Peer effects may arise
when citizens view non-payment as a violation of a community norm of
collective compliance and an individual's non-payment is observed by
others in the community.  Here, how many other taxpayers are compliant
matters to whether the citizen also pays\citeA{Posner-00}.  One might
expect this motive to be strongest when a citizen's non-payments are
publicized and the citizen is actively linked to a community group
that benefits from those payments.  Finally, citizens may pay their
tax obligation because they view it as the ``right thing to do'' as a
citizen.  Here the act of payment has value on its own; there are no
direct benefits and no one else need know.  The citizen has accepted
the social contract and bears a presumptive obligation to fulfill it
\cite{Rawls-71}.

Efforts to empirically identify the possible influence of these
non-economic motives have been mixed.  \citeA{Blumenthal-01} find no
evidence that these motives significantly influence truthful reporting
of taxable income for Minnesota taxpayers, but \citeA{Hallsworth-14}
do find a strong beneficial impact on compliance from peer motives.
A finding also supported by \citeA{delcarpio} in a recent study on
social norms and deterrence in Peru, which reported stronger 
evidence of responsivity to reminders emphasizing rates of 
compliance by taxpayers rather than rates of detection.
\citeA{Perez-Toiano-15} find that shaming penalties have a large
effect on repayment of smaller debt amounts, but no effect on larger
debt amounts.  In a study closest to our work here, \citeA{castro}
examine motives for property tax payments in a municipality in
Argentina.  They find that the economic motives from fines and
enforcement are most salient, but that the non-economic motives do
matter for selected subsamples of the population--in particular,
lower-income residents.

We conducted an earlier pilot study of property tax compliance in
Philadelphia.  The results are reported in \fullciteA{CILMS-16}.  We
find evidence that motives driven by reciprocity, peer effects, and
civic duty can positively impact property tax payment compliance. But
our sample was small and focused on repeat tax delinquents. 
In this study we specificallyexamine deterrence, reciprocity, 
peer influence, and civic duty notifications with a larger sample 
of more commonplance tax non-compliers.

  
\section{ A Tax Reminder Experiment}
 

The research setting for this experiment is the City of Philadelphia.
Notices of property tax payments are sent each year on January 1, and
the full balance of taxes are due by March 31.  If payment has not
been received by that date, or the taxpayer has not entered into a
tax-paying plan with the City, fines and interest penalties begin to
accrue.  On April 1, the Department of Revenue (DoR) begins contacting
unpaid accounts, informing taxpayers of taxes due and the accumulation
of fines and penalties for late payment.  Normally, two-thirds of the
delinquent accounts are sent to outside collection agencies acting as
co-counsel for the City; one-third of the delinquent accounts remain
within the Revenue Department for collection.  The outside collecting
agents are reimbursed at the rate of 6 percent of all delinquent
revenues collected by December 31st.  All accounts still delinquent
after that time are then assigned to new collection agents.  Our
experiment was implemented using the universe of taxpayers who owed real
estate taxes for the 2015 tax year alone.
  
Of the 579,828 properties in the city in 2015, approximately 100,000
properties, or 83 percent of all properties, were late as of April
1st.  The sample included in our experiment were the 27,264 properties
remaining with the Revenue Department still owing at least \$10 in
property taxes as of May 15, 2015.  Our sample includes only new
noncompliant taxpayers -- it excludes all chronically delinquent
taxpayers who continued to owe taxes from prior years.  Our experiment
began in mid-June 2015 and continued until December 31, 2015.  To make
sure that our experiment was not contaminated by other treatments, the
DoR agreed to postpone other enforcement activity until August 15. In
particular, no other collection agencies contacted the households in
the sample until approximately the beginning of September.

Our seven reminder letters were designed in coordination with
officials of the Department of Revenue.  Each letter was vetted by the
Department to ensure that it could be understood by a taxpayer with at
least a fourth or fifth grade level of reading comprehension.  Each
letter also provided contact information for assistance for
non-English speaking taxpayers.  The full letter templates are
included in an Appendix.  For brevity we present here the important
distinguishing feature of each letter.  Our control letter provides a
generic reminder to the taxpayer. Specifically:
 
{\it Treatment Letter 1: Control } \\ {\bf Our records indicate that
  you have a balance due of $balance$.  If you have already paid,
  thank you. If not, please pay now or contact us to arrange a payment
  plan. The fastest and easiest way to pay is online at
  \underline{www.phila.gov/pay}. Paying by E-check only costs 35\textcent -
  less than the cost of a stamp!"}

Two letters were mailed to test the efficacy of either of our 
economic penalties.  The first imposes an economic penalty only and is
called the lien letter.  The lien letter notes that the City will
impose a lien on the delinquent property which entitles the City to
deduct the amount of the lien from any future arms-length market sale
of the property.
        
{\it Treatment Letter 2: Lien } \\ {\bf Failure to pay your Real
  Estate Taxes may result in the sale of your property by the City in
  order to collect back taxes. In the past year, we have sold $N$
  properties in your neighborhood at sheriff's sale. Included in these
  $N$ are the following properties near you: $<$three properties and
  their sale dates$>$

  Pay your taxes now to prevent the sale of your property.  Our
  records indicate that you have a balance due of $balance$.}

$N$ is the number of properties sold in the recipient's neighborhood
between June 2014 and May 2015. The three listed properties in the
taxpayer's neighborhood were randomly selected from a list of
properties that had been recently sold and included tax liens on the
sale.  All delinquent taxpayers receiving the lien letter in the same
neighborhood saw the same list of three properties.\footnote{An
  initial plan to select the three lien sale properties nearest each
  delinquent property met with privacy concerns and was therefore not
  pursued.}

The second letter including an explicit mention of an economic penalty
was the sheriff's sale letter.  We view this treatment letter as the
most onerous economically.  It not only imposes the full economic
penalty of taxes plus fines plus interest at the time of sale, but it
forces the sale of the taxpayer's property.  The inconvenience and,
perhaps more importantly, the psychic costs of moving may be
significant.  
  
{\it Treatment Letter 3: Sheriff's Sale} \\ {\bf Failure to pay your
  Real Estate Taxes will result in a tax lien on your property in an
  amount equal to your back taxes plus all penalties and
  interest. When your property is sold, those delinquent tax payments
  will be deducted from the sale price. By paying your taxes now, you
  can avoid these penalties and interest. Properties near you in
  $neighborhood$ that have had liens placed on them include: $<$three
  properties and their sale dates$>$

  Pay your taxes now to avoid a lien being placed on your property.
  Our records indicate that you have a balance due of $balance$.}
  
$N$ is the number of properties sold by sheriff's sale in the recipient's
neighborhood between June 2014 and May 2015.  The three listed
properties in the taxpayer's neighborhood were randomly selected from
a list of properties that had been recently sold through a sheriff's
sale.  Again, all delinquent taxpayers receiving the sheriff's sale
letter and in the same neighborhood saw the same list of three
properties.

The next two reminder letters address the free rider motive for
non-payment.  The first letter appeals for payment from those who
might see their gain from non-payment largely in terms of their
private benefits from neighborhood services, what we call the
neighborhood letter.  

{\it Treatment Letter 4: Neighborhood } \\ {\bf We want to remind you that
  your taxes pay for essential public services in $neighborhood$, such
  as $<$two local amenities$>$, your local police officer, snow
  removal, street repairs, and trash collection. Please pay your taxes
  to help the city provide these services in your neighborhood.}

The neighborhood amenities were chosen at random for each property
from a list of City provided parks, recreation centers, and libraries
in the neighborhood of the delinquent property.  The second free rider
letter appeals for payment from those who see their gain from
non-payment in terms of their public benefits from Philadelphia-wide
services, what we call the community letter.  This letter reads:

{\it Treatment Letter 5: Community} \\ {\bf Your taxes pay for important
  services that make a city great. Your tax dollars are essential for
  ensuring all Philadelphia children receive a quality education and
  all Philadelphians feel safe in their neighborhoods. Please pay your
  taxes as soon as you can to help us pay for these important
  services.}

The final two reminder letters appeal to a taxpayer's sense of
community more generally.  The first asks the delinquent taxpayer to
recognize that he is not a contributing member of his (personally
defined) community of peer taxpayers, a letter we call the peer
letter.  

{\it Treatment Letter 6: Peer} \\ {\bf You have not paid your Real
  Estate Taxes. Almost all of your neighbors pay their fair share: 9
  out of 10 Philadelphians do so. By failing to pay, you are abusing
  the good will of your Philadelphia neighbors.}

The second letter stresses that non-payment will violate a wider
community norm of honest and responsible tax compliance needed for a
functioning democracy, a letter we call the civic duty letter. 

{\it Treatment Letter 7: Duty } \\ {\bf For democracy to work, all
  citizens need to pay their fair share of taxes for community
  services. You have not yet paid your taxes. By failing to do so, you
  are not meeting your duty as a citizen of Philadelphia.}
\footnote{We also tested one more intervention that has been
  successfully used by private firms in collecting overdue credit card
  payments.  This is to send the payment reminder in an envelope
  larger than the usually-sized envelopes used for the first mailing
  of tax bills.  Credit card firms have found that reminders mailed in
  usual envelopes (4 1/8" $\times$ 9") were often ignored, while reminders
  mailed in larger envelopes (9" $\times$ 12") resulted in greater payments.
  The total number of properties in this additional treatment was
  12,193 randomized over the seven treatment letters.  We found no
  statistically significant effect of letter size on compliance
  behavior or size of payment.  These results are available upon
  request.}
  
\section{Randomization Procedure}

Randomization took place in two stages. 
As a baseline control, we randomly removed 3,000 delinquent properties
from the possibility of receiving any reminder letter at all.  These
taxpayers became our holdout sample and allowed us to estimate the
efficacy of simply communicating with the taxpayer after the date that
taxes are due. We next grouped all remaining properties by owner
and block randomized all owners to treatments based on the total
amount of property taxes owed on all of their properties. Since most
property owners and delinquent property owners own only one property,
our main interest in this study is on unary owners, i.e. households that only
own one property in the city. Once we restrict attention to this
sample,we have 16,940 observations in the treatment group and
2,088 observations in the holdout sample.  The total sample size is
19,028.\footnote{We also trimmed the sample and excluded the 28
  owners with highest total assessed property value. None of the
  findings reported in the paper depend on this trimming.}  Table
\ref{bal_hold} checks whether the treatment and control group are
balanced based on the two most important variables, amount due and
assessed property value.

\begin{table}[ht]
\centering
\caption{Balance between Holdout and Treated Samples}\label{bal_hold}
\bigskip
\begin{tabular}{lrrc}
   \hline
Variable & Treated & Holdout & $p$-value \\ 
  Amount Due (June) & \$1,287 & \$1,233 & 0.24 \\ 
  Assessed Property Value & \$144,145 & \$142,630 & 0.93 \\ 
  \# Owners & 16,940 & 2,088 &  \\ 
   \hline
\end{tabular}
\end{table}

Table \ref{bal_hold} shows that randomization was successful in the
unary owner sample.  The average debt owed by each owner was \$1,287
in the treatment group and \$1,233 in the holdout sample. The average
assessed property value is \$144,145 in the treatment group and
\$142,630 in the control group.

Next we test whether randomization was successful among the seven
treatment groups. Table \ref{balance} shows the results for the unary
owner sample in the top panel of the table. Overall, we find no
evidence that would suggest any problems with randomization.

While the vast majority of properties in the city of Philadelphia are
owned by unary owners, approximately 10 percent of the properties are
owned by individuals or firms that own multiple properties. There is
some interest in including these multiple owners in the analysis as
well. Since we are interested in taxpayer compliance and not property
compliance, we identified owners of multiple delinquent properties by
their legal name and randomly assigned each owner to a treatment
group.\footnote{We lacked an objective identifier such as a social
  security.  There is some possibility that two or more different
  owners have the same name, but inspection by the authors found this
  to be very rare.  To the extent that it occurs, we consider this
  random noise to the experiment.} Any delinquent taxpayer holding
multiple properties within each treatment group received the same
letter for each of those properties.  Given the high correlation
between the propensity to pay taxes and total debt owed, randomization
blocks were defined according to owner-level total debt to assure
uniformity of samples along the dimension of debt owed. Each property
assigned to receive a reminder letter was equally likely to receive
each of the seven treatments.

\begin{sidewaystable}[htbp]
\centering
\caption{Balance on Observables}\label{balance}
\bigskip
\begin{tabular}{lrrrrrrrc}
\hline
\multicolumn{9}{c}{Unary Owners} \\
\hline
Variable & Control & Neighborhood & Community & Duty & Peer & Lien & Sheriff & $p$-value \\ 
\hline
Amount Due (June) & \$1,256 & \$1,289 & \$1,290 & \$1,299 & \$1,280 & \$1,280 & \$1,315 & 0.98 \\ 
Assessed Property Value & \$158,370 & \$159,079 & \$130,265 & \$165,617 & \$130,936 & \$130,642 &
 \$134,334 & 0.46 \\ 
\# Owners & 2,419 & 2,387 & 2,441 & 2,432 & 2,416 & 2,429 & 2,416 & 0.99 \\ 
\hline
\multicolumn{9}{c}{Unary and Multiple Owners} \\
\hline
Variable & Control & Neighborhood & Community & Duty & Peer & Lien & Sheriff & $p$-value \\ 
\hline
Amount Due (June) & \$1,593 & \$1,589 & \$1,583 & \$1,583 & \$1,572 & \$1,593 & \$1,590 & 1 \\ 
Assessed Property Value & \$180,664 & \$180,172 & \$153,528 & \$183,991 & \$155,438 & \$155,499 & \$157,398 & 0.48 \\ 
\% with Unary Owner & 87.6 & 86.4 & 88.4 & 88.1 & 87.5 & 88.0 & 87.5 & 0.42 \\ 
\% Overlap with Holdout & 3.69 & 3.73 & 3.40 & 3.40 & 3.55 & 3.44 & 3.29 & 0.97 \\ 
\# Properties per Owner & 1.27 & 1.32 & 1.26 & 1.26 & 1.26 & 1.26 & 1.26 & 0.67 \\ 
\# Owners & 2,762 & 2,762 & 2,762 & 2,762 & 2,762 & 2,761 & 2,762 & 1 \\ 
\hline
\multicolumn{9}{l}{\scriptsize{$p$-values in rows 1-5 are $F$-test
    $p$-values from regressing each variable on treatment dummies. A
    $\chi^2$ test was used for the count of owners.}} \\
\end{tabular}
\end{sidewaystable}

Excluding the holdout sample but including multiple owners gives us a
sample size of 19,362 observations.\footnote{Unfortunately, we were
  not able to include the holdout sample in the block-randomization
  procedure. As a consequence, we can only include the holdout sample
  into our analysis if we condition on unary ownership.} Table 2
displays the balance tests for pre-randomization characteristics
\ref{balance}.  Results confirm that randomization was also successful
in this larger sample that included multiple property owners.  There
are no statistically significant differences across reminder letters.

\section{Empirical Results}

\subsection{Short Term Impact}

In this section we focus on the short-term impact of our intervention,
which we define as the first three months after our intervention
letters were posted. During this time period, tardy taxpayers were
only exposed to our intervention. As a consequence, our estimates of
the treatment effects are not contaminated by other interventions by
the tax authority.

To gain a more complete insight into the nature of tax compliance in
Philadelphia, we consider two distinct measures of tax compliance. We
define partial compliance as a tardy taxpayer making any real estate
tax payment at all.  Partial Compliance is of interest because even
small additional payments help, but perhaps more importantly, a tax
contribution represents a willingness by the taxpayer to be engaged
with city governance.  Further, it is common for late taxpayers to pay
down their debt gradually instead of in lump sum. The ever-paid
outcome in particular does not differentiate between taxpayers that
made full repayment and those who made only a partial contribution.
Full compliance is defined as eliminating real estate tax debt.

\begin{table}[ht]
\centering
\caption{Short-Term Linear Probability Model Estimates} \label{pc_lin}
\bigskip
\begin{tabular}{l c c c c }
\hline
 & \multicolumn{2}{c}{Ever Paid} & \multicolumn{2}{c}{Paid in Full} \\
          & One Month & Three Months & One Month & Three Months \\
Holdout   & $30.5$ & $51.4$ & $23.5$ & $40.8$ \\
\hline
Control   & $3.8^{***}$  & $3.9^{***}$  & $2.2^{*}$    & $3.0^{**}$   \\
          & $(1.4)$      & $(1.5)$      & $(1.3)$      & $(1.5)$      \\
Neighborhood & $1.7$        & $2.7^{*}$    & $-0.2$       & $1.5$        \\
          & $(1.4)$      & $(1.5)$      & $(1.3)$      & $(1.5)$      \\
Community     & $3.8^{***}$  & $2.8^{*}$    & $1.3$        & $2.5^{*}$    \\
          & $(1.4)$      & $(1.5)$      & $(1.3)$      & $(1.5)$      \\
Duty      & $2.4^{*}$    & $3.6^{**}$   & $0.7$        & $2.3$        \\
          & $(1.4)$      & $(1.5)$      & $(1.3)$      & $(1.5)$      \\
Peer      & $3.9^{***}$  & $3.5^{**}$   & $1.8$        & $3.4^{**}$   \\
          & $(1.4)$      & $(1.5)$      & $(1.3)$      & $(1.5)$      \\
Lien      & $9.0^{***}$  & $9.2^{***}$  & $5.6^{***}$  & $7.2^{***}$  \\
          & $(1.4)$      & $(1.5)$      & $(1.3)$      & $(1.5)$      \\
Sheriff   & $7.4^{***}$  & $8.8^{***}$  & $4.5^{***}$  & $6.8^{***}$  \\
          & $(1.4)$      & $(1.5)$      & $(1.3)$      & $(1.5)$      \\
\hline
Num. obs. & 19028        & 19028        & 19028        & 19028        \\
\hline
\multicolumn{5}{l}{\scriptsize{$^{***}p<0.01$, $^{**}p<0.05$,
    $^*p<0.1$. Holdout values in levels; remaining figures relative to
    this}}
\end{tabular}
\end{table}

We start by considering the partial compliance results that pertain to
the sample in which we exclude owners of multiple properties.  Table
\ref{pc_lin} reports the estimated participation rates in the holdout
sample as well as the differences in participation among the seven
treatment samples. Standard errors clustered within randomization
blocks are reported in parentheses. We
find that all seven treatments increased partial compliance at the
one- and three-month snapshots. Almost all of these increases in
compliance behavior are statistically significant at standard levels
of significance.

After one month, approximately 30 percent of all taxpayers in the
holdout sample had made some contribution towards their tax
liabilities. In contrast, 40 percent of those that received the lien
letter and 37 percent of those that received the sheriff's sale
letters had made payments. The results are similar after three months
of the intervention.  The overall participation rate rose, with 51
percent of all owners in the holdout sample having made some
contribution after three months.  This is in turn dwarfed by the 61
percent of households that received the lien letter and 60 percent of
households that received the sheriff's sale letter that made some
payments in the same interval.

As shown in Table \ref{pc_lin} the results are qualitatively and
quantitatively the same if we use ``paid in full'' as our compliance
outcome. The main difference is that the neighborhood, community and
duty letters lead to a significant increase in compliance relative to
the holdout group only for partial compliance, though the sign of the
estimate is mostly the same. All other findings are similar. As a
robustness check we also estimated Logit models.  Not surprisingly,
the main findings are qualitatively and quantitatively the same.

Next we conduct some simple back-of-the-envelope calculations to
assess the impact of these estimates on revenues. Here we focus on the
results after three months. We take the median nonzero eventual
payment (i.e., the median positive remission by year's end) in each
subsample and multiply the median payment with the increase in the
compliance probability reported in Table \ref{pc_lin}. This product
can be interpreted as the impact of each treatment on revenue per
letter. To obtain the total estimated impact we then multiply the
impact per letter with the total number of individuals in each
treatment. These results are reported in Table \ref{sh_rev}.  Overall,
we find that all seven treatments generated positive revenues for the
City.  The three-month impact of these letters ranged between \$21 for
the neighborhood letter and approximately \$76 per the lien letter. We
thus conclude that receiving a reminder letter improved taxpayer
compliance in the short run, and further that the lien and sheriff's
sale letters were the most effective in inducing repayment.

\begin{table}[htbp]
\caption{Estimated Three-Month Impact on Revenue}\label{sh_rev}
\bigskip
\centering
\begin{tabular}{lcc}
  \hline
Treatment & Impact Per Letter & Total Impact \\ 
  \hline
Control & \$32.51 & \$78,634 \\ 
  Neighborhood & \$22.32 & \$53,287 \\ 
  Community & \$23.61 & \$57,623 \\ 
  Duty & \$30.05 & \$73,084 \\ 
  Peer & \$28.95 & \$69,955 \\ 
  Lien & \$76.4 & \$185,580 \\ 
  Sheriff & \$73.27 & \$177,020 \\ 
   \hline
\end{tabular}
\end{table}

Another way to determine the revenue implications of our different
treatments is to regress the total amount of revenue raised on
indicator variables for each treatment.  These regressions confirm our
estimates reported in Table \ref{sh_rev}. We find that the average
payments in the holdout sample were \$323 after one month and \$636
after three months. All of our letters, including the control
treatment, increased payments at the one- and three-month
cross-sections. The two threat letters were the only two letters that
significantly increased revenue collection. After one month the lien
(sheriff) treatment increased payments by \$90 (\$69). After three
months, the increases are approximately \$97 per letter for both of
these treatments.  This supports our assertion that the estimates
reported in Table \ref{sh_rev} are conservative estimates of the
effectiveness of our treatments.


Finally, we conducted a number of robustness checks. Recall that we
randomized the seven treatments at the ownership level. In Table
\ref{sh_lpm_rob} we replicate the analysis done above, excluding the
holdout sample and expressing estimates relative to the control
treatment. We then estimate the model using the larger sample that
also includes owners of multiple properties. Overall, we find the
results are similar to the ones reported in Table \ref{pc_lin}. If
anything, the treatment effects are stronger in the unary owner
sample. This evidence leads us to conclude that owners of multiple
properties are less likely to respond the kind of nudge strategies
explored in this paper.

\begin{table}
\caption{Robustness Analysis: Multiple Owners}
\begin{center}
\begin{tabular}{l c c c c }
\hline
 & \multicolumn{2}{c}{All Owners} & \multicolumn{2}{c}{Single-Property Owners} \\
          & One Month & Three Months & One Month & Three Months \\
\hline
Neighborhood & $-0.01$      & $-0.01$      & $-0.02$      & $-0.01$      \\
          & $(0.01)$     & $(0.01)$     & $(0.01)$     & $(0.01)$     \\
Community     & $-0.00$      & $-0.01$      & $0.00$       & $-0.01$      \\
          & $(0.01)$     & $(0.01)$     & $(0.01)$     & $(0.01)$     \\
Duty      & $-0.01$      & $-0.00$      & $-0.01$      & $-0.00$      \\
          & $(0.01)$     & $(0.01)$     & $(0.01)$     & $(0.01)$     \\
Peer      & $0.00$       & $-0.01$      & $0.00$       & $-0.00$      \\
          & $(0.01)$     & $(0.01)$     & $(0.01)$     & $(0.01)$     \\
Lien      & $0.05^{***}$ & $0.05^{***}$ & $0.05^{***}$ & $0.05^{***}$ \\
          & $(0.01)$     & $(0.01)$     & $(0.01)$     & $(0.01)$     \\
Sheriff   & $0.03^{**}$  & $0.05^{***}$ & $0.04^{**}$  & $0.05^{***}$ \\
          & $(0.01)$     & $(0.01)$     & $(0.01)$     & $(0.01)$     \\
\hline
Num. obs. & 19333        & 19333        & 16940        & 16940        \\
\hline
\multicolumn{5}{l}{\scriptsize{$^{***}p<0.001$, $^{**}p<0.05$, $^*p<0.1$}}
\end{tabular}
\label{sh_lpm_rob}
\end{center}
\end{table}


\subsection{Long-Term Impact}

Recall that all tardy tax payers were assigned to a collection agency
in the middle of August 2015 and were thereby subjected to another
enforcement activity which largely consisted of another ``threat''
treatment -- phone calls that threatened them with penalties and fines
to coax compliance.  While our experimental design is still valid for
these outcomes, it is harder to cleanly interpret the findings due to
this mixed bag of interventions.  All taxpayers that had not paid by
mid-August were subject to this uniform second treatment by a
collection agency.  Our estimates of the longer-term treatment effects
thus reflect two treatments: our initial letter treatment plus the
phone calls performed by the collection agency. Since we were not able
to randomize on the treatment by the collection agency, we can only
identify the effect of the joint treatments.

\begin{table}
\caption{Long-Term Linear Probability Model Estimates}
\begin{center}
\begin{tabular}{l c c c c }
\hline
 & \multicolumn{2}{c}{Ever Paid} & \multicolumn{2}{c}{Paid in Full} \\
          & Six Months & Tax Year 2016 & Six Months & Tax Year 2016 \\
Holdout   & $73.3$ & $65.5$ & $63.2$ & $52.5$ \\
\hline
Control   & $1.3$        & $-1.4$       & $1.5$        & $-0.7$       \\
          & $(1.3)$      & $(1.4)$      & $(1.4)$      & $(1.5)$      \\
Neighborhood & $-0.2$       & $-3.1^{**}$  & $-0.0$       & $-2.2$       \\
          & $(1.3)$      & $(1.4)$      & $(1.4)$      & $(1.5)$      \\
Community     & $0.9$        & $-1.8$       & $1.1$        & $-2.0$       \\
          & $(1.3)$      & $(1.4)$      & $(1.4)$      & $(1.5)$      \\
Duty      & $2.1$        & $-1.6$       & $1.0$        & $-1.9$       \\
          & $(1.3)$      & $(1.4)$      & $(1.4)$      & $(1.5)$      \\
Peer      & $1.3$        & $-1.9$       & $2.3$        & $-1.1$       \\
          & $(1.3)$      & $(1.4)$      & $(1.4)$      & $(1.5)$      \\
Lien      & $3.8^{***}$  & $-0.9$       & $4.8^{***}$  & $-0.7$       \\
          & $(1.3)$      & $(1.4)$      & $(1.4)$      & $(1.5)$      \\
Sheriff   & $3.8^{***}$  & $-0.6$       & $3.0^{**}$   & $-1.1$       \\
          & $(1.3)$      & $(1.4)$      & $(1.4)$      & $(1.5)$      \\
\hline
Num. obs. & 19028        & 19025        & 19028        & 19025        \\
\hline
\multicolumn{5}{l}{\scriptsize{$^{***}p<0.01$, $^{**}p<0.05$, $^*p<0.1$.
 Holdout values in levels; remaining figures relative to this}}
\end{tabular}
\label{lg_pc_lin}
\end{center}
\end{table}


Given the similarity of results, we focus on the unary owner sample.
Table \ref{lg_pc_lin} reports the estimated participation rates in the
holdout sample as well as the differences in participation in the
seven treatment samples after six months of the intervention.  We find
that only two of the treatments increased partial compliance at the
six-month juncture.  We find that 73 percent of households in the
holdout sample made some payments to the City. The only letters that
significantly improve the compliance above or below that rate were the
lien and sheriff letters, which increased compliance by 3 to 4
percentage points. The findings are similar using full repayment as
the outcome measure.

To translate these participation rates into revenue, we again take the
median nonzero payment in each subsample and multiply it with the
increase in compliance probability reported in Table \ref{lg_pc_lin}.
The results are reported in Table \ref{lg_rev}. We find that the
six-month impact of these two letters was approximately \$31 per
letter relative to the holdout sample and \$21 relative to the control
letter.\footnote{Again, we also ran a revenue regression as a
  robustness check, but it was inconclusive. Only the duty treatment
  was marginally significant, which was due to the influence of a few
  outliers.}

\begin{table}[htbp]
\caption{Estimated Six Month Impact on Revenue} \label{lg_rev}
\bigskip
\centering
\begin{tabular}{lcc}
  \hline
Treatment & Impact Per Letter & Total Impact \\ 
  \hline
Control & \$10.74 & \$25,976 \\ 
  Neighborhood & -\$1.47 & -\$3,517 \\ 
  Community & \$7.2 & \$17,568 \\ 
  Duty & \$17.32 & \$42,134 \\ 
  Peer & \$11.16 & \$26,972 \\ 
  Lien & \$31.42 & \$76,328 \\ 
  Sheriff & \$31.43 & \$75,940 \\ 
   \hline
\end{tabular}
\end{table}

We find that the impact of our six letters relative to the holdout or
control group is much attenuated after six months. This finding is
consistent with the view that the enforcement activities of the
collection agency are probably more closely aligned with our
deterrence letters than with the other five letters that we
explored. As a consequence, it is not surprising that we find some
strong convergence in the effectiveness of all treatments after six
months -- as the threat-like approach of the agency comes to the fore,
the much earlier receipt of our treatment has a dwindling impact.
However, it is also possible that some convergence of the
effectiveness of treatment would have resulted even in the absence of
the introduction of the collection agency. Since we were not allowed
to randomize on that treatment, we cannot offer a definitive
conclusion.

For revenue collecting agencies, particularly those with revenue
collection problems, acceleration of payment can be understood to be a
useful result in and of itself. Bills must be paid and debts must be
serviced on regular schedules. Relatedly, the longer that tax bills
remain unpaid, the more expensive it becomes to collect. Whether
handled internally or externally through debt collection firms,
downstream collection practices leave diminished revenues. For both of
these reasons, early collection is, \textit{ceteris paribus}, better
collection. This is not to say, however, that early collection is
social-welfare-improving. If tardy-but-eventually-compliant taxpayers
forestall early payment to cover other expenses or invest
in other assets which they eventually use to repay their tax bill with
interest, late repayment may have been individually optimal, 
in which case the welfare of the tax agency may not be synonymous
with the welfare of society. This is especially true if tax payments
are made weeks rather than months late, such that monthly payments can
still be made based on expected monthly receipts.  In the case
of chronically-delayed but consistently-paid payments, it is less
obvious that accelerating eventual or inevitable payment constitutes
something of value.

We also obtained data characterizing tax compliance of our sample of
taxpayers in the tax year 2016 which followed our 2015 intervention.
The goal was to investigate the existence of any spill-over effects of
our treatments into the next tax year. Again, our analysis is subject
to the same mulled-treatment constraint discussed above.  Focusing
again on the unary owner sample, the second and fourth columns of
Table \ref{lg_pc_lin} summarize our main findings. Overall, we find
that none of our treatments had any continued differential impact on
tax compliance in 2016. However, it is worth noting that the overall
compliance rate in 2016 was significantly higher than that in 2015,
which may be a direct consequence of our increased enforcement
activities.
 

\section{Conclusions}


We have designed and implemented a new, multi-arm field experiment
designed to test which low-cost notification strategies increase
property tax payment rates. Our results suggest that notification
strategies are effective at reducing tax delinquency and increasing
tax revenue collection.  Our different treatments are motivated by a
wide swath of theorized ``nudges'' including social norming, moral
suasion, tax morale, and deterrence. Our results suggest that all
treatments are successful in increasing tax compliance and raising
revenue for the City in the short run. Hence, a revenue director can
choose from a menu of effective messages to increase tax compliance and
revenue collection.

There are, however, some important quantitative differences in the
effectiveness of the different messages, which imply some trade-offs
faced by the revenue director. Credible threats regarding the
foreseeable consequences of nonpayment are much more effective than
other treatments that rely on moral persuasions or appeals to civic
duty. While these results differ from some recently reported studies
\cite{delcarpio,Hallsworth-14}, it is quite consistent with others 
that share a focus on estimating the impact of notification strategies
on tax delinquents rather than all potential taxpayers\cite{castro,Hallsworthb-14}.
If a revenue director wants to generate more property tax
collection in a setting such as Philadelphia, it appears that he or she should choose a
tougher and politically more costly message.\footnote{After we concluded this experiment, 
we debriefed the city about our results. The city then decided to send a version of our
sheriff's sale letter to all late tax payers in the summer of
2016. The appendix shows the letter used by the city which is 
almost identical to the letter that we used in our experiment.
Compared to the years before the experiment, in which roughly 24,000 accounts
were sent for action by outside revenue collections firms (23,187--FY'14 and 24,922--FY'15), in the year of the experiment (FY'16) only 18,004 accounts were sent
for collection. And in the year following the experiment (FY'17), ....}

The long-run effects of our intervention are somewhat harder to assess.  We
still find that credible threat notifications increase repayment rates
and revenues after six months, although the effects are smaller than
the ones observed in the short run.  All tax payers that were still
tardy three months after our experiment were contacted by a collection
agency that engaged in serious enforcement activities. We think it is
likely that this additional treatment may explain the convergence in
the effectiveness of our treatments that we observe in the data.

Overall, our findings suggests that increasing tax compliance at low
cost is possible. More research is needed on the reasons for different
behavioral responses to economic and non-economic messages. 
Perhaps the differences in the effectiveness of the various
treatments lie in alternative theories of why tax delinquents exist in
the first place? Non-economic theories of non-payment
implicitly rest on the assumption that non-compliers are not
liquidity-constrained and are merely unaware of the collective
consequences of non-payment. Under this analysis, tax delinquency is
due to discouragement, indifference, lack of appreciation, or
unawareness. Delinquents merely need to be encouraged or reminded to
participate. If tax delinquents are discouraged, providing them with
information about peer behavior, neighborhood or civic duty increases
the overall rate of tax compliance. This may reflect the fact that
delinquents are not indifferent to their peers' positive and negative
behavior. Likewise, the provision of information on public goods
assumes that recipients have not incorporated consideration of public
goods funded through tax dollars into their payment behavior.  Our
results, however, suggest that this is an incomplete explanation for
tax delinquency in Philadelphia.\footnote{We explored differences in
envelope size as a potential avenue to increase revenues. We observe
no effect of increased envelope size.  Apparently, taxpayers open
tax bills even if they do not pay them.}

Our results do clearly suggest that many tax delinquents are seemingly
unaware of the consequences of nonpayment. Our provision of detailed
information about the collection process had a clear impact on the
likelihood of repayment. This result echoes other recent findings that
clear, consistent, and timely provision of information on
consequences, particularly in the context of compliance monitoring,
can lead to notable improvements in behavioral compliance -- see also
\citeA{hawken,Bhargava-15}. Perhaps, then, the puzzle of high non-payment rates
can be understood as a case of under-enforcement. This possibility is
bolstered by the fact that, conditional on any payment, converted
defiers became near perfect compliers, making full payments in almost
all cases. This suggests that, at least for the margin affected,
liquidity constraints are not the primary reason for initial
non-payment.

Notification strategies that convey information about enforcement
activities are best viewed as complements and not substitutes to
traditional enforcement activities.  Notifications provide information
about enforcement, thus resolve some problems that arise because some
tax payers are not well informed about the consequences of
non-compliance.  Providing credible and tangible information can then
help to overcome this incomplete information problem.  However, any
threat to take a lien on a house or to start a foreclosure process can
only be credible if it is backed up by real actions. It is hard to
believe that notifications strategies that convey empty threats on
enforcement activities can be effective.

%Our research had some direct impact on the tax enforcement activities
%of the City of Philadelphia. After we finished our initial pilot
%study, the City decided to mail letters to delinquent owners of more
%than 4,000 selective properties, warning them of an impending lien
%sale. the Philadelphia Magazine reported in July 2015 that ``the
%threat was clear: If they didn't pay their delinquent taxes, the city
%would try to sell their debts to a third party. Out of the 4,000-plus
%properties, 1,419 of the owners paid their delinquent taxes in full,
%while 645 paid their 2015 taxes. The city reaped \$5.5 million in
%cash, as well as an additional \$2.2 million in expected payment
%agreements. [..] In other words, it appears that a lien sale is an incredibly effective
%tool with which to threaten property owners who have not paid their
%taxes.''

The main insights and empirical results of this study are promising
for future research. For example, an interesting additional experiment would
compare the effectiveness of our notification strategies to the effectiveness
of the enforcement activities by the collection agency. Presumably a large fraction
of the success of the collection agencies could be achieved by properly targeted notifications 
at much lower cost. Such an experiment might  allow us to design potentially more cost-effective strategies to increase property tax collection and reduce the dependency on professional collection agencies.


\newpage

%{\footnotesize \NoTitleCaseChange\citepunct{(}{and}{, }{; }{, }{)}{}{.} 

\bibliographystyle{theapa}
\bibliography{references}

%\begin{center}
%\includegraphics[width=6in, height=9in]{2016_letter.pdf}
%\end{center}

\end{document}

