\documentclass[12pt]{article}
\usepackage{amssymb}
\usepackage{theapa}
\usepackage{titlepage}
\usepackage{pdfpages}
\usepackage{amsmath}
\usepackage{setspace}

\usepackage{rotating}
\usepackage[usenames,dvipsnames]{pstricks}
\usepackage{epsfig}
\usepackage{pst-grad}
\usepackage{pst-plot}
\usepackage{color}
\usepackage{pstricks-add}
\usepackage{threeparttable}
\usepackage{array,multirow}
\usepackage{pdflscape}
\usepackage{float,lscape}
\usepackage{csquotes}
\usepackage{textcomp}
\usepackage{indentfirst}
\usepackage{longtable}


\renewcommand{\baselinestretch}{1.5}
\parindent=.2in
\evensidemargin=.05in
\oddsidemargin=-.05in
\topmargin=-0.05in
\textwidth=6.5in
\textheight=8in

\newtheorem{fact}{Stylized Fact}
\newtheorem{theorem}{Theorem}
\newtheorem{corollary}{Corollary}
\newtheorem{definition}{Definition}
\newtheorem{lemma}{Lemma}
\newtheorem{prop}{Proposition}
\newtheorem{assumption}{Assumption}
\newtheorem{remark}[theorem]{Remark}
\newtheorem{solution}[theorem]{Solution}
\renewcommand{\thefootnote}{\fnsymbol{footnote}}


\begin{document}

\title{Deterring Property Tax Delinquency in Philadelphia: \\
An Experimental Evaluation of Nudge Strategies}

\author{Michael Chirico, Robert Inman, Charles Loeffler, \\ John
  MacDonald, and Holger Sieg\thanks{We would like to thank Rob Dubow,
    Clarena Tolson, Marisa Waxman, and Darryl Watson in the Department
    of Revenue of the City of Philadelphia for their help and
    support. We would also like to thank the editor of the journal,
    Stacy Dickert-Conlin, two anonymous referees, Jeff Brown, Judd
    Kessler, Kai Konrad, Robert Moffitt, Jim Poterba, Chris
    Sanchirico, Wolfgang Sch\"on, Reed Shuldiner, and participants of
    numerous seminars for comments and suggestions.  The authors have
    received financial support for this research from the Wharton
    Initiative for Public Policy. The views expressed here are those
    of the authors and do not necessarily represent or reflect the
    views of the City of Philadelphia.}  \\ University of
  Pennsylvania}

\date{\today}

\maketitle

\begin{abstract}

Municipal governments commonly confront the problem of tardy or
delinquent property tax payments.  We implement an experiment in
property tax collection for tardy taxpayers in the City of
Philadelphia for the fiscal year, 2016.  The experiment sent one of
seven reminder letters to the tardy taxpayers, testing the efficacy of
a simple reminder, two alternative reminders stressing economic
sanctions, and four alternative reminders emphasizing either that
taxpayersÕ receive neighborhood services or city-wide services for
their tax payments, that most of their neighbors pay their taxes on
time, or that as a citizen in a democracy it is a civic duty to pay
taxes on time.  Compliance behaviors were compared to a holdout sample
that received no reminder letter.  The most effective letters were
those that threatened an economic sanction for continued
non-compliance.  These letters were particularly cost-effective in
raising additional city revenues.  There was no evidence that those
receiving a reminder for fiscal year 2016 improved their tax
compliance behavior in fiscal year 2017.


\bigskip

\noindent KEYWORDS: Tax Compliance, Property Taxation, Field
Experiment, Deterrence, Public Service Appeal, Appeal to Civic Duty.


% JEL H2, H7
\end{abstract}
\renewcommand{\thefootnote}{\arabic{footnote}}

\newpage

\section{Introduction}


Property taxation is the primary tax for most U.S. cities.  In fiscal
year 2016, 47 percent of all local government revenues and over 72
percent of local taxes came from the property tax
\cite{barnett2013state}.  Yet collection of the tax has, in many
cities, been problematic.  While some U.S. cities do an excellent job
in collecting the tax and receive over 95 percent of assessed revenues
the year the tax is due, other cities have over the last ten years
done significantly worse -- notably Flint (78\%), Cleveland (84\%),
Pittsburgh (86\%), Milwaukee (87\%), Philadelphia (88\%), Detroit
(89\%), and St. Louis (89\%).\footnote{For more details, see
  \citeA{CILMS-16}.}  In this paper, we explore some low-cost
strategies that may help cities improve the compliance of taxpayers.

This failure to collect the property tax on time creates budget
uncertainty at best and budget deficits at worst. Late payments are
costly to the city. If not enforced, delinquent taxpayers may become
permanent tax evaders. Furthermore, significant rates of delinquency
today may become a signal to other taxpayers that avoidance is
possible, encouraging further delinquency.\footnote{
  \citeA{besley2015norms} study local property taxation in England
  following the imposition of a local head tax as a replacement for
  the local property tax. In response to widespread citizen
  resistance, the poll tax was removed two years later and the
  property tax restored. However, compliance rates for the reinstated
  property tax fell by 14 percent. Though efforts to improve
  compliance emphasized high penalties it has taken nearly eighteen
  years to return to the original levels of tax compliance.} Yet
collecting the property tax should be straightforward.  In contrast to
collecting self-reported taxes on income, profits, and sales, property
tax obligations equal the city's assigned assessed value of the
property times the city chosen tax rate, and are known by both the
city and the taxpayer.\footnote{Modern tax systems make widespread use
  of third-party reporting from firms and the financial sector which
  makes income tax evasion more difficult than in the past.
  \citeA{Hallsworth-14}, \citeA{slemrod2017}, and \citeA{Alm-19}
  provide reviews of tax compliance behavior.}  There is no
uncertainty as to what is due, or when. Property tax payment is a
matter of enforcement.

The most common property tax enforcement strategy is the economic
stick: fines and penalties.  When a delinquent taxpayer does not
respond to penalties and fines, the city can issue a tax lien on the
property equal to the value of the taxes owed and accrued interest and
penalties.  Alternatively, the city may appeal to an intrinsic motive
for tax payments, perhaps to encourage a sense of community or to
foster a more positive view of elected officials responsible for
collecting taxes and providing city services.  Intrinsic motives for
paying taxes may include a desire to contribute when others
contribute, to pay a fair share of the cost of services received, or
to fulfill one's civic obligation as a citizen of the community.

To explore each of these motivations, we implemented an experiment in
property tax collection from tardy taxpayers in the City of
Philadelphia for the fiscal year 2016.  In addition, we test for a
possibly simpler explanation.  Perhaps tardy taxpayers are just
forgetful \cite{Akerlof-91} or procrastinators \cite{DR-99}?  If so, a
simple reminder letter may suffice.  Our experiment sent one of seven
different reminder letters to the tardy taxpayers. Compliance
behaviors of those receiving a reminder letter were compared to a
holdout sample that received no reminder letter.  One letter was a
simple reminder. Two letters stressed that continued non-payment leads
to fines and penalties and may lead to a lien on their home with the
possibility of a sheriff's sale of the property.  The final four
letters explored the possibility of an intrinsic, or tax morale, motive
for tax payment: that others are contributing, taxes pay for
neighborhood services, taxes pay for city-wide services, and finally,
that citizens bear a civic duty to contribute to the costs of
democratic government.

Each letter cost one dollar to send and over all seven letters,
returned on average \$37 in increased city tax revenues.  The two
letters stressing economic sanctions were the most effective,
returning \$65 in extra revenues for each letter sent.  In contrast,
adding an intrinsic message to the reminder did not increase
compliance over receiving only the simple reminder.  We do not find
any evidence that those receiving a reminder for fiscal year 2016
improved their tax compliance behavior in fiscal year 2017.

Our experimental design closely follows that used by
\citeA{Hallsworth-17} in their analysis of reminder letters to UK
citizens to encourage timely payment of income taxes already declared.
Their objective, like ours, is to improve both the speed and rate of compliance.
Like us, they find reminders that stress financial penalties encourage
greater compliance, but unlike our results, stressing to the tardy
taxpayer they are now one of a minority of all taxpayers also improves
compliance.

Beyond providing assistance to Philadelphia's Department of Revenue,
our study contributes to the growing literature on the use of
``nudges'' to encourage taxpayer compliance.  Reminders as nudges are
typically inexpensive to implement and, if successful, provide a
valuable additional tool for tax administration
\cite{keen2016optimal}.  Recent empirical research on the use of
nudges to improve tax collections have shown targeted reminders can be
effective motivators for increased compliance.  But success depends
upon the action being nudged and the exact wording of the reminder.
Most successful nudges have been applied to truthful reporting of the
taxpayer's tax base, with the reminders stressing the likelihood of a
taxpayer audit and associated economic fines on unreported income or
sales; see \citeA{Kleven-11} and \citeA{Pomeranz-15} specifically.
Less successful have been reminders that stress an intrinsic or ``tax
morale'' motive for tax reporting or payment. An exception is
\citeA{dwenger-17} who study local church tax compliance in
Bavaria. They report that a significant fraction of individuals comply
in the zero deterrence baseline where compliance should be zero absent
intrinsic motivations. Around 20 percent of individuals pay the true
taxes owed, while the remaining 80 percent of individuals evade taxes
and most of them fully evade.\footnote{See, also,
  \citeA{slemrod2001taxpayer} and \citeA{Luttmer-14} for a review.}

Four other published studies have directly evaluated the impact of
nudges on taxpayer compliance for the paying of local government
taxes.  \citeA{Torgler2004moral} finds no increased compliance
associated with tax morale messages for the payment of income taxes to
Swiss local governments.  \citeA{meiselman-18} finds improved
compliance by ``ghost" filers of Detroit's local income tax for
reminders that stress economic penalties but no improved compliance
for reminders that stress taxes are essential for Detroit's economic
future.  \citeA{Miller-Nikaj-16} examine the impact of tax lien sales
on collected city tax revenues in Ohio.  While the sales generate
direct revenue to cover past taxes owed, they also encourage
additional tax payments by other taxpayers.  They estimate that tax
lien sales increase property tax collection by 5.6 percent above what
had been generated by the sales alone, a result strikingly similar to
what we have found for our reminder letter threatening lien sales.
Finally, in the study closest to ours, \citeA{castro} examine the
payment of local property taxes to Argentine municipalities and find,
as do we, there is significantly greater compliance for those
receiving reminders stressing economic sanctions. There were only
mixed results for those receiving a tax morale message, however.  On
average, the messages stressing peer behaviors or the public service
benefits of paying taxes had no impact on payment.  But there was
significant heterogeneity in responses.  Taxpayers living inside the
city and who had previously paid their taxes reduced their rate of
payments after being reminded that 30 percent of other city taxpayers
do not pay their taxes.  Taxpayers living outside the city and who as
a group have a low rate of compliance increased their rates of payment
in response to both reminders stressing peer rates of payment and the
value of city public services.

These mixed results for the impact of nudges on taxpayer compliance
provide an important reminder that context matters.  Nudges that work
for one tax and for one government may not work in other
settings. There is no general lesson beyond the one that nudges,
in some form and in some settings, can improve compliance.  As a
strategy, nudges can be a useful policy tool for efficient tax
administration, but each nudge must be evaluated and each compared in
a specific setting of tax compliance.

With this conclusion in mind, we report the results for our evaluation
of a policy experiment to increase tax compliance in Philadelphia for
the payment of fiscal year 2016 property taxes.  The initial sample
included all taxpayers who were tardy in their tax payments for that
fiscal year.  After eliminating multiple-property owners, we obtained
a sample of 19,039 single property owners. We assigned 2,088 of these
taxpayers to the holdout sample receiving no reminder.  The remaining
16,951 tardy taxpayers received one of our seven treatment messages: a
simple reminder, either of two reminders stressing either a ``gentle''
or a ``strong'' economic sanction, and one of four reminders stressing
the payment behavior of their neighbors, the role of local taxes in
paying for neighborhood or city-wide services, that with citizenship
comes a civic duty to pay their taxes.

Overall, we find that that all reminder letters are more effective
than doing nothing.  Moreover, the two letters that stress economic
sanctions are more effective than the other five reminder letters. Our
findings do not imply that tax morale is not of primary importance for
property tax compliance. Given the low collection rates in cities such
as Philadelphia, tax morale is clearly a major concern. The only
conclusion that can be drawn from our experiment is that reminder
messages that tried to appeal to the duty to pay taxes or emphasized
the fact that most neighbors pay property taxes did not yield higher
tax collection rates than simple reminders. Our experiment does not
allow us to estimate the impact of changes in tax morale on tax
compliance behavior.

There are primarily three competing explanations for tardiness in
property tax payments: procrastination or forgetfulness, lack of
enforcement, and credit constraints. We find that 70 percent of all
tardy taxpayers fully comply with the tax laws by the end of the
fiscal year.  For the remaining 30 percent we consider the possibility
that they be constrained by their lack of credit to pay their full tax
bill.  If so, access to a tax payment plan that smooths out payments
over the fiscal year may be helpful.  The city offers such a plan.  We
find, however, that no more than 3 to 4 percent of those potentially
credit constrained taxpayers actually use the tax payment program.
Liquidity constraints may bind, but the main barrier to payment is the
taxpayer's inclination to do so.

The paper is organized as follows. Section 2 discusses details of our
field experiment including a description of the treatments. Section 3
discusses our randomization procedure.  Section 4 reports the main
empirical findings. Section 5 discusses the urban fiscal policy implications of our
experiment. Section 6 discusses the effectiveness of nudges. Section 7
offers conclusions.



\section{A Field Experiment }

The research setting for the experiment is the City of Philadelphia
for calendar year, 2015, for the payment of property taxes for fiscal
year 2016.  Notices of property tax payments are sent on
January 1, and the full balance of taxes are due by March 31.  If
payment has not been received by that date, or the taxpayer has not
entered into a tax payment plan with the City, then taxes are
considered tardy and interest and penalties begin to accrue.  On April
1, the City's Department of Revenue (DoR) begins contacting all
taxpayers with unpaid accounts, informing them of taxes due and
accumulated interest and penalties for late payment.  At this time,
the City will normally send two-thirds of the tardy accounts to
outside collection agencies acting as co-counsel for the City. The
outside collection agencies are reimbursed at the rate of six percent
of all their tardy revenues collected by December 31. The remaining
one-third of the tardy accounts remain with the DoR for
collection. All accounts still tardy on December 31 are designated as
``delinquent'' and then assigned to new outside collection
agencies. For the purposes of our experiment the City of Philadelphia
agreed to delay sending any of the tardy accounts to the collection
agencies until August 15, 2015.

Our sample for the experiment began with all taxpayers tardy as of
April 1, 2015, equal to approximately 17 percent of all taxpayers who
received tax bills for the fiscal year.  From the list of all
tardy taxpayers, we first eliminated chronically delinquent taxpayers
from our sample and those who owed less than \$10 in taxes. This
reduced the sample to 21,468 taxpayers. Of these 21,468 tardy
taxpayers, 2,429 taxpayers owned more than one property.\footnote{Tax
  compliance is by taxpayers, not properties.  It is, therefore,
  necessary to assign an owner to each property. Lacking an objective
  identifier of the owner of each property such as a social security
  or taxpayer identification number, we used the owner names attached
  to properties.  We found no instances in which property owners had
  the same legal name.} There is much heterogeneity within the
subsample of multiple property owners by the amount of taxes owed
making it difficult to achieve balance with a handful of debtors owing
substantial and substantially different amounts. We chose, therefore,
to exclude multiple property owners from the analysis reported here.
Our final sample consists of 19,039 taxpayers who owned only one
property. Overall, this sample is less affluent than the overall
sample of all property holders in Philadelphia.

Our intervention sent reminder letters to seven randomly assigned
groups of tardy taxpayers: one letter that is a simple reminder, one
sanction reminder that stresses tax liens at the time of home sale and
a second sanction reminder that threatens a sheriff's sale to pay back
taxes, and four tax morale reminders that note first, taxes pay for
neighborhood services, second, that taxes pay for city-wide services,
third, that nine of ten Philadelphians pay their taxes on time, and
fourth, that citizens have a civic duty to pay their taxes.  Reminders
are low cost interventions and take as given taxpayer preferences and
information regarding the benefits and costs of compliance.  Each
reminder is meant to trigger an immediate awareness, called saliency,
of those benefits and costs.\footnote{See Akerlof (1991) and
  O'Donoghue and Rabin (1999).}  More expensive experimental
treatments to educate taxpayers as to the possible long-run private
and wider public benefits of payment can also be imagined and have
been effective in many settings.\footnote{Effectiveness often depends
  crucially upon a preexisting social or cultural environment, in
  particular a community with homogenous values and a prior trust in
  government; see \citeA{Alm-19} for a review.  For example, the
  long-standing tradition in poorer communities of not paying utility
  bills as a protest against South Africa's apartheid regime continues
  to today, despite an extensive public message campaign featuring
  leading entertainers, sports figures, and respected public officials
  stressing this is now ``our South Africa" and that all citizens have an
  obligation to pay for public services.  Distrust of government in
  poor neighborhoods remains a significant barrier to the payment of
  fees; see Reuters News Service, November 19, 2014, ``South Africa's
  Latest Power Struggle: Unpaid Bills.'' } Our interventions were more
modest.

Our experiment began with the mailing of our experimental reminder
letters in mid-June, 2015 and continued to December 31, 2015.  Of the
tardy taxpayers with a single property, 16,951 received a reminder
letter and 2,088 taxpayers did not receive a reminder.  This sample of
2,088 taxpayers became our ``holdout'' sample and the basis for
identifying the importance of reminders in taxpaying behavior. To
ensure that our experiment was not contaminated by other treatments
not under our control, the DoR agreed to postpone all other
enforcement activities until August 15.  In particular, the outside
collection agencies were not allowed to begin their collection efforts
until after that date.  The likely earliest date that those efforts
led to any contact with a taxpayer was September 15.

Each reminder letter was approved by the City's DoR to ensure that it
could be easily understood by all taxpayers.  Each letter also
provided contact information for assistance for non-English speaking
taxpayers.  Translations were available for a number of different
languages.\footnote{Templates of the ``reminder only'' and ``lien''
  letters are attached in the appendix.  The full template for the
  other letters are available as an online appendix.}

Each reminder letter in our experiment was drafted to identify a
potential channel that may affect taxpayers compliance. For brevity
we present here the important distinguishing feature of each letter.

\noindent \textit{Reminder-only}: \textbf{Our records indicate
that you have a balance due of \textit{balance. }} If you have
already paid, thank you.  If not, please pay now or contact us
to arrange a payment plan.  The fastest and easiest way to pay is
online at  www.phila.gov/pay. Paying by E-check only costs 35 cent
-- less than the cost of a stamp!

The reminder-only letter allows us to identify the potential
importance of tax saliency to taxpayer compliance or, in simpler terms,
the effect of being reminded of the tax balance due.\footnote{ Our
  experimental design identifies the effect of saliency alone as a
  trigger for tax compliance by estimating the difference in the rate
  of compliance for our holdout sample receiving no letter compared to
  the sample receiving the simple reminder letter only.  Our reminders
  were mailed six months after the initial notification of taxes due,
  and thus could estimate the loss of saliency for this period
  only. Staggered mailings of the simple reminder letter could
  identify the rate of decline in saliency, but this was not possible
  in our experiment because of time constraints imposed by DoR.}

\noindent \textit{Reminder plus Tax Lien}: Failure to pay your Real
Estate Taxes may result in a tax lien on your property in an amount
equal to your back taxes plus all penalties and interest.  When your
property is sold, those delinquent tax payments will be deducted from
the sale price.  By paying your taxes now, you can avoid these
penalties and interest.  Properties near you in your neighborhood that
have liens placed on them include: $<$ List Three Properties and Sale
Dates $>$ \textbf{Pay your taxes now to avoid a lien being placed on
  your property.  Our records indicate that you have a balance due of
  \textit{balance}.}

\noindent \textit{Reminder plus Lien and Sheriff's Sale}: Failure to
pay your Real Estate Taxes may result in the sale of your property by
the City in order to collect back taxes.  In the past year we have
sold \textit{N} properties in your neighborhood at a Sheriff's Sale.
Included in these \textit{N} properties are the following properties
near you: $<$List Three Properties and Sale Dates$>$ \textbf{Pay your
  taxes now to prevent the sale of your property.  Our records
  indicate that you have a balance due of \textit{balance}.}

The reminder letter coupled with the threat of a lien, or a lien plus
a sheriff's sale of the taxpayer's home, increase the recognized
interest and penalties with delay -- that is, an increase in
penalties.  A taxpayer lien for all interest and penalties will be
collected at the future date of home sale, which may be a very large
obligation if the home is sold significantly in the future.  A lien
coupled with a sheriff's sale may occur sooner and thus have lower
accumulated interest and penalties, but the forced sale of one's home
is likely to have very high psychic costs.  Which of the two added
penalties is larger, and therefore likely to have a stronger impact on
compliance, will depend upon the circumstances of the individual tardy
taxpayer. To make clear these sanctions are not empty threats, both
letters list three neighborhood properties where these added
enforcement measures have been implemented.  If penalties are
perceived as significant and likely to be enforced then both letters
should increase compliance over that observed for the simple
reminder.\footnote{The point of listing neighborhood properties was to
  make it clear from locally available evidence that the City
  sanctions will be enforced. A referee suggested, however, that
  taxpayers may interpret the three listed properties as the only
  sanctioned properties and thus that while there is enforcement it is
  only rarely used.  We cannot rule out this possibility. In addition,
  our letters did not communicate when the enforcement will
  happen. which may have created some uncertainty about enforcement
  strategies. See \citeA{Alm-Jackson-McKee-92} for an experimental
  study for how institutional uncertainty may affect taxpayers
  compliance.}

Our final four reminder letters test for the potential role of ``tax
morale'' motives for compliance.  An appeal to a tax morale is meant
to cue a possible benefit from having paid one's taxes.  In contrast
to user fees, property tax payments are not tied to the citizen's
receipt of particular services during our experimental period.  In
effect, each delinquent taxpayer is a potential free rider, and the
appeal to a tax morale for payment is meant to overcome such
self-interest.

We test for the importance of four such motives: 1) the value of
knowing one is a contributor to the immediate services of one's
neighborhood; 2) the value of knowing one is a contributor to the
wider services that benefit the city as a whole; 3) the value of
knowing one is part of a collective effort with other taxpayers or
``peers'' in paying for city services; and 4) the value of knowing one
has meet one's obligations as a citizen in a democracy.  Each of these
benefits may motivate taxpayer compliance, and our reminder letters
are meant to trigger a possible recognition of the importance to the
taxpayer of each motive.  Some tardy taxpayers may respond to one
motive, some to another, and perhaps others to none at all if the
free-rider motive is decisive.  The four tax morale reminder letters
are:

\noindent \textit{Reminder Plus Appeal to Neighborhood Services}: We
want to remind you that your taxes pay for essential public services
in \textit{neighborhood name}, such as $<$List Two Local Amenities
such as a Park or a Library$>$, your local police officer, snow
removal, street repairs, and trash collection.  \textbf{Please pay
  your taxes to help the city provide these services in your
  neighborhood.} \textbf{Our records indicate that you have a balance
  due of \textit{balance}.}

\noindent \textit{Reminder Plus Appeal to City-Wide Services}: Your
taxes pay for important services that make a city great. Your tax
dollars are essential for ensuring all Philadelphia's children receive
a quality education and all Philadelphians feel safe in their
neighborhoods.  \textbf{Please pay your taxes as soon as you can to
  help us pay for these important services.  Our records indicate that
  you have a balance due of \textit{balance}.}

\noindent \textit{Reminder Plus Appeal to Peer Behavior}: You have not
paid your Real Estate Taxes.  Almost all of your neighbors pay their
fair share: 9 out of 10 Philadelphians do so.  \textbf{By failing to
  pay, you are abusing the good will of your Philadelphia neighbors.
  Our records indicate that you have a balance due of
  \textit{balance}.}

\noindent \textit{Reminder Plus Appeal to Civic Duty}: For democracy
to work, all citizens need to pay their fair share of taxes for
community services.  \textbf{By failing to do so, you are not meeting
  your duty as a citizen of Philadelphia.  Our records indicate that
  you have a balance due of \textit{balance}.}

We take as evidence that a tax morale message increases the likelihood
of tax compliance when the tax morale reminder increases the rate of
compliance above that of the reminder-only letter.  If tax morale
letters have the same impact on compliance as a reminder-only letter
then not much is gained by appealing to tax morale.


\section{Randomization Procedure}

We begin our analysis with 19,039 first-time tardy taxpayers owning
only a single property in Philadelphia.  As a baseline control we then
randomly removed 2,088 taxpayers from receiving any reminder letter at
all.  These taxpayers became our holdout sample.  We, therefore, have
16,951 taxpayers in the treatment groups and 2,088 taxpayers in the
holdout or control group. We next randomized all remaining owners by
treatments based on the total amount of property taxes owed.

Given the high correlation between the propensity to pay taxes and
total debt owed, randomization blocks were defined according to
owner-level total debt to assure uniformity of samples along the
dimension of debt owed.  Each owner assigned to receive a reminder
letter was equally likely to receive each of the seven treatments.

\begin{sidewaystable}[htbp]
\centering
\caption{Balance on Observables}
\label{balance}
\vspace{10mm}
\begin{tabular}{lrrrrrrrrc}
  \hline
Variable & 1 & 2 & 3 & 4 & 5 & 6 & 7 & 8 & $p$-value \\
   \hline
Amount Due & \$1,233 & \$1,383 & \$1,389 & \$1,613 & \$1,950 & \$1,290 & \$1,338 & \$1,316 & 0.32 \\
   & (\$1,840) & (\$6,510) & (\$4,130) & (\$13,118) & (\$25,290) & (\$2,021) & (\$3,413) & (\$2,158) &  \\
  Prop. Value & \$142 & \$163 & \$147 & \$155 & \$206 & \$130 & \$130 & \$166 & 0.29 \\
   & (\$509) & (\$1,316) & (\$699) & (\$966) & (\$2,035) & (\$181) & (\$181) & (\$1,336) &  \\
  Years Tenure & 18.7 & 18.7 & 19.0 & 18.6 & 18.5 & 18.8 & 18.9 & 18.9 & 0.96 \\
   & (15.6) & (15.2) & (15.7) & (15.5) & (15.7) & (15.6) & (15.6) & (16.0) &  \\
  Center City & 5\% & 5\% & 5\% & 5\% & 5\% & 4\% & 5\% & 5\% & 0.66 \\
  Northeast Philly & 17\% & 18\% & 16\% & 15\% & 17\% & 16\% & 18\% & 16\% &  \\
  North Philly & 22\% & 21\% & 22\% & 22\% & 21\% & 20\% & 22\% & 22\% &  \\
  Northwest Philly & 26\% & 25\% & 27\% & 28\% & 26\% & 27\% & 25\% & 25\% &  \\
  South Philly & 10\% &  9\% & 10\% & 10\% & 10\% & 10\% & 10\% & 10\% &  \\
  West Philly & 21\% & 23\% & 21\% & 21\% & 22\% & 23\% & 20\% & 22\% &  \\
  \# Owners & 2,088 & 2,420 & 2,432 & 2,419 & 2,389 & 2,441 & 2,417 & 2,433 &  \\
  \hline
\multicolumn{10}{l}{\scriptsize{$p$-values in rows 1-2 are $F$-test
    $p$-values from regressing each variable on treatment dummies. A
    $\chi^2$ test was used for the geographic distribution. }} \\
\multicolumn{10}{l}{\scriptsize{ Standard deviations in parentheses. Property values are reported in \$1000.  }} \\
\multicolumn{10}{l}{\scriptsize{1: Holdout, 2: Reminder, 3: Lien, 4: Sheriff, 5: Neighborhood, 6: Community, 7: Peer, 8: Duty}} \\
\end{tabular}
\end{sidewaystable}

Table \ref{balance} checks whether the treatment and holdout groups
are balanced based on the two most important variables, taxes due and
assessed property value.  Table \ref{balance} shows that randomization
was successful in the single property owner sample.  The average debt
owed by each owner is \$1,287 in the seven treatment groups and \$1,233 in
the holdout sample. The average assessed property value is \$144,145
in the treatment group and \$142,630 in the holdout group. The average
tenure for property ownership was 15 years across all groups.  As a
further test of our randomization procedure, we also checked to see
whether randomization achieved spatial uniformity throughout the
geographic expanse of the city. As reported in Table \ref{balance}
geographic balance was achieved. Overall, we find no evidence that
would suggest any problems with randomization.


\section{Empirical Results}

Table \ref{sh_lin} presents our core results for the one-month and
three-months period of our experiment. Our experiment started on June
15, 2015.  The one-month and three-months results are compelling since they
are not ``contaminated" by the activities of the collection
agencies. Recall that the City typically employs these collection
agencies in May of each year to improve the property tax revenue
collection, but delayed their notification activities until the end of
August in 2015 with enforcement efforts beginning in mid-September,
2015.  The difference between the one-month and three-months
illustrates some important timing issues and highlight the benefits of
being patient.  One important lesson from this study is that nudges
may be a cheaper technology to collect revenues than outsourcing the
problem to collection agencies that charge six-percent contingency
fees.

We consider three distinct outcome measures for tax compliance
behavior. First, did the taxpayer make any contribution at all towards
their tax bill; this is the \textit{ever-paid} response. Second, did
the taxpayer make a full payment of their tax bill; this is the
\textit{paid-in-full} response. Third, what was the total amount paid
by the taxpayer; this is the \textit{total-paid}.  For ease of
interpretation, Table \ref{sh_lin} presents OLS estimates for the
linear probability model.\footnote{Logit estimates are identical
in significance and interpretation to the OLS results reported here.}

The top line of Table \ref{sh_lin} reports the mean rates of
compliance for our holdout sample measured by the fraction of tardy
taxpayers by \textit{ever-paid} or \textit{paid-in-full} and the mean
payments.  We report these rates one month from the starting date of
the experiment (July 15) and three months from the starting date of
the experiment (September 15). The rate of \textit{ever-paid}
compliance for taxpayers in the holdout sample rises from 30.5 percent
after one month to 51.4 percent after three months; the rate of
\textit{paid-in-full} compliance for the holdout sample raises from
23.5 percent after one month to 40.8 percent after three months.

\begin{table}[htbp]
\caption{Short-Term Linear Probability Model Estimates}\label{sh_lin}
\begin{center}
\begin{tabular}{l c c c c c c }
\hline
 & \multicolumn{2}{c}{Ever Paid} & \multicolumn{2}{c}{Paid in Full} & \multicolumn{2}{c}{Total Paid} \\
\hline
 & One & Three & One & Three & One & Three \\
 & Month & Months & Month & Months & Month & Months \\
\hline
Holdout      & $30.5$ & $51.4$ & $23.5$ & $40.8$ & \$$324.0$ & \$$636.6$ \\
\hline
Reminder     & $3.7^{***}$  & $3.9^{***}$  & $2.2^{*}$    & $3.0^{**}$   & $36.6$        & $15.2$        \\
             & $(1.4)$      & $(1.5)$      & $(1.3)$      & $(1.5)$      & $(31.6)$      & $(43.1)$      \\
Lien         & $9.0^{***}$  & $9.2^{***}$  & $5.7^{***}$  & $7.3^{***}$  & $117.0^{***}$ & $122.7^{**}$  \\
             & $(1.4)$      & $(1.5)$      & $(1.3)$      & $(1.5)$      & $(43.9)$      & $(54.9)$      \\
Sheriff      & $7.3^{***}$  & $8.8^{***}$  & $4.5^{***}$  & $6.7^{***}$  & $68.4^{**}$   & $96.8^{*}$    \\
             & $(1.4)$      & $(1.5)$      & $(1.3)$      & $(1.5)$      & $(34.1)$      & $(49.5)$      \\
Neighbor. & $1.7$        & $2.6^{*}$    & $-0.2$       & $1.6$        & $51.0$        & $40.1$        \\
             & $(1.4)$      & $(1.5)$      & $(1.3)$      & $(1.5)$      & $(37.6)$      & $(48.8)$      \\
Community    & $3.8^{***}$  & $2.8^{*}$    & $1.3$        & $2.5^{*}$    & $41.1$        & $18.3$        \\
             & $(1.4)$      & $(1.5)$      & $(1.3)$      & $(1.5)$      & $(32.6)$      & $(45.1)$      \\
Peer         & $3.9^{***}$  & $3.5^{**}$   & $1.8$        & $3.4^{**}$   & $59.0$        & $119.6$       \\
             & $(1.4)$      & $(1.5)$      & $(1.3)$      & $(1.5)$      & $(36.6)$      & $(76.1)$      \\
Duty         & $2.4^{*}$    & $3.6^{**}$   & $0.7$        & $2.3$        & $35.8$        & $70.7$        \\
             & $(1.4)$      & $(1.5)$      & $(1.3)$      & $(1.5)$      & $(35.6)$      & $(49.2)$      \\
\hline
Num. obs.    & 19039        & 19039        & 19039        & 19039        & 19039         & 19039         \\
\hline
\multicolumn{7}{l}{\scriptsize{$^{***}p<0.01$, $^{**}p<0.05$, $^*p<0.1$. Robust standard errors.}} \\
\multicolumn{7}{l}{\scriptsize{Holdout values in levels; remaining figures relative to this.}}
\end{tabular}
\end{center}
\end{table}

The next seven rows report the additional impact on rates of
compliance and mean payments from our seven treatment letters: Reminder-only,
Reminder/Lien, Reminder/Sheriff, Reminder/Neighborhood,
Reminder/Community, Reminder/Peer, and Reminder/Duty.  Receiving the
reminder-only letter increases the rate of compliance after one month
for an \textit{ever-paid} tax payment by 3.7 percent above the
holdout's rate of compliance and by 3.9 percent after three months.
Both effects are statistically significant at the 99 percent level of
confidence.  These estimates for the reminder-only letter indicate the
relative importance of salience and the benefit of simple notification
strategies to taxpayer compliance behavior.\footnote{For evidence from
  other settings that saliency matters and reminders have significant
  impacts in inducing appropriate behaviors, see
  \citeA{Thaler-Sunstein-03}, \citeA{Karlan-16}, and
  \citeA{Kessler-Zhang-15}. For evidence that simple reminders matter
  for the payment of local taxes, see \citeA{DelCarpio-13} and for the
  payment of local fines see \citeA{Heffetz-16}.}

The simple reminder letter is particularly effective early in our
experiment, where the pure effect of a reminder increases the rate of
compliance after one month by approximately 12 percent (= 3.7/30.5).
While receipt of the reminder letter is still effective after three
months, its overall impact on compliance behavior is relatively less,
adding an additional 8 percent (= 3.9/51.4) to the rate of
\textit{ever-paid}.  The same statistical significance and declining
rate of overall impact of reminder-only on compliance is observed for
the outcome, \textit{paid-in-full}.  Here the reminder-only letter
increases the one-month rate of compliance over the holdout sample by
2.2 percent on a mean rate of holdout compliance of 23.5 percent (9.4
percent relative improvement) and the three-months rate of compliance
over the holdout sample by 3.0 percent on a mean rate of 40.8 percent
(7.4 percent relative improvement). While most of the taxpayers paid
in full -- 3 percent compared to the 3.9 percent of all taxpayers
after three months -- the additional revenues raised by the reminder
letters over that paid by those with no letter is never significant
and is quantitatively very small, on average only \$15.20 more than
the amount paid by the holdout sample after three months. The reminder
only letter had its greatest impact on taxpayers who owed the least.

Similar to results from other tax compliance studies, the impact of
adding a more substantive message to the reminder letter had mixed
results on taxpayer compliance, depending on the content of the
message.  Of the six supplemental messages, only the reminder/lien and
the reminder/sheriff letters had a statistically significant added
impact on compliance above the simple reminder.  After one month, the
sample receiving the reminder/lien letter had an additional 9.0
percent rate of \textit{ever-paid} compliance over the hold-out
sample's rate of compliance of 30.5 percent (30 percent relative
improvement) and after three months an additional 9.2 percent rate of
\textit{ever-paid} compliance over the hold-out sample's compliance
rate of 51.4 percent (18 percent relative improvement).  The impacts
are statistically significant at the 99 percent level of confidence.
The results for \textit{paid-in-full} compliance for the reminder/lien
letter are also statistically significant and add 5.7 to the rate of
compliance over the holdout sample after one month and 7.3 percent to
the rate of compliance after three months.  Importantly, the impact of
the reminder/lien letter on total taxes paid over that for the holdout
sample is statistically significant and shows increased payments of
from \$117 per letter (36 percent relative improvement) to \$122.7 per
letter (19 percent relative improvement) after one month and three months,
respectively. Though with slightly smaller impacts on compliance and
taxes paid, the reminder/sheriff letter also yields significantly
higher compliance rates and tax payments above the holdout
sample. Letters stressing sanctions have their greatest impact on
tardy taxpayers who owe more taxes.\footnote{ Left unanswered by these
  results is the question of why taxpayers respond to extrinsic
  messages that communicate pre-existing penalty information.  One
  possible explanation is that taxpayers interpret the threat of
  enforcement as new information rather than a reiteration of existing
  information. \citeA{bergolo2017tax} report evidence consistent with
  the idea that this new threat information is used to update the
  recipients perceived risk of enforcement and punishment.}  The
four tax morale letters stressing the payment's benefits for the
neighborhood (\textit{neighbor}) and city (\textit{community})
services, compliance behavior by other Philadelphians (\textit{peer}),
or civic duty (\textit{duty}) to pay ones taxes have the same effect
as a reminder-only on compliance behaviors.  There is no statistically
significant added compliance to the tax morale reminders, on average,
above that obtained from just the simple reminder letter.  Overall,
results are similar in statistical significance and impact to those in
Castro and Scartascini's (2015) study of property tax payments in
Junan, Argentina, the other major field experiment seeking to improve
property tax collection.

We need to mention here that our results in Table \ref{sh_lin} showing
the significance of reminding taxpayers of sanctions but no
significant impacts for tax morale messages stands in contrast to
results we obtained from our 2014 pilot study for this experiment,
reported in Chirico, et. al. (2016).  In the Fall, 2014, were were
asked by the City's DoR to design an experiment to test alternative
reminder letters for tardy taxpayers.  We did so, and ran a pilot
study in November, 2014 to test both the administrative feasibility of
a larger experiment and our experimental design.  The pilot sent
reminder letters to the 4,749 tardy taxpayers who had not yet paid
their 2014 Philadelphia property taxes by November 1, 2014.  It is
important to note that these taxpayers had already received a series
of simple reminder letters from the City beginning in the Spring, 2014
and had been contacted over the summer and fall by the law firms hired
by DoR to collect tardy payments.  By November, they still had not
paid.  We used three letters in our pilot, one threatening
enforcement without mentioning specifics, one stressing taxes pay for
city services, and a third stressing that it is one�s civic duty to
pay one's taxes and that 9 out 10 city taxpayers do so on time.

We focused our analysis on the 3,888 of these ``very" tardy taxpayers
owning only a single property. As a group they owed \$14.73 million
dollars, an average of \$3,790 per taxpayer. The results from our
analysis of this pilot sample of very tardy taxpayers showed no
statistically significant effect for a reminder threatening
enforcement or for that appealing to civic duty, but a strong and
significant impact on the rate of payment from those receiving the
reminder appealing to taxation to pay for quality schools and public
safety.  The impact of the public service reminder on revenues
collected was very modest, however, collecting on average only \$238
dollars towards the average amount owed of \$4,012 for those receiving
the public service reminder.  Of the 16 percent who paid, only half
paid their tax bill in full.  In the end, we concluded that most of
the very tardy taxpayers will continue to owe significant taxes by the
end of the tax year and beginning in January, 2015 will be subject to
tax liens on their homes.  In contrast to the positive impact of
sanction reminders on compliance by the tardy taxpayers in this study,
reminders of imminent sanctions had no effect on the very tardy
taxpayers in our pilot.  These contrasting behaviors are another
reminder as to the importance of heterogeneity in taxpayers' responses
to nudges for compliance.  Meeting directly with, and offering tax
payment plans, may be required for these very tardy taxpayers.


\begin{table}[htbp]
\caption{Long-Term Linear Model Estimates}\label{lt_lin}
\begin{center}
\begin{tabular}{l c c c c c c }
\hline
 & \multicolumn{3}{c}{Six Months} & \multicolumn{3}{c}{Subsequent Tax Cycle} \\
 & Ever Paid & Paid in Full & Total Paid & Ever Paid & Paid in Full & Total Paid \\
Holdout      & $73.3$ & $63.2$ & $937.9$ & $65.5$ & $52.5$ & $1043.9$ \\
\hline
Reminder     & $1.3$        & $1.5$        & $21.2$        & $-1.4$       & $-0.7$       & $-24.7$        \\
             & $(1.3)$      & $(1.4)$      & $(50.0)$      & $(1.4)$      & $(1.5)$      & $(69.1)$       \\
Lien         & $3.7^{***}$  & $4.8^{***}$  & $87.5$        & $-0.9$       & $-0.7$       & $38.9$         \\
             & $(1.3)$      & $(1.4)$      & $(58.8)$      & $(1.4)$      & $(1.5)$      & $(96.9)$       \\
Sheriff      & $3.7^{***}$  & $2.9^{**}$   & $74.5$        & $-0.6$       & $-1.1$       & $245.8$        \\
             & $(1.3)$      & $(1.4)$      & $(55.9)$      & $(1.4)$      & $(1.5)$      & $(260.6)$      \\
Neighborhood & $-0.2$       & $-0.0$       & $47.6$        & $-3.1^{**}$  & $-2.1$       & $181.3$        \\
             & $(1.3)$      & $(1.4)$      & $(55.3)$      & $(1.4)$      & $(1.5)$      & $(189.6)$      \\
Community    & $0.9$        & $1.1$        & $55.0$        & $-1.8$       & $-2.0$       & $-52.9$        \\
             & $(1.3)$      & $(1.4)$      & $(53.6)$      & $(1.4)$      & $(1.5)$      & $(66.8)$       \\
Peer         & $1.4$        & $2.3$        & $130.0$       & $-1.9$       & $-1.1$       & $-69.0$        \\
             & $(1.3)$      & $(1.4)$      & $(79.5)$      & $(1.4)$      & $(1.5)$      & $(65.9)$       \\
Duty         & $2.1$        & $1.0$        & $120.3^{**}$  & $-1.6$       & $-1.9$       & $37.1$         \\
             & $(1.3)$      & $(1.4)$      & $(57.6)$      & $(1.4)$      & $(1.5)$      & $(70.2)$       \\
\hline
Num. obs.    & 19039        & 19039        & 19039         & 19036        & 19036        & 19036          \\
\hline
\multicolumn{7}{l}{\scriptsize{$^{***}p<0.01$, $^{**}p<0.05$, $^*p<0.1$. Robust standard errors. Holdout values in levels; remaining figures relative to this.}} \\
\multicolumn{7}{l}{\scriptsize{Change in sample size between long-term and subsequent year results reflects property dissolution for three properties.}}
\end{tabular}
\end{center}
\end{table}


Table \ref{lt_lin} estimates the longer run impacts of our seven nudge
interventions on compliance.  The first three columns of Table
\ref{lt_lin} show the estimated effects of having received a letter on
compliance six months after the start of the experiment, again
compared to compliance behavior in our holdout sample.  Six-months
responses for those in the holdout sample and in our seven treatment
groups now include the possible influence of the outside collection
agencies on still delinquent taxpayers. We do not know their
``treatment'' strategies, although we conjecture that they primarily
rely on enforcement threats. The effects observed for the six-months
window, therefore, predict the impact of our treatments from our June
letters interacted with the unknown treatments by the outside
agencies.

Since all tardy taxpayers including our holdout sample now receive
some form of an enforcement threat reminder, it is not surprising that
our original reminder letter no longer has a differential impact on
payment behavior. What does continue to impact behavior, however, is
our original reminders that stressed the risk of liens and sheriff's
sales. The effects of our lien and sheriff reminders are now slightly
smaller in percentage terms, though not significantly so. Again, none
of the tax morale intrinsic nudges show a statistically significant
impact on compliance behaviors.  These taxpayers are now receiving
extrinsic reminders for the first time, just like those in the holdout
sample. They appear to respond identically, resulting in no
significant behavioral differences between those in the original
holdout sample and in the tax morale intrinsic motivation
samples.\footnote{It is our understanding from DOR that their
  treatments are a combination of simple reminders and reminders
  coupled with extrinsic messages stressing penalties, liens, and
  perhaps sheriff sales.}  This provides further evidence that
extrinsic or penalties messages are effective messages for converting
non-payers to payers.

In summary, Table \ref{lt_lin} shows that the lien and sheriff letters
still work at the extensive margin after six months, but not
necessarily at the intensive margin on amount paid.  Table
\ref{lt_lin} suggests that the duty letter may work at the intensive
margin after six months. It is the only letter that generated
significantly positive revenues after six months. This follows from
the fact that this letter was particularly effective among the top
quartile of taxpayers as shown in Table
\ref{lpm_hetero}.\footnote{Note that this effect is not significant at
  the 99 percent level of confidence which is a conservative level
  commonly used in hypothesis testing.}

The last three columns of Table \ref{lt_lin} carry our sample into the
next tax year, beginning with the receipt of a new property tax bill
in early January, 2016. We now study if having received a reminder
letter in June, 2015 improves compliance behavior for the payment of
the 2016 taxes by June of 2016.  Consistent with the importance of
saliency, none of the 2015 reminder letters appear to have ``staying
power'' into the next tax year.  Tardy Philadelphians need constant
reminders.\footnote{The new property tax bill, which is sent out at
  the beginning of the new year, may also serve as a reminder for
  tardy taxpayers to pay the bill for the previous year.}



\begin{center}
\begin{longtable}{l c c c c c c }
\caption{Treatment Effect Heterogeneity by Debt Quantile}
\label{lpm_hetero}\\
\hline
 & One & Three & Six & One & Three & Six \\
 & Month & Months & Months & Month & Months & Months \\
\hline
 & \multicolumn{3}{c}{Ever Paid} & \multicolumn{3}{c}{Total Paid} \\
\hline
\endfirsthead
\multicolumn{7}{c}{\tablename\ \thetable\ -- \textit{Continued from previous page}} \\
\hline
 & One & Three & Six & One & Three & Six \\
 & Month & Months & Months & Month & Months & Months \\
\hline
 & \multicolumn{3}{c}{Ever Paid} & \multicolumn{3}{c}{Total Paid} \\
\hline
\endhead
\hline \multicolumn{7}{r}{\textit{Continued on next page}} \\
\endfoot
\hline
\endlastfoot
\hline
Holdout in Quartile 1      & $38.1$  & $56.4$  & $74.7$ & $118.0$ & $152.0$  & $184.9$  \\
\hline
Holdout in Quartile 2      & $-9.8^{***}$  & $-11.8^{***}$ & $-7.4^{***}$ & $20.1$        & $97.5^{***}$   & $217.4^{***}$  \\
                           & $(2.9)$       & $(3.1)$       & $(2.8)$      & $(32.3)$      & $(33.6)$       & $(33.5)$       \\
Holdout in Quartile 3      & $-9.9^{***}$  & $-5.2^{*}$    & $-0.1$       & $134.1^{***}$ & $388.8^{***}$  & $658.5^{***}$  \\
                           & $(2.9)$       & $(3.1)$       & $(2.7)$      & $(38.6)$      & $(39.5)$       & $(39.3)$       \\
Holdout in Quartile 4      & $-10.7^{***}$ & $-2.5$        & $2.2$        & $691.1^{***}$ & $1494.0^{***}$ & $2193.8^{***}$ \\
                           & $(2.9)$       & $(3.1)$       & $(2.7)$      & $(92.1)$      & $(126.0)$      & $(129.1)$      \\
Reminder in Quartile 1     & $5.4^{*}$     & $2.5$         & $0.2$        & $-6.3$        & $-12.8$        & $-12.7$        \\
                           & $(2.9)$       & $(3.0)$       & $(2.6)$      & $(33.8)$      & $(34.0)$       & $(34.0)$       \\
Reminder in Quartile 2     & $1.7$         & $5.3^{*}$     & $4.8^{*}$    & $10.3$        & $26.8$         & $31.6$         \\
                           & $(2.7)$       & $(3.0)$       & $(2.7)$      & $(15.9)$      & $(19.4)$       & $(22.0)$       \\
Reminder in Quartile 3     & $0.6$         & $1.7$         & $-1.1$       & $21.0$        & $10.5$         & $-33.4$        \\
                           & $(2.7)$       & $(3.0)$       & $(2.6)$      & $(30.9)$      & $(34.8)$       & $(33.9)$       \\
Reminder in Quartile 4     & $7.2^{***}$   & $5.7^{*}$     & $0.8$        & $102.2$       & $-2.3$         & $46.8$         \\
                           & $(2.8)$       & $(3.0)$       & $(2.5)$      & $(113.8)$     & $(152.4)$      & $(167.0)$      \\
Lien in Quartile 1         & $13.5^{***}$  & $9.9^{***}$   & $3.4$        & $14.6$        & $13.4$         & $5.3$          \\
                           & $(2.9)$       & $(2.9)$       & $(2.5)$      & $(38.4)$      & $(38.9)$       & $(38.8)$       \\
Lien in Quartile 2         & $8.9^{***}$   & $13.0^{***}$  & $8.0^{***}$  & $52.7^{***}$  & $68.8^{***}$   & $52.5^{***}$   \\
                           & $(2.8)$       & $(2.9)$       & $(2.7)$      & $(16.4)$      & $(19.4)$       & $(19.0)$       \\
Lien in Quartile 3         & $6.4^{**}$    & $7.2^{**}$    & $-0.3$       & $79.9^{**}$   & $67.3^{*}$     & $-5.2$         \\
                           & $(2.8)$       & $(3.0)$       & $(2.6)$      & $(33.1)$      & $(34.8)$       & $(34.4)$       \\
Lien in Quartile 4         & $7.0^{**}$    & $6.2^{**}$    & $3.5$        & $293.6^{*}$   & $289.4$        & $226.9$        \\
                           & $(2.8)$       & $(3.0)$       & $(2.5)$      & $(163.5)$     & $(199.9)$      & $(204.4)$      \\
Sheriff in Quartile 1      & $10.7^{***}$  & $10.7^{***}$  & $4.9^{*}$    & $3.7$         & $1.2$          & $-2.9$         \\
                           & $(3.0)$       & $(2.9)$       & $(2.5)$      & $(34.4)$      & $(34.6)$       & $(34.7)$       \\
Sheriff in Quartile 2      & $7.4^{***}$   & $10.0^{***}$  & $5.4^{**}$   & $39.2^{**}$   & $50.2^{**}$    & $28.5$         \\
                           & $(2.8)$       & $(3.0)$       & $(2.7)$      & $(16.2)$      & $(19.5)$       & $(19.2)$       \\
Sheriff in Quartile 3      & $5.8^{**}$    & $7.7^{***}$   & $3.0$        & $89.0^{***}$  & $65.6^{*}$     & $13.8$         \\
                           & $(2.8)$       & $(3.0)$       & $(2.5)$      & $(32.4)$      & $(35.1)$       & $(33.8)$       \\
Sheriff in Quartile 4      & $5.1^{*}$     & $6.2^{**}$    & $1.1$        & $114.6$       & $215.6$        & $184.3$        \\
                           & $(2.8)$       & $(3.0)$       & $(2.5)$      & $(123.6)$     & $(177.4)$      & $(191.7)$      \\
Neighborhood in Quartile 1 & $3.8$         & $3.0$         & $-2.3$       & $-26.9$       & $-31.3$        & $-32.0$        \\
                           & $(3.0)$       & $(3.0)$       & $(2.7)$      & $(32.2)$      & $(32.5)$       & $(33.2)$       \\
Neighborhood in Quartile 2 & $1.4$         & $5.9^{**}$    & $2.2$        & $12.1$        & $34.5^{*}$     & $21.2$         \\
                           & $(2.7)$       & $(3.0)$       & $(2.8)$      & $(15.8)$      & $(19.5)$       & $(19.4)$       \\
Neighborhood in Quartile 3 & $-0.2$        & $0.8$         & $-1.4$       & $21.1$        & $-4.3$         & $-28.9$        \\
                           & $(2.7)$       & $(3.0)$       & $(2.6)$      & $(31.8)$      & $(34.9)$       & $(34.0)$       \\
Neighborhood in Quartile 4 & $1.6$         & $0.5$         & $0.4$        & $174.8$       & $116.3$        & $168.2$        \\
                           & $(2.7)$       & $(3.0)$       & $(2.5)$      & $(139.2)$     & $(176.3)$      & $(190.1)$      \\
Community in Quartile 1    & $5.9^{**}$    & $5.9^{**}$    & $2.7$        & $-20.2$       & $-21.2$        & $-23.4$        \\
                           & $(2.9)$       & $(2.9)$       & $(2.6)$      & $(33.1)$      & $(33.4)$       & $(33.3)$       \\
Community in Quartile 2    & $3.9$         & $3.8$         & $1.2$        & $22.3$        & $25.9$         & $17.7$         \\
                           & $(2.7)$       & $(3.0)$       & $(2.8)$      & $(16.0)$      & $(19.7)$       & $(20.4)$       \\
Community in Quartile 3    & $1.3$         & $0.8$         & $-1.5$       & $33.7$        & $2.1$          & $-28.8$        \\
                           & $(2.7)$       & $(3.0)$       & $(2.6)$      & $(31.2)$      & $(34.8)$       & $(34.1)$       \\
Community in Quartile 4    & $3.9$         & $0.5$         & $0.7$        & $105.5$       & $21.5$         & $193.0$        \\
                           & $(2.7)$       & $(3.0)$       & $(2.5)$      & $(117.9)$     & $(160.7)$      & $(181.9)$      \\
Peer in Quartile 1         & $6.2^{**}$    & $3.9$         & $-0.2$       & $-3.1$        & $-6.5$         & $2.2$          \\
                           & $(2.9)$       & $(3.0)$       & $(2.6)$      & $(33.9)$      & $(34.2)$       & $(34.9)$       \\
Peer in Quartile 2         & $2.9$         & $4.2$         & $2.4$        & $79.2$        & $80.7$         & $78.9$         \\
                           & $(2.7)$       & $(3.0)$       & $(2.8)$      & $(64.6)$      & $(66.2)$       & $(66.0)$       \\
Peer in Quartile 3         & $3.1$         & $3.6$         & $1.3$        & $57.4^{*}$    & $41.1$         & $6.6$          \\
                           & $(2.7)$       & $(3.0)$       & $(2.6)$      & $(31.7)$      & $(35.1)$       & $(33.9)$       \\
Peer in Quartile 4         & $3.2$         & $2.0$         & $1.6$        & $79.1$        & $315.5$        & $367.2$        \\
                           & $(2.7)$       & $(3.0)$       & $(2.5)$      & $(120.1)$     & $(283.8)$      & $(288.2)$      \\
Duty in Quartile 1         & $6.6^{**}$    & $6.8^{**}$    & $1.7$        & $-3.5$        & $8.2$          & $-1.8$         \\
                           & $(2.9)$       & $(2.9)$       & $(2.6)$      & $(41.0)$      & $(38.5)$       & $(38.3)$       \\
Duty in Quartile 2         & $-0.5$        & $3.1$         & $0.6$        & $1.0$         & $12.6$         & $-1.4$         \\
                           & $(2.7)$       & $(3.0)$       & $(2.8)$      & $(15.7)$      & $(19.4)$       & $(19.1)$       \\
Duty in Quartile 3         & $1.6$         & $1.5$         & $2.5$        & $42.5$        & $23.7$         & $34.6$         \\
                           & $(2.7)$       & $(3.0)$       & $(2.5)$      & $(31.5)$      & $(35.0)$       & $(34.4)$       \\
Duty in Quartile 4         & $1.9$         & $2.6$         & $3.2$        & $70.3$        & $171.5$        & $354.9^{*}$    \\
                           & $(2.7)$       & $(3.0)$       & $(2.5)$      & $(128.1)$     & $(175.4)$      & $(196.1)$      \\
\hline
Num. obs.                  & 19039         & 19039         & 19039        & 19039         & 19039          & 19039          \\
\hline
\multicolumn{7}{l}{\scriptsize{\parbox{.75\linewidth}{$^{***}p<0.01$, $^{**}p<0.05$, $^*p<0.1$. Holdout values for first quartile in levels; other holdout figures are relative to this and remaining figures are treatment effects for the stated treatment vs. holdout owners in the same quartile.}}}\\
\end{longtable}
\end{center}


Table \ref{lpm_hetero} shows the compliance behavior of tardy
taxpayers by the size of their tax bills.  Tardy taxpayers are divided
into four quartiles by taxes owed: Quartile 1 (mean owed = \$149);
Quartile 2 (mean owed = \$597), Quartile 3 (mean owed = \$1,133), and
Quartile 4 (mean owed = \$3,885).  All comparisons are for the
outcomes \textit{ever-paid} and total taxes paid relative to those
in the holdout sample in Quartile 1.  Three conclusions follow.
First, tardy taxpayers in Quartile 1 owing the least in taxes are the
most likely to make a tax payment, whether they receive a reminder
letter or not; note the significant negative effect of being in
Quartiles 2-4 of the holdout sample.  Second, receiving a
reminder/lien and reminder/sheriff letter improves \textit{ever-paid}
compliance for all four Quartiles, but the effects are greatest for
those in the lowest two Quartiles.  Third, while there is no
significant effect of the reminder/duty letter on taxpayers in
quartile 4 on their rate of compliance, there is a strong effect on
their amount paid.  If resources are limited and the objective is to
maximize additional revenue collected, then from the six-month results
for total taxes paid, the City should send the reminder/duty letter to
taxpayers in quartile 4. This finding is the strongest support for tax
morale letters in our experiment.\footnote{From Table 5, the expected
  average revenue after six months for each quartile will be the sum
  of payments by the holdout sample in that quartile plus the impact
  of each letter on payment for that quartile.  For example, payments
  after six months by taxpayers in quartile 1 receiving the lien
  letter will be \$184.90 + \$5.30 = \$190.20.  For all quartiles,
  returns after six months for the lien (sheriff) letter will be
  \$190.20 (\$182) for tardy taxpayers in quartile 1, \$269.90
  (\$245.90) for tardy taxpayers in quartile 2, \$652.80 (\$672.30)
  for tardy taxpayers in quartile 3, and \$2,419.90 (\$2377.3) for
  those in quartile 4.}

Finally, our results shed light on the importance of liquidity
constraints as a motivation for tardy tax payments. If liquidity
constraints are important, then nudges may be insufficient unless
accompanied by a way to smooth payments of the original tax
obligation. Taxpayer agreements that spread payments over several
months without penalty is one policy. Each reminder letter included a
sentence stressing the availability of taxpayer agreements to help
with payments. The results in Tables \ref{sh_lin} - \ref{lpm_hetero}
suggest, however, that liquidity constraints are not binding for most
of our tardy taxpayers (on average, 70 percent). First, from Tables
\ref{sh_lin} and \ref{lt_lin}, most taxpayers, who respond positively
to a tax nudge after one month and three months, will pay in full.
For all statistically significant nudges,
including simple reminder, the percent increase in the ``ever paid''
taxpayers that ``paid in full'' is never less than 60 percent (one
month, simple reminder), typically 75 to 80 percent, and as high as 97
percent (three months, peer reminder).

Note that Table \ref{lt_lin} suggests that compliance rates exceeded 50 percent in 2016 and were
approximately as high as the rate at the three-months point of
2015 (Table \ref{sh_lin}). This finding suggests that a large fraction of taxpayers that
were tardy in 2015 are compliers in 2016. However, 34.5 percent of
tarty taxpayers are also in non-compliance in 2016 indicating
that liquidity or cash constraints may be a serious problem for, at
most, a third of our population of tardy taxpayers.

\begin{table}[htbp]
\caption{Liquidity Constraints and Payment Agreements}\label{payment}
\begin{center}
\begin{tabular}{l c }
\hline
Holdout      & $0.9$ \\
\hline
Reminder     & $0.4$       \\
             & $(0.4)$     \\
Lien         & $1.0^{***}$ \\
             & $(0.4)$     \\
Sheriff      & $1.6^{***}$ \\
             & $(0.4)$     \\
Neighborhood & $0.6$       \\
             & $(0.4)$     \\
Community    & $0.4$       \\
             & $(0.4)$     \\
Peer         & $0.8^{**}$  \\
             & $(0.4)$     \\
Duty         & $0.8^{**}$  \\
             & $(0.4)$     \\
\hline
Num. obs.    & 19039       \\
\hline
\multicolumn{2}{l}{$^{***}p<0.01$, $^{**}p<0.05$, $^*p<0.1$. } \\
\multicolumn{2}{l}{Holdout values in levels; remaining figures relative to this}
\end{tabular}
\end{center}
\end{table}

While a liquidity constraint is not binding for the 70 percent of the
City's tardy taxpayers who make payments, it may still be a constraint
for the other 30 percent who cannot.  Offering these tardy taxpayers a
tax payment agreement to smooth their payments may be valuable.  The
City does offer such a program.  Table \ref{payment} shows what
fraction of tardy taxpayers have agreed to enroll in the program by
end of our experiment.  The share of the holdout sample that uses the
tax payment option is less than 1 percent.  Use by the recipients of
the reminder letters is never greater than an additional 1.6 percent
(reminder/sheriff).  On average, tax payer agreements are only being
used by about 1 percent of the original 19,039 tardy taxpayers or 200
taxpayers and only about 3 to 4 percent of all tardy taxpayers who
have not yet made a full tax payment.  Liquidity constraints may bind,
but the main barrier to payment is an inclination to do so.

\section{Tax Revenue Implications}

While of interest as a specification and test of a behavioral theory
of tax compliance, our results are also directly relevant for city tax
collection policies.  As a strategy for improving collection from
tardy taxpayers, our analysis informs two important policy issues.
First, cities need revenues: Do reminders improve collection, and then
do reminders with a message raise more money than a simple reminder?
Second, in light of recent municipal fiscal crises and the potential
for an unraveling of citizen commitment to local governance: Do
reminders with a message, and then which message, improve tax
collection as a ``nudge'' to citizen engagement? Table \ref{rev}
provides answers to these two questions.

\begin{table}[htbp]
\centering
\caption{Revenue Implications}\label{rev}
\begin{tabular}{lcccccc}
  \hline
\multicolumn{7}{c}{Based on 3 Months Estimates} \\
\hline
Treatment & Sample & Total Taxes & New & Revenue/ & New & New \% of Taxes \\
 & Size & Owed & Payers & Letters & Revenues & Paid \\
  \hline
Reminder & 2,420 & \$3.346 M & 94 & \$28.63 & \$69,285 & 2.1 \\
  Lien & 2,432 & \$3.378 M & 223 & \$67.44 & \$164,010 & 4.9 \\
  Sheriff & 2,419 & \$3.902 M & 212 & \$64.36 & \$155,691 & 4.0 \\
  Neighborhood & 2,389 & \$4.658 M & 63 & \$19.44 & \$46,452 & 1.0 \\
  Community & 2,441 & \$3.148 M & 69 & \$20.91 & \$51,049 & 1.6 \\
  Peer & 2,417 & \$3.233 M & 85 & \$25.79 & \$62,332 & 1.9 \\
  Duty & 2,433 & \$3.201 M & 88 & \$26.46 & \$64,369 & 2.0 \\
   \hline
Totals & 16,951 & \$24.866 M & 834 & - & \$613,188 & 2.5 \\
   \hline
\multicolumn{7}{c}{Based on 6 Months Estimates} \\
\hline
Treatment & Sample & Total Taxes & New & Revenue/ & New & New \% of Taxes \\
 & Size & Owed & Payers & Letters & Revenues & Paid \\
\hline
Reminder & 2,420 & \$3.346 M & 31 & \$9.29 & \$22,474 & 0.7 \\
  Lien & 2,432 & \$3.378 M & 91 & \$27.44 & \$66,739 & 2.0 \\
  Sheriff & 2,419 & \$3.902 M & 89 & \$27.14 & \$65,659 & 1.7 \\
  Neighborhood & 2,389 & \$4.658 M & -6 & -\$1.76 & -\$4,194 & -0.1 \\
  Community & 2,441 & \$3.148 M & 21 & \$6.38 & \$15,564 & 0.5 \\
  Peer & 2,417 & \$3.233 M & 33 & \$9.97 & \$24,091 & 0.7 \\
  Duty & 2,433 & \$3.201 M & 51 & \$15.42 & \$37,524 & 1.2 \\
  Totals & 16,951 & \$24.866 M & 310 & - & \$227,856 & 0.9 \\
\multicolumn{7}{p{1\textwidth}}{\scriptsize* Sample Size is the number of single property tax payers in the treatment group.  Total Taxes Owed is the total taxes owed by single property tax payers in the treatment group. New Payers equals the new payers after three months computed as the estimated increase in rate of compliance of those receiving the letter over those in the holdout sample as reported in Table 2; for example, for the reminder letter the number of new payers equals 94 = 0.039 x 2,420.  Revenue per letter for each treatment equals the median new revenue collected from those who received a treatment letter and made some payment (=\$735/letter) times the three month increase in compliance from each treatment letter; for example for the reminder letter the median estimated revenue per letter equals \$28.63 = 0.039 x \$735.  New revenues for each treatment equals the revenue/letter times the number of single owner properties receiving a treatment letter: for example, for the reminder letter the estimated total new revenues equals \$69,285 = \$28.63 x 2,420.  New \% of Taxes Paid equals New Revenues divided by Total Taxes Owed; for example, for the reminder letter 2.1 = \$69,285 / \$3,345,846.}
  \end{tabular}
\end{table}

Listed in Table \ref{rev} are our seven treatments, the sample size to
which each treatment applied and total taxes owed, and then estimates
of the impact of each treatment on the number new payers three months
after receipt of the treatment letter, the average additional revenue
received per letter sent, total additional revenues collected from each
treatment letter above that paid by the holdout sample, and finally,
the percent of owed taxes paid because of each treatment.  The upper
panel of Table \ref{rev} is based on the three-months estimates, while
the lower panel of Table \ref{rev} is based on the six-months results.

Note that we base our results in Table \ref{rev} on the median
estimated additional revenues per letter. These estimates of median
revenues are significantly lower than the estimates reported in Table
\ref{sh_lin}, which are estimates of mean additional revenues per
letter.  In that sense, we view our analysis  as providing
conservative estimates of the revenue implications of the different
reminder letters.

Since the three-months estimates are not contaminated by the activities
of the collection agencies we start the analysis by focusing on those
results.  For single property owners, the total number of additional
taxpayers above the holdout sample from all reminder letters is 834,
an average increase in the overall rate of compliance from receiving
one our treatment letters of 4.9 percent (834/16,951).  Table
\ref{rev} also provides an estimate of additional revenues raised by
each of our treatment letters and then the total revenue raised from
each treatment group.  From the perspective of the City's Department
of Revenue, our experiment was a good investment of resources.\footnote{
Since the costs of the experiment were financed by a grant from the
Wharton Public Policy Initiative, the City actually had zero additional
expenditures during our experiment.} Each letter cost about \$1 to process and send and raised
on average over all letters \$36.27.  The estimated benefit to cost
ratios for the seven treatments ranged from a low of \$19.19 (the
Neighborhood letter) to a high of \$67.90 (the Lien letter).  The
approximately \$17,000 spent on our experiment to mail the 16,951
treatment letters raised \$615,222 in additional city revenues: an
average benefit to cost ratio of 36.2.

Among our seven treatments, our experimental results clearly show the
power of the lien and sheriff letters compared to a simple reminder or
the tax morale nudges.  The number of taxpayers above the holdout
sample is three to four times larger and the revenue/letter is two to
three times larger with the letters stressing penalties.  As a
consequence, total additional revenues above the holdout sample from the
penalty letters and new revenues as a share of all taxes owed are
three to four times larger as well.  If we had sent only the lien or
sheriff's letter to the 16,951 taxpayers in our treatment groups we
would have raised \$1.15 million in additional revenues rather than \$615,222
-- nearly twice as much.  The paid share of taxes owed would have
risen from our experiment's average of .025 to that of the lien letter
only of .046.

The lower panel in Table \ref{rev} repeats this exercise using the
six-months estimates.  Using these results, we find qualitatively
similar effects. However, the quantitative magnitude of the relevant
effects is approximately one-third of the effects based on the three-months
estimates.  The impact of the reminders at the six-months date represent
aggregate new taxpayers and new revenues over the tax year made
possible by the experiment.  The difference between new payers and the
new revenues at the three-months and six-months dates measures the
number of taxpayers who have paid early, rather than have waited until
December to have paid.\footnote{New taxpayers and revenues at each
  date are cumulative.  Compared to the holdout sample, 94 new
  taxpayers were induced to pay their taxes after three months. After six
  months and again compared to the holdout sample, 31 new taxpayers
  were induced to pay their taxes.  This means that 63 taxpayers had
  moved from paying over the window from months four to six to paying
  by month three.  These are the taxpayers induced by the reminder to
  pay earlier.  If no taxpayers had been induced to pay earlier by the
  reminder, then the number of new taxpayers at the three-months date
  above the holdout sample would be zero while new taxpayers above
  holdout at the 6 month date would be 94. The 94 taxpayers at the end
  of our experiment would represent new taxpayers and new revenues for
  the full tax year and a net reduction of 94 taxpayers moving to tax
  delinquency because of the simple reminder.  In fact there were only
  31 new taxpayers above the holdout sample after 6 months and thus a
  net reduction in tax delinquency of 31 taxpayers because of the
  reminder.}  Additional taxpayers and revenues above the holdout sample
at the six-months date represent net gain -- not just early payments --
for the tax year ending December 31, 2015.  The revenue/letter ratio
is thus the net benefit/cost ratio measured as new, excluding early
revenues, and equals \$13.44 per letter averaged over all our letters;
see Table \ref{rev}.  Interest earnings from the early payments might
add an additional \$.50 per letter.\footnote{Early payments do provide
  benefit in the form of interest earnings over the months of early
  payment.  Assuming an 8 percent annual rate of return on city
  investments, this will add roughly .2 percent additional revenues
  from the early payments from 6 month to 3 months.  For the early
  revenues from the reminder only letter, this works out to about
  \$.40 per letter: \$.39 = .02 x (\$69,647 - \$22,474)/2,420.  When
  averaged over all reminder letters, interest earnings from our
  experiment will be \$.46 per letter: .46 = .02 x (\$615,222 -
  \$227,856)/16,951.} The financial benefits from nudges in
Philadelphia come from getting future delinquent taxpayers to pay
their taxes.

Though not possible to quantify from our results, tax nudges have a
potentially important additional benefit, particularly in taxing
jurisdictions with relatively low rates of taxpayer
participation. Besley, Jensen, and Persson (2015) provide evidence
from local taxpayers in Britain that once in a low participation
equilibrium it will be very difficult, even with penalties and
sanctions, to escape that equilibrium.  Castro and Scartascini (2015)
provide evidence from Argentina that previously paying taxpayers
reduce their rate of compliance when the rate of participation by
their peers falls to near 70 percent.  Even if nudges do not bring in
new money they are valuable if they succeed in holding a taxpaying
community in the high compliance equilibrium against pressures to free
ride; see \citeA{Wenzel-05}.  In Table \ref{lpm_hetero}, for example,
we find that tardy taxpayers in the upper quartile of taxes owed
respond favorably to the peer reminder stressing 9 of 10
Philadelphians do pay their taxes.


\section{Discussion: What Role for Nudges?}

The seven treatments are effective on the margin in increasing
compliance and in raising revenues, and the letters stressing economic
sanctions particularly so. The final column of Table \ref{rev} makes
clear that, at least in Philadelphia, our treatments will not solve the
larger problem of unpaid City property taxes.  The contributions of
each reminder letter towards total taxes owed range from a low of 1.0
percent for the neighborhood letter to a maximum of 4.9 percent for
the lien letter after the three months of our experiment.  The
reminders together raised an additional \$615,222 in property tax
revenues from the total of \$22.143 million owed in tardy payments, or
2.5 percent.  Estimates provided by the six-months results in Table
\ref{rev} show that only \$227,850 can be considered additional revenues for
the tax year; \$387,366 of the \$615,222 paid at the three-month date are
early payments that would have been received at the six-months date
without our reminders.  Importantly, since our sample was close
to the full sample of all tardy taxpayers, these revenue estimates are
close to what the City might expect in new revenues were our
experiment to become annual policy.

Should we, therefore, conclude from these modest revenue gains
that a nudge strategy should not become part of Philadelphia's fiscal
policy?  We think not.  First, our work here was an experiment looking
for an effective collection strategy, not an evaluation of an
established policy.  The reminder/lien and reminder/sheriff letters
were significantly more effective than the average reminder letter.
If those letters were to be adopted as part of the City's collection
strategy for tardy taxpayers and applied alone to the entire sample of
19,039 taxpayers, the city could expect to collect an additional
\$1.29 million (= \$67.90/letter x 19,039 letters) within three months
after mailing the reminders of which \$522,430 (= \$27.44 x 19039
letters) will be new money not available in 2016 tax year.

Second, no single policy is likely to be the best and only mean for
collecting revenues.  Any collection policy will involve a range of
collection strategies, ranging from voluntary compliance on each
annual `tax day,` to mailed reminders as here, to detailed in person
audits, to policing, trial, and impoundment of assets.  Each strategy
will have its own costs and benefits.  All strategies that provide
revenue benefits greater than collection costs should be included as
part of the final collection policy (Keen and Slemrod, 2017).  Our
reminder/lien and reminder/sheriff letters raising over \$65 in new
revenues of each \$1 of administrative costs seem to have earned their
place in Philadelphia's collection strategy.

Third, and perhaps most importantly for designing a cost-effective
strategy to collect tardy taxes, our experiment revealed an important
advantage for the City's tax administration to be patient.  After
three months of our experiment, the holdout sample of tardy taxpayers
receiving no letter increased their tax compliance by 51.4 percent in
making at least some payment and by 40.8 percent in making their full
payment.  The average payment received after three months was \$637
per taxpayer; see Table 2.  These are revenues the City received
without any expenditure of City administrative resources, just waiting
for taxpayers to recognize on their own that their property tax
payments are due.  Had the City adopted this `wait-and-see` strategy
applied to all 19,039 of the tardy, single property owner taxpayers,
then after three months 9,786 tardy taxpayers would have made some
payment.  With an average payment of \$637 per taxpayer, City revenues
would have increased by \$6.234 million, or approximately 28 percent
of revenues owed by all tardy taxpayers.  Other than a small
opportunity cost of waiting, this is ``free" money and clearly the
most efficient beginning strategy for the City for collecting tardy
property taxes.

We conclude that the efficient administrative package would combine
the waiting policy with the most efficient reminder letter stressing
economic sanctions, mailed to all tardy taxpayers.  Assuming those who
pay without a nudge will not be annoyed by receiving the reminder
letter and thus less likely to pay, the City would receive payments of
\$6.234 million from those who pay without the nudge and \$1.288
million from those who pay in response to the nudge with economic
sanctions.  This administrative package raises a total \$7.522
million, or 34 percent of all tardy revenues owed.  The cost of the
combined policy will be just a dollar per letter or \$19,039 for all
of the reminder letters.\footnote{We do not know the actual collection
  strategy used by these firms but our results suggest they are likely
  to be earning a significant profit when reimbursed at the rate of
  \$.06 for every dollar collected.  Doing nothing as revealed by our
  holdout sample will earn these agencies approximately \$6,234
  million in three months.  Assuming they have discovered the high
  marginal returns of the sanction strategy as revealed here (\$65 per
  \$1 invested) and they invest \$19,039 as we have here, the firms
  would then earn an additional \$1.235 million.  Total revenues
  collected would then be \$7.469 million earning the firms \$448,000
  when reimbursed at the rate of \$.06 for every dollar collected.
  Our efficient collection strategy raises the same revenues at a cost
  of \$19,000, saving the City \$429,000.}

Our results provide strong evidence that economic sanctions can have a
significant positive effect on the rate of taxpayer compliance.  After
three months, the lien letter added 9.2 percent (224/2434) taxpayers
above the holdout sample while the sheriff reminder letter added (8.8
percent (= 213/2419) new taxpayers after 3 months.  And while not
quite statistically significant, the peer and civic duty reminders do
encourage tax payments beyond that obtained with a simple reminder
letter, both after three months and particularly so after six
months. The payment magnitudes are comparable to those for the lien
and sheriff letters.

Perhaps the most practical consequence from our study has been to
point out \$300,000 a year in unnecessary collection costs previously
paid to city law firms.  Prior to our study, the City assigned 2/3's
of its tardy taxpayers to law firms for collection of tardy taxpayers,
or 12,700 of our (nearly complete) sample of 19,039 single owner tardy
taxpayers.  Doing nothing as revealed by our holdout sample will earn
these firms approximately \$250,000.  Of the 12,700 assigned taxpayers
approximately 51 percent would have made some payment after three
months without any intervention; see Table 2 for the holdout sample.
Each of these 6,530 taxpayers makes an average payment of \$637; total
taxes collected would then be \$4.158 million dollars, without effort!
These collected revenues will earn a payment from City of \$250,000.
If the law firms were then to adopt our sanction strategy with a
revenue:cost ratio of \$65, and they invest \$1 on each of their
12,700 assigned tardy taxpayers (the cost of the letter), they could
expect to receive an additional \$.825 million in revenues and earn an
additional \$49,500 in commissions. Total commissions paid will then
be \$299,530.  Adopting our nudge strategy, the City can raise the
same amount of three- months tax revenues, \$4.983 million, for \$12,700!


\section{Conclusion}

With the cooperation of the Department of Revenue of the City of
Philadelphia, we developed and implemented a new tax policy
experiment. We used written nudges designed to improve the collection of
property tax revenues from citizens who had not paid their fiscal year
2016 taxes by the due date of March 31, 2015.  The results reported
here are for the 19,039 taxpayers who own a single property, excluding
those who own multiple properties.  The experiment reached six
substantive conclusions: First, a simple reminder letter had a
statistically significant effect on compliance when compared to our
control group who received no reminder.  Second, the content of the
reminder letter matters.  The two letters that stress the likely
economic sanctions of continued tardy payment led to faster and higher
levels compliance than the simple reminder.  Third, adding an
intrinsic message to the reminder, one that stressed the value of
public services, neighbors' compliance, or civic duty did not increase
compliance over receiving only the simple reminder.  Fourth, most of
the taxpayers who did respond with payments, paid their full tax
obligation.  Fifth, reminders were very cost effective on the margin.
Each letter cost one dollar to send and returned on average \$37 in
increased city tax revenues.  The two letters stressing economic
sanctions were the most effective, returning \$65 to \$68 in extra
revenues for each letter sent.  Sixth, reminders had no staying power.
Having received a 2016 reminder letter had no effect on the taxpayer's
likelihood of paying their 2017 property taxes on time.

The results of our experiment suggested three policy conclusions for
the design of an efficient tax collection strategy for Philadelphia.
First, appropriately designed reminder letters as nudges contribute
positively to revenue collection and should be part of the City's
collection strategy.  Simply reminding tardy taxpayers that their
taxes are due is valuable; our average reminder letter cost \$1 to
mail and generated on average \$37 in additional revenues.  As noted,
the most effective reminders stress economic sanctions and raised \$65
to \$68 per reminder letter.  Given these high marginal returns,
nudges should be included as part of the City's collection strategy.

Second, while effective on the margin, reminder letters alone will not
solve the problem of tardy tax collections in Philadelphia.  Our study
considered the sample of tardy taxpayers that owned one property, and
our reminder letters alone succeeded in collecting only a small
fraction of what was owed.  The total owed for fiscal year 2016 was
\$24.866 million and the experiment using all our reminders raised
only \$615,222 of new revenues.  Had we used only the most effective
reminders -- the two stressing economic sanctions -- we would have
doubled collected revenues from reminders to \$1.288 million in new
revenues; still only approximately 5 percent of outstanding payments.

Third, and perhaps most interestingly, just being patient had high
returns.  Our study included a sample of holdout taxpayers who
received no reminder.  That group increased their rate of compliance
by 51.4 percent from the start of our study to its three month
conclusion date. The average payment from this sample receiving no
reminder was \$637 per taxpayer.  Total new City revenues from
this group equaled \$6.234 million dollars, all raised at no cost to
the City other than the small opportunity cost from delay.  Relying
upon tardy taxpayers to recognize and pay their taxes on their own and
then using the sanction reminder to leverage those who continue to
forget or need a nudge would together raise \$7.522 million of the
\$24.866 owed by all tardy taxpayers, or 34 percent.  The total cost
to the City of this collection strategy would be \$19,039 for the
reminder letters.  After three months, a more aggressive but also more
expensive collection strategy might be tried.

There are limitation to our study, however. First, the large
empirical literature on nudges and tax collection makes clear that
successful strategies are context specific.  Collection strategies
that work well for one tax and for one government may fail to do so
for another tax and in another fiscal setting.  Our results are for
Philadelphia taxpayers alone.  What we think does generalize, however,
is the value of repeating studies such as ours for the design of tax
administration strategies.  We feel that our seven reminder letters
and the importance of having a holdout sample provide an effective
methodology for use by other cities for understanding how best to
collect their own tardy and delinquent property taxes.  Further, at an
average cost of \$1 per reminder letter including mailing and office
expenses, there is no reason not to implement a study such as ours on
the full sample of tardy or delinquent taxpayers.

Second, that said the effectiveness of each letter is likely to depend
on the exact wording of the message and the additional information
that is provided in the letter.  For example, in our pilot study we
found that a simple deterrence message that did not provide any
specific or detailed information about enforcement was not effective
in increasing compliance.  In contrast, in this version of our
experiment reminder letters stressing sanctions did provide such
information, and these reminders were our most effective intervention.

Third, our experiment only allowed us to estimate the impact of
reminder letters conditional upon taxpayers' information and
preferences and cannot be used to evaluate the impact of more
extensive interventions.  While our sanction reminder letters did
refer to existing sanctions now in place and illustrated those
sanctions for real properties in the taxpayer's neighborhood, our tax
morale messages were less specific as to the benefits of paying taxes
and thus may suffer by comparison.  For example, the tax morale
messages did not follow a city-wide intervention advertising the value
of neighborhood or city services, have a personal plea to ``join me"
in paying city taxes, or stress the civic obligation we all have as
citizens to pay our �fair share.� These more extensive, but costly,
tax morale interventions may be very effective.

Fourth, our experiment does not allow us to determine with any
precision why approximately 1/3 of tardy taxpayers in our sample did
not pay their taxes by the end of the tax year.  They are now
delinquent.  A liquidity constraint may be the problem, but the option
to use the City's tax payment program to smooth payments available,
yet no more than 3 to 4 percent of those who did pay used the plan.
Interviewing these newly delinquent families would be valuable.

Finally, it is useful to speculate as to who pays the final burden for
our effective interventions.  General taxpayers pay for the initial
cost of our reminders.  When enforcement is successful, the burden
falls primarily on those tardy taxpayers who pay the fines and
penalties.  Those fines and penalties presumably cover their
enforcement costs and the costs associated with the delay in payment.
If enforcement is not successful then the burden of non-payment falls
on the taxpayers who do comply.  Assuming public services are
maintained, then there is a redistribution from those who pay to those
who do not.  This may be socially useful if the non-paying citizens
are facing temporary economic hardship.  A final benefit-cost analysis
will need to consider the distributive consequences of any enforcement
strategy.

Motivated by the results of our work here, Philadelphia's Department
of Revenue now uses a reminder letter stressing the risk of a tax lien
and subsequent sheriff's sale and has delayed by three months (to
mid-July) the use of outside firms for the collection of tardy of
taxes.

%{\footnotesize \NoTitleCaseChange\citepunct{(}{and}{, }{; }{, }{)}{}{.}
\newpage

\bibliographystyle{theapa}
\bibliography{references}

\newpage

\begin{appendix}


\begin{figure}[htbp!]
\begin{center}
\includegraphics[width=6in, height=8.5in]{reminder_generic.pdf}
\end{center}
\end{figure}

\begin{figure}[htbp!]
\begin{center}
\includegraphics[width=6in, height=8.0in]{reminder_lien.pdf}
\end{center}
\end{figure}

\end{appendix}


\end{document}
