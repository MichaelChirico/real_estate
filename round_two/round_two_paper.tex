\documentclass[12pt]{article}
\usepackage{amssymb}
\usepackage{theapa}
\usepackage{titlepage}
\usepackage{amsmath}
\usepackage{setspace}
\usepackage[dvips]{graphicx}
\usepackage{rotating}
\usepackage[usenames,dvipsnames]{pstricks}
\usepackage{epsfig}
\usepackage{pst-grad}
\usepackage{pst-plot}
\usepackage{color}
\usepackage{pstricks-add}
\usepackage{rotating}
\usepackage{threeparttable}
\usepackage{array,multirow}
\usepackage{pdflscape}
\usepackage{float,lscape}
\usepackage{csquotes}


\renewcommand{\baselinestretch}{1.5}
\parindent=.2in
\evensidemargin=.05in 
\oddsidemargin=-.05in 
\topmargin=-0.05in
\textwidth=6.5in 
\textheight=8in

\newtheorem{fact}{Stylized Fact}
\newtheorem{theorem}{Theorem}
\newtheorem{corollary}{Corollary}
\newtheorem{definition}{Definition}
\newtheorem{lemma}{Lemma}
\newtheorem{prop}{Proposition}
\newtheorem{assumption}{Assumption}
\newtheorem{remark}[theorem]{Remark}
\newtheorem{solution}[theorem]{Solution}
\renewcommand{\thefootnote}{\fnsymbol{footnote}}

\begin{document}

\title{Deterring Delinquency: A Field Experiment in Improving Tax Compliance Behavior}

\author{Michael Chirico, Robert Inman, Charles Loeffler, \\ 
John MacDonald, and Holger Sieg\thanks{We would like to thank Rob Dubow,
    Clarena Tolson, and Marisa Waxman in the Department of Revenue of
    the City of Philadelphia for their help and support. We thank Kent
    Smettters and the Wharton Initiative for Public Policy for funding
    this field experiment. We would also like to thank Jeff Brown, Kai
    Konrad, Robert Moffitt, Jim Poterba, Chris Sanchirico, Wolfgang
    Sch\"on, Reed Shuldiner and participants of numerous seminars for
    comments and suggestions. The views expressed here are those of
    the authors and do not necessarily represent or reflect the views
    of the City of Philadelphia.}  \\ 
University of Pennsylvania}

\date{\today}

\maketitle

\begin{abstract}

Property taxation plays a central role in the financing of municipal
government services. Local taxing authorities commonly confront
problems of property tax collection even when the tax base is known.
Using a multi-arm RCT conducted with the City of Philadelphia, we
compare the responses of delinquent taxpayers to seven different
notifications reflecting a wide variety of theorized motivations for
tax compliance. We find that all our nudge strategies significantly
outperform the ``do nothing'' alternative both in the rate of taxpayer
compliance and in the level of payments.  Among the seven nudges, the
most effective in encouraging tax payment are the two economic
strategies that threaten large financial penalties such as taking out
a lien on the property or threatening to foreclose by sheriff's sale.
Overall, the rate of return on these nudge strategies are significant
and large in economic magnitude.

\bigskip

\noindent KEYWORDS: Tax Compliance, Property Taxation, Field
Experiment, Deterrence, Public Service Appeal, Appeal to Civic Duty.

\end{abstract}
\renewcommand{\thefootnote}{\arabic{footnote}}

\newpage

\section{Introduction}

Property taxation plays a central role in the financing of municipal
government services in the United States. As a matter of practical
local government finance, it is the mainstay of city and school
district budgets.  In 2013, over 72 percent of all local government
tax revenues and nearly 50 percent of all own revenues came from
property taxation. The potential economic advantages and disadvantages
of a local property tax are well known.  With mobile households and
local zoning the tax approximates a benefit tax for the financing of
local services.  In large cities, where these assumptions are unlikely
to hold, the tax will have adverse incentive effects on new
construction and improvements in growing cities but approximates a
land tax in its incentive effects in declining or stable cities with
modest new construction \cite{Aaron-75}.  As a tax on value, the
property tax is a proportional tax on wealth and thus attractive from
the perspective of lifetime equity \cite{Mieszkowski-72}.

The collection of property taxes has one very important administrative
advantage over the collection of other taxes: the legal tax obligation
is known to the taxpayer and the taxing authorities.  Self-reporting
of tax bases, as required for income, profits, sales, and VAT
taxation, is not needed for the property tax. Each taxpayer has an
assigned tax base, the value of property, against which a common tax
rate is assessed.  This avoids problems of misreporting tax bases or
working outside the formal or  taxable economy.\footnote{See
  \citeA{Blumenthal-01}, \citeA{Kleven-11}, and \citeA{Pomeranz-15}.}

For all these virtues to be realized, it is essential that the
property tax be collected, both efficiently and fairly.  Among large
U.S. cities, this is not assured.  Because of the importance of the
tax to city budgets, even small differences in collection rates can
significantly affect the provision of local public services. While the
average rate of tax collection among a sample of large U.S. cities is
95 percent within the year of tax assessments, many cities collect
only 90 percent of taxes due, and several cities do far worse.  Among
the poorest performers are Cleveland (84\%), Detroit (68\%), Flint
(64\%), Milwaukee (86\%), and Pittsburgh (85\%). While the poorest
performers are all high poverty cities, poverty alone is an inadequate
explanation for their performance. There are many high poverty cities
that in fact collect most all of their property taxes; for example,
Baltimore (96\%), Birmingham (98\%), Dallas (98\%), Houston (98\%),
Minneapolis (98\%), and even New Orleans (95\%) \cite{CILMS-16}.
These high poverty high tax compliance cities suggest that other
factors are likely at work.  This paper explores one of these factors:
tax collection strategies. While the administrative issue for the
property tax is simple: Did the taxpayer pay the tax on time or not?
If not, then what can the tax administrator do to enforce compliance?
there is considerable disagreement on how to ensure compliance. In
this paper we explore the efficacy of seven ``nudges" for improved
property tax collection in one major U.S. city, Philadelphia.

Our first nudge strategy is a simple reminder letter to the taxpayer
that their taxes are due; the letter is identical in content to the
initial tax bill listing the tax due and  any penalties for late
payments.  The reminder letter will be our ``control nudge" and is
meant to address non-payment due to forgetfulness or oversight.

The second set of strategies are meant to address an economic motive
for non-payment.  The delinquent taxpayer is assumed to be making an
economic calculation that by not paying there is a positive
probability that delinquency will go undetected or if detected,
ignored for administrative reasons, and that the expected economic
gains of not paying exceed the expected economic costs of being caught
and fined \cite{Allingham-Sandmo-72}.  In most real world tax
settings, however, the probability of being caught and the size of the
likely sanction are both too low to rationally account for most
observed levels of taxpayer noncompliance \cite{Alm-92}.\footnote{An
  alternative specification for taxpayer utility that allows for loss
  aversion has done a better job in explaining taxpayer compliance
  among Swedish taxpayers than did the classical expected utility
  specification with always declining marginal utilities in income;
  see \citeA{Engstrom-15}. } Here we test for the effect of two nudges
with potentially large economic consequences, one where delinquent
taxes plus a graduated fine growing over time are collected as a
``lien'' on the property at sale, and a second, where the property
is seized for a ``sheriff's sale'' with a portion of the
proceeds used to pay delinquent taxes and penalties. The lien imposes
a growing real dollar future loss on the delinquent taxpayer as the
interest rate for penalties exceeds the taxpayer's alternative rate
of return. The sheriff's sale imposes an immediate economic loss,
but further, requires the delinquent taxpayer to find a new residence.
Both of these nudges threaten large economic, and in the case of the
sheriff's sale large psychic, costs for continued noncompliance.

Two additional nudges appeal to what \citeA{Luttmer-14} have called
``tax morale."  First, we remind taxpayers that their payments do
provide valuable public services.  Here we seek to address
noncompliance due to a desire by taxpayers to free-ride on the
payments of their neighbors or of Philadelphians generally.  The first
free-rider strategy seeks to motivate payment by reminding the
taxpayer that his payments go to providing services for his family and
his immediate neighbors and lists specific neighborhood amenities
likely to be affected; we call this strategy the ``neighborhood"
nudge.  The second free-rider strategy reminds the taxpayer that his
taxes support important city-wide services such as education and city
safety. We call this strategy the ``community" nudge.
  
A final set of nudges appeals to a possibly deeper motive for tax
compliance, fulfilling one's obligations to a self-identified
community of peers \cite{Posner-00} or to an abstract community of
citizens \cite{Rawls-71}.  The first of these community strategies we
call the ``peer" nudge.  The second we call the ``civic duty" nudge.  
The seven nudge strategies for increased taxpayer compliance are first
compared to the alternative of doing nothing beyond sending the first
tax bill.  We then compare the seven nudge strategies among themselves
to see which are most effective in encouraging taxpayer compliance.

Our experiment started in the beginning of June 2015. No other enforcement
activities were undertaken by the City until the middle of August
2015. It is, therefore, useful to distinguish between short-term and
long-term impact of our intervention. Short term outcomes are those
that measure compliance up to three months into the experiment. These
outcomes are clean measures of the impact of the intervention since no
other enforcement activities took place during that time period.  In
the short term, we find that most of our nudge strategies
significantly outperform the ``do nothing'' alternative both in the
rate of taxpayer compliance and in the level of payments, conditional
on compliance.  Second, among the seven nudges, the most effective in
encouraging tax payment are the two economic strategies that threaten
large financial (``lien'') or financial and psychic (sheriff's sale)
penalties.

After one month, approximately 30 percent of all taxpayers in the
hold-out sample had made some contribution towards their tax
liabilities. In contrast 40 percent of taxpayers that received the
lien letter and 37 percent that received the sheriff sale letters had
made payments. The results are similar after three months of the
intervention.  51 percent of all households in the hold-out sample had
made some contribution after three months (61 percent for the lien and
60 percent for the sheriff sale letters) Reminder letters also
improved the level of tax payments, given that the taxpayer
complied. The three-month impact of these two letters was
approximately \$75 per letter. We thus conclude that receiving a
reminder letter improved taxpayer compliance int he short run, with
the lien and sheriff sale letters the most effective.

We also consider long term outcomes measured six month after our
experiment. While our experimental design is still valid for these
outcomes, it is harder to interpret the findings since other
treatments occurred during that time period such as enforcement
activities by collection agencies. In particular, the city hires two
collection agencies. These collections primarily use phone calls to
contact tardy tax payers and threaten them with penalties and fines to
obtain compliance. All taxpayers that have not paid the taxes are
subject to this uniform second treatment by the two collection
agencies.  Our estimates of the longer term treatment effects thus
reflect two treatments: our initial letter treatment plus the phone
calls performed by the collection agencies. Since we were not able to
randomize on the treatment by the collection agencies, we can only
identify the effect of the joint treatments.

Our findings with respect to longer run outcomes indicate that there
was at least some convergence in the effectiveness of the nudge
strategies. After 6 months, 73 percent of households in the hold-out
sample made some payments to the City. The only letters that
significantly improve the compliance above that rate were the lien and
sheriff letters, which increased compliance by 3 to 4 percentage
points. The six-month impact of these two letters was approximately
\$31 per letter relative to the hold-out sample and \$21relative to
the control letter. We thus see a fair amount of convergence in the
effectiveness of the different treatments at the six month stage. We
do not now whether this convergence in the effectiveness is due to
additional treatment by the collection agencies or whether this
convergence would have also occurred in the absence of the second
treatment.

We find that the impact of our six letters relative to the hold-out or
control group is much reduced after six month. This finding is
consistent with the view that the enforcement activities of the
collection agencies are probably more closely aligned with our
deterrence letters than with the other five letter that we
explored. As a consequence it is not surprising that we find some
strong convergence in the effectiveness of all treatments after six
months.

Finally, we investigated whether our experiment had any spill-over
effects into the following tax year. Looking a compliance rates by
March 31, 2016 we find no significant impact of our letters on
compliance in the following year

Ignoring the opportunity costs of time, the cost of our experiment was
\$23,000.  Our of-the-envelop calculations suggest that our experiment
generated approximately \$690,000 after three months. We estimate that
if extended to the whole of our sample of delinquent taxpayers, the
two most effective nudges -- the lien or sheriff sales reminders --
have the potential to increase collected revenues during each tax year
by as much as 3 to 4 million dollars.

The rest of the paper is organized as follows. Section 2 contains a
brief literature review. Section 3 discusses details of our field
experiment including a detailed description of the treatments and the
randomization procedure. Section 4 reports the main empirical
findings. Section 5 discusses the policy impact of our experiment and
discussions on enforcement activities in the City of
Philadelphia. Section 6 offers some conclusions.
    

\section{Literature Review}

Our study is related to different branches of the empirical literature
on tax compliance. Our study is related to different branches of the
empirical literature on tax compliance. Early empirical studies
focused the effectiveness of penalties and fines and found little
impact of such penalties on aggregate tax compliance
\cite{Slemrod-07}.  More recent, nuanced studies, have found an impact
of fines on both the level and speed of tax payments.
\citeA{Fellner-13} find that a reminder letter for payment of the
Austrian TV license fee that explicitly threatens legal action if the
resident does not provide the required information for assessment
performed significantly better than the standard reminder letter
informing residents that they had not yet returned the required forms.
\citeA{Wenzel-Taylor-04} find that including a letter reminding
taxpayers that their statement of rental income can be audited and
that faulty reporting may lead to fines significantly reduced
deductions when compared to forms submitted by taxpayers who did not
receive the threatening letter \citeA{Hallsworth-14} find the speed
with which taxpayers pay their liabilities can also be improved with
increased fines.  In contrast, \cite{Ariel-12} find that an increase
in fines reduced corporate tax compliance in Israel.

A range of alternative theories have been suggested to explain
taxpayer compliance behaviors.  First, perhaps taxpayers are honest
but simply do not understand what their true obligations are. Tax
forms can be complicated.  A recent study by \citeA{kosonen} of
Finnish small business owners who faced a well-publicized change in
their VAT tax rate from 9 percent to 23 percent found that explicitly
mentioning the rate change as part of general questionnaire regarding
tax administration for small businesses significantly increased tax
compliance to the higher rate.

Second, true obligations may be known to the taxpayer, but they may
choose to cheat.  They can do so in two ways.  When taxes are
self-assessed (e.g. income tax, VAT, profits), taxpayers can
under-report incomes or sales and over-report costs and purchases; or,
they can simply not pay by working outside the formal economy.
Studying compliance behavior of Danish and Chilean taxpayers,
\citeA{Kleven-11} for Denmark and \citeA{Pomeranz-15} for Chile found
that taxpayers reported incomes and value-added sales increased as the
ability of the tax administration to independently assess those income
and sales improved via outside reporting.  Reported taxable incomes by
Minnesota residents were also found to increase when the probability
of an official tax audit was increased \cite{Blumenthal-01}.

When tax obligations are known to both the taxpayer and the tax
authorities -- as is the case for government-assessed property
taxation -- citizens may still choose to cheat if the chances they
will be detected, prosecuted, and fined are low.  From the economic
model of tax compliance, as first specified by
\citeA{Allingham-Sandmo-72}, taxpayers make their decision to comply
by balancing the economic savings from non-payment against the
uncertain costs they bear from being caught and fined.  In most
studied instances of tax compliance, however, the probability of being
caught and the associated fines are too low to rationally account for
the observed high rates of tax compliance.  Nor can the answer be
found in any plausible estimate of taxpayer risk aversion
\cite{Alm-92}. Efforts to understand taxpayer compliance need to
consider explanations beyond the narrow framework of individual
utility maximization under uncertainty.

There are two extensions of the usual framework to consider.  The
first re-specifies the taxpayer's utility from income to allow for
non-convex reactions to equal gains and losses.  A recent study of
Swedish taxpayers by \citeA{Engstrom-15} finds loss aversion as
defined by prospect theory can account for taxpayer compliance in a
way that classical utility maximizing behavior with risk aversion
cannot.  Taxpayers facing a loss from a \$1000 tax payment were
significantly more likely to overstate allowed deductions than
taxpayers facing a \$1000 refund for the same deductions.

The second approach retains the classic specification for taxpayer
welfare from income, but adds one or more additional motives for
payment, called ``tax morale" by \citeA{Luttmer-14}.  They include
reciprocity or payment for public goods received; norm behavior or
peer effects; and civic duty.  Reciprocity argues that citizens
understand that to not pay their taxes will mean less public services.
In this case, government services along with after-tax income
determine taxpayer welfare.  One would expect this motive to be
strongest when tax payments are directly linked by the taxpayer to
services received, for example, local street repairs.  Peer effects
may arise when citizens view non-payment as a violation of a community
norm of cooperative behavior and an individual's non-payment is
observed by others in the community.  Here, how many other taxpayers
are compliant matters to whether the citizen also pays; see
\citeA{Posner-00}.  One might expect this motive to be strongest when
a citizen's non-payments are publicized and the citizen is actively
involved in, or exposed to, a community group that benefits from those
payments--for example, a neighborhood school association or community
oriented church group.  Finally, citizens may pay their full tax
obligation because it is the ``right thing to do" as a citizen.  Here
the act of payment has value on its own; there are no direct benefits
and no one else need know.  The citizen has accepted the democratic
contract and bears a presumptive obligation to fulfill that contract;
see \citeA{Rawls-71}.

Efforts to empirically identify the possible influence of these
non-economic motives have been mixed.  \citeA{Blumenthal-01} find no
evidence that these motives significantly influence truthful reporting
of taxable income for Minnesota taxpayers, but \citeA{Hallsworth-14}
do find a strong beneficial impact on compliance from peer motives.
In a study closest to our work here, \citeA{castro} examine motives
for property tax payments in a municipality in Argentina.  They find
that the economic motives from fines and enforcement are most salient,
but that the non-economic motives do matter for selected subsamples of
the population--in particular, lower-income residents.  Finally, in
our earlier, pilot study of property tax compliance in Philadelphia we
did find evidence that motives driven by reciprocity, peer effects,
and civic duty can positively impact payment probabilities. But our
sample size was small and our framing of the alternative motives was
not as clear as we would have liked; see \citeA{CILMS-16}.  We view
our work here as chance to pursue all these motives (deterrence,
reciprocity, peer influence, and civic duty) with a larger sample and
with a sharper experimental design.

  
\section{ A Tax Reminder Experiment}
  
\subsection{Treatments}

The research setting for this experiment is the City of Philadelphia.
Notices of property tax payments are sent each year on January 1, and
the full balance of taxes are due by March 31.  If payment has not
been received by that date, or the taxpayer has not entered into a
taxpaying plan with the City, fines and interest penalties begin to
accrue.  On April 1, the Department of Revenue (DoR) begins contacting
unpaid accounts informing taxpayers of taxes due and the accumulation
of fines and penalties for late payment.  Normally, two-thirds of the
delinquent accounts are sent to outside collection agencies acting as
co-counsel for the City; one-third of the delinquent accounts remain
within the Revenue Department for collection.  The outside collecting
agents are reimbursed at the rate of 6 percent of all delinquent
revenues collected by December 31st.  All accounts still delinquent
after that time are then assigned to new collection agents.  Our
experiment was implemented using the City's share of delinquent
taxpayers for the tax year, 2015.
  
Of the 579,828 properties in the city in 2015, approximately 100,000
properties, or 83 percent of all properties, were delinquent as of
April 1st.  The sample included in our experiment were the 27,264
properties remaining with the Revenue Department and still owing at
least \$10 in property taxes as on May 15, 2015.  Our sample includes
only new delinquent taxpayers; it excludes all chronically delinquent
taxpayers who owe taxes from prior years.  Our experiment began in
mid-June, 2015 and continued until December 31, 2015.  To make sure
that our experiment as not contaminated by other treatments, the DoR
agreed to postpone other enforcement activity until August 15. In
particular, no other collection agencies contacted the households in
the sample until approximately the beginning of September.

Our seven reminder letters were designed in coordination with
officials of the Department of Revenue.  Each letter was vetted by the
Department to ensure that it could be understood by a taxpayer with at
least a fourth or fifth grade level of reading comprehension.  Each
letter also provided contact information for assistance for
non-English speaking taxpayers.  The full letter templates are
included in an Appendix.  Here we present the important distinguishing
feature of each letter.  Our control letter provides a generic
reminder to the taxpayer. Specifically:
 
{\it Treatment Letter 1: Control } \\ {\bf Our records indicate that
  you have a balance due of $balance$.  If you have already paid,
  thank you. If not, please pay now or contact us to arrange a payment
  plan. The fastest and easiest way to pay is online at
  \underline{www.phila.gov/pay}. Paying by E-check only costs 35c -
  less than the cost of a stamp!"}

Two letters were mailed to test the efficacy of either of our two
economic penalties.  The first imposes an economic penalty only and is
called the lien letter.  The lien letter notes that the City will
impose a lien on the delinquent property which entitles the City to
deduct the amount of the lien from any future arms-length market sale
of the property.
        
{\it Treatment Letter 2: Lien } \\ {\bf Failure to pay your Real
  Estate Taxes may result in the sale of your property by the City in
  order to collect back taxes. In the past year, we have sold $N$
  properties in your neighborhood at Sheriff's Sale. Included in these
  $N$ are the following properties near you: $<$three properties and
  their sale dates$>$

  Pay your taxes now to prevent the sale of your property.  Our
  records indicate that you have a balance due of $balance$.}

$N$ is the number of properties sold in neighborhood between June 2014
and May 2015. The three listed properties in the taxpayer's
neighborhood were randomly selected from a list of properties that had
been recently sold and included tax liens on the sale.  All delinquent
taxpayers receiving the lien letter and in the same neighborhood saw
the same list of three properties.\footnote{An initial plan to select
  the three lien sale properties nearest each delinquent property met
  with privacy concerns and was therefore not pursued.}

The second letter including an explicit mention of an economic penalty
was the sheriff's sale letter.  We view this treatment letter as the
most onerous economically.  It not only imposes the full economic
penalty of taxes plus fines plus interest at the time of sale, but it
forces the sale of the taxpayer's property.  The inconvenience and,
perhaps more importantly, the psychic costs of moving may be
significant.  
	
{\it Treatment Letter 3: Sheriff's Sale} \\ {\bf Failure to pay your
  Real Estate Taxes will result in a tax lien on your property in an
  amount equal to your back taxes plus all penalties and
  interest. When your property is sold, those delinquent tax payments
  will be deducted from the sale price. By paying your taxes now, you
  can avoid these penalties and interest. Properties near you in
  $neighborhood$ that have had liens placed on them include: $<$three
  properties and their sale dates$>$

  Pay your taxes now to avoid a lien being placed on your property.
  Our records indicate that you have a balance due of $balance$.}
	
N is the number of properties sold by sheriff's sale in each
neighborhood between June, 2014 and May, 2015.  The three listed
properties in the taxpayer's neighborhood were randomly selected from
a list of properties that had been recently sold through a sheriff's
sale.  Again, all delinquent taxpayers receiving the sheriff's sale
letter and in the same neighborhood saw the same list of three
properties.

The next two reminder letters address the free rider motive for
non-payment.  The first letter appeals for payment from those who
might see their gain from non-payment largely in terms of their
private benefits from neighborhood services, what we call the
neighborhood letter.  

{\it Treatment Letter 4: Amenity } \\ {\bf We want to remind you that
  your taxes pay for essential public services in $neighborhood$, such
  as $<$two local amenities$>$, your local police officer, snow
  removal, street repairs, and trash collection. Please pay your taxes
  to help the city provide these services in your neighborhood.}

The neighborhood amenities were chosen at random for each property
from a list of City provided parks, recreation centers, and libraries
in the neighborhood of the delinquent property.  The second free rider
letter appeals for payment from those who see their gain from
non-payment in terms of their public benefits from Philadelphia-wide
services, what we call the community letter.  This letter reads:

{\it Treatment Letter 5: Community} \\ {\bf Your taxes pay for important
  services that make a city great. Your tax dollars are essential for
  ensuring all Philadelphia children receive a quality education and
  all Philadelphians feel safe in their neighborhoods. Please pay your
  taxes as soon as you can to help us pay for these important
  services.}

The final two reminder letters appeal to a taxpayer's sense of
community more generally.  The first asks the delinquent taxpayer to
recognize that he is not a contributing member of his (personally
defined) community of peer taxpayers, a letter we call the peer
letter.  

{\it Treatment Letter 6: Peer} \\ {\bf You have not paid your Real
  Estate Taxes. Almost all of your neighbors pay their fair share: 9
  out of 10 Philadelphians do so. By failing to pay, you are abusing
  the good will of your Philadelphia neighbors.}

The second letter stresses that non-payment will violate a wider
community norm of honest and responsible tax compliance needed for a
functioning democracy, a letter we call the civic duty letter. 

{\it Treatment Letter 7: Duty } {\bf For democracy to work, all
  citizens need to pay their fair share of taxes for community
  services. You have not yet paid your taxes. By failing to do so, you
  are not meeting your duty as a citizen of Philadelphia.}


As a baseline control, we randomly removed 3,000 delinquent taxpayers
from the possibility of receiving any reminder letter at all.  These
taxpayers became our holdout sample and allowed us to estimate the
efficacy of simply communicating with the taxpayer after the date that
taxes are due.\footnote{We tested one more intervention that has been
  successfully used by private firms in collecting overdue credit card
  payments.  This is to send the payment reminder in an envelope
  larger than the usually sized envelopes used for the first mailing
  of tax bills.  Credit card firms have found that reminders mailed in
  usual envelopes (4 1/8" by 9 ?") were often ignored, while reminders
  mailed in larger envelopes (9" by 12") resulted in greater payments.
  The total number of properties in this additional treatment was
  12,193 randomized over the seven treatment letters.  We found no
  statistically significant effect of letter size on compliance
  behavior or size of payment.  These results are available upon
  request.}
	
\subsection{Randomization Procedure}

Randomization took place in two stages.  First a number of eligible
properties were randomly assigned to the Holdout Sample.  Our main
interest in this study is on unique owners, i.e. households that only
own one property in the city. Once we restrict attention to this
sample than we have 16,940 observations in the treatment group and
2,088 observations in the hold-out sample.  The total sample size is
19,028.\footnote{We also trimmed the sample and excluded 28
  observations with highest assessed property value. None of the
  findings reported in the paper depend on this trimming.}  Table
\ref{bal_hold} checks whether the treatment and control group are
balanced based on the two most important variables, amount due and
assessed property value.

\begin{table}[ht]
\centering
\caption{Balance between Holdout and Treated Samples}\label{bal_hold}
\bigskip
\begin{tabular}{lrrc}
   \hline
Variable & Treated & Holdout & $p$-value \\ 
  Amount Due (June) & \$1,287 & \$1,233 & 0.24 \\ 
  Assessed Property Value & \$144,145 & \$142,630 & 0.93 \\ 
  \# Owners & 16,940 & 2,088 &  \\ 
   \hline
\end{tabular}
\end{table}

Table \ref{bal_hold} shows that randomization was successful in the
unique owner sample.  The average debt owed by each owner was \$1,287
in the treatment group and \$1,233 in the hold-out sample. The average
assessed property value is \$144,145 in the treatment group and
\$142,630 in the control group.

Next we test whether randomization was successful among the seven
treatment groups. Table \ref{balance} shows the results for the unique
owner sample in the top panel of the table. Overall, we find no
evidence that would suggest any problems with randomization.

While the vast majority of properties in the city of Philadelphia are
owned by unique owners, approximately 10 percent of the properties are
owned by individuals or firms that own multiple properties. There is
some interest in including these multiple owners in the analysis as
well. Since we are interested in taxpayer compliance and not property
compliance, we identified owners of multiple delinquent properties by
their legal name and sent those owners one treatment letter.  We
identified multiple property owners by matching the legal name
associated with each property.\footnote{We lacked an objective
  identifier such as a social security number so identification was by
  the owner's legal name.  There is possibility that two or more
  different owners might have the same name, but inspection by the
  authors found this to be very rare.  We consider this random noise
  to the experiment.}

\begin{sidewaystable}[htbp]
\centering
\caption{Balance on Observables}\label{balance}
\bigskip
\begin{tabular}{lrrrrrrrc}
\hline
\multicolumn{9}{c}{Unique Owners} \\
\hline
Variable & Control & Amenities & Moral & Duty & Peer & Lien & Sheriff & $p$-value \\ 
\hline
Amount Due (June) & \$1,256 & \$1,289 & \$1,290 & \$1,299 & \$1,280 & \$1,280 & \$1,315 & 0.98 \\ 
Assessed Property Value & \$158,370 & \$159,079 & \$130,265 & \$165,617 & \$130,936 & \$130,642 &
 \$134,334 & 0.46 \\ 
\# Owners & 2,419 & 2,387 & 2,441 & 2,432 & 2,416 & 2,429 & 2,416 & 0.99 \\ 
\hline
\multicolumn{9}{c}{Unique and Multiple Owners} \\
\hline
Variable & Control & Amenities & Moral & Duty & Peer & Lien & Sheriff & $p$-value \\ 
\hline
Amount Due (June) & \$1,593 & \$1,589 & \$1,583 & \$1,583 & \$1,572 & \$1,593 & \$1,590 & 1 \\ 
Assessed Property Value & \$180,664 & \$180,172 & \$153,528 & \$183,991 & \$155,438 & \$155,499 & \$157,398 & 0.48 \\ 
\% with Unique Owner & 87.6 & 86.4 & 88.4 & 88.1 & 87.5 & 88.0 & 87.5 & 0.42 \\ 
\% Overlap with Holdout & 3.69 & 3.73 & 3.40 & 3.40 & 3.55 & 3.44 & 3.29 & 0.97 \\ 
\# Properties per Owner & 1.27 & 1.32 & 1.26 & 1.26 & 1.26 & 1.26 & 1.26 & 0.67 \\ 
\# Owners & 2,762 & 2,762 & 2,762 & 2,762 & 2,762 & 2,761 & 2,762 & 1 \\ 
\hline
\multicolumn{9}{l}{\scriptsize{$p$-values in rows 1-5 are $F$-test
    $p$-values from regressing each variable on treatment dummies. A
    $\chi^2$ test was used for the count of owners.}} \\
\end{tabular}
\end{sidewaystable}

Having identified owners, we then randomly assigned each owner to a
treatment group.  Any delinquent taxpayer holding multiple properties
within each treatment group received the same letter for each of those
properties.  Given the high correlation between the propensity to pay
taxes and total debt-owed, the treatment groups were defined according
to owner-level total debt to assure uniformity of samples along the
dimension of debt owed.  Standard errors in our analysis are clustered
by the treatment group as randomized.  Each property assigned to
receive a reminder letter was equally likely to receive one of the
seven treatments.

Excluding the hold-out sample, but including multiple owners gives us
a sample size of 19362 observations.\footnote{Unfortunately, we were
  not able to include the hold-out sample in the block-randomization
  procedure. As a consequence, we can only include the hold-out sample
  into our analysis if we condition on unique ownership.} Balance
tests for pre-randomization characteristics are reported in Table
\ref{balance}.  Results confirm that randomization was also successful
in this larger sample that included multiple property owners.  There
are no significant differences across reminder letters.

\section{Empirical Results}

\subsection{Short Term Impact}

In this section we focus on the short term impact of our
intervention. We define the short term as the first three months after
our intervention. During this time period tardy tax payers were only
exposed to our intervention. As a consequence, our estimates of the
treatment effects are not contaminated by other interventions.

To gain insight into the nature of tax compliance in Philadelphia, we
consider two discrete measures of tax compliance. We define partial
compliance if the the tardy taxpayer makes any payment at all and
zero.  Partial Compliance is of interest because even small additional
payments help, but perhaps more importantly, a tax contribution
represents a willingness by the taxpayer to be engaged with city
governance.  The ever-paid outcome does not differentiate between
taxpayers that made full payment and those who made only a partial
contribution.  Full compliance is defined as making a full payment.

\begin{table}[ht]
\centering
\caption{Short Term Linear Probability Model Estimates} \label{pc_lin}
\bigskip
\begin{tabular}{l c c c c }
\hline
 & \multicolumn{2}{c}{Ever Paid} & \multicolumn{2}{c}{Paid in Full} \\
          & One Month & Three Months & One Month & Three Months \\
Holdout   & $30.5$ & $51.4$ & $23.5$ & $40.8$ \\
\hline
          & $(1.0)$      & $(1.1)$      & $(1.0)$      & $(1.1)$      \\
Control   & $3.8^{***}$  & $3.9^{***}$  & $2.2^{*}$    & $3.0^{**}$   \\
          & $(1.4)$      & $(1.5)$      & $(1.3)$      & $(1.5)$      \\
Amenities & $1.7$        & $2.7^{*}$    & $-0.2$       & $1.5$        \\
          & $(1.4)$      & $(1.5)$      & $(1.3)$      & $(1.5)$      \\
Moral     & $3.8^{***}$  & $2.8^{*}$    & $1.3$        & $2.5^{*}$    \\
          & $(1.4)$      & $(1.5)$      & $(1.3)$      & $(1.5)$      \\
Duty      & $2.4^{*}$    & $3.6^{**}$   & $0.7$        & $2.3$        \\
          & $(1.4)$      & $(1.5)$      & $(1.3)$      & $(1.5)$      \\
Peer      & $3.9^{***}$  & $3.5^{**}$   & $1.8$        & $3.4^{**}$   \\
          & $(1.4)$      & $(1.5)$      & $(1.3)$      & $(1.5)$      \\
Lien      & $9.0^{***}$  & $9.2^{***}$  & $5.6^{***}$  & $7.2^{***}$  \\
          & $(1.4)$      & $(1.5)$      & $(1.3)$      & $(1.5)$      \\
Sheriff   & $7.4^{***}$  & $8.8^{***}$  & $4.5^{***}$  & $6.8^{***}$  \\
          & $(1.4)$      & $(1.5)$      & $(1.3)$      & $(1.5)$      \\
\hline
Num. obs. & 19028        & 19028        & 19028        & 19028        \\
\hline
\multicolumn{5}{l}{\scriptsize{$^{***}p<0.01$, $^{**}p<0.05$, $^*p<0.1$. Holdout values in levels; remaining figures relative to this}}
\end{tabular}
\end{table}

We start and consider the partial compliance results that pertain to
the sample in which we exclude owners of multiple properties.  Table
\ref{pc_lin} reports the estimated participation rates in the hold out
sample as well as the differences in the participation in the seven
treatment samples. Robust standard errors are reported in
parentheses. We find that all seven treatment increased partial
compliance at the one month and three month date. Almost all of these
increases in compliance behavior are statistically significant at
standard levels of significance.

After one month, approximately 30 percent of all taxpayers in the
hold-out sample had made some contribution towards their tax
liabilities. In contrast 40 percent of taxpayers that received the
lien letter and 37 percent that received the sheriff sale letters had
made payments. The results are similar after three months of the
intervention.  51 percent of all households in the hold-out sample had
made some contribution after three months. In contrast 61 percent of
households that received the Lien letter and 60 percent of households
that received the Sheriff Sale letter made some payments after three
months.

As shown in Table \ref{pc_lin} the results are qualitatively and
quantitatively the same if we use ``paid in full" as our compliance
outcome. The main difference is that the Amenity, Community and Duty
letters do lead to a significant increase in compliance relative to
the hold-out group. All other findings are similar.

\begin{table}[htbp]
\caption{Short Term Logistic Model Estimates}\label{sh_logit}
\bigskip
\centering
\begin{tabular}{l c c c c }
\hline
 & \multicolumn{2}{c}{Ever Paid} & \multicolumn{2}{c}{Paid in Full} \\
               & One Month & Three Months & One Month & Three Months \\
\hline
Control        & $0.17^{***}$ & $0.16^{***}$ & $0.12^{*}$   & $0.12^{**}$  \\
               & $(0.06)$     & $(0.06)$     & $(0.07)$     & $(0.06)$     \\
Amenities      & $0.08$       & $0.11^{*}$   & $-0.01$      & $0.06$       \\
               & $(0.06)$     & $(0.06)$     & $(0.07)$     & $(0.06)$     \\
Moral          & $0.17^{***}$ & $0.11^{*}$   & $0.07$       & $0.10^{*}$   \\
               & $(0.06)$     & $(0.06)$     & $(0.07)$     & $(0.06)$     \\
Duty           & $0.11^{*}$   & $0.15^{**}$  & $0.04$       & $0.09$       \\
               & $(0.06)$     & $(0.06)$     & $(0.07)$     & $(0.06)$     \\
Peer           & $0.18^{***}$ & $0.14^{**}$  & $0.10$       & $0.14^{**}$  \\
               & $(0.06)$     & $(0.06)$     & $(0.07)$     & $(0.06)$     \\
Lien           & $0.40^{***}$ & $0.37^{***}$ & $0.29^{***}$ & $0.29^{***}$ \\
               & $(0.06)$     & $(0.06)$     & $(0.07)$     & $(0.06)$     \\
Sheriff        & $0.33^{***}$ & $0.36^{***}$ & $0.24^{***}$ & $0.27^{***}$ \\
               & $(0.06)$     & $(0.06)$     & $(0.07)$     & $(0.06)$     \\
\hline
Log Likelihood & -12238.56    & -13026.43    & -10794.78    & -13038.73    \\
Num. obs.      & 19028        & 19028        & 19028        & 19028        \\
\hline
\multicolumn{5}{l}{\scriptsize{$^{***}p<0.01$, $^{**}p<0.05$, $^*p<0.1$}}
\end{tabular}
\end{table}

As a robustness check we also estimated Logit models. Table
\ref{sh_logit} summarizes the estimates and the estimated standard
errors for the three samples that we considered above. We report
robust standard errors that are clustered at the treatment level. The main
findings are qualitatively and quantitatively the same.

Next we conduct some simple of the envelop calculations to assess the
impact of these estimates on revenues. Here focus on the results after
six month. We take the median payment in each subsample and multiply
the median payment with the increase in the compliance probability
reported in Table \ref{pc_lin}. This product can be interpreted as the
impact of each treatment on revenue per letter. To obtain the total
estimated impact we then multiply the impact per letter with the total
number of individuals in the sample. The results are reported in Table
\ref{sh_rev}.  Overall, we find that six of the seven treatments
generated positive revenues for the city.  The three-month impact of
these letters ranged between \$21 for the Amenity letter to
approximately \$76 per the Lien letter. We thus conclude that
receiving a reminder letter improved taxpayer compliance int he short
run, with the lien and sheriff sale letters the most effective.

\begin{table}[htbp]
\caption{Estimated Three Month Impact on Revenue}\label{sh_rev}
\bigskip
\centering
\begin{tabular}{lcc}
  \hline
Treatment & Impact Per Letter & Total Impact \\ 
  \hline
Control & \$32.51 & \$78,634 \\ 
  Amenities & \$22.32 & \$53,287 \\ 
  Moral & \$23.61 & \$57,623 \\ 
  Duty & \$30.05 & \$73,084 \\ 
  Peer & \$28.95 & \$69,955 \\ 
  Lien & \$76.4 & \$185,580 \\ 
  Sheriff & \$73.27 & \$177,020 \\ 
   \hline
\end{tabular}
\end{table}

Another way to determine the revenue implications of our different
treatments is to regress the total amount of revenue raised on the
different indictor variables.  These regressions confirm our estimates
reported in Table \ref{sh_rev}. We find that the average payments in
the hold-out sample were \$323 after one months and \$636 after three
months. All our letters including the control treatment increased
payments at the one and three months level. The two threat letters
were the only two letter that significantly increased revenue
collection. After one month the lien (sheriff) treatment increased
payments by \$90 (69). After three months, the increases are
approximately \$97 per letter for both treatments. We thus conclude
that the estimates reported in Table \ref{sh_rev} are conservative
estimates of the effectiveness of our treatments.


Finally, we conducted a number of robustness checks. Recall that we
randomized the seven treatments at the ownership level. In Table
\ref{sh_logit_rob} we replicate the analysis done above excluding the
hold-out sample. We then estimate the model using the larger sample
that also includes owners of multiple properties. Overall, we find the
results are similar to the ones reported in Table \ref{sh_logit}. If
anything, the treatment effects are stronger in the single owner
sample. We thus conclude that owners of multiple properties are less
likely to respond the kind of nudge strategies explored in this paper.

\begin{table}[htbp]
\caption{Robustness Analysis: Multiple Owners}\label{sh_logit_rob}
\begin{center}
\begin{tabular}{l c c c c }
\hline
 & \multicolumn{2}{c}{All Owners} & \multicolumn{2}{c}{Single-Property Owners} \\
               & One Month & Three Months & One Month & Three Months \\
\hline
Amenities      & $-0.05$      & $-0.03$      & $-0.09$      & $-0.05$      \\
               & $(0.06)$     & $(0.05)$     & $(0.06)$     & $(0.06)$     \\
Moral          & $-0.02$      & $-0.06$      & $0.00$       & $-0.04$      \\
               & $(0.06)$     & $(0.05)$     & $(0.06)$     & $(0.06)$     \\
Duty           & $-0.06$      & $-0.01$      & $-0.06$      & $-0.01$      \\
               & $(0.06)$     & $(0.05)$     & $(0.06)$     & $(0.06)$     \\
Peer           & $0.01$       & $-0.03$      & $0.01$       & $-0.02$      \\
               & $(0.06)$     & $(0.05)$     & $(0.06)$     & $(0.06)$     \\
Lien           & $0.21^{***}$ & $0.20^{***}$ & $0.23^{***}$ & $0.22^{***}$ \\
               & $(0.06)$     & $(0.05)$     & $(0.06)$     & $(0.06)$     \\
Sheriff        & $0.15^{**}$  & $0.19^{***}$ & $0.16^{**}$  & $0.20^{***}$ \\
               & $(0.06)$     & $(0.05)$     & $(0.06)$     & $(0.06)$     \\
\hline
Log Likelihood & -12582.62    & -13167.95    & -10954.22    & -11580.00    \\
Num. obs.      & 19333        & 19333        & 16940        & 16940        \\
\hline
\multicolumn{5}{l}{\scriptsize{$^{***}p<0.001$, $^{**}p<0.05$, $^*p<0.1$}}
\end{tabular}
\end{center}
\end{table}


\subsection{Long Term Impact}

Next we focus on the six month results. Recall that all tardy tax
payers were assigned to a collection agency in the middle of August
2015 and thus received another enforcement activity which largely
consisted of another "threat" treatment.  While our experimental
design is still valid for these outcomes, it is harder to interpret
the findings since other treatments occurred during that time period
such as enforcement activities by collection agencies. In particular,
the city hires two collection agencies. These collections primarily
use phone calls to contact tardy tax payers and threaten them with
penalties and fines to obtain compliance. All taxpayers that have not
paid the taxes are subject to this uniform second treatment by the two
collection agencies.  Our estimates of the longer term treatment
effects thus reflect two treatments: our initial letter treatment plus
the phone calls performed by the collection agencies. Since we were
not able to randomize on the treatment by the collection agencies, we
can only identify the effect of the joint treatments.

\begin{table}
\caption{Long Term Linear Probability Model Estimates}
\begin{center}
\begin{tabular}{l c c c c }
\hline
 & \multicolumn{2}{c}{Ever Paid} & \multicolumn{2}{c}{Paid in Full} \\
          & Six Months & Twelve Months & Six Months & Twelve Months \\
Holdout   & $73.3$ & $65.5$ & $63.2$ & $52.5$ \\
\hline
          & $(0.9)$      & $(1.1)$      & $(1.0)$      & $(1.1)$      \\
Control   & $1.3$        & $-1.4$       & $1.5$        & $-0.7$       \\
          & $(1.3)$      & $(1.4)$      & $(1.4)$      & $(1.5)$      \\
Amenities & $-0.2$       & $-3.1^{**}$  & $-0.0$       & $-2.2$       \\
          & $(1.3)$      & $(1.4)$      & $(1.4)$      & $(1.5)$      \\
Moral     & $0.9$        & $-1.8$       & $1.1$        & $-2.0$       \\
          & $(1.3)$      & $(1.4)$      & $(1.4)$      & $(1.5)$      \\
Duty      & $2.1$        & $-1.6$       & $1.0$        & $-1.9$       \\
          & $(1.3)$      & $(1.4)$      & $(1.4)$      & $(1.5)$      \\
Peer      & $1.3$        & $-1.9$       & $2.3$        & $-1.1$       \\
          & $(1.3)$      & $(1.4)$      & $(1.4)$      & $(1.5)$      \\
Lien      & $3.8^{***}$  & $-0.9$       & $4.8^{***}$  & $-0.7$       \\
          & $(1.3)$      & $(1.4)$      & $(1.4)$      & $(1.5)$      \\
Sheriff   & $3.8^{***}$  & $-0.6$       & $3.0^{**}$   & $-1.1$       \\
          & $(1.3)$      & $(1.4)$      & $(1.4)$      & $(1.5)$      \\
\hline
Num. obs. & 19028        & 19025        & 19028        & 19025        \\
\hline
\multicolumn{5}{l}{\scriptsize{$^{***}p<0.01$, $^{**}p<0.05$, $^*p<0.1$. Holdout values in levels; remaining figures relative to this}}
\end{tabular}
\label{lg_pc_lin}
\end{center}
\end{table}


We start by analyzing the unique owner sample.  Table \ref{lg_pc_lin}
reports the estimated participation rates in the hold out sample as
well as the differences in the participation in the seven treatment
samples after six month of the intervention. Robust standard errors
are reported in parentheses. We find that only two of the treatment
increased partial compliance at the six month date.  We find that 73
percent of households in the hold-out sample made some payments to the
City. The only letters that significantly improve the compliance about
that rate were the lien and sheriff letters, which increased
compliance by 3 to 4 percentage points. The findings are similar using
paid in full as the outcome measure.

To translate participation rates into revenue, we take again the
median payment in each subsample and multiply the median payment with
the increase in the compliance probability reported in Table
\ref{lg_pc_lin}.  The results are reported in Table \ref{lg_rev}. We
find that the six-month impact of these two letters was approximately
\$31 per letter relative to the hold-out sample and \$21 relative to
the control letter.\footnote{Again,we also ran revenue regression as a 
robustness check, but these regressions were inconclusive. Only one 
the coefficient of the Duty treatment was marginally significant which we
explain by some outliers.}

\begin{table}[htbp]
\caption{Estimated Long Term Impact on Revenue} \label{lg_rev}
\bigskip
\centering
\begin{tabular}{lcc}
  \hline
Treatment & Impact Per Letter & Total Impact \\ 
  \hline
Control & \$10.74 & \$25,976 \\ 
  Amenities & -\$1.47 & -\$3,517 \\ 
  Moral & \$7.2 & \$17,568 \\ 
  Duty & \$17.32 & \$42,134 \\ 
  Peer & \$11.16 & \$26,972 \\ 
  Lien & \$31.42 & \$76,328 \\ 
  Sheriff & \$31.43 & \$75,940 \\ 
   \hline
\end{tabular}
\end{table}

We find that the impact of our six letters relative to the hold-out or
control group is much reduced after six month. This finding is
consistent with the view that the enforcement activities of the
collection agencies are probably more closely aligned with our
deterrence letters than with the other five letter that we
explored. As a consequence it is not surprising that we find some
strong convergence in the effectiveness of all treatments after six
months. However, it is also possible that some convergence of the
effectiveness of treatment would have resulted in the absence of the
treatment by the collection agencies. Since we were not allowed to
randomize on that treatment, we cannot offer a definite conclusion.

For revenue collecting agencies, particularly those with revenue
collection problems, acceleration of payment can be understood to be a
useful result in and of itself. Bills must be paid and debts must be
serviced on regular schedules. Relatedly, the longer that tax bills
remain unpaid, the more expensive it becomes to collect. Whether
handled internally or externally through debt collection firms,
downstream collection practices leave diminished revenues. For both of
these reasons, early collection is, \textit{ceteris paribus}, better
collection. This is not to say, however, that early collection is
social welfare improving. If tardy but eventually-compliant tax
delinquents forestall early payment to cover other expenses or invest
in other assets which they eventually use to repay their tax bill with
interest, then the welfare of the tax agency may not be synonymous
with the welfare of society. This is especially true if tax payments
are made weeks rather than months late, such that monthly payments can
still be made based on expected monthly receipts. However, in the case
of persistently-delayed but consistently-paid payments, it is less
obvious that accelerating eventual or inevitable payment constitutes
something of value.

\subsection{Spill-overs Effects into the Next Tax Year}

We also obtained data characterizing tax compliance of tax payers in
the tax year 2016 which followed our intervention which took place in
2015.  Here we investigate the existence of any spill-over effects of
our treatments in the next tax year. Again our analysis is subject to
the same constrained as discussed above.  Focusing again on the unique
owner sample, the next table summarizes our main findings using our
linear probability model.

-- to be continued --

\section{Policy Impact}

Here we want to briefly discuss enforcement activities of the city in
2016 that were motivated by our experimental findings.

-- to be continued --

\section{Conclusions}


We have designed and implemented a new, multi-arm, field experiment
designed to test which low-cost notification strategies increase
property tax payment rates. Our results suggest that 
notification strategies are effective at reducing tax delinquency and
increasing tax revenue collection.  Our different treatments are
motivated by a wide swath of theorized ``nudges" including social
norming, moral suasion, tax morale, and deterrence. Our results
suggest that all treatments are successful in increasing tax
compliance and raising revenue for the city in the short run. Hence, a revenue
director can chose from a menu of effective messages to increase
tax compliance and revenue collection.

There are, however, some important quantitative differences in the 
effectiveness of the different messages, which imply some
trade-offs faced by the revenue director. Credible threats regarding the
foreseeable consequences of nonpayment are much more effective than
other treatments that rely on moral persuasions or appeal to civic
duty. If a revenue director wants to generate more revenue, he or she
needs to chose a tougher and politically more costly message. 

The long run effects of our intervention are harder to assess.
We still find that credible threat notifications increase repayment
rates and revenues although the effects are smaller than the ones
observed in the short run.  All tax payers that were still tardy three
months after our experiment were contacted by a collection agency
that engaged in serious enforcement activities. It is likely that this
additional treatment may explain the convergence in the effectiveness
of our treatments that we observe in the data.

Overall, our findings suggests that increasing repayment at low cost
is possible. More research is needed on the reasons for
different behavioral responses to consequentialist and
non-consequentialist messages. Perhaps, the differences in the
effectiveness of different treatments lie in alternative theories of
why tax delinquents exist in the first place. Non-consequentialist
theories of non-payment implicitly rest on the assumption that
non-compliers are not liquidity-constrained and are merely unaware of
the collective consequences of non-payment. Under this analysis, tax
delinquency is due to discouragement, indifference, lack of
appreciation, or unawareness. Delinquents merely need to be encouraged
or reminded to participate. If tax delinquents are discouraged, providing them with information
about peer behavior, amenities or civic increases the overall
repayment rates the overall rate of tax compliance. This may reflect
the fact that delinquents are not indifferent to their peers' positive
and negative behavior. Likewise, the provision of information on
public goods assumes that recipients have not incorporated
consideration of public goods funded through tax dollars into their
payment behavior.  Our results, however, suggest that this is an
incomplete explanation for tax delinquency.\footnote{We explored
  differences in envelop size as a potential avenue to increase
  revenues. We observe no effect of increased envelope size.
  Apparently, taxpayers open tax bills even if they do not pay them.}

Our results clearly suggest that many tax delinquents are seemingly unaware of the
consequences of nonpayment. Our provision of detailed information
about the collection process had a clear impact on the likelihood of
repayment. This result echoes other recent findings that clear,
consistent, and timely provision of information on consequences,
particularly in the context of compliance monitoring, can lead to
notable improvements in behavioral compliance \citeA{hawken}. Perhaps,
then, the puzzle of high non-payment rates can be understood as a case
of under-enforcement. This possibility is reinforced by the fact that
conditional on any payment, converted defiers became near perfect
compliers, making full payments in almost all cases. This suggests
that at least for the margin affected, liquidity-constraints are not
the primary reason for initial non-payment.

\newpage
{\footnotesize \NoTitleCaseChange\citepunct{(}{and}{, }{; }{, }{)}{}{.} 
\bibliographystyle{theapa}
\bibliography{references}

\end{document}

