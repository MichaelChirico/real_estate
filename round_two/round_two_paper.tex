\documentclass[12pt]{article}
\usepackage{amssymb}
\usepackage{theapa}
\usepackage{titlepage}
\usepackage{pdfpages}
\usepackage{amsmath}
\usepackage{setspace}

\usepackage{rotating}
\usepackage[usenames,dvipsnames]{pstricks}
\usepackage{epsfig}
\usepackage{pst-grad}
\usepackage{pst-plot}
\usepackage{color}
\usepackage{pstricks-add}
\usepackage{threeparttable}
\usepackage{array,multirow}
\usepackage{pdflscape}
\usepackage{float,lscape}
\usepackage{csquotes}
\usepackage{textcomp}
\usepackage{indentfirst}

\renewcommand{\baselinestretch}{1.5}
\parindent=.2in
\evensidemargin=.05in
\oddsidemargin=-.05in
\topmargin=-0.05in
\textwidth=6.5in
\textheight=8in

\newtheorem{fact}{Stylized Fact}
\newtheorem{theorem}{Theorem}
\newtheorem{corollary}{Corollary}
\newtheorem{definition}{Definition}
\newtheorem{lemma}{Lemma}
\newtheorem{prop}{Proposition}
\newtheorem{assumption}{Assumption}
\newtheorem{remark}[theorem]{Remark}
\newtheorem{solution}[theorem]{Solution}
\renewcommand{\thefootnote}{\fnsymbol{footnote}}


\begin{document}

\title{Deterring Delinquency: A Field Experiment in Improving Property Tax Compliance Behavior}

\author{Michael Chirico, Robert Inman, Charles Loeffler, \\
John MacDonald, and Holger Sieg\thanks{We would like to thank Rob
  Dubow, Clarena Tolson, Marisa Waxman, and Darryl Watson in the
  Department of Revenue of the City of Philadelphia for their help and
  support. We thank Kent Smetters and the Wharton Initiative for
  Public Policy for funding this field experiment. We would also like
  to thank Jeff Brown, Stefano DellaVigna, Kai Konrad, Robert Moffitt,
  Jim Poterba, Chris Sanchirico, Wolfgang Sch\"on, Reed Shuldiner and
  participants of numerous seminars for comments and suggestions. The
  views expressed here are those of the authors and do not necessarily
  represent or reflect the views of the City of Philadelphia.}
\\
University of Pennsylvania}

\date{\today}

\maketitle

\begin{abstract}

Municipal governments commonly confront the problem of untimely property
tax collection.  We implemented an experiment in tax collection for
approximately 19,000 tardy taxpayers in the City of Philadelphia for
the fiscal year 2015 payments.  The experiment sent one of seven
reminder letters to tardy taxpayers, testing the efficacy of a simple
reminder, two alternative penalty reminders, and four alternative
intrinsic motivation reminders appealing to the taxpayer's sense of
received economic benefits from neighborhood or city-wide services,
peer behavior, or civic duty. Compliance behaviors are compared to a
holdout sample that received no reminder letter.  The most effective letters
were those that threatened a penalty for continued non-compliance,
with the largest impact on those in the lower two quartiles of taxes
owed.  There is no evidence that reminder letters sent for fiscal year
2015 had any impact on payments in the subsequent fiscal year.

\bigskip

\noindent KEYWORDS: Tax Compliance, Property Taxation, Field
Experiment, Deterrence, Public Service Appeal, Appeal to Civic Duty.

\end{abstract}
\renewcommand{\thefootnote}{\arabic{footnote}}

\newpage

\section{Introduction}

Property taxation is the primary tax for most U.S. cities.  In fiscal
year 2013, 30 percent of all local government revenues and over 73
percent of local taxes came from the property tax
\cite{barnett2013state}.  Yet collection of the tax has, in many
cities, been problematic.  While some U.S. cities do an excellent job
in collecting the tax and receive over 95 percent of assessed revenues
the year the tax is due, other cities have over the last ten years
done significantly worse -- notably Flint (78\%), Cleveland (84\%),
Pittsburgh (86\%), Milwaukee (87\%), Philadelphia (88\%), Detroit
(89\%), and St. Louis (89\%) (see Chirico et al., \citeyear{CILMS-16}.  While Flint, Detroit,
Cleveland and Milwaukee are relatively poor cities, Philadelphia and Pittsburgh are not.
Among the list of cities with outstanding tax collection records are
Buffalo, Birmingham, Houston, and New Orleans.  While city poverty is
important, it cannot be the whole explanation for low rates of
collection.  Poor tax administration is likely to be an important
contributing factor.

This failure to collect the property tax on time creates budget
uncertainty at best and budget deficits at worst. Late payments are costly to the city. If not enforced, delinquent taxpayers may become
permanent tax evaders. Furthermore, significant rates of delinquency today may become
a signal to other taxpayers that avoidance is possible, encouraging further delinquency.\footnote{See Besley et al.,'s \citeyear{besley2015norms} study of local
property taxation in England following imposition of a local head tax as a replacement
for the local property tax. In response to widespread citizen resistance, the poll tax
was removed two years later and the property tax restored. But compliance rates for the
reinstated property tax fell by 14 percent. Though efforts to improve
compliance emphasized high penalties it has taken nearly eighteen years to return to the
original levels of tax compliance.} Yet collecting the
property tax should be straightforward.  In contrast to collecting
self-reported taxes on income, profits, and sales,
property tax obligations equal the city's assigned assessed value
of the property times the city chosen tax rate, and are known by both
the city and the taxpayer.  There is no uncertainty as to what is due,
or when.\footnote{Up until recently, the literature on tax compliance has
  primarily focused on taxpayers reporting of income or sales under the
  threat of a tax audit. See \citeA{Slemrod-07,slemrod2017} for reviews as well as
  recent research of \citeA{Kleven-11} and \citeA{Pomeranz-15}.}
Payment is primarily a matter of enforcement.  The most common
enforcement strategy is the economic stick: fines and penalties.
Failure to pay property taxes in time leads to interest penalties
sufficiently large that there is no arbitrage advantage to waiting,
and perhaps to a significant late fine as well. It is only possible to
avoid payment by abandoning the property, a costly option for most homeowners.

When a delinquent taxpayer does not respond to penalties and fines,
the city can issue a tax lien on the property equal to the value of the
taxes owed and accrued interest and penalties.  A lien does not
impose an immediate direct cost on the taxpayer since payment to the holder
of the tax lien will not occur until the sale of the property.\footnote{
A city can sell a tax lien to a private firm to increase the city's current
revenue collections from delinquent taxes. Selling liens  to ``vulture investors,'' however,
can be politically costly.} The owner of the lien, typically the city,
can start forced sale of the property through a foreclosure process. The home
is then sold at auction with proceeds of the sale used to pay taxes, interest, and
penalties due with any remaining proceeds from the sale returned to the property owner.
In Philadelphia, the foreclosure process typically takes from nine months to a year for completion.

Despite the significant penalties associated with tax delinquency, a
significant number of taxpayers in many cities still do not pay their property taxes
on time. Understanding why taxpayers may be delinquent has concrete policy implications for the design and operation of property tax regimes.
 The motivation for late payment could simply be economic\textemdash the homeowner could be cash constrained\textemdash or behavioral. Cash constrained households may be
helped by being offered payment plans. Households who do not pay on time for behavioral
reasons may need a ``nudge.'' The nudge can be as simple as a gentle reminder that the
tax payment is overdue or a reminder coupled with a more direct appeal to an individualized
motivation for paying taxes. For households who are procrastinators (\citeauthor{DR-99}, 1999) or have limited attentional capacities to attend to all of their obligations (\citeauthor{gabaix_2017}, 2017), a simple reminder may be sufficient \cite{thaler,karlan2016}. But added motivation may be necessary too.
Delinquent taxpayers may need to be reminded both that (i) taxes are due and (ii) a failure to pay
has consequences (see Fellner et al. \citeyear{Fellner-13}; Bergolo et al. \citeyear{bergolo2017tax}; and Meiselman \citeyear{meiselman2017ghostbusting}). Those consequences can be extrinsic as in the classic work
of Allingham and Sandmo \citeyear{Allingham-Sandmo-72} and involves interests and penalties that accrue
with continued 	late payment with the possibility of a lien or foreclosure sale. Or the reminder might appeal to an intrinsic ``tax morale'' motive for paying ones taxes (see Luttmer and Singhal \citeyear{Luttmer-14}), by stressing the
benefits of government services that follow from payment, or the fact that most of one's neighbors pay their taxes, or
that with citizenship comes an obligation to pay taxes to support democratic governance.

Knowing if a  simple reminder is sufficient to significantly improve collection or that a reminder coupled with
a motivational appeal will be needed is important for city tax administrators. Messages that ``work'' can be implemented
to lower the cost of tax administration and as a consequence significantly improve the overall efficiency performance of
city taxation \cite{keen2016optimal}. The recent wave of new research on message design to improve
tax collection has provided useful, but conflicting insights into how tax collection might be improved by direct
communication with taxpayers.\footnote{The focus here is on research designed to increase individual tax compliance. Recent work
directed at increasing tax compliance by firms, in addition to individuals, is fully reviewed in Slemrod \citeyear{slemrod2017}.}
For example, Slemrod, Blumenthal, and Christian \citeyear{slemrod2001taxpayer} find an ``extrinsic'' reminder sent to Minnesota income
taxpayers that their state taxes would be ``closely examined'' significantly increased reported income and compliance over those not receiving a reminder, but a similar controlled experiment also with Minnesota taxpayers by the same researchers (\citeauthor{Blumenthal-01}, 2001) found that ``intrinsic'' motivations that your taxes pay for community services or that most taxpayers truthfully report their taxable income had no statistically significant effect on compliance. \citeauthor{torgler2004moral} \citeyear{torgler2004moral} also found no gains in tax compliance with normative appeals, here for the payment  of income taxes to Swiss local governments. So too for Austrian homeowners in their payment of their annual TV and radio fee and for Detroit residents who had not filed local income tax returns (see Fellner et al. \citeyear{Fellner-13} and Meiselman \citeyear{meiselman2017ghostbusting}). Only letters that stressed the likelihood of detection and fines had an impact on the payment of broadcasting fees by Austrian homeowners \cite{Fellner-13} and only the threat of economic penalties for non-filing improved income tax compliance in Detroit \cite{meiselman2017ghostbusting}. In both experiments, appeals to neighbors' behaviors had no effect on compliance.

Until recently, intrinsic
explanations for tax compliance had a strong theoretical foundation
but scant empirical support in field settings, with most field
experimental studies showing little evidence that appeals to these
motivations did much to explain why the marginal taxpayer (or
non-payer) did or did not pay their taxes. A number of
recent papers, however, have now provided experimental
evidence that intrinsic behavioral motivations may be important.
In their study of UK taxpayers, Hallsworth et al. \citeyear{Hallsworth-17} found that
intrinsic messages that are descriptive (how others behave) rather than injunctive (how one should behave) are more effective for encouraging prompt tax payments as are messages that
relate specifically to the taxpayers' own circumstances ("proximal") rather than to those of a generic taxpayer. Del Carpio \citeyear{delcarpio} in a
study of property taxes, found that inrinsic, norm-based messages significantly improved compliance over that obtained with simple reminders or extrinsic, economic-based messages. Who gets what message also matters. In their study of German citizens payment of their church tax, Dwenger et al. \citeyear{dwenger_2017} found that citizens who already voluntarily pay some of their tax are more receptive to intrinsic appeals to fully pay, while citizens who have always evaded the tax are only responsive to threats of economic penalties. Slemrod's
\citeyear{slemrod2017} recent review identified this seemingly conflicting findings
relative to other work \cite{Blumenthal-01,torgler2004moral,Fellner-13,bergolo2017tax,castro} showing either no effect of intrinsic messages
or the positive effect of extrinsic messages as an open question. Whether these contrasting results are in fact conflicting, the impact of any message will most likely be context specific, dependent not just on the content of the message and th attributes of the taxpayer, but also upon the wider context in which the message occurs, whether it applies to general taxation or the payment of bill, the service being provided, and finally (and particularly for tax morale messages) the social and political environment in which messaging occurs.

To gain new insights, we designed a field experiment to compare the
response of delinquent property owners to a variety of both intrinsic
and extrinsic motivational messages. This paper uses a field experiment involving a sample of 19,000 first-time delinquent Philadelphia taxpayers to examine the effectiveness of seven alternative message strategies for improving property tax compliance: 1) a simple reminder letter that tax payment is late and is now due; 2) an extrinsic reminder letter that payment is late and if taxes are not paid a lien can be placed on the home that will require the payment of all taxes, interests and penalties when the home is sold; 3) an extrinsic reminder that payment is late and if taxes are not paid the home can be sold in a sheriff's sale to pay all taxes, interests and penalties when the home is sold; 4) an intrinsic (tax morale) reminder that resident's taxes pay for neighborhood services with specific (proximal) mention of service's in each taxpayer's own neighborhood; 6) an intrinsic (descriptive, not injunctive) reminder that nine out of ten of the taxpayer's neighbors pay their taxes on time; and finally 7) an intrinsic (injunctive) reminder that the taxpayer, like all Philadelphians, has a civic duty to pay their fair share of city taxes. These letters and a holdout option were randomly assigned across all delinquent taxpayers.

Our experiment supports six conclusions.
First, consistent with O'Donoghue and Rabin \citeyear{DR-99} and Gabaix \citeyear{gabaix_2017} saliency of the tax obligation matters. A simple reminder
letter has a statistically significant effect on compliance when compared to the absence of a reminder.  Second, tardy Philadelphia taxpayers need constant reminders; there is no evidnece that reminders in 2015 improved compliance by these taxpayers in 2016. Third, the content of reminders matters. As in Fellner
\citeyear{Fellner-13}, Bergolo \citeyear{bergolo2017tax}, and
Mieselman \citeyear{meiselman2017ghostbusting} we find a clearly written extrinsic message that explains the unpleasant economic consequences of continued non-payment leads to faster and higher levels of compliance than a simple reminder. Fourth, as in past property tax studies \cite{delcarpio} adding any intrinsic message related to the value of public services, neighbor compliance behavior, or civic duty accelerates payment response but does not significantly increase compliance or revenues above that of a generic reminder. Fifth, some (but not most) of the taxpayers who did respond with payments only partially paid what was owed. Sixth, reminder letters are a very cost-effective tool for increasing city revenues, particularly those stressing economic penalties. Letters with these messages brough in approximately \$65 for each dollar of administrative costs.

The rest of the paper is organized as follows. Section 2 discusses details
of our field experiment including a description of the
treatments and the randomization procedure. Section 3 discusses our
randomization procedure.  Section 4 reports the main empirical
findings. Section 5 discusses the urban fiscal policy implications of
our experiment. Section 6 offers conclusions.


\section{A Field Experiment }

The research setting for the experiment is the City of Philadelphia
for calendar year, 2015.  Notices of property tax payments are sent on
January 1, and the full balance of taxes are due by March 31.  If
payment has not been received by that date, or the taxpayer has not
entered into a tax payment plan with the City, then taxes are
considered tardy and interest and penalties begin to accrue.  On April
1, the City's Department of Revenue (DoR) begins contacting all
taxpayers with unpaid accounts, informing them of taxes due and
accumulated interest and penalties for late payment.  At this time,
the City will normally send two-thirds of the tardy accounts to
outside collection agencies acting as co-counsel for the City. The
outside collection agencies are reimbursed at the rate of six percent
of all their tardy revenues collected by December 31. The remaining
one-third of the tardy accounts remain with the DoR for
collection. All accounts still tardy on December 31 are designated as
``delinquent'' and then assigned to new outside collection
agencies. For the purposes of our experiment the City of Philadelphia
agreed to delay sending any of the tardy accounts to the collection
agencies until August 15, 2015.

Our experiment was implemented with those taxpayers newly tardy on
March 31, 2015. Of the 579,828 properties in the city receiving 2015
tax bills, approximately 100,000 or 17 percent were late in payment as
of April 1. Of these 100,000 properties, 27,264 (owned by 21,468
taxpayers) had tax obligations of more than \$10 as of May 15, 2015,
but had not owed property taxes from prior years. Our experiment
excludes all chronically delinquent taxpayers who owed taxes from
prior years. Of the 21,468 tardy taxpayers, 2,429 taxpayers owned more
than one property. While all 21,468 taxpayers were included in our
experiment, we focus our empirical work on the 19,333 taxpayers who
owned only one property.\footnote{As a robustness check we repeated
  our empirical analysis for the full sample of 21,468 and the results
  are identical those we report in Sections IV and V below. }

Our experiment began with the mailing of our experimental reminder
letters in mid-June, 2015 and continued to December 31, 2015. Prior to
this point in time all property owners had received identical tax
notices in December 2014.  Of the tardy taxpayers with a single
property, 16,940 received a standard or experimental reminder letter
and 2,088 taxpayers did not receive a reminder.\footnote{We estimate
  that 7-8\% of mailed reminders were returned by July.}  This sample
of 2,088 taxpayers became our ``holdout'' sample and the basis for
identifying the importance of saliency in taxpaying behavior. To
ensure that our experiment was not contaminated by other treatments
not under our control, the DoR agreed to postpone all other
enforcement activities until August 15.  In particular, the outside
collection agencies were not allowed to begin their collection efforts
until after that date.  The likely earliest date that those efforts
led to any contact with a taxpayer is September 1.

Each reminder letter was approved by City's DoR to ensure that it
could be understood by a taxpayer with at least a fourth or fifth
grade level of English reading comprehension.  Each letter also
provided contact information for assistance for non-English speaking
taxpayers.  Translation were available for a number of different
languages.\footnote{Templates of the ``reminder only'' and ``lien''
  letters are attached in the appendix.  The full template for the
  other letters are available as an online appendix.}

Each reminder letter in our experiment was drafted to identify a
potential channel that may affect tax payers compliance. For brevity
we present here the important distinguishing feature of each letter.

\bigskip

\noindent \textit{Reminder-only}: \textbf{Our records indicate
that you have a balance due of \textit{balance. }} If you have
already paid, thank you.  If not, please pay now or contact us
to arrange a payment plan.  The fastest and easiest way to pay is
online at  www.phila.gov/pay. Paying by E-check only costs 35 cent
-- less than the cost of a stamp!

\bigskip

The reminder-only letter allows us to identify the potential
importance of tax saliency to taxpayer compliance.\footnote{Our
  experimental design can identify the presence of saliency by an
  increase in compliance for those receiving a reminder letter, but
  time staggered reminder letters at a two-week or monthly interval
  would be needed to identify the actual rate of saliency.  This was
  not possible within the time constraints imposed by DoR on our
  experiment.  }

\bigskip

\noindent \textit{Reminder plus Tax Lien}: Failure to pay your Real
Estate Taxes may result in a tax lien on your property in an amount
equal to your back taxes plus all penalties and interest.  When your
property is sold, those delinquent tax payments will be deducted from
the sale price.  By paying your taxes now, you can avoid these
penalties and interest.  Properties near you in your neighborhood that
have liens placed on them include: $<$ List Three Properties and Sale
Dates $>$ \textbf{Pay your taxes now to avoid a lien being placed on
  your property.  Our records indicate that you have a balance due of
  \textit{balance}.  }

\bigskip

\noindent \textit{Reminder plus Lien and Sheriff's Sale}: Failure to
pay your Real Estate Taxes may result in the sale of your property by
the City in order to collect back taxes.  In the past year we have
sold \textit{N} properties in your neighborhood at a Sheriff's Sale.
Included in these \textit{N} properties are the following properties
near you: $<$List Three Properties and Sale Dates$>$ \textbf{Pay your
  taxes now to prevent the sale of your property.  Our records
  indicate that you have a balance due of \textit{balance}.}

\bigskip

The reminder letter coupled with the threat of a lien, or a lien plus
a sheriff's sale of the taxpayer's home, increase the expected
interest and penalties to the costs of delay -- that is, an increase
in penalties.  Both letters make clear that interest and
penalties are not an empty threat and will be collected by listing
neighborhood properties where these added enforcement measures have
been implemented.  A taxpayer lien for all interest and penalties will
be collected at the future date of home sale, which may be a very
large obligation if the home is sold significantly in the future.  A
lien coupled with a sheriff's sale may occur sooner and thus have
lower accumulated interest and penalties, but the forced sale of one's
home is likely to have very high psychic costs.  Which of the two
added penalties is larger, and therefore likely to have a stronger
impact on compliance, will depend upon the circumstances of the
individual tardy taxpayer.  However, both letters should increase
compliance over the holdout cohort from the reminder effect on
saliency and from the added expected penalty, and both letters should
increase compliance over the reminder-only letter from the added
expected penalty.

Our final four reminder letters test for the potential role of ``tax
morale'' motives for compliance.  An appeal to a tax morale is meant
to cue a possible benefit from having paid one's taxes, apart from the
actual receipt of services those payments may make possible.  In
contrast to user fees, property tax payments are not tied to the
citizen's receipt of particular services during our experimental
period.  In effect, each delinquent taxpayer is a potential free
rider, and the appeal to a tax morale for payment is meant to overcome
such self-interest.

We test for the importance of four such motives: 1) the value of
knowing one is a contributor to the immediate services of one's
neighborhood; 2) the value of knowing one is a contributor to
the wider services that benefit the city as a whole; 3) the
value of knowing one is part of a collective effort with other
taxpayers or ``peers'' in paying for city services; and 4)
the value of knowing one has meet one's obligations as a citizen in a
democracy.  Each of these benefits may motivate taxpayer
compliance, and our reminder letters are meant to trigger a possible
recognition of the importance of each motive.  Some tardy
taxpayers may respond to one motive, some to another, and perhaps
others to none at all if the free-rider motive is decisive.  The four
tax morale reminder letters are:

\bigskip

\noindent \textit{Reminder Plus Appeal to Neighborhood Services}: We
want to remind you that your taxes pay for essential public services
in \textit{neighborhood name}, such as $<$List Two Local Amenities
such as a Park or a Library$>$, your local police officer, snow
removal, street repairs, and trash collection.  \textbf{Please pay
  your taxes to help the city provide these services in your
  neighborhood.} \textbf{Our records indicate that you have a balance
  due of \textit{balance}.}

\bigskip

\noindent \textit{Reminder Plus Appeal to City-Wide Services}: Your
taxes pay for important services that make a city great. Your tax
dollars are essential for ensuring all Philadelphia's children receive
a quality education and all Philadelphians feel safe in their
neighborhoods.  \textbf{Please pay your taxes as soon as you can to
  help us pay for these important services.  Our records indicate that
  you have a balance due of \textit{balance}.}

\bigskip

\noindent \textit{Reminder Plus Appeal to Peer Behavior}: You have not
paid your Real Estate Taxes.  Almost all of your neighbors pay their
fair share: 9 out of 10 Philadelphians do so.  \textbf{By failing to
  pay, you are abusing the good will of your Philadelphia neighbors.
  Our records indicate that you have a balance due of
  \textit{balance}.}

\bigskip

\noindent \textit{Reminder Plus Appeal to Civic Duty}: For democracy
to work, all citizens need to pay their fair share of taxes for
community services.  \textbf{By failing to do so, you are not meeting
  your duty as a citizen of Philadelphia.  Our records indicate that
  you have a balance due of \textit{balance}.}

\bigskip

We take as evidence that an increase in tax morale increases the
likelihood of tax compliance when a tax morale reminder letter
increases the rate of compliance above that of those receiving a
reminder-only letter.  If none of the tax morale letters impact
compliance above a reminder-only letter then, at least on the margin
for paying the property tax, the free-rider motivation is decisive for
tardy Philadelphia taxpayers.  In this case, increased enforcement
will need to appeal to reminders and penalties.


\section{Randomization Procedure}

Randomization took place in two stages.  As a baseline control, we
randomly removed 3,000 tardy properties from the possibility of
receiving any reminder letter at all, representing 2,088 property
owners.  These taxpayers (N=2,088) became our holdout sample and
allowed us to estimate the efficacy of simply communicating with the
taxpayer after the date that taxes are due. We next grouped all
remaining properties by owner and randomized all owners to treatments
based on the total amount of property taxes owed on all of their
properties.

While the vast majority of properties in the city of Philadelphia are
owned by those with just one property, approximately 10 percent of the
properties are owned by individuals or firms that own two or more
properties. Since we are interested in taxpayer compliance and not
property compliance, we identified owners of multiple properties by
their legal name and randomly assigned each owner to a treatment
group.\footnote{We lacked an objective identifier such as a social
  security number.  There is some possibility that two or more different
  owners have the same name, but inspection by the authors found this
  to be very rare.  To the extent that it occurs, we consider this
  random noise to the experiment.} Any tardy taxpayer holding multiple
properties within each treatment group received the same letter for
each of those properties.  Given the high correlation between the
propensity to pay taxes and total debt owed, randomization blocks were
defined according to owner-level total debt to assure uniformity of
samples along the dimension of debt owed. Each property assigned to
receive a reminder letter was equally likely to receive each of the
seven treatments. Since most tardy property owners own only one
property, our main interest in this study will be households that only
own one property in the city. Once we restrict attention to this
sample, we have 16,940 taxpayers in the treatment group and 2,088
taxpayers in the holdout sample.  The total sample size for single
property owners is 19,028. Table \ref{balance} checks whether the
treatment and holdout groups are balanced based on the two most
important variables, taxes due and assessed property value.

Table \ref{balance} shows that randomization was successful in the
single property owner sample.  The average debt owed by each owner was
\$1,287 in the treatment group and \$1,233 in the holdout sample. The
average assessed property value is \$144,145 in the treatment group
and \$142,630 in the holdout group. The average tenure was 15 years
across all groups.  As a further test of our randomization procedure,
we also checked to see whether randomization achieved spatial
uniformity throughout the geographic expanse of the city. As reported
in Table \ref{balance} geographic balance was achieved.

Next we test whether randomization was successful among the seven
experimental treatment groups. Table \ref{balance} shows the results
for the single property owner sample. Overall, we find no evidence
that would suggest any problems with randomization. Results for
multiple property owners, which do not differ from results for single
property owners, are reported in Table \ref{balance2} in the appendix.

\begin{sidewaystable}[htbp]
\centering
\caption{Balance on Observables (Unary Owners)}
\label{balance}
\vspace{10mm}
\begin{tabular}{lrrrrrrrrc}
  \hline
Variable & 1 & 2 & 3 & 4 & 5 & 6 & 7 & 8 & $p$-value \\
   \hline
Amount Due & \$1,233 & \$1,383 & \$1,389 & \$1,613 & \$1,950 & \$1,290 & \$1,338 & \$1,316 & 0.32 \\
  (June) & (\$1,840) & (\$6,510) & (\$4,130) & (\$13,118) & (\$25,290) & (\$2,021) & (\$3,413) & (\$2,158) &  \\
  Assessed Property & \$142 & \$163 & \$147 & \$155 & \$206 & \$130 & \$130 & \$166 & 0.29 \\
  Value & (\$509) & (\$1,316) & (\$699) & (\$966) & (\$2,035) & (\$181) & (\$181) & (\$1,336) &  \\
  Ownership Tenure & 18.7 & 18.7 & 19.0 & 18.6 & 18.5 & 18.8 & 18.9 & 18.9 & 0.96 \\
  Years & (15.6) & (15.2) & (15.7) & (15.5) & (15.7) & (15.6) & (15.6) & (16.0) &  \\
  Center City & 5\% & 5\% & 5\% & 5\% & 5\% & 4\% & 5\% & 5\% & 0.66 \\
  Northeast Philly & 17\% & 18\% & 16\% & 15\% & 17\% & 16\% & 18\% & 16\% &  \\
  North Philly & 22\% & 21\% & 22\% & 22\% & 21\% & 20\% & 22\% & 22\% &  \\
  Northwest Philly & 26\% & 25\% & 27\% & 28\% & 26\% & 27\% & 25\% & 25\% &  \\
  South Philly & 10\% &  9\% & 10\% & 10\% & 10\% & 10\% & 10\% & 10\% &  \\
  West Philly & 21\% & 23\% & 21\% & 21\% & 22\% & 23\% & 20\% & 22\% &  \\
  \# Owners & 2,088 & 2,420 & 2,432 & 2,419 & 2,389 & 2,441 & 2,417 & 2,433 &  \\
  \hline
\multicolumn{10}{l}{\scriptsize{$p$-values in rows 1-2 are $F$-test
    $p$-values from regressing each variable on treatment dummies. A
    $\chi^2$ test was used for the geographic distribution. }} \\
\multicolumn{10}{l}{\scriptsize{ Standard deviations in parentheses. Property values are reported in \$1000 }} \\
\multicolumn{10}{l}{\scriptsize{1: Holdout, 2: Reminder, 3: Lien, 4: Sheriff, 5: Neighborhood, 6: Community, 7: Peer, 8: Duty}} \\
\end{tabular}
\end{sidewaystable}

\section{Empirical Results}

Table \ref{sh_lin} presents our core results for the three month
period of our experiment unaffected by the intervention of the
two outside collection agencies hired by the City to begin their own
enforcement efforts in September, 2015. We consider three distinct
measures of tax compliance behavior. First, did the taxpayer make any
contribution at all towards their tax bill; this is the
\textit{ever-paid} response. Second, did the taxpayer make a full
payment of their tax bill; this is the \textit{paid-in-full}
response. Third, what was the total amount paid by the taxpayer; this is the \textit{total-paid}.  The sample in Table 2  includes only the 19,028 taxpayers
who own a single property.\footnote{We have repeated our analysis for
  the sample of taxpayers, including multi-property owning
  taxpayers. Results for the full sample are identical to those
  reported here for single property owners. We limit our reported
  results here and our discussion to the single property owner sample. For
  comparison, results for the sample with multiple property owners are
  reported in Appendix Tables \ref{sh_lpm_mult} and
  \ref{sh_logit_rob}.} For ease of interpretation, Table \ref{sh_lin}
presents OLS estimates for the linear probability model; logit
estimates are available in Tables \ref{sh_logit} and
\ref{sh_logit_rob} in the appendix and are identical in significance
and interpretation to the OLS results reported here.

The top line of Table \ref{sh_lin} reports the mean rate of compliance
of our holdout sample for \textit{ever-paid} or \textit{paid-in-full}
one month from the starting date of the experiment (July 15) and for
the three months to the ending date of the experiment (September
15). The rate of \textit{ever-paid} compliance for taxpayers in the
holdout sample rises from 30.5 percent after one month to 51.4 percent
after three months; the rate of \textit{paid-in-full} compliance for
the holdout sample raises from 23.5 percent after one month to 40.8
percent after three months.

\begin{table}[htbp]
\caption{Short-Term Linear Probability Model Estimates}
\begin{center}
\begin{tabular}{l c c c c c c }
\hline
 & \multicolumn{2}{c}{Ever Paid} & \multicolumn{2}{c}{Paid in Full} & \multicolumn{2}{c}{Total Paid} \\
\hline
 & One & Three & One & Three & One & Three \\
 & Month & Months & Month & Months & Month & Months \\
\hline
Holdout      & $30.5$ & $51.4$ & $23.5$ & $40.8$ & \$$324.0$ & \$$636.6$ \\
\hline
Reminder     & $3.7^{***}$  & $3.9^{***}$  & $2.2^{*}$    & $3.0^{**}$   & $36.6$        & $15.2$        \\
             & $(1.4)$      & $(1.5)$      & $(1.3)$      & $(1.5)$      & $(31.6)$      & $(43.1)$      \\
Lien         & $9.0^{***}$  & $9.2^{***}$  & $5.7^{***}$  & $7.3^{***}$  & $117.0^{***}$ & $122.7^{**}$  \\
             & $(1.4)$      & $(1.5)$      & $(1.3)$      & $(1.5)$      & $(43.9)$      & $(54.9)$      \\
Sheriff      & $7.3^{***}$  & $8.8^{***}$  & $4.5^{***}$  & $6.7^{***}$  & $68.4^{**}$   & $96.8^{*}$    \\
             & $(1.4)$      & $(1.5)$      & $(1.3)$      & $(1.5)$      & $(34.1)$      & $(49.5)$      \\
Neighbor. & $1.7$        & $2.6^{*}$    & $-0.2$       & $1.6$        & $51.0$        & $40.1$        \\
             & $(1.4)$      & $(1.5)$      & $(1.3)$      & $(1.5)$      & $(37.6)$      & $(48.8)$      \\
Community    & $3.8^{***}$  & $2.8^{*}$    & $1.3$        & $2.5^{*}$    & $41.1$        & $18.3$        \\
             & $(1.4)$      & $(1.5)$      & $(1.3)$      & $(1.5)$      & $(32.6)$      & $(45.1)$      \\
Peer         & $3.9^{***}$  & $3.5^{**}$   & $1.8$        & $3.4^{**}$   & $59.0$        & $119.6$       \\
             & $(1.4)$      & $(1.5)$      & $(1.3)$      & $(1.5)$      & $(36.6)$      & $(76.1)$      \\
Duty         & $2.4^{*}$    & $3.6^{**}$   & $0.7$        & $2.3$        & $35.8$        & $70.7$        \\
             & $(1.4)$      & $(1.5)$      & $(1.3)$      & $(1.5)$      & $(35.6)$      & $(49.2)$      \\
\hline
Num. obs.    & 19039        & 19039        & 19039        & 19039        & 19039         & 19039         \\
\hline
\multicolumn{7}{l}{\scriptsize{$^{***}p<0.01$, $^{**}p<0.05$, $^*p<0.1$. Robust standard errors.}} \\
\multicolumn{7}{l}{\scriptsize{Holdout values in levels; remaining figures relative to this.}}
\end{tabular}
\label{sh_lin}
\end{center}
\end{table}

The next seven rows report the additional impact on compliance from
our seven treatment letters: Reminder-only, Reminder/Lien,
Reminder/Sheriff, Reminder/Neighborhood, Reminder/Community,
Reminder/Peer, and Reminder/Duty.  Receiving the reminder-only letter
increases the rate of compliance after one month for an
\textit{ever-paid} tax payment by 3.7 percent above the holdout's rate
of compliance and by 3.9 percent after three months.  Both effects are
statistically significant at the 99 percent level of confidence.
These estimates for the reminder-only letter indicate the relative
importance of salience and the benefit of simple notification strategies to taxpayer compliance
behavior.  Our letter is particularly effective early in our
experiment, where the pure effect of a reminder increases the rate of
compliance after one month by approximately 12 percent (\= 3.7/30.5).
While receipt of the reminder letter is still effective after three
months, its relative impact on compliance behavior is less, adding an
additional 8 percent (= 3.9/51.4) to the rate of \textit{ever-paid}.
The same statistical significance and declining rate of impact of reminder-only on
compliance is observed for the outcome, \textit{paid-in-full}.  Here
the reminder-only letter increases the one month rate of compliance
over the holdout sample by 2.2 percent on a mean rate of holdout
compliance of 23.5 percent (9.4 percent improvement) and the three
month rate of compliance over the holdout sample by 3.0 percent on a
mean rate of 40.8 percent (7.4 percent improvement). While most of the new taxpayers paid in full\texttt{--}3 percent
compared to the 3.9 percent of all new payers after three months\texttt{--}the additional revenues raised by the
reminder letters over that paid by those with no letter is never significant and is quantitatively very small,
on average only \$15.20 more than the amount paid by the holdout sample after three months.

Adding a more substantive message to the reminder letter produced a
mixed impact on taxpayer compliance, depending on the content of the
message.  Table 2 reports the joint effects of receiving a reminder
and a message.  Of the six messages, only the reminder/lien and
reminder/sheriff letters had a statistically robust \textit{added}
impact on compliance.  After one month, the sample receiving the
reminder/lien letter had an additional 9.0 percent rate of
\textit{ever-paid} compliance over the holdout sample's compliance
rate of 30.5 percent rate (30 percent improvement) and after three
months, an additional 9.2 percent rate of \textit{ever-paid}
compliance over the holdout sample's compliance rate of 51.4 percent
(18 percent improvement).  The impact is statistically significant at
the 99 percent level of confidence.  The results for {\it
  paid-in-full} compliance for the reminder/lien letter are also
quantitatively important and statistically significant, adding 5.7
additional compliance over the holdout sample's one month mean rate of
23.5 percent (24 percent improvement) and an additional 7.3 percent
compliance to holdout sample's three month mean compliance rate of
40.8 percent (18 percent improvement).  Comparable impacts are
observed for the sample receiving the reminder/sheriff letter, where
we observe a 24 percent (=7.3/30.5) improvement in the rate of
ever-paid compliance after one month, a 17 percent (= 8.8/51.4)
improvement in \textit{ever-paid} compliance after three months, a 19
percent (= 4.5/23.5) improvement in \textit{paid-in-full} compliance
after one month, and an 17 percent (= 6.7/40.8) improvement in
compliance after three months. In contrast to the reminder-only
letters, both the lien and the sheriff letters improved revenue
collection. The lien letter increased {\it total paid} by 36 percent
after one month and 21 percent after three month.  The sheriff letter
increased {\it total paid} by 21 percent after one month and 15
percent after three months.

\begin{table}[htbp]
\caption{Short-term Results: Relative to Generic Reminder}
\begin{center}
\begin{tabular}{l c c c c c c }
\hline
 & \multicolumn{2}{c}{Ever Paid} & \multicolumn{2}{c}{Paid in Full} & \multicolumn{2}{c}{Total Paid} \\
\hline
 & One & Three & One & Three & One & Three \\
 & Month & Months & Month & Months & Month & Months \\
\hline
Reminder     & $34.3$ & $55.3$ & $25.7$ & $43.8$ & $360.6$ & $651.8$ \\
\hline
Lien         & $5.3^{***}$  & $5.3^{***}$  & $3.5^{***}$  & $4.2^{***}$  & $80.4^{*}$    & $107.5^{**}$  \\
             & $(1.3)$      & $(1.4)$      & $(1.2)$      & $(1.4)$      & $(41.6)$      & $(45.7)$      \\
Sheriff      & $3.6^{***}$  & $4.9^{***}$  & $2.3^{*}$    & $3.7^{***}$  & $31.8$        & $81.5^{*}$    \\
             & $(1.4)$      & $(1.4)$      & $(1.2)$      & $(1.4)$      & $(27.9)$      & $(42.4)$      \\
Neighborhood & $-2.1$       & $-1.2$       & $-2.5^{**}$  & $-1.5$       & $14.4$        & $24.8$        \\
             & $(1.3)$      & $(1.4)$      & $(1.2)$      & $(1.4)$      & $(34.6)$      & $(40.6)$      \\
Community    & $0.1$        & $-1.0$       & $-0.9$       & $-0.5$       & $4.4$         & $3.0$         \\
             & $(1.3)$      & $(1.4)$      & $(1.2)$      & $(1.4)$      & $(24.4)$      & $(32.3)$      \\
Peer         & $0.1$        & $-0.4$       & $-0.4$       & $0.3$        & $22.4$        & $104.3$       \\
             & $(1.3)$      & $(1.4)$      & $(1.2)$      & $(1.4)$      & $(35.4)$      & $(71.8)$      \\
Duty         & $-1.3$       & $-0.3$       & $-1.6$       & $-0.8$       & $-0.8$        & $55.4$        \\
             & $(1.3)$      & $(1.4)$      & $(1.2)$      & $(1.4)$      & $(32.4)$      & $(39.9)$      \\
\hline
Num. obs.    & 16951        & 16951        & 16951        & 16951        & 16951         & 16951         \\
\hline
\multicolumn{7}{l}{\scriptsize{$^{***}p<0.01$, $^{**}p<0.05$, $^*p<0.1$. Standard errors clustered by block.}} \\
\multicolumn{7}{l}{\scriptsize{Reminder values in levels; remaining figures relative to this.}}
\end{tabular}
\label{sh_lpm_rob}
\end{center}
\end{table}

No consistent improvements in compliance above the reminder-only
letter are observed for those receiving a reminder letter with a ``tax
morale'' message.  This is seen most clearly in Table \ref{sh_lpm_rob}
where we compare compliance in the reminder-only sample to that of the
samples receiving one of the six message letters. In this comparison,
both the reminder/lien and the reminder/sheriff letters stressing the
penalties of noncompliance have statistically significant and policy
relevant additional impacts on compliance above reminder-only, both
for the \textit{ever-paid} and \textit{paid-in-full} outcomes and at
the one month and three month intervals. The lien letter adds more
than a 5 percent increase in the rate of compliance above the
reminder-only letter for \textit{ever-paid} and about 4 percent to the
rate of compliance for \textit{paid-in-full}. These effects represent
a 10 to 15 percent improvement in the rates of compliance over those
obtained with the reminder-only letter.  The sheriff letter also
offers a significant improvement over the reminder-only letter, though
the effects are slightly lower than those obtained with the lien
letter.  Compliance rates for \textit{ever-paid} increase by 3 to 5
percent and for \textit{paid-in-full} by to 2 to 4 percent above those
achieved with the simple reminder.  These effects represent a 9 to 11
percent improvement in compliance performance over what had been
obtained with a reminder only. Table \ref{sh_lpm_rob} also shows most
clearly the inability of the tax morale reminders to induce greater
compliance from Philadelphia's tardy taxpayers.  Among those
reminders, only the neighborhood letter is ever statistically
significant and its effect is negative (!) for those paying in
full.\footnote{\label{fn:nudges}Our results for both the positive
  impact of penalties and mixed effectiveness of tax morale messages
  are consistent with most of the current literature on ``nudges'' and
  tax compliance; see \cite{Hallsworth-14} for a thorough review. However, our pilot study
  \cite{CILMS-16} for this project did find a role for a community or
  duty letter in increasing compliance.  The control group in the
  pilot study received a reminder-only letter.  Three other groups
  received either a penalty letter, a community letter -- your taxes
  pay for city schools, police services, and fire fighters -- or a
  combined peer/duty letter -- 9 out of 10 Philadelphians pay their
  taxes; paying your taxes is your duty. In our pilot the penalty
  letter had no additional effect on compliance over that of the
  reminder-only letter.  The community letter increased the rate of
  compliance above the reminder letter by 4 percent, but the effect
  was not quite statistically significant.  The combined peer/duty
  letter increased rate of compliance above the simple reminder letter
  by 2 percent and the effect was statistically significant at a 95
  percent level of confidence.

It is worth speculating as to why our results here differ from those
in our pilot study.  First, the pilot was run on a much smaller sample
(3,900 single property taxpayers) and thus the results were less
precisely estimated.  Second, and more importantly, the sample for the
pilot study included only taxpayers who had not yet paid by the middle of
November, 2014 (the time of our pilot), and thus are very close to
being what the City will classify as a ``delinquent'' taxpayer as
those who have not paid by December 31 of the tax year.  The sample
therefore consisted of the ``most-tardy'' of tardy taxpayers.  Of
these ``nearly delinquent'' taxpayers who did make a contribution in our
pilot study, the contributions were typically only partial payments of
\$50 to \$150, suggesting these very tardy households may be seriously cash
constrained.  One might then imagine that for this sample of taxpayers penalties are irrelevant; they cannot pay in full in any case.
But a morale nudge might induce some payment in the spirit of a
``charitable contribution.''  Consistent with this possible
explanation is the fact that the average rate of compliance of this
sample over the six weeks of our pilot was only 15 percent and the
moral nudges boosted the rate of those making even some contribution
to no more than 20 percent.  It would be very valuable to design a
larger experiment that seeks a compliance strategy for these very
tardy or delinquent taxpayers.}

Our results are similar in statistical significance and impact to those in
Castro and Scartascini's \citeyear{castro} study of property tax
payments in Junin Argentina, the other major field experiment seeking
to improve property tax collection.  For Philadelphians at least, and
for the residents of Junin, it is reminders and penalties that improve
compliance among tardy property taxpayers.

\begin{table}[htbp]
\caption{Long-Term Linear Model Estimates}
\begin{center}
\begin{tabular}{l c c c c c c }
\hline
 & \multicolumn{3}{c}{Six Months} & \multicolumn{3}{c}{Subsequent Tax Cycle} \\
 & Ever Paid & Paid in Full & Total Paid & Ever Paid & Paid in Full & Total Paid \\
Holdout      & $73.3$ & $63.2$ & $937.9$ & $65.5$ & $52.5$ & $1043.9$ \\
\hline
Reminder     & $1.3$        & $1.5$        & $21.2$        & $-1.4$       & $-0.7$       & $-24.7$        \\
             & $(1.3)$      & $(1.4)$      & $(50.0)$      & $(1.4)$      & $(1.5)$      & $(69.1)$       \\
Lien         & $3.7^{***}$  & $4.8^{***}$  & $87.5$        & $-0.9$       & $-0.7$       & $38.9$         \\
             & $(1.3)$      & $(1.4)$      & $(58.8)$      & $(1.4)$      & $(1.5)$      & $(96.9)$       \\
Sheriff      & $3.7^{***}$  & $2.9^{**}$   & $74.5$        & $-0.6$       & $-1.1$       & $245.8$        \\
             & $(1.3)$      & $(1.4)$      & $(55.9)$      & $(1.4)$      & $(1.5)$      & $(260.6)$      \\
Neighborhood & $-0.2$       & $-0.0$       & $47.6$        & $-3.1^{**}$  & $-2.1$       & $181.3$        \\
             & $(1.3)$      & $(1.4)$      & $(55.3)$      & $(1.4)$      & $(1.5)$      & $(189.6)$      \\
Community    & $0.9$        & $1.1$        & $55.0$        & $-1.8$       & $-2.0$       & $-52.9$        \\
             & $(1.3)$      & $(1.4)$      & $(53.6)$      & $(1.4)$      & $(1.5)$      & $(66.8)$       \\
Peer         & $1.4$        & $2.3$        & $130.0$       & $-1.9$       & $-1.1$       & $-69.0$        \\
             & $(1.3)$      & $(1.4)$      & $(79.5)$      & $(1.4)$      & $(1.5)$      & $(65.9)$       \\
Duty         & $2.1$        & $1.0$        & $120.3^{**}$  & $-1.6$       & $-1.9$       & $37.1$         \\
             & $(1.3)$      & $(1.4)$      & $(57.6)$      & $(1.4)$      & $(1.5)$      & $(70.2)$       \\
\hline
Num. obs.    & 19039        & 19039        & 19039         & 19036        & 19036        & 19036          \\
\hline
\multicolumn{7}{l}{\scriptsize{$^{***}p<0.01$, $^{**}p<0.05$, $^*p<0.1$. Robust standard errors. Holdout values in levels; remaining figures relative to this.}} \\
\multicolumn{7}{l}{\scriptsize{Change in sample size between long-term and subsequent year results reflects property dissolution for three properties.}}
\end{tabular}
\label{ltmpme}
\end{center}
\end{table}

Table \ref{ltmpme} estimates the longer run impacts of our seven nudge
interventions on compliance.  The letters were sent on June 15th and
received soon thereafter.  The first two columns of Table \ref{ltmpme}
show the estimated effects of having received a letter on compliance
six months later, again compared to compliance behavior in our holdout
sample.  Six month responses for those in the holdout sample and in our
seven treatment groups now include the possible influence of the outside
collection agencies on still delinquent taxpayers. We do not know their ``treatment'' strategies. The
effects observed for the six month window therefore predict the impact of our
``pure'' treatments from our June letters interacted with the unknown treatments by the
outside agencies.\footnote{It is our understanding from DOR that their treatment is
a combination of simple reminders and reminders coupled with extrinsic messages stressing
penalties, liens, and perhaps sheriff sales.} Since all delinquent taxpayers now
receive some form of a reminder, it is not surprising that our original reminder
no longer has a differential impact on payment behavior. What does continue to
impact behavior, however, is our original reminders that stressed the risk of liens
and sheriff's sales. The effects of our lien and sheriff reminders are now slightly
smaller in percentage terms, though not significantly so. Again, none of the tax
morale intrinsic nudges show a statistically significant impact on compliance behaviors.
Those taxpayers are now receiving extrinsic reminders for the first time, just like those
in the holdout sample. They appear to respond identically, resulting in no significant
behavioral differences between those in the original holdout sample and in the tax morale
intrinsic motivation samples. This provides further evidence that extrinsic (penalties) messages
are the only effect messages for converting non-payers to payers.

Left unanswered by these results is the question of why taxpayers respond
to extrinsic messages that communicate pre-existing penalty information.
One possible explanation is that taxpayers interpret the threat of enforcement as new
information rather than a reiteration of existing information. The
best evidence to date on this possibility is provided by a survey of risk perception
accompanying Bergolo et al. \citeyear{bergolo2017tax}. They report evidence consistent with the idea that
this new threat information is used to update the recipients perceived
risk of enforcement and punishment.

The last two columns of Table \ref{ltmpme} carry our sample into the
next tax year, beginning with the receipt of a new property tax bill
in early January, 2016, and asks if having received a reminder letter
in June, 2015 improves compliance behavior for the payment of the 2016
taxes by June of 2016.  Consistent with the importance of saliency,
none of the 2015 reminder letters appear to have ``staying power''
into the next tax year.  Tardy Philadelphians need constant
reminders.

\begin{table}[htbp]
\caption{Treatment Effect Heterogeneity by Debt Quantile}
\begin{center}
\begin{tabular}{l c c c c c c }
\hline
 & \multicolumn{3}{c}{Ever Paid} & \multicolumn{3}{c}{Total Paid} \\
\hline
 & One & Three & Six & One & Three & Six \\
 & Month & Months & Months & Month & Months & Months \\
\hline
Holdout in Quartile 1 & $38.1$  & $56.4$  & $74.7$ & $118.0$ & $152.0$  & $184.9$  \\
\hline
Holdout in Quartile 2 & $-9.8^{***}$  & $-11.8^{***}$ & $-7.4^{***}$ & $20.1$        & $97.5^{***}$   & $217.4^{***}$  \\
                      & $(2.9)$       & $(3.1)$       & $(2.8)$      & $(32.3)$      & $(33.6)$       & $(33.5)$       \\
Holdout in Quartile 3 & $-9.9^{***}$  & $-5.2^{*}$    & $-0.1$       & $134.1^{***}$ & $388.8^{***}$  & $658.5^{***}$  \\
                      & $(2.9)$       & $(3.1)$       & $(2.7)$      & $(38.6)$      & $(39.5)$       & $(39.3)$       \\
Holdout in Quartile 4 & $-10.7^{***}$ & $-2.5$        & $2.2$        & $691.1^{***}$ & $1494.0^{***}$ & $2193.8^{***}$ \\
                      & $(2.9)$       & $(3.1)$       & $(2.7)$      & $(92.1)$      & $(126.0)$      & $(129.1)$      \\
Lien in Quartile 1    & $13.5^{***}$  & $9.9^{***}$   & $3.4$        & $14.6$        & $13.4$         & $5.3$          \\
                      & $(2.9)$       & $(2.9)$       & $(2.5)$      & $(38.4)$      & $(38.9)$       & $(38.8)$       \\
Lien in Quartile 2    & $8.9^{***}$   & $13.0^{***}$  & $8.0^{***}$  & $52.7^{***}$  & $68.8^{***}$   & $52.5^{***}$   \\
                      & $(2.8)$       & $(2.9)$       & $(2.7)$      & $(16.4)$      & $(19.4)$       & $(19.0)$       \\
Lien in Quartile 3    & $6.4^{**}$    & $7.2^{**}$    & $-0.3$       & $79.9^{**}$   & $67.3^{*}$     & $-5.2$         \\
                      & $(2.8)$       & $(3.0)$       & $(2.6)$      & $(33.1)$      & $(34.8)$       & $(34.4)$       \\
Lien in Quartile 4    & $7.0^{**}$    & $6.2^{**}$    & $3.5$        & $293.6^{*}$   & $289.4$        & $226.9$        \\
                      & $(2.8)$       & $(3.0)$       & $(2.5)$      & $(163.5)$     & $(199.9)$      & $(204.4)$      \\
Sheriff in Quartile 1 & $10.7^{***}$  & $10.7^{***}$  & $4.9^{*}$    & $3.7$         & $1.2$          & $-2.9$         \\
                      & $(3.0)$       & $(2.9)$       & $(2.5)$      & $(34.4)$      & $(34.6)$       & $(34.7)$       \\
Sheriff in Quartile 2 & $7.4^{***}$   & $10.0^{***}$  & $5.4^{**}$   & $39.2^{**}$   & $50.2^{**}$    & $28.5$         \\
                      & $(2.8)$       & $(3.0)$       & $(2.7)$      & $(16.2)$      & $(19.5)$       & $(19.2)$       \\
Sheriff in Quartile 3 & $5.8^{**}$    & $7.7^{***}$   & $3.0$        & $89.0^{***}$  & $65.6^{*}$     & $13.8$         \\
                      & $(2.8)$       & $(3.0)$       & $(2.5)$      & $(32.4)$      & $(35.1)$       & $(33.8)$       \\
Sheriff in Quartile 4 & $5.1^{*}$     & $6.2^{**}$    & $1.1$        & $114.6$       & $215.6$        & $184.3$        \\
                      & $(2.8)$       & $(3.0)$       & $(2.5)$      & $(123.6)$     & $(177.4)$      & $(191.7)$      \\
\hline
Num. obs.             & 19039         & 19039         & 19039        & 19039         & 19039          & 19039          \\
\hline
\multicolumn{7}{l}{\scriptsize{\parbox{.75\linewidth}{$^{***}p<0.01$, $^{**}p<0.05$, $^*p<0.1$. Holdout values for first quartile in levels; other holdout figures are relative to this and remaining figures are treatment effects for the stated treatment vs. holdout owners in the same quartile.}}}
\end{tabular}
\label{lpm_hetero}
\end{center}
\end{table}

Finally, Table \ref{lpm_hetero} shows the compliance behavior of tardy
taxpayers by the size of their tax bill.  Tardy taxpayers are divided
into four quartiles by taxes owed: Quartile 1 (mean owed = \$149);
Quartile 2 (mean owed = \$597), Quartile 3 (mean owed = \$1,133), and
Quartile 4 (mean owed = \$3,885).  All comparisons are for the
outcome ever paid relative to those in the holdout sample (no reminder
letter) in Quartile 1. (Similar results hold for paid in full.)  Three
conclusions follow.  First, tardy taxpayers owing the least in taxes
are the ones most likely to make a contribution to taxes owed, whether
they receive a reminder letter or not. Second, receiving a lien or
sheriff reminder letter impacts compliance behavior for all levels of
taxes owed, but again, the effects are greatest for the taxpayers who
the owe the least, at least in the short-run. By the end of the
experimental phase of the study (3 months) and continuing until the
end of six months, the effect of reminder and lien reminder letters
was greatest for the second quartile. Third, if enforcement resources
are limited and the tax authority's objective is to maximize revenue collected,
then reminders should be directed at those who owe the most in tardy taxes.

From Table 5, the expected average revenue after six months for each quartile will be
the sum of payments by the holdout sample in that quartile plus the
impact of each letter on payment for that quartile.  For example,
payments after six months by taxpayers in quartile 1 receiving the
lien letter will be \$184.90 + \$5.30 = \$190.20.  For all quartiles,
returns after six months for the lien (sheriff) letter will be
\$190.20 (\$182) for tardy taxpayers in quartile 1, \$269.90
(\$245.90) for tardy taxpayers in quartile 2, \$652.80 (\$672.30) for
tardy taxpayers in quartile 3, and \$2,419.90 (\$2377.3) for those in
quartile 4.

Our results also shed light on the importance of liquidity constraints as a motivation for tardy tax payments. If liquidity constraints are important, then nudges may be insufficient unless accompanied by a way to smooth payments of the original tax obligation. Taxpayer agreements that spread payments over several months (typically, three to six months) without penalty provide for payment smoothing. Each reminder letter included a sentence stressing the availability of taxpayer agreements to help with payments. The results in Tables 2 and 5 suggest, however, that liquidity constraints are not binding for most of our tardy taxpayers and that nudges have a valuable role to play in eliciting full payment. First, from Table 2, most taxpayers who respond positively to a tax nudge after one and three months and who make some payment will pay in full. For all statistically significant nudges, including simple reminder, the percent increase in the ``ever paid'' taxpayers that ``paid in full'' is never less than 60 perent (1 month, simple reminder), typically 75 to 80 percent, and as high as 97 percent (three month, peer reminder). Second, from Table 5, taxpayers in the 1st quartile, who owe the least in taxes, have the highest rate of payment, even without the nudge. Further  when taxpayers respond to a nudge, again the share who ``pay in full'' (not shown in Table 5) is over 70 percent of those who ``ever paid'' for the two strongest nudges, the lien and sheriff letters.

Still approximately thirty percent of those tardy taxpayers who respond to a nudge do not, or cannot, make full payment. One likely explanation for these taxpayers is a liquidity constraint. Offering these incomplete taxpayers a tax payment agreement can ease this constraint. Table \ref{waterrelholdout} shows what fraction of taxpayers signed such an agreement during the tax year of the experiment. All recipients of reminder letters were more likely to
  be in payment agreements than non-recipients (1.3\% versus 0.9\%). However, recipients
  of the sheriff's letter were slightly more likely (2.5\% versus 1.3\%)
  to be in an agreement (See Table A5). As a further check on the role of liquidity constraints, we
  examined whether there was any evidence of substitution from other
  municipal bills towards the property tax payment. We did this by
  calculating the probability that individuals would fail to pay their
  water bills and end up past due on these separate accounts. Receipt
  of any letter slightly increased the odds that recipients would be
  past due on their water bills by the end of six months (1.7\% versus
  1.1\%). However, conditional on receiving a reminder letter (of any
  kind) there was no difference in the probability of being past
  due except for a marginally significant difference for the peer letter.
  We interpret these results as suggesting that while most of those who respond to the reminder and extrinsic message letters within the first and third months of our experiment pay in full and do not appear liquidity constrained, those who do not pay in full do appear to benefit from the availability of tax agreements.

\begin{table}[htb]
\caption{Liquidity Linear Probability Model Estimates}
\begin{center}
\begin{tabular}{l c c }
\hline
 & Payment Agreement & Water Delinquency \\
Holdout      & $0.9$ & $1.1$ \\
\hline
Reminder     & $0.4$       & $0.7^{*}$   \\
             & $(0.4)$     & $(0.4)$     \\
Lien         & $1.0^{***}$ & $0.7^{*}$   \\
             & $(0.4)$     & $(0.4)$     \\
Sheriff      & $1.6^{***}$ & $0.7^{*}$   \\
             & $(0.4)$     & $(0.4)$     \\
Neighborhood & $0.6$       & $1.3^{***}$ \\
             & $(0.4)$     & $(0.4)$     \\
Community    & $0.4$       & $1.0^{**}$  \\
             & $(0.4)$     & $(0.4)$     \\
Peer         & $0.8^{**}$  & $1.3^{***}$ \\
             & $(0.4)$     & $(0.4)$     \\
Duty         & $0.8^{**}$  & $0.9^{**}$  \\
             & $(0.4)$     & $(0.4)$     \\
\hline
Num. obs.    & 19039       & 19039       \\
\hline
\multicolumn{3}{l}{\scriptsize{$^{***}p<0.01$, $^{**}p<0.05$, $^*p<0.1$. Holdout values in levels; remaining figures relative to this}}
\end{tabular}
\label{waterrelholdout}
\end{center}
\end{table}

\section{Tax Revenue Implications}

While of interest as a specification and test of a behavioral theory
of tax compliance, our results are also relevant for city tax
collection policies.  As a strategy for improving collection from
tardy taxpayers, our analysis informs two important policy issues.
First, cities need revenues: Do reminders improve collection, and then
do reminders with a message raise more money than a simple reminder?
Second, in light of the recent municipal fiscal crises and the
potential for an unraveling of citizen commitment to local governance:
Do reminders with a message, and then which message, improve tax
collection as a ``nudge'' to citizen engagement? Table \ref{sh_rev}
provides answers to these two questions.


\begin{table}[htbp]
\centering
\caption{Three Month Impact of Collection ``Nudges''*}
\label{sh_rev}
\begin{tabular}{lcccccc}
  \hline
Treatment & Sample & Total Taxes & New & Revenue/ & New & New \% of Taxes \\
 & Size & Owed & Payers & Letters & Revenues & Paid \\
  \hline
Reminder & 2,420 & \$3.346 M & 94 & \$28.63 & \$69,285 & 2.1 \\
  Lien & 2,432 & \$3.378 M & 223 & \$67.44 & \$164,010 & 4.9 \\
  Sheriff & 2,419 & \$3.902 M & 212 & \$64.36 & \$155,691 & 4.0 \\
  Neighborhood & 2,389 & \$4.658 M & 63 & \$19.44 & \$46,452 & 1.0 \\
  Community & 2,441 & \$3.148 M & 69 & \$20.91 & \$51,049 & 1.6 \\
  Peer & 2,417 & \$3.233 M & 85 & \$25.79 & \$62,332 & 1.9 \\
  Duty & 2,433 & \$3.201 M & 88 & \$26.46 & \$64,369 & 2.0 \\
   \hline
Totals & 16,951 & \$24.866 M & 834 & - & \$613,188 & 2.5 \\
   \hline
\multicolumn{7}{p{1\textwidth}}{\scriptsize* Sample Size is the number of single property tax payers in the treatment group.  Total Taxes Owed is the total taxes owed by single property tax payers in the treatment group. New Payers equals the new payers after three months computed as the estimated increase in rate of compliance of those receiving the letter over those in the holdout sample as reported in Table 2; for example, for the reminder letter the number of new payers equals 94 = 0.039 x 2,420.  Revenue per letter for each treatment equals the median new revenue collected from those who received a treatment letter and made some payment (=\$735/letter) times the three month increase in compliance from each treatment letter; for example for the reminder letter the median estimated revenue per letter equals \$28.63 = 0.039 x \$735.  New revenues for each treatment equals the revenue/letter times the number of single owner properties receiving a treatment letter: for example, for the reminder letter the estimated total new revenues equals \$69,285 = \$28.63 x 2,420.  New \% of Taxes Paid equals New Revenues divided by Total Taxes Owed; for example, for the reminder letter 2.1 = \$69,285 / \$3,345,846.}
\end{tabular}
\end{table}

Listed in Table \ref{sh_rev} are our seven treatments, the sample size
to which each treatment applied and total taxes owed, and then
estimates of the impact of each treatment on the number new payers
three months after receipt of the treatment letter, the average new
revenue received per letter sent, total new revenues collected from
each treatment letter above that paid by the holdout sample, and
finally, the percent of owed taxes paid because of each treatment.

For single property owners, the total number of new taxpayers above
the holdout sample from all reminder letters is 838, an average
increase in the overall rate of compliance from receiving one our
treatment letters of 4.9 percent (838/16,940).  Table \ref{sh_rev}
also provides an estimate of additional revenues raised by each of our
treatment letters and then the total revenue raised from each
treatment group.  From the perspective of the City's Department of
Revenue, our experiment was a good investment of Department resources.
Each letter cost about \$1 to process and send.  Thus estimated
benefit to cost ratios for the seven treatments ranged from a low of
\$19.77 (the Neighborhood letter) to a high of \$67.67 (the Lien
letter).  The approximately \$17,000 spent on our experiment to mail
the 16,940 treatment letters raised \$615,752 in additional city
revenues: an average benefit to cost ratio of 36.3.

Among our seven treatments, our experimental results clearly show the
power of the lien and sheriff letters compared to a simple reminder or
the tax morale nudges.  The number of new taxpayers above the holdout
sample is three to four times larger and the revenue/letter is two to
three times larger with the letters stressing penalties.  As a
consequence, total new revenues (above the holdout sample) from the
penalty letters and new revenues as a share of all taxes owed are
three to four times larger as well.  If we had sent only the lien or
sheriff's letter to the 16,940 taxpayers in our treatment groups we
would have raised \$1.15 million in new revenues rather than \$616,752
-- nearly twice as much.  The paid share of taxes owed would have
risen from our experiment's average of .028 to lien letter only of
.053.

While the seven treatments are effective on the margin and the penalty
letters particularly so, the final column makes clear that at least in
Philadelphia, our treatments will not completely solve the larger
problem of unpaid City property taxes.  The treatments encourage a 3
to 9 percent higher rate of compliance above the holdout sample, and
the typical new taxpayer pays on average about 60 percent of what they
owe.\footnote{The median taxpayer in our sample who pays taxes, pays
  \$738 towards the (average) tax bill of about \$1200, or 60
  percent.}  Thus the contribution towards total taxes owed will range
from a low of 1.5 percent for the neighborhood letter to a maximum of
5.3 percent for the lien letter.  Nudges help, and money is money, but
at least in Philadelphia, they alone will only partially solve the
large problem of tardy and then delinquent tax payments.

\section{Conclusion}

Municipal governments commonly confront the problem of timely property
tax collection.  The failure to collect property taxes on time creates
budget uncertainty at best and budget deficits at worst.  Collecting
the tax should be straightforward.  Both taxpayers and the city know
the amount due, and bills are sent at the beginning of year with
payment due (typically) within three months. Timely payment is
primarily a matter of enforcement.  Yet many cities face the problem
of tardy, and perhaps finally delinquent, tax payment.  We implemented
an experiment in tax collection for all tardy
taxpayers in the City of Philadelphia for the fiscal year 2015
payments.  The experiment sent one of seven reminder letters to tardy
taxpayers, testing the efficacy of a simple reminder, two alternative
penalty ``nudges,'' and four alternative ``nudges'' appealing to the
taxpayer's sense of received economic benefits from neighborhood or
city-wide services, peer behavior, or civic duty. Compliance behaviors
are compared to a holdout sample that received no letter.

Our empirical analysis reached six conclusions.  First, there is
strong evidence that salience is important.  A simple reminder will
improve compliance.  The rate of compliance rose by 4 percent with a
simple reminder above that of our holdout sample that received no
reminder.\footnote{These results are consistent with Akerlof's
  \citeyear{Akerlof-91} work on procrastination.  Taxpayers may have a
  limited capacity to remember and process tax (and benefit)
  information when making their spending and financial decisions.  An
  explicit reminder that brings that information to the fore can
  encourage payment.  In this regard our results are consistent with
  those in Chetty, Looney, and Kroft \citeyear{chetty2009salience} on
  the role of saliency in the payment of sales taxation and the
  results in Bhargava and Manoli \citeyear{Bhargava-15} on the take-up
  rate for welfare benefits.} Second, there is no evidence that a letter
  received in 2015 improved compliance behavior into the next
  tax year, even for those who responded to the letter in 2015.
Third, adding an extrinsic message to the reminder stressing the likelihood of a tax
lien or a sheriff's sale significantly increased the rate of taxpayer compliance and
collected revenues above that obtained with a simple reminder. Fourth, adding any of
our intrinsic messages stressing the value of public services, neighbor compliance behavior,
or civic duty did not significantly increase compliance or revenues above that of the simple
reminder. Fifth, some of the taxpayers who did respond with payments only partially paid
what was owed, and 27 percent of all tardy taxpayers were still in arrears at the end of the
calendar year. Sixth, reminder letters are a very cost-effective tool for increasing city revenues,
particularly those stressing economic penalties. The lien and sheriff sale
reminders brought in approximately \$65 for each dollar of administrative costs.

Several less direct conclusions also bear consideration. First, though effective on the margin, nudges strategies will not entirely solve collection problems. As passive nudges, reminders
leave the compliance decision in the hands of the tardy taxpayer
alone. Though
clearly more costly, both in dollars and perhaps political capital,
more activist enforcement strategies involving audits, larger
financial penalties, and perhaps even public shaming of violators may
be needed for more significant compliance and revenue effects \cite{ortega2015don}. Second, the recent wave of messaging tax experiments have yet to report clear
evidence that intrinsic messaging strategies exceed basic reminder
messages by an economically significant degree, except perhaps over very
short follow-up periods.\footnote{Del Carpio \citeyear{delcarpio}
  reports large improvements from social norm messaging over a control
  of no reminder.  However, this effect of norms falls below
  conventional levels of significance when norms are compared to a
  reminder baseline.} Tax administration will continue to require
  a mix of generic and extrinsic reminders. Finally, while nudges, reminders, and other notification strategies may
yield increased compliance and revenue in a highly cost-effective manner,
possible adverse consequences are always a possible risk. In the
present study, however, the absence of significant adverse effects in
either the year of the experiment or in the following tax cycle suggests that
concerns about crowd out
\cite{frey1997constitution} or cratering effects \cite{slemrod2017} may
be less than feared.


\nocite{dellavigna2012testing}

%{\footnotesize \NoTitleCaseChange\citepunct{(}{and}{, }{; }{, }{)}{}{.}
\newpage

\bibliographystyle{theapa}
\bibliography{references}

\bigskip

\bigskip

\bigskip

\newpage

\begin{appendix}

\section{Appendix: Additional Figures and Tables}

The appendix contains Tables \ref{balance2} and \ref{sh_lpm_mult}
which summarizes additional balance tests and robustness analyses
using all owners (including multiple property owners).Tables
\ref{sh_logit} and \ref{sh_logit_rob} report estimates based on Logit
models for single property owners and single plus multiple property owners. Table \ref{liquidversuscontrol} reports estimates for payment agreements and payment substitution versus the simple reminder baseline.

\setcounter{table}{0}
\renewcommand{\thetable}{A\arabic{table}}


\begin{table}[htbp]
\caption{Robustness Analysis: Relative to Reminder (All Owners)}
\begin{center}
\begin{tabular}{l c c c c }
\hline
 & \multicolumn{2}{c}{Ever Paid} & \multicolumn{2}{c}{Paid in Full} \\
 & One Month & Three Months & One Month & Three Months \\
Reminder     & $34.9$ & $56.4$ & $23.9$ & $41.8$ \\
\hline
Lien         & $4.9^{***}$  & $4.8^{***}$  & $3.4^{***}$  & $4.0^{***}$  \\
             & $(1.3)$      & $(1.3)$      & $(1.2)$      & $(1.3)$      \\
Sheriff      & $3.4^{***}$  & $4.5^{***}$  & $2.3^{**}$   & $3.6^{***}$  \\
             & $(1.3)$      & $(1.3)$      & $(1.2)$      & $(1.3)$      \\
Neighborhood & $-1.0$       & $-0.8$       & $-1.2$       & $-0.4$       \\
             & $(1.3)$      & $(1.3)$      & $(1.2)$      & $(1.3)$      \\
Community    & $-0.4$       & $-1.4$       & $-0.6$       & $-0.2$       \\
             & $(1.3)$      & $(1.3)$      & $(1.2)$      & $(1.3)$      \\
Peer         & $0.4$        & $-0.8$       & $0.5$        & $0.8$        \\
             & $(1.3)$      & $(1.3)$      & $(1.2)$      & $(1.3)$      \\
Duty         & $-1.2$       & $-0.2$       & $-1.0$       & $-0.8$       \\
             & $(1.3)$      & $(1.3)$      & $(1.2)$      & $(1.3)$      \\
\hline
Num. obs.    & 19361        & 19361        & 19361        & 19361        \\
\hline
\multicolumn{5}{l}{\scriptsize{$^{***}p<0.01$, $^{**}p<0.05$, $^*p<0.1$. Reminder values in levels; remaining figures relative to this.}}
\end{tabular}
\label{sh_lpm_mult}
\end{center}
\end{table}

\begin{sidewaystable}[htbp]
\centering
\caption{Balance on Observables}
\label{balance2}
\begin{tabular}{lrrrrrrrc}
\hline
\multicolumn{9}{c}{Single Property Owners} \\
  \hline
Variable & Reminder & Lien & Sheriff & Neighborhood & Community & Peer & Duty & $p$-value \\
   \hline
Amount Due (June) & \$1,383 & \$1,389 & \$1,613 & \$1,950 & \$1,290 & \$1,338 & \$1,316 & 0.38 \\
  Assessed Property Value & \$163,084 & \$147,573 & \$155,597 & \$206,214 & \$130,265 & \$130,936 & \$166,791 & 0.28 \\
  \# Owners & 2,420 & 2,432 & 2,419 & 2,389 & 2,441 & 2,417 & 2,433 & 0.99 \\
  \hline
\multicolumn{9}{c}{Single and Multiple Property Owners} \\
  \hline
Variable & Reminder & Lien & Sheriff & Neighborhood & Community & Peer & Duty & $p$-value \\
   \hline
Amount Due (June) & \$1,847 & \$1,735 & \$1,887 & \$2,209 & \$1,954 & \$1,772 & \$1,700 & 0.78 \\
  Assessed Property Value & \$195,029 & \$173,690 & \$178,556 & \$224,412 & \$220,963 & \$165,957 & \$191,199 & 0.76 \\
  \% with Single Property Owner & 87.5 & 88.0 & 87.5 & 86.4 & 88.3 & 87.4 & 88.0 & 0.45 \\
  \% Overlap with Holdout & 3.72 & 3.47 & 3.29 & 3.80 & 3.47 & 3.58 & 3.47 & 0.96 \\
  \# Properties per Owner & 1.33 & 1.32 & 1.26 & 1.36 & 1.29 & 1.26 & 1.27 & 0.55 \\
  \# Owners & 2,766 & 2,765 & 2,766 & 2,766 & 2,766 & 2,766 & 2,766 & 1 \\
  \hline
\multicolumn{9}{l}{\scriptsize{$p$-values in rows 1-5 are $F$-test $p$-values from regressing each variable on treatment dummies. A $\chi^2$ test was used for the count of owners.}} \\
\end{tabular}
\end{sidewaystable}


\begin{table}[htbp]
\caption{Short-Term Logistic Model Estimates (Single Property Owners)}
\begin{center}
\begin{tabular}{l c c c c }
\hline
 & \multicolumn{2}{c}{Ever Paid} & \multicolumn{2}{c}{Paid in Full} \\
 & One Month & Three Months & One Month & Three Months \\
Holdout        & $-0.8$ & $0.1$       & $-1.2$ & $-0.4$ \\
\hline
Reminder       & $0.2^{***}$  & $0.2^{***}$ & $0.1^{*}$    & $0.1^{**}$   \\
               & $(0.1)$      & $(0.1)$     & $(0.1)$      & $(0.1)$      \\
Lien           & $0.4^{***}$  & $0.4^{***}$ & $0.3^{***}$  & $0.3^{***}$  \\
               & $(0.1)$      & $(0.1)$     & $(0.1)$      & $(0.1)$      \\
Sheriff        & $0.3^{***}$  & $0.4^{***}$ & $0.2^{***}$  & $0.3^{***}$  \\
               & $(0.1)$      & $(0.1)$     & $(0.1)$      & $(0.1)$      \\
Neighborhood   & $0.1$        & $0.1^{*}$   & $-0.0$       & $0.1$        \\
               & $(0.1)$      & $(0.1)$     & $(0.1)$      & $(0.1)$      \\
Community      & $0.2^{***}$  & $0.1^{*}$   & $0.1$        & $0.1^{*}$    \\
               & $(0.1)$      & $(0.1)$     & $(0.1)$      & $(0.1)$      \\
Peer           & $0.2^{***}$  & $0.1^{**}$  & $0.1$        & $0.1^{**}$   \\
               & $(0.1)$      & $(0.1)$     & $(0.1)$      & $(0.1)$      \\
Duty           & $0.1^{*}$    & $0.1^{**}$  & $0.0$        & $0.1$        \\
               & $(0.1)$      & $(0.1)$     & $(0.1)$      & $(0.1)$      \\
\hline
AIC            & 24503.9      & 26086.6     & 21618.1      & 26109.6      \\
BIC            & 24566.7      & 26149.5     & 21680.9      & 26172.5      \\
Log Likelihood & -12243.9     & -13035.3    & -10801.1     & -13046.8     \\
Deviance       & 24487.9      & 26070.6     & 21602.1      & 26093.6      \\
Num. obs.      & 19039        & 19039       & 19039        & 19039        \\
\hline
\multicolumn{5}{l}{\scriptsize{$^{***}p<0.01$, $^{**}p<0.05$, $^*p<0.1$. Holdout values in levels; remaining figures relative to this}}
\end{tabular}
\label{sh_logit}
\end{center}
\end{table}

\begin{table}[htbp]
\caption{Logit Estimates Including Multiple Owners}\label{sh_logit_rob}
\begin{center}
\begin{tabular}{l c c c c }
\hline
 & \multicolumn{2}{c}{All Owners} & \multicolumn{2}{c}{Single Property Owners} \\
 & One Month & Three Months & One Month & Three Months \\
\hline
Lien           & $0.21^{***}$ & $0.20^{***}$ & $0.23^{***}$ & $0.22^{***}$ \\
               & $(0.06)$     & $(0.05)$     & $(0.06)$     & $(0.06)$     \\
Sheriff        & $0.15^{**}$  & $0.19^{***}$ & $0.16^{**}$  & $0.20^{***}$ \\
               & $(0.06)$     & $(0.05)$     & $(0.06)$     & $(0.06)$     \\
Neighborhood   & $-0.05$      & $-0.03$      & $-0.09$      & $-0.05$      \\
               & $(0.06)$     & $(0.05)$     & $(0.06)$     & $(0.06)$     \\
Community      & $-0.02$      & $-0.06$      & $0.00$       & $-0.04$      \\
               & $(0.06)$     & $(0.05)$     & $(0.06)$     & $(0.06)$     \\
Peer           & $0.01$       & $-0.03$      & $0.01$       & $-0.02$      \\
               & $(0.06)$     & $(0.05)$     & $(0.06)$     & $(0.06)$     \\
Duty           & $-0.06$      & $-0.01$      & $-0.06$      & $-0.01$      \\
               & $(0.06)$     & $(0.05)$     & $(0.06)$     & $(0.06)$     \\
\hline
AIC            & 25179.24     & 26349.91     & 21922.44     & 23174.00     \\
BIC            & 25234.33     & 26405.00     & 21976.61     & 23228.16     \\
Log Likelihood & -12582.62    & -13167.95    & -10954.22    & -11580.00    \\
Deviance       & 25165.24     & 26335.91     & 21908.44     & 23160.00     \\
Num. obs.      & 19333        & 19333        & 16940        & 16940        \\
\hline
\multicolumn{5}{l}{\scriptsize{$^{***}p<0.001$, $^{**}p<0.05$, $^*p<0.1$}}
\end{tabular}
\end{center}
\end{table}

\begin{table}[htbp!]
\caption{Six-Month Liquidity Linear Probability Model Estimates}\label{liquidversuscontrol}
\begin{center}
\begin{tabular}{l c c }
\hline
 & \multicolumn{1}{c}{Payment Agreement} & \multicolumn{1}{c}{Water Delinquency} \\
Reminder     & $1.3$ & $1.7$ \\
\hline
Lien         & $0.6$       & $-0.0$      \\
             & $(0.4)$     & $(0.4)$     \\
Sheriff      & $1.2^{***}$ & $0.0$       \\
             & $(0.4)$     & $(0.4)$     \\
Neighborhood & $0.1$       & $0.6$       \\
             & $(0.4)$     & $(0.4)$     \\
Community    & $-0.1$      & $0.4$       \\
             & $(0.4)$     & $(0.4)$     \\
Peer         & $0.4$       & $0.7^{*}$   \\
             & $(0.4)$     & $(0.4)$     \\
Duty         & $0.4$       & $0.2$       \\
             & $(0.4)$     & $(0.4)$     \\
\hline
Num. obs.    & 16951       & 16951       \\
\hline
\multicolumn{3}{l}{\scriptsize{$^{***}p<0.01$, $^{**}p<0.05$, $^*p<0.1$. Reminder values in levels; remaining figures relative to this.}}
\end{tabular}
\end{center}
\end{table}
\pagebreak

\begin{figure}[htbp!]
\begin{center}
\includegraphics[width=6in, height=8.5in]{reminder_generic.pdf}
\end{center}
\end{figure}

\begin{figure}[htbp!]
\begin{center}
\includegraphics[width=6in, height=8.0in]{reminder_lien.pdf}
\end{center}
\end{figure}

\end{appendix}



\end{document}

Money may not be all that matters with tax collection, however.
Voluntarily paying one's taxes on time is a signal that one believes
in what government is trying to do; see Posner
\citeyear{Posner-00}. From the U.S. Colonies' resistance to British
taxation in the 1760's to the boycotts of the apartheid government's
imposition of utility taxes on the residents of Soweto in the 1980's,
refusing to pay one's taxes is a rejection of government's
performance.  In signaling games where there is a cost to
non-compliance, the more who indicate they favor your contrarian
position, the more likely you are to publicly express that position
too; see Lohmann \citeyear{lohmann1994dynamics} and Benabou and Tirole
\citeyear{benabou2011laws}.  In our case, what may have once been a
strong tax compliance outcome can unravel to a new, non-compliance
equilibrium when government no longer performs as needed for a
majority of citizens; see Besley, Jensen, and Persson
\citeyear{besley2015norms}.  Recently, such an unraveling towards a
low compliance equilibrium can be observed in Detroit. The city's rate
of taxpayer compliance for property tax collections fell from a ten
year average of .90 from 2000-2010 to a compliance rate of .68 by 2014
(Chirico, et. al., 2015).  In 2013, 47 percent of Detroit's properties
were classified as delinquent.\footnote{See Reese and Sands
  \citeyear{reese2013no} who conclude from their review of the
  economic and political events leading to the Detroit fiscal crisis
  that ``it is not surprising that many view the social contract
  between property taxpayers and city government as broken.''  (p. 9)
  Another example of this can be seen in the 1990 taxpayer revolt to
  Prime Minister Thatcher's introduction of a local poll (head) tax;
  see Besley, Jensen and Persson \citeyear{besley2015norms}.  The
  regressive poll tax replaced a proportional property tax.  In
  response to widespread citizen resistance, the poll tax was removed
  two years later and the property tax restored.  But compliance rates
  for the restored property tax were 14 percent lower than before: .83
  vs. .97.  Efforts to restore compliance since then have stressed
  high penalties but it has taken nearly eighteen years before
  returning to the original rates of payment.  Expected penalties may
  be no substitute for good governance for ensuring voluntary taxpayer
  compliance.} While nudges help, a high initial value of \textit{V}
reflecting government benefits significantly greater than tax costs
may be the most important determinant of the aggregate rate of
taxpayer compliance and commitment to city government; see Haughwout,
Inman, Craig and Luce \citeyear{haughwout2004local}.
