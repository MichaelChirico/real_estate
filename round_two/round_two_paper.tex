\documentclass[12pt]{article}
\usepackage{amssymb}
\usepackage{theapa}
\usepackage{titlepage}
\usepackage{pdfpages}
\usepackage{amsmath}
\usepackage{setspace}

\usepackage{rotating}
\usepackage[usenames,dvipsnames]{pstricks}
\usepackage{epsfig}
\usepackage{pst-grad}
\usepackage{pst-plot}
\usepackage{color}
\usepackage{pstricks-add}
\usepackage{threeparttable}
\usepackage{array,multirow}
\usepackage{pdflscape}
\usepackage{float,lscape}
\usepackage{csquotes}
\usepackage{textcomp}
\usepackage{indentfirst}

\renewcommand{\baselinestretch}{1.5}
\parindent=.2in
\evensidemargin=.05in 
\oddsidemargin=-.05in 
\topmargin=-0.05in
\textwidth=6.5in 
\textheight=8in

\newtheorem{fact}{Stylized Fact}
\newtheorem{theorem}{Theorem}
\newtheorem{corollary}{Corollary}
\newtheorem{definition}{Definition}
\newtheorem{lemma}{Lemma}
\newtheorem{prop}{Proposition}
\newtheorem{assumption}{Assumption}
\newtheorem{remark}[theorem]{Remark}
\newtheorem{solution}[theorem]{Solution}
\renewcommand{\thefootnote}{\fnsymbol{footnote}}


\begin{document}

\title{Deterring Tax Delinquency in Philadelphia}

\author{Michael Chirico, Robert Inman, Charles Loeffler, \\ 
John MacDonald, and Holger Sieg\thanks{We would like to thank Rob
  Dubow, Clarena Tolson, Marisa Waxman, and Darryl Watson in the
  Department of Revenue of the City of Philadelphia for their help and
  support. The authors have received financial support for this research from
   the Wharton Initiative for Public Policy. We would also like
  to thank Jeff Brown, Stefano DellaVigna, Kai Konrad, Robert Moffitt,
  Jim Poterba, Chris Sanchirico, Wolfgang Sch\"on, Reed Shuldiner and
  participants of numerous seminars for comments and suggestions. The
  views expressed here are those of the authors and do not necessarily
  represent or reflect the views of the City of Philadelphia.}
\\ 
University of Pennsylvania}
	
\date{\today}

\maketitle

\begin{abstract}

Municipal governments commonly confront the problem of tardy or
delinquent property tax payments.  We implement an experiment in tax
collection for tardy taxpayers in the City of
Philadelphia for the fiscal year, 2016.  The experiment sent one of
seven reminder letters to the tardy taxpayers, testing the efficacy of
a simple reminder, two alternative reminders stressing economic
sanctions, and four alternative reminders emphasizing either that
taxpayers� receive neighborhood services or city-wide services for
their tax payments, that most of their neighbors pay their taxes on
time, or that as a citizen in a democracy it is a civic duty to pay
taxes on time.  Compliance behaviors were compared to a holdout sample
that received no reminder letter.  The most effective letters were
those that threatened an economic sanction for continued
non-compliance.  These letters were particularly cost-effective in
raising additional city revenues.  There was no evidence that those
receiving a reminder for fiscal year 2016 improved their tax
compliance behavior in fiscal year 2017.


\bigskip

\noindent KEYWORDS: Tax Compliance, Property Taxation, Field
Experiment, Deterrence, Public Service Appeal, Appeal to Civic Duty.


% JEL H2, H7
\end{abstract}
\renewcommand{\thefootnote}{\arabic{footnote}}

\newpage

\section{Introduction}

Property taxation is the primary tax for most U.S. cities.  In fiscal
year 2013, 30 percent of all local government revenues and over 73
percent of local taxes came from the property tax
\cite{barnett2013state}.  Yet collection of the tax has, in many
cities, been problematic.  While some U.S. cities do an excellent job
in collecting the tax and receive over 95 percent of assessed revenues
the year the tax is due, other cities have over the last ten years
done significantly worse -- notably Flint (78\%), Cleveland (84\%),
Pittsburgh (86\%), Milwaukee (87\%), Philadelphia (88\%), Detroit
(89\%), and St. Louis (89\%).\footnote{For details, see Chirico et
  al., \citeyear{CILMS-16}.}  While Flint, Detroit, Cleveland and
Milwaukee are relatively poor cities, Philadelphia and Pittsburgh are
not.  Among the list of cities with outstanding tax collection records
are Buffalo, Birmingham, Houston, and New Orleans.  While city poverty
is important, it cannot be the whole explanation for low rates of
collection.  Poor tax administration is likely to be an important
contributing factor.

This failure to collect the property tax on time creates budget
uncertainty at best and budget deficits at worst. Late payments are
costly to the city. If not enforced, delinquent taxpayers may become
permanent tax evaders. Furthermore, significant rates of delinquency
today may become a signal to other taxpayers that avoidance is
possible, encouraging further delinquency.\footnote{See \citeA{besley2015norms} study of local property taxation
  in England following imposition of a local head tax as a replacement
  for the local property tax. In response to widespread citizen
  resistance, the poll tax was removed two years later and the
  property tax restored. But compliance rates for the reinstated
  property tax fell by 14 percent. Though efforts to improve
  compliance emphasized high penalties it has taken nearly eighteen
  years to return to the original levels of tax compliance.} Yet
collecting the property tax should be straightforward.  In contrast to
collecting self-reported taxes on income, profits, and sales, property
tax obligations equal the city's assigned assessed value of the
property times the city chosen tax rate, and are known by both the
city and the taxpayer.  There is no uncertainty as to what is due, or
when. Property tax payment is a matter of enforcement.

Following the prescriptions of the economic theory of tax enforcement
as outlined in \citeA{Allingham-Sandmo-72}, the most common
enforcement strategy is the economic stick: fines and penalties.
Failure to pay property taxes in time leads to interest penalties
sufficiently large that there is no arbitrage advantage to waiting,
and perhaps to a significant late fine as well.  When a delinquent
taxpayer does not respond to penalties and fines, the city can issue a
tax lien on the property equal to the value of the taxes owed and
accrued interest and penalties.  A lien does not impose an immediate
direct cost on the taxpayer since payment to the holder of the tax
lien will not occur until the sale of the property.\footnote{A city
  can sell a tax lien to a private firm to increase the city's current
  revenue collections from delinquent taxes. Selling liens to
  ``vulture investors,'' however, can be politically costly.} The
owner of the lien, typically the city, can start forced sale of the
property through a foreclosure process. The home is then sold at
auction with proceeds of the sale used to pay taxes, interest, and
penalties due with any remaining proceeds from the sale returned to
the property owner.  It is only possible to avoid payment by
abandoning the property, a costly option for most homeowners.

Despite the significant penalties associated with late payments many
taxpayers do not pay on time.  Understanding why taxpayers may be
tardy has policy implications for the design and implementation of
property tax administration.  Taxpayer motivation for late payments
may be simply economic -- the homeowner could be cash constrained -- or
behavioral.  Cash constrained households can be helped by offering
payment plans.  Those who do not pay on time for behavioral reasons,
however, may need a ``nudge."  The nudge can be a simple reminder, a
reminder that also stresses the fines and penalties associated with
late payments, or a reminder that appeals to an intrinsic motive for
tax payment such as ``most of your neighbors pay their taxes on time,"
``taxes provide valuable city services," or ``with citizenship comes
an obligation to pay one's taxes."  Reminders that work can be
implemented to improve city tax collection, and if collected revenues
exceed the cost of the reminder provide a valuable additional tool for
efficient tax administration \cite{keen2016optimal}.

Recent empirical research on the use of nudges to improve tax
collections have shown targeted reminders can be effective motivators
for increased taxpayer compliance. But success depends upon the action
being nudged and the exact wording of the reminder.  Most successful
nudges have been applied to truthful reporting of the taxpayer's tax
base, with the reminders stressing the likelihood of a taxpayer audit
and associated economic fines on unreported income or sales; see
\citeA{Kleven-11} and \citeA{Pomeranz-15} specifically, and
\citeA{Alm-92}, \citeA{Hallsworth-14}, and \citeA{slemrod2017} for
general reviews.  Less successful have been reminders that stress an
intrinsic or ``tax morale'' motive for tax reporting or payment; see
\citeA{slemrod2001taxpayer} and \citeA{Luttmer-14} for a review.  That
said, well-targeted tax morale messages have worked, in particular
those that stress compliance behavior by the taxpayers' peers; see
\citeA{dwenger-17} and \citeA{Hallsworth-17}.

Only four published studies have directly evaluated the impact of
nudges on taxpayer compliance for the paying of local government
taxes.  \citeA{Torgler2004moral} finds no increased compliance
associated with tax morale messages for the payment of income taxes to
Swiss local governments. \citeA{dwenger-17} find increased compliance
associated with tax morale messages stressing peer behavior in the
payment of the local church tax in Germany.  \citeA{meiselman-18} finds
improved compliance by ``ghost" filers of Detroit's local income tax
for reminders that stress economic penalties but no improved
compliance for reminders that stress taxes are essential for Detroit's
economic future.  \citeA{castro} study of the payment of local
property tax to Argentine municipalities finds significantly increased
compliance for those receiving reminders stressing economic sanctions.
There were only mixed results for those receiving a tax morale
message, however.  On average, the intrinsic messages stressing peer
behaviors or the public service benefits of paying taxes had no impact
on payment, but behavior differed by whether one had paid taxes in the
past or owned property in, but lived outside, the city.
Interestingly, being reminded that 30 percent of taxpayers did not pay
their taxes reduced the rate of compliance among those who usually pay
their taxes.  But, taxpayers who live outside the city and had
initially lower average rates of compliance increased their rates of
payment in response to both the peer and public service reminders.

The mixed results for the impact of nudges on taxpayer compliance is
an important reminder that context matters.  Nudges that work for one
tax and for one government may not work in other settings.  This fact
is nowhere more clearly evident than in the careful decomposition by
Castro and Scartascini (2015) of the impact of nudges on compliance
for property taxpayers in Argentina.  That there are no general
lessons beyond that that nudges, in some form and in some settings,
can improve compliance is perhaps the empirical literature's most
important lesson.  The implied recommendation is that nudges can be a
useful policy tool for efficient tax administration, but each nudge
must be evaluated and compared in a specific setting of tax
compliance.  

With this conclusion in mind, we report the results for
our evaluation of a policy experiment to increase tax compliance in
Philadelphia for the payment of fiscal year 2016 property taxes.  The
initial sample included all 21,468 taxpayers who were tardy in their tax
payments for that fiscal year.  Taxpayers who received a reminder
letter were mailed one of seven letters: a simple reminder, either of
two reminders stressing either a ``gentle'' or ``strong'' economic
sanction, and one of four reminders stressing a role of taxes in
paying for neighborhood or city-wide services, the payment behavior of
their neighbors, or a civic duty of citizens to pay their taxes. Finally,
a last set of taxpayers received no reminder and served as
our control group. Our main evaluation is based on the sub-sample
of single property owners, i.e. we exclude taxpayers that owned 
multiple properties. There are 19,039 single-property owners in our sample.
2,088 of these taxpayers were assigned to the holdout sample, while 16,951 
were assigned to one of the seven treatment groups.

The evaluation reached six conclusions.  First, a simple reminder
letter had a statistically significant effect on compliance when
compared to our control group who received no reminder.  Second, the
content of the reminder letter matters.  The two letters that stress
the likely economic sanctions of continued tardy payment led to faster
and higher levels compliance than the simple reminder.  Third, adding
an intrinsic message to the reminder, one that stressed the value of
public services, neighbors' compliance, or civic duty did not increase
compliance over receiving only the simple reminder.  Fourth, most of
the taxpayers who did respond with payments, paid their full tax
obligation.  Fifth, reminders were very cost effective on the margin.
Each letter cost one dollar to send and returned on average \$37 in
increased city tax revenues.  The two letters stressing economic
sanctions were the most effective, returning \$65 in extra revenues
for each letter sent.  Sixth, reminders had no staying power.  Having
received a 2016 reminder letter had no effect on the taxpayer�s
likelihood of paying their 2017 property taxes on time.

The rest of the paper is organized as follows. Section 2 discusses
details of our field experiment including a description of the
treatments and the randomization procedure. Section 3 discusses our
randomization procedure.  Section 4 reports the main empirical
findings. Section 5 discusses the urban fiscal policy implications of
our experiment. Section 6 discusses the effectiveness of nudges. Section
7 offers conclusions.


\section{A Field Experiment }

The research setting for the experiment is the City of Philadelphia
for calendar year, 2015, for the payment of property taxes for fiscal 
year 2016.  Notices of property tax payments are sent on
January 1, and the full balance of taxes are due by March 31.  If
payment has not been received by that date, or the taxpayer has not
entered into a tax payment plan with the City, then taxes are
considered tardy and interest and penalties begin to accrue.  On April
1, the City's Department of Revenue (DoR) begins contacting all
taxpayers with unpaid accounts, informing them of taxes due and
accumulated interest and penalties for late payment.  At this time,
the City will normally send two-thirds of the tardy accounts to
outside collection agencies acting as co-counsel for the City. The
outside collection agencies are reimbursed at the rate of six percent
of all their tardy revenues collected by December 31. The remaining
one-third of the tardy accounts remain with the DoR for
collection. All accounts still tardy on December 31 are designated as
``delinquent'' and then assigned to new outside collection
agencies. For the purposes of our experiment the City of Philadelphia
agreed to delay sending any of the tardy accounts to the collection
agencies until August 15, 2015.

Our experiment was implemented with those taxpayers newly tardy on
March 31, 2015. Of the 579,828 properties in the city receiving 2015
tax bills, approximately 100,000 or 17 percent were late in payment as
of April 1. Of these 100,000 properties, 27,264 (owned by 21,468
taxpayers) had tax obligations of more than \$10 as of May 15, 2015,
but had not owed property taxes from prior years. Our experiment
excludes all chronically delinquent taxpayers who owed taxes from
prior years. Of the 21,468 tardy taxpayers, 2,429 taxpayers owned more
than one property. While all 21,468 taxpayers were included in our
experiment, we focus our empirical work on the 19,039 taxpayers who
owned only one property.\footnote{As a robustness check we repeated
  our empirical analysis for the full sample of 21,468 and the results
  are similar those we report in Sections IV and V below.}

Our experiment began with the mailing of our experimental reminder
letters in mid-June, 2015 and continued to December 31, 2015.  Of the
tardy taxpayers with a single property, 16,951 received a reminder
letter and 2,088 taxpayers did not receive a reminder.  This sample of
2,088 taxpayers became our ``holdout'' sample and the basis for
identifying the importance of reminders in taxpaying behavior. To
ensure that our experiment was not contaminated by other treatments
not under our control, the DoR agreed to postpone all other
enforcement activities until August 15.  In particular, the outside
collection agencies were not allowed to begin their collection efforts
until after that date.  The likely earliest date that those efforts
led to any contact with a taxpayer is September 1.

Each reminder letter was approved by City's DoR to ensure that it
could be understood by a taxpayer with at least a fourth or fifth
grade level of English reading comprehension.  Each letter also
provided contact information for assistance for non-English speaking
taxpayers.  Translation were available for a number of different
languages.\footnote{Templates of the ``reminder only'' and ``lien''
  letters are attached in the appendix.  The full template for the
  other letters are available as an online appendix.}

Each reminder letter in our experiment was drafted to identify a
potential channel that may affect taxpayers compliance. For brevity
we present here the important distinguishing feature of each letter.

\bigskip

\noindent \textit{Reminder-only}: \textbf{Our records indicate 
that you have a balance due of \textit{balance. }} If you have 
already paid, thank you.  If not, please pay now or contact us 
to arrange a payment plan.  The fastest and easiest way to pay is 
online at  www.phila.gov/pay. Paying by E-check only costs 35 cent 
-- less than the cost of a stamp!

\bigskip

The reminder-only letter allows us to identify the potential
importance of tax saliency to taxpayer compliance.\footnote{ On the
  potential importance of saliency for individual choices, see
  \cite{DR-99} and \cite{gabaix_2017}.  Our experimental design
  identifies the effect of saliency alone as a trigger for tax
  compliance by estimating the difference in the rate of compliance
  for our holdout sample receiving no letter compared to the sample
  receiving the simple reminder letter only.  Our reminders were
  mailed six months after the initial notification of taxes due, and
  thus could estimate the loss of saliency for this period
  only. Staggered mailings of the simple reminder letter could
  identify the rate of decline in saliency, but this was not possible
  in our experiment because of time constraints imposed by DoR. }
   
\bigskip

\noindent \textit{Reminder plus Tax Lien}: Failure to pay your Real
Estate Taxes may result in a tax lien on your property in an amount
equal to your back taxes plus all penalties and interest.  When your
property is sold, those delinquent tax payments will be deducted from
the sale price.  By paying your taxes now, you can avoid these
penalties and interest.  Properties near you in your neighborhood that
have liens placed on them include: $<$ List Three Properties and Sale
Dates $>$ \textbf{Pay your taxes now to avoid a lien being placed on
  your property.  Our records indicate that you have a balance due of
  \textit{balance}.  }

\bigskip

\noindent \textit{Reminder plus Lien and Sheriff's Sale}: Failure to
pay your Real Estate Taxes may result in the sale of your property by
the City in order to collect back taxes.  In the past year we have
sold \textit{N} properties in your neighborhood at a Sheriff's Sale.
Included in these \textit{N} properties are the following properties
near you: $<$List Three Properties and Sale Dates$>$ \textbf{Pay your
  taxes now to prevent the sale of your property.  Our records
  indicate that you have a balance due of \textit{balance}.}

\bigskip

The reminder letter coupled with the threat of a lien, or a lien plus
a sheriff's sale of the taxpayer's home, increase the expected
interest and penalties to the costs of delay -- that is, an increase
in penalties.  Both letters make clear that interest and
penalties are not an empty threat and will be collected by listing
neighborhood properties where these added enforcement measures have
been implemented.  A taxpayer lien for all interest and penalties will
be collected at the future date of home sale, which may be a very
large obligation if the home is sold significantly in the future.  A
lien coupled with a sheriff's sale may occur sooner and thus have
lower accumulated interest and penalties, but the forced sale of one's
home is likely to have very high psychic costs.  Which of the two
added penalties is larger, and therefore likely to have a stronger
impact on compliance, will depend upon the circumstances of the
individual tardy taxpayer.  However, both letters should increase
compliance over the holdout cohort from (i) the reminder effect on
saliency and (ii) from the added expected penalty, and both letters should
increase compliance over the reminder-only letter from the added
expected penalty.

Our final four reminder letters test for the potential role of ``tax
morale'' motives for compliance.  An appeal to a tax morale is meant
to cue a possible benefit from having paid one's taxes.  In
contrast to user fees, property tax payments are not tied to the
citizen's receipt of particular services during our experimental
period.  In effect, each delinquent taxpayer is a potential free
rider, and the appeal to a tax morale for payment is meant to overcome
such self-interest. 

We test for the importance of four such motives: 1) the value of
knowing one is a contributor to the immediate services of one's
neighborhood; 2) the value of knowing one is a contributor to
the wider services that benefit the city as a whole; 3) the
value of knowing one is part of a collective effort with other
taxpayers or ``peers'' in paying for city services; and 4)
the value of knowing one has meet one's obligations as a citizen in a
democracy.  Each of these benefits may motivate taxpayer
compliance, and our reminder letters are meant to trigger a possible
recognition of the importance of each motive.  Some tardy
taxpayers may respond to one motive, some to another, and perhaps
others to none at all if the free-rider motive is decisive.  The four
tax morale reminder letters are:

\bigskip

\noindent \textit{Reminder Plus Appeal to Neighborhood Services}: We
want to remind you that your taxes pay for essential public services
in \textit{neighborhood name}, such as $<$List Two Local Amenities
such as a Park or a Library$>$, your local police officer, snow
removal, street repairs, and trash collection.  \textbf{Please pay
  your taxes to help the city provide these services in your
  neighborhood.} \textbf{Our records indicate that you have a balance
  due of \textit{balance}.}

\bigskip

\noindent \textit{Reminder Plus Appeal to City-Wide Services}: Your
taxes pay for important services that make a city great. Your tax
dollars are essential for ensuring all Philadelphia's children receive
a quality education and all Philadelphians feel safe in their
neighborhoods.  \textbf{Please pay your taxes as soon as you can to
  help us pay for these important services.  Our records indicate that
  you have a balance due of \textit{balance}.}

\bigskip

\noindent \textit{Reminder Plus Appeal to Peer Behavior}: You have not
paid your Real Estate Taxes.  Almost all of your neighbors pay their
fair share: 9 out of 10 Philadelphians do so.  \textbf{By failing to
  pay, you are abusing the good will of your Philadelphia neighbors.
  Our records indicate that you have a balance due of
  \textit{balance}.}

\bigskip

\noindent \textit{Reminder Plus Appeal to Civic Duty}: For democracy
to work, all citizens need to pay their fair share of taxes for
community services.  \textbf{By failing to do so, you are not meeting
  your duty as a citizen of Philadelphia.  Our records indicate that
  you have a balance due of \textit{balance}.}

\bigskip

We take as evidence that an increase in tax morale increases the
likelihood of tax compliance when a tax morale reminder letter
increases the rate of compliance above that of those receiving a
reminder-only letter.  If none of the tax morale letters impact
compliance above a reminder-only letter then, at least on the margin
for paying the property tax, the free-rider motivation is decisive for
tardy Philadelphia taxpayers.  In this case, increased enforcement
will need to appeal to reminders and penalties.

  
\section{Randomization Procedure}

Randomization took place in two stages.  As a baseline control, we
randomly removed 3,000 out of 27,264 tardy properties from the possibility of
receiving any reminder letter at all.  Taxpayers that owned one
of these properties  became our holdout sample. Hence, we can
estimate the efficacy of simply communicating with the
taxpayer after the date that taxes are due. We next grouped all
remaining properties by owner and randomized all owners to treatments
based on the total amount of property taxes owed on all of their
properties.

While the vast majority of properties in the city of Philadelphia are
owned by those with just one property, approximately 10 percent of the
properties are owned by individuals or firms that own two or more
properties. Since we are interested in taxpayer compliance and not
property compliance, we identified owners of multiple properties by
their legal name and randomly assigned each owner to a treatment
group.\footnote{We lacked an objective identifier such as a social
  security number.  There is some possibility that two or more different
  owners have the same name, but inspection by the authors found this
  to be very rare.  To the extent that it occurs, we consider this
  random noise to the experiment.} Any tardy taxpayer holding multiple
properties within each treatment group received the same letter for
each of those properties.  Given the high correlation between the
propensity to pay taxes and total debt owed, randomization blocks were
defined according to owner-level total debt to assure uniformity of
samples along the dimension of debt owed. Each property assigned to
receive a reminder letter was equally likely to receive each of the
seven treatments. 

Since most tardy property owners own only one
property, our main interest in this study will be households that only
own one property in the city. Once we restrict attention to the 19,039 
single property owners. We have 16,951 taxpayers in the treatment group and 2,088
taxpayers in the holdout sample.

Table \ref{balance} checks whether the
treatment and holdout groups are balanced based on the two most
important variables, taxes due and assessed property value.
Table \ref{balance} shows that randomization was successful in the
single property owner sample.  The average debt owed by each owner was
\$1,287 in the treatment group and \$1,233 in the holdout sample. The
average assessed property value is \$144,145 in the treatment group
and \$142,630 in the holdout group. The average tenure was 15 years
across all groups.  As a further test of our randomization procedure,
we also checked to see whether randomization achieved spatial
uniformity throughout the geographic expanse of the city. As reported
in Table \ref{balance} geographic balance was achieved. Overall, we find no evidence
that would suggest any problems with randomization.

\begin{sidewaystable}[htbp]
\centering
\caption{Balance on Observables}
\label{balance}
\vspace{10mm}
\begin{tabular}{lrrrrrrrrc}
\hline 
Variable & 1 & 2 & 3 & 4 & 5 & 6 & 7 & 8 & $p$-value \\
 \hline 
 Amount Due & \$1,233 & \$1,383 & \$1,389 & \$1,613 & \$1,950 & \$1,290 & \$1,338 & \$1,316 & 0.32 \\ 
 & (\$1,840) & (\$6,510) & (\$4,130) & (\$13,118) & (\$25,290) & (\$2,021) & (\$3,413) & (\$2,158) & \\ 
Prop. Value & \$142 & \$163 & \$147 &
\$155 & \$206 & \$130 & \$130 & \$166 & 0.29 \\ 
 & (\$509) & (\$1,316) & (\$699) & (\$966) & (\$2,035) & (\$181) & (\$181) & (\$1,336) & \\ 
Years Tenure & 18.7 & 18.7 & 19.0 & 18.6 & 18.5 & 18.8 & 18.9 & 18.9 & 0.96 \\  
& (15.6) & (15.2) & (15.7) & (15.5) & (15.7) & (15.6) & (15.6) & (16.0) & \\
 Center City & 5\% & 5\% & 5\% & 5\% &
5\% & 4\% & 5\% & 5\% & 0.66 \\ Northeast Philly & 17\% & 18\% &
16\% & 15\% & 17\% & 16\% & 18\% & 16\% & \\ North Philly & 22\%
& 21\% & 22\% & 22\% & 21\% & 20\% & 22\% & 22\% & \\ Northwest
Philly & 26\% & 25\% & 27\% & 28\% & 26\% & 27\% & 25\% & 25\% &
\\ South Philly & 10\% & 9\% & 10\% & 10\% & 10\% & 10\% & 10\%
& 10\% & \\ 
West Philly & 21\% & 23\% & 21\% & 21\% & 22\% &
23\% & 20\% & 22\% & \\ \# Owners & 2,088 & 2,420 & 2,432 & 2,419 &
2,389 & 2,441 & 2,417 & 2,433 & \\ \hline
\multicolumn{10}{l}{\scriptsize{$p$-values in rows 1-2 are $F$-test
    $p$-values from regressing each variable on treatment dummies. A
    $\chi^2$ test was used for the geographic distribution.}} \\
    \multicolumn{10}{l}{\scriptsize{ Standard deviations in parentheses. Property values are reported in \$1000. }} \\
 \multicolumn{10}{l}{\scriptsize{1: Holdout, 2 : Reminder, 3: Lien, 4:  Sheriff,
5: Neighborhood,  6: Community, 7: Peer, 8:  Duty}} \\
\end{tabular}
\end{sidewaystable}


\section{Empirical Results}

Table \ref{sh_lin} presents our core results for the three month
period of our experiment unaffected by the intervention of the two
outside collection agencies hired by the City to begin their own
enforcement efforts in September, 2015. We consider three distinct
measures of tax compliance behavior. First, did the taxpayer make any
contribution at all towards their tax bill; this is the
\textit{ever-paid} response. Second, did the taxpayer make a full
payment of their tax bill; this is the \textit{paid-in-full}
response. Third, what was the total amount paid by the taxpayer; this
is the \textit{total-paid}.  The sample in Table 2 includes only the
19,039 taxpayers who own a single property. For ease of interpretation, Table
\ref{sh_lin} presents OLS estimates for the linear probability model.\footnote{
logit estimates are available upon request and are identical in significance
and interpretation to the OLS results reported here.}

The top line of Table \ref{sh_lin} reports the mean rate of compliance
of our holdout sample for \textit{ever-paid} or \textit{paid-in-full}
one month from the starting date of the experiment (July 15) and for
the three months to the ending date of the experiment (September
15). The rate of \textit{ever-paid} compliance for taxpayers in the
holdout sample rises from 30.5 percent after one month to 51.4 percent
after three months; the rate of \textit{paid-in-full} compliance for
the holdout sample raises from 23.5 percent after one month to 40.8
percent after three months. 

\begin{table}[htbp]
\caption{Short-Term Linear Probability Model Estimates}\label{sh_lin}
\begin{center}
\begin{tabular}{l c c c c c c }
\hline
 & \multicolumn{2}{c}{Ever Paid} & \multicolumn{2}{c}{Paid in Full} & \multicolumn{2}{c}{Total Paid} \\
  \hline
 & One  & Three  & One & Three  & One & Three \\
 & Month & Months & Month & Months & Month & Months \\
 \hline
 Holdout      & $30.5$ & $51.4$ & $23.5$ & $40.8$ & \$$324.0$ & \$$636.6$ \\
\hline
Reminder     & $3.7^{***}$  & $3.9^{***}$  & $2.2^{*}$    & $3.0^{**}$   & $36.6$        & $15.2$        \\
             & $(1.4)$      & $(1.5)$      & $(1.3)$      & $(1.5)$      & $(31.6)$      & $(43.1)$      \\
Lien         & $9.0^{***}$  & $9.2^{***}$  & $5.7^{***}$  & $7.3^{***}$  & $117.0^{***}$ & $122.7^{**}$  \\
             & $(1.4)$      & $(1.5)$      & $(1.3)$      & $(1.5)$      & $(43.9)$      & $(54.9)$      \\
Sheriff      & $7.3^{***}$  & $8.8^{***}$  & $4.5^{***}$  & $6.7^{***}$  & $68.4^{**}$   & $96.8^{*}$    \\
             & $(1.4)$      & $(1.5)$      & $(1.3)$      & $(1.5)$      & $(34.1)$      & $(49.5)$      \\
Neighbor. & $1.7$        & $2.6^{*}$    & $-0.2$       & $1.6$        & $51.0$        & $40.1$        \\
             & $(1.4)$      & $(1.5)$      & $(1.3)$      & $(1.5)$      & $(37.6)$      & $(48.8)$      \\
Community    & $3.8^{***}$  & $2.8^{*}$    & $1.3$        & $2.5^{*}$    & $41.1$        & $18.3$        \\
             & $(1.4)$      & $(1.5)$      & $(1.3)$      & $(1.5)$      & $(32.6)$      & $(45.1)$      \\
Peer         & $3.9^{***}$  & $3.5^{**}$   & $1.8$        & $3.4^{**}$   & $59.0$        & $119.6$       \\
             & $(1.4)$      & $(1.5)$      & $(1.3)$      & $(1.5)$      & $(36.6)$      & $(76.1)$      \\
Duty         & $2.4^{*}$    & $3.6^{**}$   & $0.7$        & $2.3$        & $35.8$        & $70.7$        \\
             & $(1.4)$      & $(1.5)$      & $(1.3)$      & $(1.5)$      & $(35.6)$      & $(49.2)$      \\
\hline
Num. obs.    & 19039        & 19039        & 19039        & 19039        & 19039         & 19039         \\
\hline
\multicolumn{7}{l}{\scriptsize{$^{***}p<0.01$, $^{**}p<0.05$, $^*p<0.1$. Robust standard errors.}}  \\
\multicolumn{7}{l}{\scriptsize{Holdout values in levels; remaining figures relative to this.}}
\end{tabular}
\end{center}
\end{table}

The next seven rows report the additional impact on compliance from
our seven treatment letters: Reminder-only, Reminder/Lien,
Reminder/Sheriff, Reminder/Neighborhood, Reminder/Community,
Reminder/Peer, and Reminder/Duty.  Receiving the reminder-only letter
increases the rate of compliance after one month for an
\textit{ever-paid} tax payment by 3.7 percent above the holdout's rate
of compliance and by 3.9 percent after three months.  Both effects are
statistically significant at the 99 percent level of confidence.
These estimates for the reminder-only letter indicate the relative
importance of salience and the benefit of simple notification
strategies to taxpayer compliance behavior.\footnote{For evidence from
  other settings that saliency matters and reminders have significant
  impacts in inducing appropriate behaviors, see
  \citeA{Thaler-Sunstein-03} and \citeA{Karlan-16}. For evidence that
  simple reminders matter for the payment of local taxes, see
  \citeA{DelCarpio-13} and for the payment of local fines see
  \citeA{Heffetz-16}.}

Our letter is particularly effective early in our experiment, where
the pure effect of a reminder increases the rate of compliance after
one month by approximately 12 percent (= 3.7/30.5).  While receipt of
the reminder letter is still effective after three months, its
relative impact on compliance behavior is less, adding an additional 8
percent (= 3.9/51.4) to the rate of \textit{ever-paid}.  The same
statistical significance and declining rate of impact of reminder-only
on compliance is observed for the outcome, \textit{paid-in-full}.
Here the reminder-only letter increases the one month rate of
compliance over the holdout sample by 2.2 percent on a mean rate of
holdout compliance of 23.5 percent (9.4 percent improvement) and the
three month rate of compliance over the holdout sample by 3.0 percent
on a mean rate of 40.8 percent (7.4 percent improvement). While most
of the new taxpayers paid in full -- 3 percent compared to the 3.9
percent of all new payers after three months -- the additional
revenues raised by the reminder letters over that paid by those with
no letter is never significant and is quantitatively very small, on
average only \$15.20 more than the amount paid by the holdout sample
after three months.


\noindent WHY ARE WE LOSING 3 OBSERVATIONS WHEN WE LOOK AT THE SUBSEQUENT TAX CYCLE? NEED TO EXPLAIN!

\begin{table}[htbp]
\caption{Long-Term Linear Model Estimates}\label{lt_lin}
\begin{center}
\begin{tabular}{l c c c c c c }
\hline
 & \multicolumn{3}{c}{Six Months} & \multicolumn{3}{c}{Subsequent Tax Cycle} \\
 & Ever Paid & Paid in Full & Total Paid & Ever Paid & Paid in Full & Total Paid \\
Holdout      & $73.3$ & $63.2$ & $937.9$ & $65.5$ & $52.5$ & $1043.9$ \\
\hline
Reminder     & $1.3$        & $1.5$        & $21.2$        & $-1.4$       & $-0.7$       & $-24.7$        \\
             & $(1.3)$      & $(1.4)$      & $(50.0)$      & $(1.4)$      & $(1.5)$      & $(69.1)$       \\
Lien         & $3.7^{***}$  & $4.8^{***}$  & $87.5$        & $-0.9$       & $-0.7$       & $38.9$         \\
             & $(1.3)$      & $(1.4)$      & $(58.8)$      & $(1.4)$      & $(1.5)$      & $(96.9)$       \\
Sheriff      & $3.7^{***}$  & $2.9^{**}$   & $74.5$        & $-0.6$       & $-1.1$       & $245.8$        \\
             & $(1.3)$      & $(1.4)$      & $(55.9)$      & $(1.4)$      & $(1.5)$      & $(260.6)$      \\
Neighborhood & $-0.2$       & $-0.0$       & $47.6$        & $-3.1^{**}$  & $-2.1$       & $181.3$        \\
             & $(1.3)$      & $(1.4)$      & $(55.3)$      & $(1.4)$      & $(1.5)$      & $(189.6)$      \\
Community    & $0.9$        & $1.1$        & $55.0$        & $-1.8$       & $-2.0$       & $-52.9$        \\
             & $(1.3)$      & $(1.4)$      & $(53.6)$      & $(1.4)$      & $(1.5)$      & $(66.8)$       \\
Peer         & $1.4$        & $2.3$        & $130.0$       & $-1.9$       & $-1.1$       & $-69.0$        \\
             & $(1.3)$      & $(1.4)$      & $(79.5)$      & $(1.4)$      & $(1.5)$      & $(65.9)$       \\
Duty         & $2.1$        & $1.0$        & $120.3^{**}$  & $-1.6$       & $-1.9$       & $37.1$         \\
             & $(1.3)$      & $(1.4)$      & $(57.6)$      & $(1.4)$      & $(1.5)$      & $(70.2)$       \\
\hline
Num. obs.    & 19039        & 19039        & 19039         & 19036        & 19036        & 19036          \\
\hline
\multicolumn{7}{l}{\scriptsize{$^{***}p<0.01$, $^{**}p<0.05$,
    $^*p<0.1$. Robust standard errors. Holdout values in levels;
    remaining figures relative to this.}} \\
\multicolumn{7}{l}{\scriptsize{Change in sample size between long-term
    and subsequent year results reflects property dissolution for
    three properties.}}
\end{tabular}
\end{center}
\end{table}

Table \ref{lt_lin} estimates the longer run impacts of our seven nudge
interventions on compliance.  Our reminder letters were sent on June
15th and received soon thereafter.  The first three columns of Table
\ref{lt_lin} show the estimated effects of having received a letter on
compliance six months later, again compared to compliance behavior in
our holdout sample.  Six month responses for those in the holdout
sample and in our seven treatment groups now include the possible
influence of the outside collection agencies on still delinquent
taxpayers. We do not know their ``treatment'' strategies. The effects
observed for the six month window therefore predict the impact of our
``pure'' treatments from our June letters interacted with the unknown
treatments by the outside agencies. Since all tardy taxpayers
including our holdout sample now receive some form of a reminder, it
is not surprising that our original reminder letter no longer has a
differential impact on payment behavior. What does continue to impact
behavior, however, is our original reminders that stressed the risk of
liens and sheriff's sales. The effects of our lien and sheriff
reminders are now slightly smaller in percentage terms, though not
significantly so. Again, none of the tax morale intrinsic nudges show
a statistically significant impact on compliance behaviors.  Those
taxpayers are now receiving extrinsic reminders for the first time,
just like those in the holdout sample. They appear to respond
identically, resulting in no significant behavioral differences
between those in the original holdout sample and in the tax morale
intrinsic motivation samples.\footnote{It is our understanding from
  DOR that their treatments are a combination of simple reminders and
  reminders coupled with extrinsic messages stressing penalties,
  liens, and perhaps sheriff sales.}  This provides further evidence
that extrinsic (penalties) messages are the only effective messages for
converting non-payers to payers.

Left unanswered by these results is the question of why taxpayers
respond to extrinsic messages that communicate pre-existing penalty
information.  One possible explanation is that taxpayers interpret the
threat of enforcement as new information rather than a reiteration of
existing information. The best evidence to date of this possibility is
provided by a survey of risk perception accompanying Bergolo et
al. \citeyear{bergolo2017tax}. They report evidence consistent with
the idea that this new threat information is used to update the
recipients perceived risk of enforcement and punishment.

The last three columns of Table \ref{lt_lin} carry our sample into the
next tax year, beginning with the receipt of a new property tax bill
in early January, 2016, and asks if having received a reminder letter
in June, 2015 improves compliance behavior for the payment of the 2016
taxes by June of 2016.  Consistent with the importance of saliency,
none of the 2015 reminder letters appear to have ``staying power''
into the next tax year.  Tardy Philadelphians need constant
reminders. 

\noindent TABLE \ref{lpm_hetero}  NEEDS TO BE UPDATED TO INCLUDE OTHER 5 TREATMENTS

\begin{table}[htbp]
\caption{Treatment Effect Heterogeneity by Debt Quantile}
\begin{center}
\begin{tabular}{l c c c c c c }
\hline
 & \multicolumn{3}{c}{Ever Paid} & \multicolumn{3}{c}{Total Paid} \\
   \hline
   & One  & Three  & Six & One & Three  & Six \\
 & Month & Months & Month & Months & Month & Months \\
 \hline
 Holdout in Quartile 1 & $38.1$  & $56.4$  & $74.7$ & $118.0$ & $152.0$  & $184.9$  \\
\hline
Holdout in Quartile 2 & $-9.8^{***}$  & $-11.8^{***}$ & $-7.4^{***}$ & $20.1$        & $97.5^{***}$   & $217.4^{***}$  \\
                      & $(2.9)$       & $(3.1)$       & $(2.8)$      & $(32.3)$      & $(33.6)$       & $(33.5)$       \\
Holdout in Quartile 3 & $-9.9^{***}$  & $-5.2^{*}$    & $-0.1$       & $134.1^{***}$ & $388.8^{***}$  & $658.5^{***}$  \\
                      & $(2.9)$       & $(3.1)$       & $(2.7)$      & $(38.6)$      & $(39.5)$       & $(39.3)$       \\
Holdout in Quartile 4 & $-10.7^{***}$ & $-2.5$        & $2.2$        & $691.1^{***}$ & $1494.0^{***}$ & $2193.8^{***}$ \\
                      & $(2.9)$       & $(3.1)$       & $(2.7)$      & $(92.1)$      & $(126.0)$      & $(129.1)$      \\
Lien in Quartile 1    & $13.5^{***}$  & $9.9^{***}$   & $3.4$        & $14.6$        & $13.4$         & $5.3$          \\
                      & $(2.9)$       & $(2.9)$       & $(2.5)$      & $(38.4)$      & $(38.9)$       & $(38.8)$       \\
Lien in Quartile 2    & $8.9^{***}$   & $13.0^{***}$  & $8.0^{***}$  & $52.7^{***}$  & $68.8^{***}$   & $52.5^{***}$   \\
                      & $(2.8)$       & $(2.9)$       & $(2.7)$      & $(16.4)$      & $(19.4)$       & $(19.0)$       \\
Lien in Quartile 3    & $6.4^{**}$    & $7.2^{**}$    & $-0.3$       & $79.9^{**}$   & $67.3^{*}$     & $-5.2$         \\
                      & $(2.8)$       & $(3.0)$       & $(2.6)$      & $(33.1)$      & $(34.8)$       & $(34.4)$       \\
Lien in Quartile 4    & $7.0^{**}$    & $6.2^{**}$    & $3.5$        & $293.6^{*}$   & $289.4$        & $226.9$        \\
                      & $(2.8)$       & $(3.0)$       & $(2.5)$      & $(163.5)$     & $(199.9)$      & $(204.4)$      \\
Sheriff in Quartile 1 & $10.7^{***}$  & $10.7^{***}$  & $4.9^{*}$    & $3.7$         & $1.2$          & $-2.9$         \\
                      & $(3.0)$       & $(2.9)$       & $(2.5)$      & $(34.4)$      & $(34.6)$       & $(34.7)$       \\
Sheriff in Quartile 2 & $7.4^{***}$   & $10.0^{***}$  & $5.4^{**}$   & $39.2^{**}$   & $50.2^{**}$    & $28.5$         \\
                      & $(2.8)$       & $(3.0)$       & $(2.7)$      & $(16.2)$      & $(19.5)$       & $(19.2)$       \\
Sheriff in Quartile 3 & $5.8^{**}$    & $7.7^{***}$   & $3.0$        & $89.0^{***}$  & $65.6^{*}$     & $13.8$         \\
                      & $(2.8)$       & $(3.0)$       & $(2.5)$      & $(32.4)$      & $(35.1)$       & $(33.8)$       \\
Sheriff in Quartile 4 & $5.1^{*}$     & $6.2^{**}$    & $1.1$        & $114.6$       & $215.6$        & $184.3$        \\
                      & $(2.8)$       & $(3.0)$       & $(2.5)$      & $(123.6)$     & $(177.4)$      & $(191.7)$      \\
\hline
Num. obs.             & 19039         & 19039         & 19039        & 19039         & 19039          & 19039          \\
\hline
\multicolumn{7}{l}{\scriptsize{\parbox{.75\linewidth}{$^{***}p<0.01$,
      $^{**}p<0.05$, $^*p<0.1$. Holdout values for first quartile in
      levels; other holdout figures are relative to this and remaining
      figures are treatment effects for the stated treatment
      vs. holdout owners in the same quartile.}}}
\end{tabular}
\label{lpm_hetero}
\end{center}
\end{table}

Table \ref{lpm_hetero} shows the compliance behavior of tardy
taxpayers by the size of their tax bill.  Tardy taxpayers are divided
into four quartiles by taxes owed: Quartile 1 (mean owed = \$149);
Quartile 2 (mean owed = \$597), Quartile 3 (mean owed = \$1,133), and
Quartile 4 (mean owed = \$3,885).  All comparisons are for the
outcomes \textit{ever-paid} and total (taxes) paid relative to those in the
holdout sample in Quartile 1.  Three conclusions follow.  First, tardy
taxpayers in Quartile 1 owing the least in taxes are the most likely
to make a tax payment, whether they receive a reminder letter or not;
note the significant negative effect of being in Quartiles 2-4 of the
holdout sample.  Second, receiving a reminder/lien and
reminder/sheriff letter improves \textit{ever-paid} compliance for all four
Quartiles, but the effects are greatest for those in the lowest two
Quartiles.  Third, if resources are limited and the objective is to
maximize additional revenue collected, then from the results for
total-(taxes)-paid the City should send the reminder/lien letter to
taxpayers in Quartile 4, those who the most.\footnote{From Table 5,
  the expected average revenue after six months for each quartile will
  be the sum of payments by the holdout sample in that quartile plus
  the impact of each letter on payment for that quartile.  For
  example, payments after six months by taxpayers in quartile 1
  receiving the lien letter will be \$184.90 + \$5.30 = \$190.20.  For
  all quartiles, returns after six months for the lien (sheriff)
  letter will be \$190.20 (\$182) for tardy taxpayers in quartile 1,
  \$269.90 (\$245.90) for tardy taxpayers in quartile 2, \$652.80
  (\$672.30) for tardy taxpayers in quartile 3, and \$2,419.90
  (\$2377.3) for those in quartile 4.}

Finally, our results shed light on the importance of liquidity
constraints as a motivation for tardy tax payments. If liquidity
constraints are important, then nudges may be insufficient unless
accompanied by a way to smooth payments of the original tax
obligation. Taxpayer agreements that spread payments over several
months (typically, three to six months) without penalty provide for
payment smoothing. Each reminder letter included a sentence stressing
the availability of taxpayer agreements to help with payments. The
results in Tables 2 and 5 suggest, however, that liquidity constraints
are not binding for most of our tardy taxpayers. First, from Table 2,
most taxpayers who respond positively to a tax nudge after one and
three months and who make some payment will pay in full. For all
statistically significant nudges, including simple reminder, the
percent increase in the ``ever paid'' taxpayers that ``paid in full''
is never less than 60 percent (1 month, simple reminder), typically 75
to 80 percent, and as high as 97 percent (three month, peer
reminder). Second, from Table 5, taxpayers in the 1st quartile, who
owe the least in taxes, have the highest rate of payment, even without
the nudge. Further when taxpayers respond to a nudge, again the share
who ``pay in full'' (not shown in Table 5) is over 70 percent of those
who ``ever paid'' for the two strongest nudges, the lien and sheriff
letters.

Still approximately thirty percent of those tardy taxpayers who
respond to a nudge do not, or cannot, make full payment. One likely
explanation for these taxpayers is a liquidity constraint. Offering
these incomplete taxpayers a tax payment agreement can ease this
constraint. Table \ref{waterrelholdout} shows what fraction of tardy
taxpayers are in a tax payment agreement by the end of our experiment.
Recipients of the reminder/lien and reminder/sheriff letters are a bit
more likely to have chosen payment agreements than those in our holdout
sample as are those who receive the reminder/peer and reminder/duty
letters.  That said, agreements are only being used by about 1 percent
of the initial 19,039 tardy taxpayers (approximately 200 taxpayers)
and only about 3 to 4 percent of all tardy taxpayers who have not yet
made a full tax payment.
  
\begin{table}[htb]
\caption{Liquidity Linear Probability Model Estimates}\label{waterrelholdout}
\begin{center}
\begin{tabular}{l c}
\hline
 & \multicolumn{1}{c}{Payment Agreement}  \\
Holdout      & $0.9$ \\
\hline
Reminder     & $0.4$     \\
             & $(0.4)$      \\
Lien         & $1.0^{***}$    \\
             & $(0.4)$        \\
Sheriff      & $1.6^{***}$  \\
             & $(0.4)$         \\
Neighborhood & $0.6$        \\
             & $(0.4)$          \\
Community    & $0.4$    \\
             & $(0.4)$         \\
Peer         & $0.8^{**}$  \\
             & $(0.4)$        \\
Duty         & $0.8^{**}$    \\
             & $(0.4)$          \\
\hline
Num. obs.    & 19039            \\
\hline
\multicolumn{2}{l}{\scriptsize{$^{***}p<0.01$, $^{**}p<0.05$, $^*p<0.1$. 
Holdout values in levels; remaining figures relative to this}}
\end{tabular}
\end{center}
\end{table}

\section{Tax Revenue Implications}

While of interest as a specification and test of a behavioral theory
of tax compliance, our results are directly relevant for city tax
collection policies.  As a strategy for improving collection from
tardy taxpayers, our analysis informs two important policy issues.
First, cities need revenues: Do reminders improve collection, and then
do reminders with a message raise more money than a simple reminder?
Second, in light of recent municipal fiscal crises and the
potential for an unraveling of citizen commitment to local governance:
Do reminders with a message, and then which message, improve tax
collection as a ``nudge'' to citizen engagement? Table \ref{sh_rev}
provides answers to these two questions.


\begin{table}[htbp]
\centering
\caption{Three Month Impact of Collection ``Nudges''*: NEEDS TO BE UPDATED} 
\label{sh_rev}
\begin{tabular}{lcccccc}
  \hline
Treatment & Sample & Total Taxes & New  & Revenue/ & New  & New \% of Taxes 
\\ 
& Size & Owed & Payers & Letters & Revenues & Paid\\
\hline
Reminder & 2,420 & \$3.346 M & & & & \\
Lien & 2,432 & \$3.378 M &  & & & \\
Sheriff & 2,419 & \$3.902 M &  & & & \\
Neighborhood & 2,389 & \$4.658 M &  & & & \\
Community & 2,441 & \$3.148 M &  & & & \\
Peer & 2,417 & \$3.233 M &  & & & \\
Duty & 2,433 & \$3.201 M &  & & & \\
\hline
Totals & 16,951 & \$24.866 M &  & & & \\
  \hline
\end{tabular}
\end{table}

Listed in Table \ref{sh_rev} are our seven treatments, the sample size
to which each treatment applied and total taxes owed, and then
estimates of the impact of each treatment on the number new payers
three months after receipt of the treatment letter, the average new
revenue received per letter sent, total new revenues collected from
each treatment letter above that paid by the holdout sample, and
finally, the percent of owed taxes paid because of each treatment.

For single property owners, the total number of new taxpayers above
the holdout sample from all reminder letters is 838, an average
increase in the overall rate of compliance from receiving one our
treatment letters of 4.9 percent (838/16,940).  Table \ref{sh_rev}
also provides an estimate of additional revenues raised by each of our
treatment letters and then the total revenue raised from each
treatment group.  From the perspective of the City's Department of
Revenue, our experiment was a good investment of Department resources.
Each letter cost about \$1 to process and send and raised on average
over all letters \$37.34.  The estimated benefit to cost ratios for
the seven treatments ranged from a low of \$19.77 (the Neighborhood
letter) to a high of \$67.67 (the Lien letter).  The approximately
\$17,000 spent on our experiment to mail the 16,940 treatment letters
raised \$615,752 in additional city revenues: an average benefit to
cost ratio of 36.3.

Among our seven treatments, our experimental results clearly show the
power of the lien and sheriff letters compared to a simple reminder or
the tax morale nudges.  The number of new taxpayers above the holdout
sample is three to four times larger and the revenue/letter is two to
three times larger with the letters stressing penalties.  As a
consequence, total new revenues (above the holdout sample) from the
penalty letters and new revenues as a share of all taxes owed are
three to four times larger as well.  If we had sent only the lien or
sheriff's letter to the 16,940 taxpayers in our treatment groups we
would have raised \$1.15 million in new revenues rather than \$616,752
-- nearly twice as much.  The paid share of taxes owed would have
risen from our experiment's average of .028 to that of the lien letter only of
.053.


\begin{table}[htbp]
\centering
\caption{Six Month Impact of Collection ``Nudges''*} \label{lg_rev}
\begin{tabular}{lcccccc}
  \hline
Treatment & Sample & Total Taxes & New & Revenue/ & New & New \% of Taxes \\ 
 & Size & Owed & Payers & Letters & Revenues & Paid \\
  \hline
Reminder & 2,420 & \$3.346 M & 31 & \$9.29 & \$22,474 & 0.700 \\ 
  Lien & 2,432 & \$3.378 M & 91 & \$27.44 & \$66,739 & 2.000 \\ 
  Sheriff & 2,419 & \$3.902 M & 89 & \$27.14 & \$65,659 & 1.700 \\ 
  Neighborhood & 2,389 & \$4.658 M & -6 & -\$1.76 & -\$4,194 & -0.100 \\ 
  Community & 2,441 & \$3.148 M & 21 & \$6.38 & \$15,564 & 0.500 \\ 
  Peer & 2,417 & \$3.233 M & 33 & \$9.97 & \$24,091 & 0.700 \\ 
  Duty & 2,433 & \$3.201 M & 51 & \$15.42 & \$37,524 & 1.200 \\ 
   \hline
Totals & 16,951 & \$24.866 M & 310 & - & \$227,856 & 0.900 \\ 
   \hline
\multicolumn{7}{p{1\textwidth}}{\scriptsize* Sample Size is the number of single property tax payers in the treatment group.  Total Taxes Owed is the total taxes owed by single property tax payers in the treatment group. New Payers equals the new payers after three months computed as the estimated increase in rate of compliance of those receiving the letter over those in the holdout sample as reported in Table 2; for example, for the reminder letter the number of new payers equals 31 = 0.013 x 2,420.  Revenue per letter for each treatment equals the median new revenue collected from those who received a treatment letter and made some payment (=\$735/letter) times the three month increase in compliance from each treatment letter; for example for the reminder letter the median estimated revenue per letter equals \$9.29 = 0.013 x \$735.  New revenues for each treatment equals the revenue/letter times the number of single owner properties receiving a treatment letter: for example, for the reminder letter the estimated total new revenues equals \$22,474 = \$9.29 x 2,420.  New \% of Taxes Paid equals New Revenues divided by Total Taxes Owed; for example, for the reminder letter 0.7 = \$22,474 / \$3,345,846.}
\end{tabular}
\end{table}

\noindent ADD DISCUSSION OF TABLE \ref{lg_rev}.


\section{Discussion: What Role for Nudges?}

While the seven treatments are effective on the margin in increasing
compliance and in raising revenues, and the letters stressing economic
sanctions particularly so, the final column of Table 7 makes clear
that at least in Philadelphia our treatments will not solve the larger
problem of unpaid City property taxes.  The contributions of each
reminder letter towards total taxes owed range from a low of 1.5
percent for the neighborhood letter to a maximum of 5.3 percent for
the lien letter after the three months of our experiment.  The
reminders together raised an additional \$616,000 in property tax
revenues from the total of \$22,143 million owed in tardy payments, or
2.8 percent.  Importantly, since our sample was very close to the full
sample of all tardy taxpayers, these revenue estimates are very close
to what the City might expect in new revenues were our experiment to
become annual policy.  

Should we conclude from these very modest revenue gains that a nudge
strategy should not become part of Philadelphia's fiscal policy?  We
think not.  First, our work here was an experiment looking for an
effective collection strategy, not an evaluation of an established
policy.  The reminder/lien and reminder/sheriff letters were
significantly more effective than the average reminder letter.  If
those letters were to be adopted as part of the City's collection
strategy for tardy taxpayers and applied alone to the entire sample of
19,039 taxpayers, the city could expect to collect an additional
\$1.288 million (= \$67.67/letter x 19,039 letters) within three
months after mailing the reminders.  

Second, no one policy is likely
to be the best and only means for collecting revenues.  Any collection
policy will involve a range of collection strategies, ranging from
voluntary compliance on each annual `tax day,` to mailed reminders as
here, to detailed in person audits, to policing, trial, and
impoundment of assets.  Each strategy will have its own costs and
benefits.  All strategies that provide revenue benefits greater than
collection costs should be included as part of the final collection
policy (Keen and Slemrod, 2017).  Our reminder/lien and
reminder/sheriff letters raising over \$65 in new revenues of each \$1
of administrative costs seem to have earned their place in
Philadelphia's collection strategy.

Third, and perhaps most importantly for designing a cost-effective
strategy to collect tardy taxes, our experiment revealed an important
advantage to being patient.  After three months of our experiment, the
holdout sample of tardy taxpayers receiving no letter increased their
tax compliance by 51.4 percent in making at least some payment and by
40.8 percent in making their full payment.  The average payment
received after three months was \$637 per new taxpayer; see Table 2.
These are revenues the City received without any expenditure of City
administrative resources, just waiting for taxpayers to recognize on
their own that their property tax payments are due.  Had the City
adopted this `wait-and-see` strategy applied to all 19,039 of the
tardy, single property owner taxpayers, then after three months 9,786
tardy taxpayers would have made some payment.  With an average payment
of \$637 per new taxpayer, City revenues would have increased by
\$6.234 million, or approximately 28 percent of revenues owed by all
tardy taxpayers.  Other than a small opportunity cost of waiting, this
is �free� money and clearly the most efficient beginning strategy for
the City for collecting tardy property taxes.  The efficient
administrative package would combine the waiting policy with the most
efficient reminder letter stressing economic sanctions, mailed to all
tardy taxpayers.  Assuming those who pay without a nudge will not be
annoyed by receiving the reminder letter (and thus less likely to pay)
the City would receive payments of \$6.234 million from those who pay
without the nudge and \$1.288 million from those who pay in response
the nudge with economic sanctions.  This administrative package raises
a total \$7.522 million, or 34 percent of all tardy revenues owed.
The cost of the combined policy will be just the \$19,039 for the
reminder letters.  After three months (say September), the City can
then move to more aggressive strategies run perhaps by outside
collection agencies.  Today the City uses such agencies beginning
immediately in May after taxes become tardy.  The collection agencies
are reimbursed at the rate of \$.06 for each dollar collected, for an
implied City revenue to cost ratio of \$16.67, lower than any of our
nudges including the simple reminder.\footnote{Currently the City
  allocates 100 percent of tardy taxpayers to outside collection
  agencies beginning in early May.  Adapting our efficient collection
  strategy rather than allocating to outside agencies may save the
  City approximately \$400,000 a year. We do not know the actual
  collection strategy used by these firms but our results suggest they
  are likely to be earning a significant profit when reimbursed at the
  rate of \$.06 for every dollar collected.  Doing nothing as revealed
  by our holdout sample will earn these agencies approximately \$6,234
  million in three months.  Assuming they have discovered the high
  marginal returns of the sanction strategy as revealed here (\$65 per
  \$1 invested) and they invest \$19,000 as we have here, the firms
  would then earn an additional \$1.235 million.  Total revenues
  collected would then be \$7.469 million earning the firms \$448,000
  when reimbursed at the rate of \$.06 for every dollar collected.
  Our efficient collection strategy raises the same revenues at a cost
  of \$19,000, saving the City \$429,000.}

Finally, nudge strategies can have important effects on the aggregate
rate of taxpayer participation, but the direction of the impact is not
obvious and may depend upon the current level of participation.  The
results of Hallsworth, et.  al.  (2017) and Dwenger, et.  al.  (2017)
found announcing high rates of peer participation encouraged
additional participation by delinquent taxpayers, but in contrast,
Castro and Scartascini (2015) found announcing a relatively low rate
of peer participation (70 percent) discouraged future participation by
those who had been paying their taxes in the past.  Without a
realistic threat of large penalties, paying ones taxes may be seen as
a voluntary contribution decided in response to the equilibrium
behaviors of other taxpayers, Such games can have both low and high
participation equilibria \cite{Bergstrom-Blume-Varian-86}.  Nudge
strategies that exogenously increase participation may be able to move
the outcome from the low to the high participation equilibrium. See,
for example \citeA{Wenzel-05} and Besley, et.  al (2015).  For cities
in a low participation equilibrium the initial nudge might emphasize
economic sanctions.  As participation increases, however, the reminder
letter might place a greater emphasis on peer behavior or civic duty,
perhaps with a soft reminder that compounding economic sanctions will
apply after a future date.

Our results provide strong evidence that economic sanctions can have a
significant positive effect on the rate of taxpayer participation.
The lien reminder added 9.2 percent (= 224/2429) new taxpayers above
the holdout sample while the sheriff reminder letter added 8.8 percent
(=213/2416) new taxpayers.  And while not quite statistically
significant, the peer and civic duty reminders do encourage tax
payments beyond that obtained with a simple reminder letter, both
after three months and particularly so after six months.\footnote{The
  six month results for the peer and civic duty reminders reported in
  Table 6 are suggestive of the potential usefulness of the joint
  peer/duty reminder coupled with the threat of latter sanctions.
  This is exactly what had happened to taxpayers in peer and duty
  subsamples who received the peer and duty letters in the first three
  months of our experiment, and then the possible threat of sanction
  by the collection agencies in the second three months, from
  September to December. } The payment magnitudes are comparable to
those for the lien and sheriff letters.  Further, those who are likely
to be credit constrained respond favorably to peer and duty reminders,
and at almost the same rate as they do to the sanction reminders, by
entering tax payment agreements as shown in Table 6.\footnote{ A
  result consistent with our pilot study's finding that ``very tardy"
  taxpayers, also likely to be credit constrained, responded
  positively to a peer/duty reminder.}

\section{Conclusion}

With the cooperation of the Department of Revenue of the City of
Philadelphia, we developed and implemented a policy experiment in the
use of written nudges to improve the collection of property tax
revenues from citizens who had not paid their fiscal year 2016 taxes
by the due date of March 31, 2015.  The experiment entailed the full
sample of 21,500 ``tardy" taxpayers.  The results reported here are
for the 19,039 taxpayers who own a single property, excluding those
who own multiple properties; the results here generalize to the full
sample.  The experiment reached six substantive conclusions: First, a
simple reminder letter had a statistically significant effect on
compliance when compared to our control group who received no
reminder.  Second, the content of the reminder letter matters.  The
two letters that stress the likely economic sanctions of continued
tardy payment led to faster and higher levels compliance than the
simple reminder.  Third, adding an intrinsic message to the reminder,
one that stressed the value of public services, neighbors' compliance,
or civic duty did not increase compliance over receiving only the
simple reminder.  Fourth, most of the taxpayers who did respond with
payments, paid their full tax obligation.  Fifth, reminders were very
cost effective on the margin.  Each letter cost one dollar to send and
returned on average \$37 in increased city tax revenues.  The two
letters stressing economic sanctions were the most effective,
returning \$65 to \$67 in extra revenues for each letter sent.  Sixth,
reminders had no staying power.  Having received a 2016 reminder
letter had no effect on the taxpayer�s likelihood of paying their 2017
property taxes on time.

The results of our experiment suggested three policy conclusions for
the design of an efficient tax collection strategy for Philadelphia
First, appropriately designed reminder letters as nudges contribute
positively to revenue collection and should be part of the City's
collection strategy.  Simply reminding tardy taxpayers that their
taxes are due is valuable; our average reminder letter cost \$1 to
mail and generated on average \$37 in additional revenues.  As noted,
the most effective reminders stress economic sanctions and raised \$65
to \$67 per reminder letter.  Given these high marginal returns,
nudges should be included as part of the City's collection strategy.

Second, while effective on the margin, reminder letters alone will not
solve the problem of tardy tax collections in Philadelphia.  Our study
had the full sample of tardy taxpayers, and our reminder letters alone
succeeded in collecting only a small fraction of what was owed.  The
total owed for fiscal year 2016 was \$22.143 million and the
experiment using all our reminders raised only \$616,000 of new
revenues.  Had we used only the most effective reminders -- the two
stressing economic sanctions -- we would have doubled collected
revenues from reminders to \$1.288 million in new revenues; still only
6 percent of outstanding payments.

Third, and perhaps most interestingly, just being patient had high
returns.  Our study included a sample of holdout taxpayers who
received no reminder.  That group increased their rate of compliance
by 51.4 percent from the start of our study to its three month
conclusion date. The average payment from this sample receiving no
reminder was \$637 per new taxpayer.  Total new City revenues from
this group equaled \$6.234 million dollars, all raised at no cost to
the City other than the small opportunity cost from delay.  Relying
upon tardy taxpayers to recognize and pay their taxes on their own and
then using the sanction reminder to leverage those who continue to
forget or need a nudge would together raise \$7.522 million of the
\$22.143 owed by all tardy taxpayers, or 34 percent.  The total cost
to the City of this collection strategy would be \$19,039 for the
reminder letters.  After three months, a more aggressive but also more
expensive collection strategy might be tried.


It is useful to stress again here an important motivation for why we
ran our study. As the large empirical literature on nudges and tax
collection now makes very clear, successful strategies are context
specific.  Collection strategies that work well for one tax and for
one government may fail to do so for another tax and in another fiscal
setting.  Our results are for Philadelphia taxpayers alone.  What we
think does generalize, however, is the value of repeating studies such
as ours for the design of tax administration strategies.  We feel that
our seven reminder letters and the importance of having a holdout
sample provide an effective methodology for use by other cities for
understanding how best to collect their own tardy and delinquent
property taxes.  Further, at an average cost of \$1 per reminder
letter including mailing and office expenses, there is no reason not
to implement a study such as ours on the full sample of tardy or
delinquent taxpayers.
 
Motivated by the results of our work here, Philadelphia's Department
of Revenue now uses a reminder letter stressing the risk of a tax lien
and subsequent sheriff's sale and has delayed by three months (to
mid-July) the use of outside agencies for the collection of tardy
taxes. 

%{\footnotesize \NoTitleCaseChange\citepunct{(}{and}{, }{; }{, }{)}{}{.} 
\newpage

\bibliographystyle{theapa}
\bibliography{references}

\newpage

\begin{appendix}

\section{Multiple Owner Sample}

\setcounter{table}{0}
\renewcommand{\thetable}{A\arabic{table}}

UPDATE TABLE  \ref{balance2} TO INCLUDE HOLDOUT SAMPLE, AND NOT JUST THE 7 TREATMENT GROUPS. JUST AS WE DO IN TABLE 1.

ALTERNATIVELY JUST DO BALANCE ANALYSIS USING THE SAMPLE THAT CONSISTS OF OWNERS OF MULTIPLE PROPERTIES, I.E. EXCLUDE THE SINGLE OWNERS FROM THE SAMPLE.

IF WE LOSE BALANCE FOR THESE SAMPLES, WE ARE DONE. WE JUST REPORT THAT FINDING IN A FOOTNOTE. 

IF NOT, REDO THE ANALYSIS FOR TABLES 2 AND 3 FOR MULTIPLE-OWNER SAMPLE.


\begin{sidewaystable}[htbp]
\centering
\caption{Balance on Observables} \label{balance2}
\begin{tabular}{lrrrrrrrc}
\hline
\multicolumn{9}{c}{Single Property Owners} \\
  \hline
Variable & Reminder & Lien & Sheriff & Neighborhood & Community & Peer & Duty & $p$-value \\ 
   \hline
Amount Due (June) & \$1,383 & \$1,389 & \$1,613 & \$1,950 & \$1,290 & \$1,338 & \$1,316 & 0.38 \\ 
  Assessed Property Value & \$163,084 & \$147,573 & \$155,597 & \$206,214 & \$130,265 & \$130,936 & \$166,791 & 0.28 \\ 
  \# Owners & 2,420 & 2,432 & 2,419 & 2,389 & 2,441 & 2,417 & 2,433 & 0.99 \\ 
  \hline
\multicolumn{9}{c}{Single and Multiple Property Owners} \\
  \hline
Variable & Reminder & Lien & Sheriff & Neighborhood & Community & Peer & Duty & $p$-value \\ 
   \hline
Amount Due (June) & \$1,847 & \$1,735 & \$1,887 & \$2,209 & \$1,954 & \$1,772 & \$1,700 & 0.78 \\ 
  Assessed Property Value & \$195,029 & \$173,690 & \$178,556 & \$224,412 & \$220,963 & \$165,957 & \$191,199 & 0.76 \\ 
  \% with Single Property Owner & 87.5 & 88.0 & 87.5 & 86.4 & 88.3 & 87.4 & 88.0 & 0.45 \\ 
  \% Overlap with Holdout & 3.72 & 3.47 & 3.29 & 3.80 & 3.47 & 3.58 & 3.47 & 0.96 \\ 
  \# Properties per Owner & 1.33 & 1.32 & 1.26 & 1.36 & 1.29 & 1.26 & 1.27 & 0.55 \\ 
  \# Owners & 2,766 & 2,765 & 2,766 & 2,766 & 2,766 & 2,766 & 2,766 & 1 \\ 
  \hline
\multicolumn{9}{l}{\scriptsize{$p$-values in rows 1-5 are $F$-test $p$-values from regressing each variable on treatment dummies. A $\chi^2$ test was used for the count of owners.}} \\
\end{tabular}
\end{sidewaystable}

\pagebreak

\begin{figure}[htbp!]
\begin{center}
\includegraphics[width=6in, height=8.5in]{reminder_generic.pdf}
\end{center}
\end{figure}

\begin{figure}[htbp!]
\begin{center}
\includegraphics[width=6in, height=8.0in]{reminder_lien.pdf}
\end{center}
\end{figure}

\end{appendix}


\end{document}




\begin{table}[htbp]
\caption{Robustness Analysis: Relative to Reminder (All Owners)}
\begin{center}
\begin{tabular}{l c c c c }
\hline
 & \multicolumn{2}{c}{Ever Paid} & \multicolumn{2}{c}{Paid in Full} \\
 & One Month & Three Months & One Month & Three Months \\
Reminder     & $34.9$ & $56.4$ & $23.9$ & $41.8$ \\
\hline
Lien         & $4.9^{***}$  & $4.8^{***}$  & $3.4^{***}$  & $4.0^{***}$  \\
             & $(1.3)$      & $(1.3)$      & $(1.2)$      & $(1.3)$      \\
Sheriff      & $3.4^{***}$  & $4.5^{***}$  & $2.3^{**}$   & $3.6^{***}$  \\
             & $(1.3)$      & $(1.3)$      & $(1.2)$      & $(1.3)$      \\
Neighborhood & $-1.0$       & $-0.8$       & $-1.2$       & $-0.4$       \\
             & $(1.3)$      & $(1.3)$      & $(1.2)$      & $(1.3)$      \\
Community    & $-0.4$       & $-1.4$       & $-0.6$       & $-0.2$       \\
             & $(1.3)$      & $(1.3)$      & $(1.2)$      & $(1.3)$      \\
Peer         & $0.4$        & $-0.8$       & $0.5$        & $0.8$        \\
             & $(1.3)$      & $(1.3)$      & $(1.2)$      & $(1.3)$      \\
Duty         & $-1.2$       & $-0.2$       & $-1.0$       & $-0.8$       \\
             & $(1.3)$      & $(1.3)$      & $(1.2)$      & $(1.3)$      \\
\hline
Num. obs.    & 19361        & 19361        & 19361        & 19361        \\
\hline
\multicolumn{5}{l}{\scriptsize{$^{***}p<0.01$, $^{**}p<0.05$, $^*p<0.1$. Reminder values in levels; remaining figures relative to this.}}
\end{tabular}
\label{sh_lpm_mult}
\end{center}
\end{table}





\begin{table}[htbp!]
\caption{Six-Month Liquidity Linear Probability Model Estimates}\label{liquidversuscontrol}
\begin{center}
\begin{tabular}{l c c }
\hline
 & \multicolumn{1}{c}{Payment Agreement} & \multicolumn{1}{c}{Water Delinquency} \\
Reminder     & $1.3$ & $1.7$ \\
\hline
Lien         & $0.6$       & $-0.0$      \\
             & $(0.4)$     & $(0.4)$     \\
Sheriff      & $1.2^{***}$ & $0.0$       \\
             & $(0.4)$     & $(0.4)$     \\
Neighborhood & $0.1$       & $0.6$       \\
             & $(0.4)$     & $(0.4)$     \\
Community    & $-0.1$      & $0.4$       \\
             & $(0.4)$     & $(0.4)$     \\
Peer         & $0.4$       & $0.7^{*}$   \\
             & $(0.4)$     & $(0.4)$     \\
Duty         & $0.4$       & $0.2$       \\
             & $(0.4)$     & $(0.4)$     \\
\hline
Num. obs.    & 16951       & 16951       \\
\hline
\multicolumn{3}{l}{\scriptsize{$^{***}p<0.01$, $^{**}p<0.05$, $^*p<0.1$. Reminder values in levels; remaining figures relative to this.}}
\end{tabular}
\end{center}
\end{table}


