\documentclass[12pt,titlepage]{article}

\renewcommand\baselinestretch{1.5}
\setlength{\parskip}{0.08in}
\setlength{\medskipamount}{0.05in}
\textheight 8.1 in
\textwidth 6.0 in
\topmargin 0.25in
\tolerance=11000

\usepackage[utf8]{inputenc}
\usepackage{csquotes}
\usepackage[pdfencoding=auto,unicode=true]{hyperref}
\usepackage{graphicx}
\graphicspath{{images/analysis/}{images/balance/}}

\begin{document}

\title{Deterring Delinquency: A Field Experiment in Improving Tax Compliance Behavior}

\author{Michael Chirico, Robert Inman, Charles Loeffler, \\
        John MacDonald, and Holger Sieg\thanks{We would like to
        thank Rob Dubow (Director of Finance), Clarena Tolson (Revenue
        Commissioner), and Marisa Waxman (Deputy Commissioner for
        Assessment of Properties) in the Department of Revenue of the City
        of Philadelphia for their help and support.}  \\
        University of Pennsylvania}

\date{\today}

\maketitle

\begin{abstract}

Taxing authorities commonly confront problems of revenue collection
even when the tax base is known. This predicament, along with the more
commonly considered case of tax avoidance in the presence of non-reporting,
has led to extensive theorization of taxpayer motivation. Using a multi-arm
RCT conducted with the City of Philadelphia, we compare the responses of
taxpayers to notifications reflecting a wide variety of these theorized motivations
for tax compliance. We find that while nudge-like reminders can accelerate
tax payment, they do not increase ultimate payment rates above
the counterfactual without taxpayer contact.
Similarly, appeals to intrinsic motivations appear no better than generic
reminders at accelerating or increasing payment. Only credible and
consequentialist threats produce net increases in repayment rates. These
results suggest that tax delinquency is less a function of awareness
deficits or abstract normative commitments, as commonly theorized, and
more a product of inefficient or poorly-communicated tax enforcement information.

\noindent KEYWORDS: Tax Compliance, Property Taxation, Field
Experiment, Deterrence, Public Service Appeal, Appeal to Civic Duty.

\end{abstract}

\newpage

\section{Introduction}

The property tax plays a key role in how municipalities in the
United States fund operations from personnel to infrastructure.
Roughly 21 percent of all state and local governments’ revenues
come from property taxes.  Importantly, taxes owed are computed
by the state and local government directly from assessed market
values, and thus, unlike most other taxes (\textit{e.g.}, income, sales,
VAT, or profits), no taxpayer reporting is needed to compute tax
obligations.  The only issue for tax compliance, therefore, is
collecting a known tax obligation.  Since the tax obligation is
known, we would expect very high rates of tax collection, and
indeed, in most US jurisdictions this is the case.  The average
rate of collection is close to 95 percent.  That said, there are
still important US cities where the 10 year average rate of property
tax collection is 90 percent or lower, and there is no obvious
economic correlation to the observed rates of collection.  There
are economically distressed cities with 10 year average collection
rates below 90 percent (Cleveland, Detroit, Milwaukee, St. Louis),
but equally disadvantaged cities with collection rates above 95
percent (Birmingham, Buffalo, Memphis).   There are economically
successful cities with collection rates near 100 percent (Atlanta,
Boston, Denver, Minneapolis,  San Francisco) but then equally
successful cities with 10 year collection rates at or below 90
percent (New York City, Philadelphia, and Pittsburgh). For
reasons of economic efficiency, tax fairness, and fiscal
sustainability, it is important to understand why some cities are
successful in collecting their property taxes, and others not.  

If the explanation for variation in tax collections is not likely
to be found in the overall performance of the city’s economy,
perhaps the answer lies in how tax collections are administered
and how taxpayers react to such efforts. A range of theories
have been suggested to explain taxpayer compliance behaviors. 
First, perhaps taxpayers are honest but simply do not understand
what their true obligations are. Tax forms can be complicated. 
A recent study by Kosonen and Ropponen (2015) of Finnish small
business owners who faced a well-publicized change in their VAT
tax rate from 9 percent to 23 percent found that explicitly
mentioning the rate change as part of general questionnaire
regarding tax administration for small businesses significantly
increased tax compliance to the higher rate. 

Second, true obligations may be known to the taxpayer, but they may
choose to cheat.  They can do so in two ways.  When taxes are
self-assessed (\textit{e.g.}, income tax, VAT, profits), taxpayers can
under-report incomes or sales and over-report costs and purchases; 
or, they can simply not pay by working outside the formal economy. 
Studying compliance behavior of Danish and Chilean taxpayers, 
Kleven and his colleagues (2011) for Denmark and Pomeranz (2015)
for Chile found that taxpayers’ reported incomes and value-added
sales increased as the ability of the tax administration to
independently assess those income and sales improved via outside
reporting.   Reported taxable incomes by Minnesota residents were
also found to increase when the probability of an official tax audit
was increased (Blumenthal, et al., 2001).  

When tax obligations are known to both the taxpayer and the tax
authorities – as is the case for government-assessed property
taxation – citizens may still choose to cheat if the chances they
will be detected, prosecuted, and fined are low.  From the economic
model of tax compliance, as first specified by Allingham and Sandmo (1972),
taxpayers make their decision to comply by balancing the economic
savings from non-payment against the uncertain costs they bear from
being caught and fined.  In most studied instances of tax compliance,
however, the probability of being caught and the associated fines are
too low to rationally account for the observed high rates of tax
compliance.    Nor can the answer be found in any plausible estimate
of taxpayer risk aversion (Alm, McClelland, and Schulze 1992). 
Efforts to understand taxpayer compliance need to consider explanations
beyond the narrow framework of individual utility maximization under uncertainty.  

There are two extensions of the usual framework to consider. 
The first re-specifies the taxpayer’s utility from income to allow for
nonconvex reactions to equal gains and losses.  A recent study of 
Swedish taxpayers by Engstr{\H o}m et. al. (2015) finds loss aversion 
as defined by prospect theory can account for taxpayer compliance in 
a way that classical utility maximizing behavior with risk aversion 
cannot.  Taxpayers facing a loss from a \$1000 tax payment were
significantly more likely to overstate allowed deductions than
taxpayers facing a \$1000 refund for the same deductions.  
The second approach retains the classic specification for 
taxpayer welfare from income, but adds one or more additional 
motives for payment, called “tax morale” by Luttmer and Singhal (2014).  
They include reciprocity or payment for public goods received; 
norm behavior or peer effects; and civic duty.   Reciprocity argues 
that citizens understand that to not pay their taxes will mean less 
public services.   In this case, government services along with 
after-tax income determine taxpayer welfare.  One would expect this 
motive to be strongest when tax payments are directly linked by the 
taxpayer to services received, for example, local street repairs.   
Peer effects may arise when citizens view non-payment as a violation 
of a community norm of cooperative behavior and an individual’s 
non-payment is observed by others in the community.  Here, how many 
other taxpayers are compliant matters to whether the citizen 
also pays; see Posner (2000).   One might expect this motive to be 
strongest when a citizen’s non-payments are publicized and the citizen 
is actively involved in, or exposed to,  a community group that benefits 
from those payments--for example, a neighborhood school association or 
community oriented church group.  Finally, citizens may pay their full 
tax obligation because it is the “right thing to do” as a citizen.  
Here the act of payment has value on its own; there are no direct 
benefits and no one else need know.  The citizen has accepted the 
democratic contract and bears a presumptive obligation to fulfill 
that contract; see Rawls (1971, 350–355).  

Efforts to empirically identify the possible influence of these non-economic 
motives have been mixed.   Blumenthal et. al. (2001) find no evidence that 
these motives significantly influence truthful reporting of taxable income 
for Minnesota taxpayers, but Hallsworth, et. al. (2014) do find a strong 
beneficial impact on compliance from peer motives.  In a study closest to 
our work here, Castro and Scartascini (2015) examine motives for property 
tax payments in a municipality in Argentina.  They find that the economic 
motives from fines and enforcement are most salient, but that the non-economic 
motives do matter for selected subsamples of the population--in particular, 
lower-income residents.   Finally, in our earlier, pilot study of property tax 
compliance in Philadelphia we did find evidence that motives driven by 
reciprocity, peer effects, and civic duty can positively impact payment 
probabilities. But our sample size was small and our framing of the 
alternative motives was not as clear as we would have liked; see 
Chirico, et. al. (2015).  We view our work here as chance to pursue all 
these motives – deterrence, reciprocity, peer influence, and civic duty – 
with a larger sample and with a sharper experimental design.     

In conjunction with the City of Philadelphia, we conducted a multi-arm 
notification-based field experiment testing each of these alternatives. The 
results of this experiment demonstrate that the problem of tax delinquency 
can be improved by sending messages with more specific information on
a) the consequences of non-payment, b) the implied likelihood of 
consequences of continued non-payment, c) the certainty of eventual 
payment. Furthermore, we show that there is minimal evidence to support 
using social information, moral suasion, or other strategies. Finally, 
we show that notification/reminder strategies are not social 
welfare-improving, as they merely accelerate payment. 

% -- Charles --
%Specific Findings
%* Among first time property tax delinquents, we estimate that credible 
%  deterrent letters increase tax compliance by high-single digits. 
%*That this effect is social welfare improving rather than simply shifting.
%*And that affected population is not liquidity constrained.
%*Social norming, moral suasion, and feature engineering have no or minimal effects 

We interpret these results as being consistent with recent research 
on the importance of crafting salient and enforceable threats for 
deterrence to be efficacious (Hawken and Kleiman 2009). Further, these results indicate
that, under circumstances under which continued delinquency is preferable 
from the delinquent’s perspective and consequences of delinquency are 
perceived to be minimal or absent, non-coercive nudges or appeals are 
unlikely to shift behavior. 

\section{Research Setting -- Property Tax in Philadelphia}

The research setting for this experiment is the City of
Philadelphia. In Philadelphia, where, in recent years, just under
18\% of property owners were delinquent, owing 
\$292.3 million in unpaid taxes and another \$223.2 million in interest,
fees and accumulated charges (PEW Charitable Trusts 2013).
\footnote{
	In Detroit, nearly half of all property owners were delinquent
    on their property tax bills in 2011 (MacDonald and Wilkinson 2013).
}
, tax payments are due on January 1st of each year. If the full balance on a
property tax account is not paid in full by March 31st, fees and interest
begin accruing. The City’s Department of Revenue, along with outside
collection agencies acting as co-counsel for the City, begin contacting
unpaid accounts on April 1st. Normally, 2/3 of properties are sent to
co-counsel and 1/3 are kept in-house. If the outside firms are able to
collect the outstanding tax from property owners by December 31st of the
tax year, under their contract with the city, they are entitled to 6\% of
the collected revenue. After the new year, remaining delinquent properties
are reassigned to different firms for collection.

\section{Philadelphia Property Tax Notification Experiment}

% -- Charles --
%Various theories have been offered for why taxpayers would not pay these
%well-reported tax debts. These include discontent with provided public
%goods, low tax morale, missing social norms, or perceived unfairness of 
%taxation (Erard and Feinstein 1994; Kirchler 2007; Torgler 2007). To see 
%whether these perceived deficiencies in non-financial factors among 
%non-taxpayers do in fact explain this form of non-hidden tax delinquency, 
%most recent taxpayer field experiments focus on motivating improved tax 
%compliance via appeals to these instrinsic motivations while contrasting 
%their effects with those produced by more conventional extrinsic 
%motivations (Blumenthal et al., 2001). In the literature, this distinction 
%is also referred to as deterrence versus non-deterrence messaging 
%(Hallsworth et al. 2014).
%
%Social norming or moral persuasion have been suggested as ways of achieving 
%this goal since both involve messaging strategies targeted at possible 
%motivational reasons for tax delinquency (Posner 2000; Traxler 2010; 
%Wenzel 2005). Also, additional notifications and reminders have been proposed 
%as well as manipulation of other non-economic communication factors, 
%including shape, color, and other non-prose cues with the expectation that 
%recipients will be more likely to see and respond to communications in 
%these formats (Thaler and Sunstein 2003). Finally, the use of enhanced 
%sanctions, threats, and even public shaming have been considered. However, 
%with exception of Chirico et al. (2015) and Del Carpio (2014), little work 
%has examined which motivations are at play in the case of non-hidden tax 
%delinquency and consequently what the most effective intervention strategies 
%would be to reduce it.

Of the 579,828 properties in the city in 2015, roughly 100,000 were
% -- Charles --
%citation for 100k number:
%http://www.phillymag.com/citified/2015/07/01/philadelphia-tax-delinquency-lien-sale/
delinquent on their real estate tax bills in the Spring of 2015 when
the tax experiment began. We included in our sample the universe of 27,264
properties in Philadelphia still owing at least \$10 in Real Estate
Taxes to the City as of May 15, 2015
% -- Michael --
%Couldn't find any specific date for the validity of the balances;
%I'm going off an e-mail containing data we received May 15
%(title “File is Ready!!!”)
\footnote{
	Given that taxes start accruing penalties as of 
	March 31st each year, this means these properties were 
	nearing three months late by the onset of the 
	experiment in mid-June.
}
and excluding chronic delinquents who additionally owed taxes for 
prior years (see Chirico et al. 2015 for more on this population). 
% -- Charles --
% Add debt statistics for properties here.
In coordination with officials at the Department of Revenue, we designed 
seven letter templates probing a variety of tax compliance motivations. 
%TO DO: Number appendices, if appropriate
The templates are included in the Appendix; here we delineate the important 
distinguishing features of each. First, we designed a Control template, which 
served simply as a minimalist reminder to taxpayers about their outstanding 
balance, as well as providing contact information and a notice about 
available help for English non-natives.

Two more templates were designed with the aim of deterrence in mind, under 
the belief that taxpayers can be convinced to pay by spelling out clearly 
and in tangible terms the potential consequences of nonpayment. The first 
of the deterrence letters, which we call the Sheriff treatment, emphasizes 
the City's most stringent debt recovery mechanism, the Sheriff's Sale, 
through which the City can auction delinquent properties to the public. 
This treatment contains the following text: 

\blockquote{
	Failure to pay your Real Estate Taxes may result in the sale of 
	your property by the City in order to collect back taxes. In the 
	past year, we have sold $N$ properties in your neighborhood at 
	Sheriff's Sale.	Included in these $N$ are the following properties 
	near you: <three properties and their sale dates> 

	Pay your taxes now to prevent the sale of your property. 
	Our records indicate that you have a balance due of $balance$.”
}

$N$ is the number of properties sold in neighborhood between June 2014
and May 2015.
\footnote{\label{fn:neighborhoods}
	Data retrieved from
	\url{
	http://www.officeofphiladelphiasheriff.com/real-estate/sheriffs-sale-webapp
	}
	and then geocoded by the authors; in total, there were 876 properties 
	sold by the City during the year through May 2015. $neighborhood$ is defined 
	in high-density neighorboods, where at least 8 Sheriff's sales took place,
	as the Azavea neighborhood in which the property is located. The Azavea 
	neighborhoods are a commonly-used partition of Philadelphia into 158 
	colloquially-known (\textit{i.e.}, residents of a neighborhood generally refer to 
	their own area similarly) regions, as specifically created and maintained 
	by the GIS firm Azavea, Inc.; see 
	\url{
	https://www.opendataphilly.org/dataset/philadelphia-neighborhoods
	}
	. Roughly half of properties were in high-density neighborhoods; for the rest, where 
	there were few local Sheriff's sales to offer as warnings to recipients, 
	$neighborhood$ is defined at a slightly higher level of aggregation, namely, 
	the “Azavea quadrant,” as created by the authors to combine Azavea 
	neighborhoods into a total of six larger meta-neighborhoods, namely, 
	Northeast, North, Northwest, West, South, and Center City Philadelphia. 
	%TO DO: Make sure we actually include this map (currently absent)
	See results for the exact composition of these “quadrants.”
}
To generate the listed properties, for each neighborhood
in our sample, we selected three of the properties sold therein in the
previous year at random, meaning residents in the same neighborhood
saw the same set of three properties.
\footnote{
	Initial plans to mention the three nearest sales to each property, 
	to choose three sales at random for each property (not neighborhood), 
	and other variations of these approaches were met with concerns 
	about privacy and abandoned.
}

The second deterrence treatment, which we call the Lien treatment, emphasizes 
a more mild version of debt recovery punishment at the City's disposal, 
namely, that the City can exact a lien on any delinquent property which 
entitles the City to deduce the amount of the lien from any future arms-length 
market transaction involving the property; the text included is as follows: 

\blockquote{
	Failure to pay your Real Estate Taxes will result in a tax lien on your 
	property in an amount equal to your back taxes plus all penalties and 
	interest. When your property is sold, those delinquent tax payments will 
	be deducted from the sale price. By paying your taxes now, you can avoid 
	these penalties and interest. Properties near you in $neighborhood$ that 
	have had liens placed on them include: 
	<three properties and their sale dates> 

	Pay your taxes now to avoid a lien being placed on your property. 
	Our records indicate that you have a balance due of $balance$.
}



The variables here are defined as they are in the Sheriff treatment. 
In particular, the triplet of properties assigned to each neighborhood is 
identical across these two treatments. 

The next pair of templates target appeals to taxpayers' sense of citizenship. 
The first of these, which we'll refer to herein as Peer, implicitly posits 
that one's predilections for tax compliance arise from a desire for conformity, 
and as such aim to highlight that the late payee is behaving abnormally with 
respect to their peers. The specific text reads:

\blockquote{
	
	You have not paid your Real Estate Taxes. Almost all of your neighbors 
	pay their fair share—9 out of 10 Philadelphians do so. By failing to pay, 
	you are abusing the good will of your Philadelphia neighbors.
}
 
We test an alternative manifestation of the idea that citizens pay their 
taxes out of a sense of citizenship by formulating in our fourth treatment 
letter, dubbed Duty, an appeal to the delinquent’s sense of duty, in whichever 
abstract sense the the reader of the letter chooses to ascribe to the notion. 
Specifically, we exhorted tax nonpayers in this treatment with the following:

\blockquote{
	For democracy to work, all citizens need to pay their fair share of taxes 
	for community services. You have not yet paid your taxes. By failing to 
	do so, you are not meeting your duty as a citizen of Philadelphia.
}

The final pair of letter templates were designed to elicit payment from 
the out-of-hock by highlighting the \textit{quid pro quo} nature of
public good provision – the City provides a bundle of goods and services
in exchange for tax payments by residents, especially real estate taxes.
The first such treatment, which we simply call Moral, targets an appeal 
to public good provision writ large, and reads:

\blockquote{
	Your taxes pay for important services that make a city great. Your 
	tax dollars are essential for ensuring all Philadelphia children 
	receive a quality education and all Philadelphians feel safe in 
	their neighborhoods. Please pay your taxes as soon as you can 
	to help us pay for these important services. 
}

The final letter, which we call Amenities, took instead a more 
localized approach to this appeal, taking the further step of mentioning 
specific local public amenities in the vicinity of the letter recipient's property 
which are in part financed by property taxes. The wording for this treatment is:

\blockquote{
	We want to remind you that your taxes pay for essential public services 
	in $neighborhood$, such as <two local amenities>, your local police officer, 
	snow removal, street repairs, and trash collection. Please pay your taxes 
	to help the city provide these services in your neighborhood.
}

The amenities were chosen at random for each property from a list of 
City-supplied parks, recreation centers, libraries, etc.
\footnote{
	We again use a bifurcation of the definition of $neighborhood$ based on
	the density of amenities in the Azavea neighborhood. Here, we only 
	required that a neighborhood had four amenities supplied to us by the
	City in order to be classified as high-density. Slightly less than 
	half (48\%) of properties were in high-density neighborhoods; the 
	remaining 52\% were assigned amenities at random from their quadrant, 
	as discussed above in Footnote ~\ref{fn:neighborhoods}. 
}

In addition to varying the content of the letters delivered to our sample,
we undertook two additional interventions. The first intervention 
investigated the effectiveness of another commonly-touted “nudge” to payment,
namely, manipulation of the envelope itself. The hypothesis behind 
intentionally tailoring the design of the envelope is that a non-negligible 
portion of recipients simply discard the letter prior to opening it 
(and in doing so can never be exposed to the notification treatments); 
by capturing the attention of these casual disposers, the well-designed 
envelope exposes them to the letter contents, thereafter translating into 
higher compliance rates. We tested this hypothesis by delivering half of 
the letters in a standard-sized windowed envelope
(roughly $4\frac18$'' x $9\frac12$''), a treatment we refer to as Small. 
The other half of our sample received full-sheet-sized envelopes (9'' x 12''), 
the Big treatment. 
\footnote{
	The letters in small envelopes were stuffed and delivered by the 
	Department of Revenue's in-house mail room. As they lack the 
	infrastructure to stuff (at scale) large-format envelopes, the authors 
	elected to contract the services of a printing shop, Lawler Direct, Inc., 
	to handle stuffing and delivery of this treatment group. Some 
	coordination issues led to the letters handled by the City being 
	delivered roughly two weeks earlier than the Lawler sample. 
}

For the final intervention, we randomly removed 3,000 properties from 
the possibility of receiving any letter altogether, placing these 
properties in a holdout group. These properties were left entirely 
% -- Michael --
% Need to be careful about wording here, given GRB, etc.
% -- Charles --
% ?
uncontacted during our treatment period. These properties serve to 
elucidate, through comparison to our control condition, our 
understanding of whether or not properties can be induced to act simply 
through the receipt of any sort of correspondence from the City. 
For more details and results of these additional experiments, see Appendix A.

\section{Randomization Procedure}

% -- Charles --
%Experimental Setup
%Blocked randomization
%Timing of experiment, roll-out
%Description
%Randomization Tests
%# properties & owners by letter
%Log balance by treatment (box plots)

Randomization took place in two stages. In the first stage, 3,000 
of the 27,264 eligible properties were randomly assigned 
to the Holdout group. Of the remaining 24,264 properties, 89\% were 
owned by unique owners (owning only one property in Philadelphia) and 
the remaining 11\% were held by owners with multiple holdings in our sample. To avoid 
randomizing separate properties of non-unique property owners into 
multiple treatments, we identified multiple property owners by 
matching the legal name associated with each property. Average 
owner-level debt was \$1810, well above the median of \$907 due to 
upper-tail outliers. In the second randomization stage, owners were 
assigned treatments in blocks – that is, any owner holding multiple 
properties with unpaid real estate tax bills received the same letter 
(and envelope) for each of those properties.
\footnote{
	Lacking an objective identifier of an individual, \textit{e.g.}, social 
	security number, we elected to group properties based on 
	the owner's legal name. This undoubtedly led to distinct 
	individuals with identical names being grouped together, but 
	inspection by the authors found this to be rare; we consider this 
	as random noise and ignore it henceforth.
}
The blocks were defined according to owner-level total debt, given 
the high correlation between propensity to pay and total debt owed, 
to help control variability in the sample and assure equality of 
samples along this very important dimension.
\footnote{
	More specifically, all owners were ordered by total debt and assigned 
	sequentially in groups of 14 to randomization blocks. Within each 
	randomization block, a permutation of the 7x2 = 14 treatments described 
	above was assigned to the properties. The final block, having only 13 
	leftover properties, was assigned a random choice of 13 treatments.
}
In order to account for blocking in subsequent analysis, standard errors 
are clustered at the block level (as randomized) for all regression equations. 

Each of the seven letter templates described above was assigned at random 
to envelopes of each size--in other words, each property was equally 
likely to receive any one of fourteen possible envelope+letter combinations. 
Balance tests for pre-randomization characteristics, 
%TO DO: Should replace manual numbering with programmatic numbering via \ref
% -- Charles --
%Mike—Could you produce a balance table to match what I have as Table 3 of 
%the first paper? Treatments are columns and property characteristics 
%are rows? We’ll need a second one for the holdout comparison.
reported in Table 1 and Figure \ref{fig:box_bal}, confirm that randomization was successful, 
with no discernable differences between any of our treatment groups.
\footnote{For balance tests for the stage 1 holdout sample, see Appendix.}
While large and multiple property owners introduced small variations 
between groups, the randomization procedures worked as designed with 
blocking serving to limit the influence of these small number of property 
owners to the first two blocks. Excluding these produces perfect balance 
on all variables for the 99.9\% of all properties in the remaining blocks.

\section{Results}
% -- Charles --
%Mike—Can you clean up the references to the tables per Holger’s request.
%Basically, numbering needs to align. Currently, I’ve just pasted things 
%in based on numbering before we figured out order.

\subsection{Intervention \#1: Targeted Phrasing vs. Control}

%TO DO: Programmatic numbering
Table 1 and Figure \ref{fig:box_bal} report the main experimental intervention results 
at one-, three-, and six-month intervals. At the one month mark, 
several weeks after receiving the experimental or control notifications, 
35\% of delinquents in the control condition have made at least some payments 
towards their outstanding bill, with 24\% having paid the bill in its entirety.
%TO DO: Make sure we're consistent with treatment names
All intrinsic motivation notifications (\textit{i.e.}, Amenity, Civic Responsibility, 
Moral, and Peer) are within 1 percent of the control repayment rate and 
statistically insignificant. By comparison, notifications emphasizing 
extrinsic/consequentialist factors both show short-term differences of 
4.9\% (Lien) and 3.4\% (Sheriff's sale). Both are statistically significant. 
Results from logit models using paid-in-full (including fees) instead of any 
%TO DO: Actual number
%TO DO: Programmatic numbering
payment reported Table 2, reveal that the majority (\#\#\%) of the estimated 
short-term effect of payment comes from recipients paying off their entire balance.

%TO DO: Programmatic numbering
Three-month results, reported in Column 2 of Table 1, show a very 
similar picture. As can be seen visually in Figure 1, baseline (control) 
repayment rates have risen to 56\% (any payment) and 42\% (full payment). 
However, intrinsic motivation treatments continue to have no discernable 
effect on repayment rates, while extrinsic notifications have produced a 
4.8\% (Lien) and 4.5\% (Sheriff's sale) response as compared to control. These 
results retain statistical significance and full repayment continues to 
contribute to the majority of the observed effects.

%TO DO: Programmatic numbering
Six-month results, reported in Column 3 of Table 1, indicate that final 
participation rates for first-time delinquents are 76\% within six months. 
Nudge-like treatment notifications remain insignificant and barely 
distinguishable from the control response. Consequentialist notifications 
are 2.4\% (Lien) and 1.9\% (Sheriff's sale) higher than control. The lien treatment 
thereby is the only treatment to retain significance at conventional 
levels through the end of the follow-up period.
\footnote{
	We terminate follow-up at year's end, when the onset of 2016 tax liabilities 
	begin to muddy the interpretation of repayment and outstanding debt.
}
Comparing full payment 
outcomes instead of all payments reveals that the lien consequentialist 
treatment is 4\% higher than the control mean of 63\% with no other 
treatments having discernable or significant differences.

To provide an estimate of the revenue increases resulting from experimental 
treatments, we fit OLS models for amount paid regressed on treatment status.
The results of these regressions are reported in 
%TO DO: Programmatic numbering
Table 3. Nonparametric tests with bootstrapped standard errors within 
randomization blocks are reported visually in Figure \ref{fig:tp_time_7_own}. 
These results indicate that the lien treatment was effective at boasting 
revenue collected within the first month of the experiment. Lien recipients 
%TO DO: Replace with true number
paid \$\#\#.\#\#, on average, during this period. However, no statistically 
significant different in repayment occurred over the longer period of 3 or 6 months.

%possible insert sensitivity/heterogeneity results:
%	Amount owed
%	Owner occupied versus not

\subsection{Intervention \#2: Mailer versus No-Mailer}

%TO DO: Programmatic numbering
Table 4 reports the comparison of sending a generic reminder letter 
(control) versus sending no notification letter whatsoever. This 
comparison is restricted to unique property owners.
\footnote{
	Most multiple owners of a Holdout property also own a treated property, which
	serves to preclude a proper apples-to-apples comparison of such owners, so
	we focus instead on contrasting only unique owners in each group.
}
As can be seen visually in Figure \ref{fig:ep_time_ch_own}, 
a significant gap (4\%) in the probability 
of any payment quickly emerged after the letters were mailed. Until 
late August, corresponding to the time when still-delinquent properties 
in the experiment were released to tax collection agents, this gap of 
4\% persisted. After this time, hold-out property owners began paying 
at higher rates, leading to a convergence of trends—suggesting that 
while generic notification does accelerate payment, some
combination of tardy participants and other enforcement actions are 
likely to produce a similar ultimate result. 

\subsection{Intervention \#3: Standard Envelope versus Larger Envelope}

%TO DO: Programmatic numbering
Table 5 and Figure \ref{fig:ep_time_2_own} report the results of a third tested
intervention--modifying the size of the notification envelope.
Previous research has hypothesized that increasing the visibility or visual 
salience of a notification could increase uptake or response.
However, the results of this particular version of increased visual salience 
suggest that tax authorities sending tax bills are likely to receive little 
benefit from the increased postage cost more eye-catching mailers. 
At one, three and six months post-treatment, payment probabilities 
were similar and statistically indistinguishable. 

\section{Fiscal Analysis}
% -- Charles --
%Mike—This is another thing for you to insert based on our emails post-briefing. 

\section{Conclusions}

In this paper, we report the results of a multi-arm field experiment 
designed to test which low-cost notification strategies increase tax 
payment rates. In the context of this field experiment in municipal 
tax collection, our results suggest that simple notification strategies, 
much discussed in recent studies, can accelerate participation but are 
ineffective at reducing tax delinquency. This finding covers a 
wide swath of theorized “nudges,” including social norming, moral suasion, 
and tax morale. In addition, simply notifying delinquents more often was 
found to be an accelerator of payment while leaving long-term levels 
unchanged. Finally, we observe no effect of increased envelope size. 
Apparently, taxpayers open tax bills even if they do not pay them.
These non-findings are contrasted with the single persistent effect 
that delinquency levels can be lowered quickly and permanently using 
credible threats regarding the foreseeable consequences of nonpayment. 

For revenue collecting agencies, particularly those with revenue collection 
problems, acceleration of payment can be understood to be a useful result 
in and of itself. Bills must be paid and debts must be serviced on regular 
schedules. Relatedly, the longer that tax bills remain unpaid, the more 
expensive it becomes to collect. Whether handled internally or externally 
through debt collection firms, downstream collection practices leave 
diminished revenues. For both of these reasons, early collection is, 
\textit{ceteris paribus}, better collection. This is not to say, 
however, that early collection is social welfare improving. If tardy 
but eventually-compliant tax delinquents forestall early payment to 
cover other expenses or invest in other assets which they eventually 
use to repay their tax bill with interest, then the welfare of the tax 
agency may not be synonymous with the welfare of society. This is 
especially true if tax payments are made weeks rather than months late, 
such that monthly payments can still be made based on expected monthly 
receipts. However, in the case of persistently-delayed but consistently-paid 
payments, it is less obvious that accelerating eventual or inevitable 
payment constitutes something of value. 

For this reason, the fact that credible threat notifications increase 
repayment rates and convert, in the language of Rubin causal framework, 
defiers into compliers is particularly notable. It suggests that 
increasing repayment at low cost is possible, and that more research is 
needed on the reasons for different behavioral responses to 
consequentialist and non-consequentialist messages. Perhaps, the difference 
between treatments that merely accelerate payment and those that increase 
payment lie in alternative theories of why tax delinquents exist in the 
first place. Non-consequentialist theories of non-payment implicitly rest 
on the assumption that non-compliers are not liquidity-constrained and are
merely unaware of the collective consequences of non-payment. Under this 
analysis, tax delinquency is due to discouragement, indifference, 
lack of appreciation, or unawareness. Delinquents merely need to be 
encouraged or reminded to participate. Our results, however, suggest 
that this is an incomplete explanation for tax delinquency. 

Tax delinquents are not simply discouraged. Providing them with 
% -- Charles --
%Could add sentence here about the envelope experiment.
information about peer behavior, amenities, moral arguments, or civic 
duties does nothing to increase the overall repayment rates. This may 
reflect the fact that delinquents are indifferent to or already aware 
of their peers’ positive and negative behavior. Likewise, the provision 
of information on public goods assumes that recipients have not 
incorporated consideration of public goods funded through tax dollars 
into their payment behavior. If services are considered paid for by 
other taxes or the quality of the public services are considered to be 
sub-par, then the rationale for funding these initiatives may not be 
particularly compelling.  Similarly, social contract theories of citizen- 
and state-shared responsibilities offer an idealized vision of civic 
responsibilities that ignores the reality that both sides are chronically 
dissatisfied with the performance of other parties.

At the same time, our results suggest that if tax delinquents are not 
simply discouraged, they are also seemingly unaware of the consequences 
of nonpayment. Our provision of simple information about the collection 
process had a clear impact on the likelihood of repayment. This result 
echoes other recent findings that clear, consistent, and timely provision 
of information on consequences, particularly in the context of compliance 
monitoring, can lead to notable improvements in behavioral compliance 
(Hawken and Kleiman 2009). Perhaps, then, the puzzle of high non-payment 
rates despite perfect public information on noncompliance can be understood 
as a case of under-enforcement. This possibility is reinforced by the fact 
that conditional on any payment, converted defiers became near perfect 
compliers, making full payments in almost all cases. This suggests that at 
least for the margin affected, liquidity-constraints are not the primary 
reason for initial non-payment.

%TO DO: Automate this with bibTeX
\section{References}
Allingham, Micahel G., and Agnar Sandmo. 1972. “Income Tax Evasion: A Theoretical Analysis.” Journal of Public Economics 1: 323–38.

Alm, James, Gary H. McClelland, and William D. Schulze. 1992. “Why Do People Pay Taxes?” Journal of Public Economics 48 (1): 21–38. doi:10.1016/0047-2727(92)90040-M.

Blumenthal, Marsha, Charles Christian, Joel Slemrod, and Matthew G. Smith. 2001. “Do Normative Appeals Affect Tax Compliance? Evidence from a Controlled Experiment in Minnesota.” National Tax Journal 54 (1): 125–38.

Castro, Lucio, and Carlos Scartascini. 2015. “Tax Compliance and Enforcement in the Pampas Evidence from a Field Experiment.” Journal of Economic Behavior \& Organization 116 (August): 65–82. doi:10.1016/j.jebo.2015.04.002.

Chirico, Michael, Robert P. Inman, Charles Loeffler, John MacDonald, and Holger Sieg. 2015. “An Experimental Evaluation of Notification Strategies to Increase Property Tax Compliance: Free-Riding in the City of Brotherly Love.” In Tax Policy and the Economy, Volume 30. University of Chicago Press. \url{http://www.nber.org/chapters/c13690.pdf}.

Del Carpio, Lucia. 2014. “Are the Neighbors Cheating? Evidence from a Social Norm Experiment on Property Taxes in Peru.” Work. Pap., INSEAD.

Engstr{\H o}m, Per, Katarina Nordblom, Henry Ohlsson, and Annika Persson. 2015. “Tax Compliance and Loss Aversion.” American Economic Journal: Economic Policy 7 (4): 132–64.

Erard, Brian, and Jonathan S. Feinstein. 1994. “Honesty and Evasion in the Tax Compliance Game.” The RAND Journal of Economics 25 (1): 1–19. doi:10.2307/2555850.

Hallsworth, Michael, John List, Robert Metcalfe, and Ivo Vlaev. 2014. “The Behavioralist as Tax Collector: Using Natural Field Experiments to Enhance Tax Compliance.” National Bureau of Economic Research.

Hawken, Angela, and Mark Kleiman. 2009. “Managing Drug Involved Probationers with Swift and Certain Sanctions: Evaluating Hawaii’s HOPE.” Washington, D.C.: National Institute of Justice. \url{https://www.ncjrs.gov/App/Publications/abstract.aspx?ID=251050}.

Kirchler, Erich. 2007. The Economic Psychology of Tax Behaviour. Cambridge University Press.

Kleven, Henrik Jacobsen, Martin B. Knudsen, Claus Thustrup Kreiner, Søren Pedersen, and Emmanuel Saez. 2011. “Unwilling or Unable to Cheat? Evidence From a Tax Audit Experiment in Denmark.” Econometrica 79 (3): 651–92. doi:10.3982/ECTA9113.

Kosonen, Tuomas, and Olli Ropponen. 2015. “The Role of Information in Tax Compliance: Evidence from a Natural Field Experiment.” Economics Letters 129 (April): 18–21. doi:10.1016/j.econlet.2015.01.026.

Luttmer, Erzo F. P., and Monica Singhal. 2014. “Tax Morale.” Journal of Economic Perspectives 28 (4): 149–68. doi:10.1257/jep.28.4.149.

MacDonald, Christine, and Mike Wilkinson. 2013. “Half of Detroit Property Owners Don’t Pay Taxes.” Detroit News, The (MI), February 21, 2-dot edition.

OECD. 2011. “Tax Administration in OECD and Selected Non-OECD Countries: Comparative Information Series (2010).” Organization for Economic Cooperation and Development, Center for Tax Policy and Administration.

PEW Charitable Trusts. 2013. “Delinquent Property Tax in Philadelphia: Stark Challenges and Realistic Goals.” Philadelphia, PA: The Pew Charitable Trusts. \url{http://www.pewtrusts.org/~/media/legacy/uploadedfiles/wwwpewtrustsorg/reports/philadelphia_research_initiative/philadelphiapropertytaxdelinquencyreportpdf.pdf}.

Pomeranz, Dina. 2015. “No Taxation without Information: Deterrence and Self-Enforcement in the Value Added Tax.” American Economic Review 105 (8): 2539–69. doi:10.1257/aer.20130393.

Posner, Eric A. 2000. “Law and Social Norms: The Case of Tax Compliance.” Virginia Law Review 86 (8): 1781–1819. doi:10.2307/1073829.

Rawls, John. 1971. A Theory of Justice. Cambridge, Mass.: Belknap Press of Harvard University Press.

Slemrod, Joel. 2007. “Cheating Ourselves: The Economics of Tax Evasion.” The Journal of Economic Perspectives 21 (1): 25–48.

Thaler, Richard H., and Cass R. Sunstein. 2003. “Libertarian Paternalism.” The American Economic Review 93 (2): 175–79.

Torgler, Benno. 2007. Tax Compliance and Tax Morale: A Theoretical and Empirical Analysis. Edward Elgar Publishing.

Traxler, Christian. 2010. “Social Norms and Conditional Cooperative Taxpayers.” European Journal of Political Economy 26 (1): 89–103. doi:10.1016/j.ejpoleco.2009.11.001.

Wenzel, Michael. 2005. “Misperceptions of Social Norms about Tax Compliance: From Theory to Intervention.” Journal of Economic Psychology 26 (6): 862–83. doi:10.1016/j.joep.2005.02.002.

\section{Tables and Figures}

%Table 1:  Paid Full @ 1,3,6 months, main treatments vs. Control

%Table 2: Total Paid @ 1,3,6 months, main treatments vs. Control

%Figure 1
\begin{figure}[htpb]
\begin{center}
\caption{Sample Balance on Initial Debt across Main Treatments}
\label{fig:box_bal}
\bigskip
\includegraphics[width=6in]{dist_log_due_by_trt_7_box}
\end{center}
\end{figure}

%Figure 2
\begin{figure}[htpb]
\begin{center}
\caption{Partial Participation in Main Treatments across Time}
\label{fig:ep_time_7_own}
\bigskip
\includegraphics[width=6in]{bar_plot_ever_paid_julsepdec_7_own}
\end{center}
\end{figure}

%Figure 3
\begin{figure}[htpb]
\begin{center}
\caption{Average Repayment in Main Treatments over Time}
\label{fig:tp_time_7_own}
\bigskip
\includegraphics[width=6in]{bar_plot_aver_paid_julsepdec_7_own}
\end{center}
\end{figure}

%Figure 4
\begin{figure}[htpb]
\begin{center}
\caption{Comparison of Partial Participation between Control and Holdout over Time}
\label{fig:ep_time_ch_own}
\bigskip
\includegraphics[width=6in]{cum_haz_ever_paid_control_holdout_own}
\end{center}
\end{figure}

%Figure 5
\begin{figure}[htpb]
\begin{center}
\caption{Partial Participation in Envelope Treatments across Time}
\label{fig:ep_time_2_own}
\bigskip
\includegraphics[width=6in]{bar_plot_ever_paid_julsepdec_2_own}
\end{center}
\end{figure}

\section{Appendix}

%Table A1: Control vs. Holdout (Unique Property Owners)

\end{document}