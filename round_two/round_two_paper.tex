\documentclass[12pt,titlepage]{article}

\renewcommand\baselinestretch{1.5}
\setlength{\parskip}{0.08in}
\setlength{\medskipamount}{0.05in}
\textheight 8.1 in
\textwidth 6.0 in
\topmargin 0.25in
\tolerance=11000

\usepackage[pdfencoding=auto,unicode=true]{hyperref}
\usepackage[utf8]{inputenc}
\usepackage{csquotes}
\usepackage{graphicx}
\usepackage{rotating}
\usepackage{natbib}
\bibliographystyle{aer}
\graphicspath{{images/analysis/}{images/balance/}}
\renewcommand{\thefootnote}{\fnsymbol{footnote}}

\begin{document}

\title{Deterring Delinquency: A Field Experiment in Improving Tax Compliance Behavior}

\author{Michael Chirico, Robert Inman, Charles Loeffler, \\
        John MacDonald, and Holger Sieg\thanks{We would like to
        thank Frank Breslin (Revenue Commissioner), Rob Dubow (Director of Finance), Clarena Tolson (Deputy Managing Director for Transportation and Infrastructure),  Marisa Waxman (Deputy Commissioner of Policy and Analysis) and the senior staff in the Department of Revenue of the City
        of Philadelphia for their help and support.  The Penn Wharton Initiative for Public Policy provided funding for the experiment. We would also like to thank Jeff Brown, Chris
    Sanchirico, Wolfgang Sch\"on, Reed Shuldiner and participants of
    numerous seminars for comments and suggestions. The views
    expressed here are those of the authors and do not necessarily
    represent or reflect the views of the City of Philadelphia.}  \\
        University of Pennsylvania}

\date{\today}

\maketitle

\begin{abstract}

Taxing authorities commonly confront problems of revenue collection
even when the tax base is known.  The problem of tax avoidance has lead to 
extensive theorizing about the motivations of taxpayers.  This paper analyzes the role of different
tax compliance notification strategies in Philadelphia, PA through a randomized experiment of over 20,000 properties 
that were delinquent paying their city real estate tax bills.  Notifications that focus on civic duties, 
public goods, and moral appeal accelerate tax payment but they do not increase payment rates compared to 
a generic reminder of an outstanding balance.  Announcing the threat of a property lien or sale 
increases payment rates. We also find that the size of envelopes has no effect on tax payments.
These findings suggest that deterrence focused property tax notification strategies are more effective
than appeals to civic duties and normative commitments to the provision of public goods.

\noindent KEYWORDS: Tax Compliance, Property Taxation, Field
Experiment, Deterrence, Public Service Appeal, Appeal to Civic Duty.

\end{abstract}
\renewcommand{\thefootnote}{\arabic{footnote}}

\newpage

\section{Introduction}

Property tax plays a key role in how municipalities in the
United States fund operations from personnel to infrastructure.
Roughly 21 percent of all state and local governments’ revenues
come from property taxes.  Property taxees are asssessed based
on market values. Thus, unlike most other taxes that rely on some 
form of self-reporting  (\textit{e.g.}, income, sales,
VAT, or profits), compliance is about collecting a known tax obligation.  
Not surprisingly then the average rate of property tax collection is close
to 95 percent, significantly higher than other forms of tax collection.  
Still, even small differences in property tax collection rates can fundamentally
affect the provision of public goods in economically distressed cities that 
rely heavily on this source of revenue.  Fiscal sustainability, economic efficiency,
and equity are all important motivations for studying the effectiveness of different
strategies for compelling full payment of property taxes.  

A range of theories have been suggested to explain tax compliances. 
First, tax avoidance may be caused by unawareness, as tax forms can be complicated. 
\citet{kosonen}, for example, report that small business owners were more likely to comply
with a publicized change in their VAT tax rate from 9 percent to 23 percent when they 
were clearly explained the change in a general questionnaire
regarding tax administration. 

Second, true obligations may be known to the taxpayer, but they may
choose to cheat.  They can do so in two ways.  When taxes are
self-assessed (\textit{e.g.}, income tax, VAT, profits), taxpayers can
under-report incomes or sales and over-report costs and purchases. 
Individuals and firms can also evade taxes by working outside the formal economy. 
\citet{kleven}  and \citet{pomeranz}
found in Denmark and Chile that reported incomes and value-added
sales increased with an improvement in the ability of tax administrators to
independently assess incomes and sales. \citep{blumenthal} found that taxable 
income by Minnesota residents increased when the probability of an audit increased.  

When tax obligations are known the tax administrators -- as is the case for propery taxes - 
citizens may not comply if the certainty savings from non-payment exceed the uncertain costs they bear from
being caught and fined \citet{allingham}.  In most cases, however, the probability of 
being caught and sanctioned is too low to rationally account for the observed high 
property tax compliance.    Risk aversion also does not offer a good candidate explantion 
for tax compliance\citep{alm}. Efforts to understand taxpayer 
compliance need to consider extensions beyond the narrow framework of 
individual utility maximization under uncertainty.  

There are two extensions of the usual framework to consider. 
The first re-specifies the taxpayer’s utility from income to allow for
non-convex reactions to equal gains and losses.   \citet{engstrom} find among
Swedish taxpayers that loss aversion can account for compliance in 
a way that classical utility maximizing behavior with risk aversion 
cannot.  Taxpayers facing a loss from a \$1000 tax payment were
significantly more likely to overstate allowed deductions than
taxpayers facing a \$1000 refund for the same deductions.  

Second, \citet{luttmer} suggests ``tax morale" is a motivation for compliance.  
They include reciprocity or payment for public goods received; 
norm behavior or peer effects; and civic duty.   Reciprocity argues 
that citizens understand that to not pay their taxes will mean less 
public services.   In this case, government services along with 
after-tax income determine taxpayer welfare.  One would expect this 
motive to be strongest when tax payments are directly linked by the 
taxpayer to services received, for example, local street repairs.   
Peer effects may arise when citizens view non-payment as a violation 
of a community norm of cooperative behavior and an individual’s 
non-payment is observed by others in the community.  Here, how many 
other taxpayers are compliant matters to whether the citizen 
also pays; \citet{posner}.   One might expect this motive to be 
strongest when a citizen’s non-payments are publicized and the citizen 
is actively involved in, or exposed to,  a community group that benefits 
from those payments--for example, a neighborhood school association or 
community oriented church group.  

Finally, citizens may comply with tax obligiations because it is the ``right thing to do" as a citizen.  
Here the act of payment has value on its own; there are no direct 
benefits and no one else need know.  The citizen has accepted the 
democratic contract and bears a presumptive obligation to fulfill 
that contract; see \citet[350-355]{rawls}.  

Efforts to empirically identify the possible influence of these non-economic 
motives have been mixed.   \citet{blumenthal} find no evidence that 
these motives significantly influence truthful reporting of taxable income 
for Minnesota taxpayers, but \citet{hallsworth} find a strong 
beneficial impact on compliance from peer motives.  In a study closest to 
our work here, \citet{castro} examine motives for property 
tax payments in a municipality in Argentina.  They find that the economic 
motives from fines and enforcement are most salient, but that the non-economic 
motives do matter for selected subsamples of the population in 
lower-incomes.   Finally, in our earlier pilot study of property tax 
compliance in Philadelphia, we did find evidence that motives driven by 
reciprocity, peer effects, and civic duty impact payment 
probabilities. But our sample size was small and our framing of the 
alternative motives was not as clear as we would have liked\citet{chirico}.  
We view our work here as chance to pursue all these motives – deterrence, reciprocity, peer influence, and civic duty – 
with a larger sample and with a stronger experimental design.     

In conjunction with the City of Philadelphia, we conducted a multiple 
notification field experiment testing each of these alternatives. The 
results of this experiment demonstrate property tax compliance 
can be improved by sending messages with more specific information on
a) the consequences of non-payment, b) the implied likelihood of 
consequences of continued non-payment, c) the certainty of eventual 
payment. Furthermore, we show that there is minimal evidence to support 
using social information, moral suasion, or other strategies. Finally, 
we show that notification/reminder strategies are not social 
welfare-improving, as they merely accelerate payment. 

% -- Charles --
%Specific Findings
%* Among first time property tax delinquents, we estimate that credible 
%  deterrent letters increase tax compliance by high-single digits. 
%*That this effect is social welfare improving rather than simply shifting.
%*And that affected population is not liquidity constrained.
%*Social norming, moral suasion, and feature engineering have no or minimal effects 

We interpret these results as being consistent with recent research 
on the importance of crafting salient and enforceable threats for 
deterrence to be efficacious \citep{hawken}. Further, these results indicate
that, under circumstances under which continued delinquency is preferable 
for those who are not compliant with paying property taxes and the consequences of delinquency are 
perceived to be minimal, non-coercive nudges or appeals to ciivic duties or public goods are 
unlikely to shift behavior. 

\section{A Tax Notification Experiment}

The research setting for this experiment is the City of
Philadelphia, where just under
18\% of property owners were delinquent and owed 
\$292.3 million in unpaid taxes and another \$223.2 million in interest,
fees and accumulated charges \citep{pew}.
\footnote{
	In Detroit, nearly half of all property owners were delinquent
    on their property tax bills in 2011 \citep{macdonald}.
}
Property tax payments are due on January 1st of each year. If the full balance on a
property tax account is not paid  by March 31st, fees and interest
begin accruing. The City's Department of Revenue, along with outside
collection agencies acting as co-counsel for the City, begin contacting
unpaid accounts on April 1st. Normally, 2/3 of properties are sent to
co-counsel and 1/3 are kept in-house with the Department of Revenue. If the outside firms are able to
collect the outstanding tax from property owners by December 31st of the
tax year, under their contract with the city, they are entitled to 6\% of
the collected revenue. After the new year, remaining delinquent properties
are reassigned to different firms for collection.



% -- Charles --
%Various theories have been offered for why taxpayers would not pay these
%well-reported tax debts. These include discontent with provided public
%goods, low tax morale, missing social norms, or perceived unfairness of 
%taxation (Erard and Feinstein 1994; Kirchler 2007; Torgler 2007). To see 
%whether these perceived deficiencies in non-financial factors among 
%non-taxpayers do in fact explain this form of non-hidden tax delinquency, 
%most recent taxpayer field experiments focus on motivating improved tax 
%compliance via appeals to these instrinsic motivations while contrasting 
%their effects with those produced by more conventional extrinsic 
%motivations (Blumenthal et al., 2001). In the literature, this distinction 
%is also referred to as deterrence versus non-deterrence messaging 
%(Hallsworth et al. 2014).
%
%Social norming or moral persuasion have been suggested as ways of achieving 
%this goal since both involve messaging strategies targeted at possible 
%motivational reasons for tax delinquency (Posner 2000; Traxler 2010; 
%Wenzel 2005). Also, additional notifications and reminders have been proposed 
%as well as manipulation of other non-economic communication factors, 
%including shape, color, and other non-prose cues with the expectation that 
%recipients will be more likely to see and respond to communications in 
%these formats (Thaler and Sunstein 2003). Finally, the use of enhanced 
%sanctions, threats, and even public shaming have been considered. However, 
%with exception of Chirico et al. (2015) and Del Carpio (2014), little work 
%has examined which motivations are at play in the case of non-hidden tax 
%delinquency and consequently what the most effective intervention strategies 
%would be to reduce it.

Of the 579,828 properties in the city in 2015, roughly 100,000 were
% -- Charles --
%citation for 100k number:
%http://www.phillymag.com/citified/2015/07/01/philadelphia-tax-delinquency-lien-sale/
delinquent on their real estate tax bills in the Spring of 2015 when
the experiment began. We included in our sample the universe of 27,264
properties in Philadelphia still owing at least \$10 in Real Estate
Taxes to the City as of May 15, 2015\footnote{
	Given that taxes start accruing penalties as of 
	March 31st each year, this means these properties were 
	nearing three months late by the onset of the 
	experiment in mid-June.}
and excluding chronic delinquents who additionally owed taxes for 
prior years (see citep*{chirico} for more on this population). 
% -- Charles --
% Add debt statistics for properties here.
In coordination with officials at the Department of Revenue, we designed 
seven letter templates probing a variety of tax compliance motivations. 
%TO DO: Number appendices, if appropriate
The templates are included in the Appendix; here we delineate the important 
distinguishing features of each. First, we designed a 
\hypertarget{control}{Control} template, which served simply as a generic 
reminder to taxpayers about their outstanding balance, as well as 
providing contact information and a notice about 
available help for English non-natives. Specifically, we printed:

\blockquote{
	Dear $owner\_name$,

	Our records indicate that you have a balance due of $balance$.

	If you have already paid, thank you. If not, please pay now or contact us
	to arrange a payment plan. The fastest and easiest way to pay is online at
	\underline{www.phila.gov/pay}. Paying by E-check only costs 35c - 
	less than the cost of a stamp!
}

Two more templates were designed with the aim of deterrence in mind, under 
the belief that taxpayers can be convinced to pay by spelling out clearly 
and in tangible terms the potential consequences of nonpayment. The first 
of the deterrence letters, which we call the \hypertarget{sheriff}{Sheriff}
treatment, emphasizes the City's most stringent debt recovery mechanism, 
the Sheriff's Sale, through which the City can auction delinquent properties 
to the public. This treatment contains the following text: 

\blockquote{
	Failure to pay your Real Estate Taxes may result in the sale of 
	your property by the City in order to collect back taxes. In the 
	past year, we have sold $N$ properties in your neighborhood at 
	Sheriff's Sale.	Included in these $N$ are the following properties 
	near you: $<$three properties and their sale dates$>$ 

	Pay your taxes now to prevent the sale of your property. 
	Our records indicate that you have a balance due of $balance$.”
}

$N$ is the number of properties sold in neighborhood between June 2014
and May 2015.
\footnote{\label{fn:neighborhoods}
	Data retrieved from
	\url{
	http://www.officeofphiladelphiasheriff.com/real-estate/sheriffs-sale-webapp
	}
	and then geocoded by the authors; in total, there were 876 properties 
	sold by the City during the year through May 2015. $neighborhood$ is defined 
	in high-density neighborhoods, where at least 8 Sheriff's sales took place,
	as the Azavea neighborhood in which the property is located. The Azavea 
	neighborhoods are a commonly-used partition of Philadelphia into 158 
	colloquially-known (\textit{i.e.}, residents of a neighborhood generally refer to 
	their own area similarly) regions, as specifically created and maintained 
	by the GIS firm Azavea, Inc.; see 
	\url{https://www.opendataphilly.org/dataset/philadelphia-neighborhoods}. 
	Roughly half of properties were in high-density neighborhoods; for the rest, where 
	there were few local Sheriff's sales to offer as warnings to recipients, 
	$neighborhood$ is defined at a slightly higher level of aggregation, namely, 
	the “Azavea quadrant,” as created by the authors to combine Azavea 
	neighborhoods into a total of six larger meta-neighborhoods, namely, 
	Northeast, North, Northwest, West, South, and Center City Philadelphia. 
	%TO DO: Make sure we actually include this map (currently absent)
	See results for the exact composition of these ``quadrants."
}
To generate the listed properties, for each neighborhood
in our sample, we selected three of the properties sold therein in the
previous year at random, meaning residents in the same neighborhood
saw the same set of three properties.
\footnote{
	Initial plans to mention the three nearest sales to each property, 
	to choose three sales at random for each property (not neighborhood), 
	and other variations of these approaches were met with concerns 
	about privacy and abandoned.
}

The second deterrence treatment, which we call the \hypertarget{lien}{Lien} 
treatment, emphasizes a more mild version of debt recovery punishment at the City's disposal, 
namely, that the City can exact a lien on any delinquent property which 
entitles the City to deduce the amount of the lien from any future arms-length 
market transaction involving the property; the text included is as follows: 

\blockquote{
	Failure to pay your Real Estate Taxes will result in a tax lien on your 
	property in an amount equal to your back taxes plus all penalties and 
	interest. When your property is sold, those delinquent tax payments will 
	be deducted from the sale price. By paying your taxes now, you can avoid 
	these penalties and interest. Properties near you in $neighborhood$ that 
	have had liens placed on them include: 
	$<$three properties and their sale dates$>$ 

	Pay your taxes now to avoid a lien being placed on your property. 
	Our records indicate that you have a balance due of $balance$.
}



The variables here are defined as they are in the Sheriff treatment. 
In particular, the triplet of properties assigned to each neighborhood is 
identical across these two treatments. 

The next pair of templates target appeals to taxpayers' sense of citizenship. 
The first of these, which we'll refer to herein as \hypertarget{peer}{Peer},
implicitly posits that one's predilections for tax compliance arise from a 
desire for conformity, and as such aim to highlight that the late payee 
is behaving abnormally with respect to their peers. The specific text reads:

\blockquote{
	
	You have not paid your Real Estate Taxes. Almost all of your neighbors 
	pay their fair share—9 out of 10 Philadelphians do so. By failing to pay, 
	you are abusing the good will of your Philadelphia neighbors.
}
 
We test an alternative manifestation of the idea that citizens pay their 
taxes out of a sense of citizenship by formulating in our fourth treatment 
letter, dubbed \hypertarget{duty}{Duty}, an appeal to the delinquent’s sense 
of duty, in whichever abstract sense the the reader of the letter chooses 
to ascribe to the notion. Specifically, we exhorted tax non-payers in this 
treatment with the following:

\blockquote{
	For democracy to work, all citizens need to pay their fair share of taxes 
	for community services. You have not yet paid your taxes. By failing to 
	do so, you are not meeting your duty as a citizen of Philadelphia.
}

The final pair of letter templates were designed to elicit payment from 
the out-of-hock by highlighting the \textit{quid pro quo} nature of
public good provision – the City provides a bundle of goods and services
in exchange for tax payments by residents, especially real estate taxes.
The first such treatment, which we simply call \hypertarget{moral}{Moral},
targets an appeal to public good provision writ large, and reads:

\blockquote{
	Your taxes pay for important services that make a city great. Your 
	tax dollars are essential for ensuring all Philadelphia children 
	receive a quality education and all Philadelphians feel safe in 
	their neighborhoods. Please pay your taxes as soon as you can 
	to help us pay for these important services. 
}

The final letter, which we call \hypertarget{amenities}{Amenities},
took instead a more localized approach to this appeal, taking the 
further step of mentioning specific local public amenities in the 
vicinity of the letter recipient's property which are in part financed 
by property taxes. The wording for this treatment is:

\blockquote{
	We want to remind you that your taxes pay for essential public services 
	in $neighborhood$, such as $<$two local amenities$>$, your local police officer, 
	snow removal, street repairs, and trash collection. Please pay your taxes 
	to help the city provide these services in your neighborhood.
}

The amenities were chosen at random for each property from a list of 
City-supplied parks, recreation centers, libraries, etc.
\footnote{
	We again use a bifurcation of the definition of $neighborhood$ based on
	the density of amenities in the Azavea neighborhood. Here, we only 
	required that a neighborhood had four amenities supplied to us by the
	City in order to be classified as high-density. Slightly less than 
	half (48\%) of properties were in high-density neighborhoods; the 
	remaining 52\% were assigned amenities at random from their quadrant, 
	as discussed above in Footnote ~\ref{fn:neighborhoods}. 
}

In addition to varying the content of the letters delivered to our sample,
we undertook two additional interventions. The first intervention 
investigated the effectiveness of another commonly-touted  ``nudge" to payment,
namely, manipulation of the envelope itself. The hypothesis behind 
intentionally tailoring the design of the envelope is that a non-negligible 
portion of recipients simply discard the letter prior to opening it 
(and in doing so can never be exposed to the notification treatments); 
by capturing the attention of these casual disposers, the well-designed 
envelope exposes them to the letter contents, thereafter translating into 
higher compliance rates. We tested this hypothesis by delivering half of 
the letters in a standard-sized windowed envelope
(roughly $4\frac18$'' x $9\frac12$''), a treatment we refer to as Small. 
The other half of our sample received full-sheet-sized envelopes (9'' x 12''), 
the Big treatment. 
\footnote{
	The letters in small envelopes were stuffed and delivered by the 
	Department of Revenue's in-house mail room. As they lack the 
	infrastructure to stuff (at scale) large-format envelopes, the authors 
	elected to contract the services of a printing shop, Lawler Direct, Inc., 
	to handle stuffing and delivery of this treatment group. Some 
	coordination issues led to the letters handled by the City being 
	delivered roughly two weeks earlier than the Lawler sample. 
}

For the final intervention, we randomly removed 3,000 properties from 
the possibility of receiving any letter altogether, placing these 
properties in a holdout group. These properties were left entirely 
% -- Michael --
% Need to be careful about wording here, given GRB, etc.
% -- Charles --
% ?
uncontacted during our treatment period. These properties serve to 
elucidate, through comparison to our control condition, our 
understanding of whether or not properties can be induced to act simply 
through the receipt of any sort of correspondence from the City. 
For more details and results of these additional experiments, see Appendix A.

\section{Randomization Procedure}

% -- Charles --
%Experimental Setup
%Blocked randomization
%Timing of experiment, roll-out
%Description
%Randomization Tests
%# properties & owners by letter
%Log balance by treatment (box plots)

Randomization took place in two stages. In the first stage, 3,000 
of the 27,264 eligible properties were randomly assigned 
to the Holdout group. Of the remaining 24,264 properties, 89\% were 
owned by unique owners (owning only one property in Philadelphia) and 
the remaining 11\% were held by owners with multiple holdings in our sample. To avoid 
randomizing separate properties of non-unique property owners into 
multiple treatments, we identified multiple property owners by 
matching the legal name associated with each property. Average 
owner-level debt was \$1810, well above the median of \$907 due to 
upper-tail outliers. In the second randomization stage, owners were 
assigned treatments in blocks – that is, any owner holding multiple 
properties with unpaid real estate tax bills received the same letter 
(and envelope) for each of those properties.
\footnote{
	Lacking an objective identifier of an individual, \textit{e.g.}, social 
	security number, we elected to group properties based on 
	the owner's legal name. This undoubtedly led to distinct 
	individuals with identical names being grouped together, but 
	inspection by the authors found this to be rare; we consider this 
	as random noise and ignore it henceforth.
}
The blocks were defined according to owner-level total debt, given 
the high correlation between propensity to pay and total debt owed, 
to help control variability in the sample and assure equality of 
samples along this very important dimension.
\footnote{
	More specifically, all owners were ordered by total debt and assigned 
	sequentially in groups of 14 to randomization blocks. Within each 
	randomization block, a permutation of the 7x2 = 14 treatments described 
	above was assigned to the properties. The final block, having only 13 
	leftover properties, was assigned a random choice of 13 treatments.
}
In order to account for blocking in subsequent analysis, standard errors 
are clustered at the block level (as randomized) for all regression equations. 

\begin{sidewaystable}[ht]
\centering
\begin{tabular}{|r|lllllll|l|}
   \hline
Variable & Control & Amenities & Moral & Duty & Peer & Lien & Sheriff & $p$-value \\ 
   \hline
Amount Due (June) & \$1,847 & \$2,209 & \$1,954 & \$1,700 & \$1,772 & \$1,735 & \$1,887 & 0.78 \\ 
  Assessed Property Value & \$195,029 & \$224,412 & \$220,963 & \$191,199 & \$165,957 & \$173,690 & \$178,556 & 0.76 \\ 
  \% with Unique Owner & 87.5 & 86.4 & 88.3 & 88.0 & 87.4 & 88.0 & 87.5 & 0.45 \\ 
  \% Overlap with Holdout & 3.72 & 3.80 & 3.47 & 3.47 & 3.58 & 3.47 & 3.29 & 0.96 \\ 
  \# Properties per Owner & 1.33 & 1.36 & 1.29 & 1.27 & 1.26 & 1.32 & 1.26 & 0.55 \\ 
  \# Owners & 2,766 & 2,766 & 2,766 & 2,766 & 2,766 & 2,765 & 2,766 & 1 \\ 
   \hline 
 \multicolumn{9}{l}{\scriptsize{$p$-values in rows 1-5 are $F$-test $p$-values from regressing each variable on treatment dummies. A $\chi^2$ test was used for the count of owners.}} \\ 
\end{tabular}
\caption{Balance on Observables} 
\label{tbl:balance}
\end{sidewaystable}

\begin{sidewaystable}[ht]
\centering
\begin{tabular}{|r|lllllll|l|}
   \hline
Variable & Control & Amenities & Moral & Duty & Peer & Lien & Sheriff & $p$-value \\ 
   \hline
Amount Due (June) & \$1,383 & \$1,950 & \$1,290 & \$1,316 & \$1,338 & \$1,389 & \$1,613 & 0.38 \\ 
  Assessed Property Value & \$163,084 & \$206,214 & \$130,265 & \$166,791 & \$130,936 & \$147,573 & \$155,597 & 0.28 \\ 
  \# Owners & 2,420 & 2,389 & 2,441 & 2,433 & 2,417 & 2,432 & 2,419 & 0.99 \\ 
   \hline 
 \multicolumn{9}{l}{\scriptsize{$p$-values in rows 1-5 are $F$-test $p$-values from regressing each variable on treatment dummies. A $\chi^2$ test was used for the count of owners.}} \\ 
\end{tabular}
\caption{Balance on Observables (Unique Owners)} 
\label{tbl:balance_unq_own}
\end{sidewaystable}

\begin{figure}[htpb]
\begin{center}
\caption{Sample Balance on Initial Debt across Main Treatments}
\label{fig:box_bal}
\bigskip
\includegraphics[width=6in]{dist_log_due_by_trt_7_box}
\end{center}
\end{figure}

Each of the seven letter templates described above was assigned at random 
to envelopes of each size--in other words, each property was equally 
likely to receive any one of fourteen possible envelope+letter combinations. 
Balance tests for pre-randomization characteristics, 
%TO DO: Add balance table for holdout comparison
reported in Table \ref{tbl:balance} and Figure \ref{fig:box_bal}, 
confirm that randomization was successful, 
with no discernible differences between any of our treatment groups.
\footnote{For balance tests for the stage 1 holdout sample, see Appendix.}
While large and multiple property owners introduced small variations 
between groups, the randomization procedures worked as designed with 
blocking serving to limit the influence of these small number of property 
owners to the first two blocks. Excluding these produces perfect balance 
on all variables for the 99.9\% of all properties in the remaining blocks.

\section{Results}

\subsection{Intervention \#1: Targeted Phrasing vs. Control}

Table \ref{tbl:reg7_ep} and Figure \ref{fig:box_bal} report the main 
experimental intervention results at one-, three-, and six-month 
intervals. At the one month mark, several weeks after receiving the 
experimental or control notifications, 35\% of delinquents in the 
control condition have made at least some payments towards their 
outstanding bill, with 24\% having paid the bill in its entirety. All 
intrinsic motivation notifications (\textit{i.e.}, \hyperlink{amenities}{Amenity}, 
\hyperlink{duty}{Duty}, \hyperlink{moral}{Moral}, and \hyperlink{peer}{Peer}) 
are within 1 percent of the control repayment rate and 
statistically insignificant. By comparison, notifications emphasizing 
extrinsic/consequentialist factors both show short-term differences of 
4.9\% (\hyperlink{lien}{Lien}) and 3.4\% (\hyperlink{sheriff}{Sheriff's sale}). 
Both are statistically significant. 

\begin{sidewaystable}[htbp]
\caption{Estimated Average Treatment Effects: Ever Paid}
\begin{center}
\begin{tabular}{|l| c c c| c c c| }
\hline
               & One Month & Three Months & Six Months & One Month & Three Months & Six Months \\
\hline
 & \multicolumn{3}{c}{All Owners} & \multicolumn{3}{|c|}{Single-Property Owners} \\
Amenities      & $-0.04$      & $-0.03$      & $-0.09$     & $-0.09$      & $-0.05$      & $-0.08$     \\
               & $(0.06)$     & $(0.05)$     & $(0.06)$    & $(0.06)$     & $(0.06)$     & $(0.06)$    \\
Moral          & $-0.02$      & $-0.06$      & $-0.02$     & $0.00$       & $-0.04$      & $-0.02$     \\
               & $(0.06)$     & $(0.05)$     & $(0.06)$    & $(0.06)$     & $(0.06)$     & $(0.07)$    \\
Duty           & $-0.05$      & $-0.01$      & $0.04$      & $-0.06$      & $-0.01$      & $0.04$      \\
               & $(0.05)$     & $(0.05)$     & $(0.06)$    & $(0.06)$     & $(0.06)$     & $(0.06)$    \\
Peer           & $0.02$       & $-0.03$      & $-0.02$     & $0.01$       & $-0.02$      & $0.00$      \\
               & $(0.06)$     & $(0.05)$     & $(0.06)$    & $(0.06)$     & $(0.06)$     & $(0.06)$    \\
Lien           & $0.21^{***}$ & $0.20^{***}$ & $0.13^{**}$ & $0.23^{***}$ & $0.22^{***}$ & $0.13^{**}$ \\
               & $(0.05)$     & $(0.06)$     & $(0.06)$    & $(0.06)$     & $(0.06)$     & $(0.07)$    \\
Sheriff        & $0.15^{***}$ & $0.19^{***}$ & $0.11^{*}$  & $0.15^{***}$ & $0.20^{***}$ & $0.13^{**}$ \\
               & $(0.06)$     & $(0.05)$     & $(0.06)$    & $(0.06)$     & $(0.06)$     & $(0.07)$    \\
\hline
Log Likelihood & -12599.09    & -13189.11    & -10624.92   & -10959.60    & -11588.89    & -9495.38    \\
Num. obs.      & 19361        & 19361        & 19361       & 16951        & 16951        & 16951       \\
\hline
\multicolumn{7}{l}{\scriptsize{$^{***}p<0.01$, $^{**}p<0.05$, $^*p<0.1$}}
\end{tabular}
\label{tbl:reg7_ep}
\end{center}
\end{sidewaystable}

\begin{sidewaystable}[htbp]
\caption{Estimated Average Treatment Effects (Excluding top 28 Owners): Ever Paid}
\begin{center}
\begin{tabular}{|l| c c c| c c c| }
\hline
 & \multicolumn{3}{c}{All Owners} & \multicolumn{3}{|c|}{Single-Property Owners} \\
 & \multicolumn{3}{c}{All Owners} & \multicolumn{3}{|c|}{Single-Property Owners} \\
               & One Month & Three Months & Six Months & One Month & Three Months & Six Months \\
\hline
Amenities      & $-0.05$      & $-0.03$      & $-0.09$     & $-0.09$      & $-0.05$      & $-0.08$     \\
               & $(0.05)$     & $(0.05)$     & $(0.06)$    & $(0.06)$     & $(0.06)$     & $(0.07)$    \\
Moral          & $-0.02$      & $-0.06$      & $-0.03$     & $0.00$       & $-0.04$      & $-0.02$     \\
               & $(0.06)$     & $(0.05)$     & $(0.06)$    & $(0.06)$     & $(0.06)$     & $(0.07)$    \\
Duty           & $-0.06$      & $-0.01$      & $0.03$      & $-0.06$      & $-0.01$      & $0.04$      \\
               & $(0.05)$     & $(0.05)$     & $(0.06)$    & $(0.06)$     & $(0.06)$     & $(0.07)$    \\
Peer           & $0.01$       & $-0.03$      & $-0.03$     & $0.01$       & $-0.02$      & $0.00$      \\
               & $(0.06)$     & $(0.05)$     & $(0.06)$    & $(0.06)$     & $(0.06)$     & $(0.07)$    \\
Lien           & $0.21^{***}$ & $0.20^{***}$ & $0.13^{**}$ & $0.23^{***}$ & $0.22^{***}$ & $0.14^{**}$ \\
               & $(0.05)$     & $(0.05)$     & $(0.06)$    & $(0.06)$     & $(0.06)$     & $(0.06)$    \\
Sheriff        & $0.15^{***}$ & $0.19^{***}$ & $0.11^{*}$  & $0.16^{***}$ & $0.20^{***}$ & $0.14^{**}$ \\
               & $(0.05)$     & $(0.05)$     & $(0.06)$    & $(0.06)$     & $(0.06)$     & $(0.07)$    \\
\hline
Log Likelihood & -12582.62    & -13167.95    & -10601.90   & -10954.22    & -11580.00    & -9483.18    \\
Num. obs.      & 19333        & 19333        & 19333       & 16940        & 16940        & 16940       \\
\hline
\multicolumn{7}{l}{\scriptsize{$^{***}p<0.01$, $^{**}p<0.05$, $^*p<0.1$}}
\end{tabular}
\label{tbl:reg7_ep_x28}
\end{center}
\end{sidewaystable}

The corresponding marginal effects are reported in Table \ref{tbl:marg}.

\begin{table}[ht]
\centering
\caption{Participation Rates by Treatment over Time, Unique Owners vs. Holdout} 
\label{tbl:marg_ep}
\begin{tabular}{rlll}
  \hline
Treatment & One Month & Three Months & Six Months \\ 
 Holdout & 30.5 & 51.4 & 73.3 \\ 
   \hline
Control & 3.7*** & 3.9*** & 1.3 \\ 
   & (1.4) & (1.5) & (1.3) \\ 
  Amenities & 1.7 & 2.6* & -0.2 \\ 
   & (1.4) & (1.5) & (1.3) \\ 
  Moral & 3.8*** & 2.8* & 0.9 \\ 
   & (1.4) & (1.5) & (1.3) \\ 
  Duty & 2.4* & 3.6** & 2.1* \\ 
   & (1.4) & (1.5) & (1.3) \\ 
  Peer & 3.9*** & 3.5** & 1.4 \\ 
   & (1.4) & (1.5) & (1.3) \\ 
  Lien & 9*** & 9.2*** & 3.7*** \\ 
   & (1.4) & (1.5) & (1.3) \\ 
  Sheriff & 7.3*** & 8.8*** & 3.7*** \\ 
   & (1.4) & (1.5) & (1.3) \\ 
   
 \hline 
 \multicolumn{4}{l}{\scriptsize{Holdout values are in levels; remaining figures are relative to Holdout}} \\ 
\end{tabular}
\end{table}

Results from logit models using 
paid-in-full (including fees) instead of any payment, reported Table 
\ref{tbl:reg7_pf}, reveal that the majority (69\%) of the estimated 
short-term effect of payment comes from recipients paying off their entire balance.
Three-month results, reported in Column 2 of Table \ref{tbl:reg7_pf}, show a very 
similar picture. As can be seen visually in Figure 1, baseline (control) 
repayment rates have risen to 56\% (any payment) and 42\% (full payment). 
However, intrinsic motivation treatments continue to have no discernible 
effect on repayment rates, while extrinsic notifications have produced a 
4.8\% (Lien) and 4.5\% (Sheriff's sale) response as compared to control. These 
results retain statistical significance and full repayment continues to 
contribute to the majority of the observed effects.

\begin{sidewaystable}[htbp]
\caption{Estimated Average Treatment Effects: Paid Full}
\begin{center}
\begin{tabular}{|l| c c c| c c c| }
\hline
               & One Month & Three Months & Six Months & One Month & Three Months & Six Months \\
\hline
 & \multicolumn{3}{c}{All Owners} & \multicolumn{3}{|c|}{Single-Property Owners} \\
Amenities      & $-0.07$      & $-0.02$      & $-0.04$      & $-0.13^{**}$ & $-0.06$      & $-0.07$     \\
               & $(0.06)$     & $(0.05)$     & $(0.05)$     & $(0.06)$     & $(0.06)$     & $(0.06)$    \\
Moral          & $-0.03$      & $-0.01$      & $-0.01$      & $-0.05$      & $-0.02$      & $-0.02$     \\
               & $(0.06)$     & $(0.05)$     & $(0.05)$     & $(0.06)$     & $(0.06)$     & $(0.06)$    \\
Duty           & $-0.05$      & $-0.03$      & $-0.01$      & $-0.08$      & $-0.03$      & $-0.02$     \\
               & $(0.06)$     & $(0.05)$     & $(0.05)$     & $(0.06)$     & $(0.06)$     & $(0.06)$    \\
Peer           & $0.03$       & $0.03$       & $0.05$       & $-0.02$      & $0.01$       & $0.04$      \\
               & $(0.06)$     & $(0.05)$     & $(0.06)$     & $(0.07)$     & $(0.06)$     & $(0.06)$    \\
Lien           & $0.18^{***}$ & $0.16^{***}$ & $0.16^{***}$ & $0.17^{***}$ & $0.17^{***}$ & $0.15^{**}$ \\
               & $(0.06)$     & $(0.05)$     & $(0.06)$     & $(0.06)$     & $(0.06)$     & $(0.06)$    \\
Sheriff        & $0.12^{**}$  & $0.15^{***}$ & $0.05$       & $0.12^{*}$   & $0.15^{***}$ & $0.06$      \\
               & $(0.06)$     & $(0.05)$     & $(0.05)$     & $(0.06)$     & $(0.06)$     & $(0.06)$    \\
\hline
Log Likelihood & -10741.91    & -13206.17    & -12689.74    & -9662.20     & -11635.40    & -10953.77   \\
Num. obs.      & 19361        & 19361        & 19361        & 16951        & 16951        & 16951       \\
\hline
\multicolumn{7}{l}{\scriptsize{$^{***}p<0.01$, $^{**}p<0.05$, $^*p<0.1$}}
\end{tabular}
\label{tbl:reg7_pf}
\end{center}
\end{sidewaystable}



Six-month results, reported in Column 3 of Table \ref{tbl:reg7_pf}, indicate that final 
participation rates for first-time delinquents are 76\% within six months. 
Nudge-like treatment notifications remain insignificant and barely 
distinguishable from the control response. Consequentialist notifications 
are 2.4\% (Lien) and 1.9\% (Sheriff's sale) higher than Control. The Lien treatment 
thereby is the only treatment to retain significance at conventional 
levels through the end of the follow-up period.
\footnote{
	We terminate follow-up at year's end, when the onset of 2016 tax liabilities 
	begin to muddy the interpretation of repayment and outstanding debt.
}

\begin{sidewaystable}[htbp]
\caption{Estimated Average Treatment Effects: Total Paid}
\begin{center}
\begin{tabular}{|l| c c c| c c c| }
\hline
          & One Month & Three Months & Six Months & One Month & Three Months & Six Months \\
\hline
 & \multicolumn{3}{c}{All Owners} & \multicolumn{3}{|c|}{Single-Property Owners} \\
Amenities & $23.34$      & $-43.30$   & $-15.64$   & $14.39$     & $24.83$       & $26.38$      \\
          & $(35.29)$    & $(47.21)$  & $(52.89)$  & $(34.41)$   & $(40.28)$     & $(39.02)$    \\
Moral     & $41.37$      & $53.50$    & $81.04$    & $4.44$      & $3.03$        & $33.85$      \\
          & $(60.47)$    & $(118.63)$ & $(119.12)$ & $(24.17)$   & $(32.25)$     & $(37.55)$    \\
Duty      & $4.66$       & $-18.82$   & $81.10$    & $-0.82$     & $55.43$       & $99.06^{**}$ \\
          & $(35.15)$    & $(40.50)$  & $(61.58)$  & $(32.29)$   & $(39.59)$     & $(41.78)$    \\
Peer      & $124.26$     & $124.54$   & $113.29$   & $22.37$     & $104.33$      & $108.76$     \\
          & $(82.15)$    & $(133.17)$ & $(132.88)$ & $(35.08)$   & $(71.58)$     & $(71.02)$    \\
Lien      & $83.05^{**}$ & $64.47$    & $21.37$    & $80.37^{*}$ & $107.49^{**}$ & $66.30$      \\
          & $(41.77)$    & $(42.56)$  & $(47.86)$  & $(42.05)$   & $(45.68)$     & $(43.75)$    \\
Sheriff   & $14.81$      & $29.72$    & $-51.82$   & $31.78$     & $81.55^{*}$   & $53.28$      \\
          & $(28.91)$    & $(43.23)$  & $(49.64)$  & $(27.79)$   & $(42.89)$     & $(42.21)$    \\
\hline
Num. obs. & 19361        & 19361      & 19361      & 16951       & 16951         & 16951        \\
\hline
\multicolumn{7}{l}{\scriptsize{$^{***}p<0.01$, $^{**}p<0.05$, $^*p<0.1$}}
\end{tabular}
\label{tbl:reg7_tp}
\end{center}
\end{sidewaystable}

Comparing full payment 
outcomes instead of all payments reveals that the Lien consequentialist 
treatment is 4\% higher than the control mean of 63\% with no other 
treatments having discernible or significant differences.

To provide an estimate of the revenue increases resulting from experimental 
treatments, we fit OLS models for amount paid regressed on treatment status.
The results of these regressions are reported in 
Table \ref{tbl:reg7_tp}. Nonparametric tests with bootstrapped standard errors within 
randomization blocks are reported visually in Figure \ref{fig:tp_time_7_own}. 
These results indicate that the lien treatment was effective at boosting 
revenue collected within the first month of the experiment. Lien recipients 
paid on average \$507.36 during this period, or \$83.05 more than control. However, no statistically 
significant different in repayment occurred over the longer period of 3 or 6 months.

%possible insert sensitivity/heterogeneity results:
%	Amount owed
%	Owner occupied versus not

\subsection{Intervention \#2: Mailer versus No-Mailer}

Table \ref{tbl:reg8_ep} reports the comparison of sending a generic reminder letter 
(control) versus sending no notification letter whatsoever. This 
comparison is restricted to unique property owners.
\footnote{
	Most multiple owners of a Holdout property also own a treated property, which
	serves to preclude a proper apples-to-apples comparison of such owners, so
	we focus instead on contrasting only unique owners in each group.
}

\begin{table}[htbp]
\caption{Estimated Average Treatment Effects: Ever Paid, vs. Holdout}
\begin{center}
\begin{tabular}{l c c c }
\hline
               & One Month & Three Months & Six Months \\
\hline
Control        & $0.17^{***}$ & $0.16^{**}$  & $0.07$       \\
               & $(0.06)$     & $(0.06)$     & $(0.07)$     \\
Amenities      & $0.08$       & $0.11^{*}$   & $-0.01$      \\
               & $(0.06)$     & $(0.06)$     & $(0.07)$     \\
Moral          & $0.17^{***}$ & $0.11^{*}$   & $0.04$       \\
               & $(0.06)$     & $(0.06)$     & $(0.07)$     \\
Duty           & $0.11^{*}$   & $0.14^{**}$  & $0.11$       \\
               & $(0.06)$     & $(0.06)$     & $(0.07)$     \\
Peer           & $0.18^{***}$ & $0.14^{**}$  & $0.07$       \\
               & $(0.06)$     & $(0.06)$     & $(0.07)$     \\
Lien           & $0.40^{***}$ & $0.37^{***}$ & $0.20^{***}$ \\
               & $(0.06)$     & $(0.06)$     & $(0.07)$     \\
Sheriff        & $0.33^{***}$ & $0.36^{***}$ & $0.20^{***}$ \\
               & $(0.06)$     & $(0.06)$     & $(0.07)$     \\
\hline
AIC            & 24503.88     & 26086.64     & 21428.89     \\
BIC            & 24566.71     & 26149.47     & 21491.72     \\
Log Likelihood & -12243.94    & -13035.32    & -10706.44    \\
Deviance       & 24487.88     & 26070.64     & 21412.89     \\
Num. obs.      & 19039        & 19039        & 19039        \\
\hline
\multicolumn{4}{l}{\scriptsize{$^{***}p<0.01$, $^{**}p<0.05$, $^*p<0.1$}}
\end{tabular}
\label{tbl:reg8_ep}
\end{center}
\end{table}

\begin{table}[htbp]
\caption{Estimated Average Treatment Effects: Ever Paid, vs. Holdout}
\begin{center}
\begin{tabular}{l c c c }
\hline
               & One Month & Three Months & Six Months \\
\hline
Control        & $0.12^{*}$   & $0.12^{**}$  & $0.07$       \\
               & $(0.07)$     & $(0.06)$     & $(0.06)$     \\
Amenities      & $-0.01$      & $0.06$       & $-0.00$      \\
               & $(0.07)$     & $(0.06)$     & $(0.06)$     \\
Moral          & $0.07$       & $0.10^{*}$   & $0.05$       \\
               & $(0.07)$     & $(0.06)$     & $(0.06)$     \\
Duty           & $0.04$       & $0.09$       & $0.04$       \\
               & $(0.07)$     & $(0.06)$     & $(0.06)$     \\
Peer           & $0.10$       & $0.14^{**}$  & $0.10$       \\
               & $(0.07)$     & $(0.06)$     & $(0.06)$     \\
Lien           & $0.29^{***}$ & $0.30^{***}$ & $0.21^{***}$ \\
               & $(0.07)$     & $(0.06)$     & $(0.06)$     \\
Sheriff        & $0.24^{***}$ & $0.27^{***}$ & $0.13^{**}$  \\
               & $(0.07)$     & $(0.06)$     & $(0.06)$     \\
\hline
AIC            & 21618.11     & 26109.62     & 24671.52     \\
BIC            & 21680.95     & 26172.45     & 24734.36     \\
Log Likelihood & -10801.06    & -13046.81    & -12327.76    \\
Deviance       & 21602.11     & 26093.62     & 24655.52     \\
Num. obs.      & 19039        & 19039        & 19039        \\
\hline
\multicolumn{4}{l}{\scriptsize{$^{***}p<0.01$, $^{**}p<0.05$, $^*p<0.1$}}
\end{tabular}
\label{tbl:reg8_ep}
\end{center}
\end{table}

\begin{figure}[htpb]
\begin{center}
\caption{Comparison of Partial Participation between Control and Holdout over Time}
\label{fig:ep_time_ch_own}
\bigskip
\includegraphics[width=6in]{cum_haz_ever_paid_control_holdout_own}
\end{center}
\end{figure}


As can be seen visually in Figure \ref{fig:ep_time_ch_own}, 
a significant gap (4\%) in the probability 
of any payment quickly emerged after the letters were mailed. Until 
late August, corresponding to the time when still-delinquent properties 
in the experiment were released to tax collection agents, this gap of 
4\% persisted. After this time, hold-out property owners began paying 
at higher rates, leading to a convergence of trends suggesting that 
while generic notification does accelerate payment, some
combination of tardy participants and other enforcement actions are 
likely to produce a similar ultimate result. 

\subsection{Intervention \#3: Standard Envelope versus Larger Envelope}

Table \ref{tbl:reg2_ep} and Figure \ref{fig:ep_time_2_own} 
report the results of a third tested
intervention--modifying the size of the notification envelope.
Previous research has hypothesized that increasing the visibility or visual 
salience of a notification could increase uptake or response.
However, the results of this particular version of increased visual salience 
suggest that tax authorities sending tax bills are likely to receive little 
benefit from the increased postage cost more eye-catching mailers. 
At one, three and six months post-treatment, payment probabilities 
were similar and statistically indistinguishable. 

\section{Fiscal Analysis}

\textit{Prima facie}, mere acceleration of payment may seem like an
economic null result for the City. However, the timing of payments received
by the City can be important for a variety of reasons, the most salient of
which is resorption of revenue lost to the law firms subcontracted by DoR.
As mentioned, the City hands over enforcement rights of its properties to
outside law firms who use their own means to collect late and delinquent
tax balances in exchange for fees. For properties which are simply late
(not delinquent), the City pays these firms by awarding 6\% of receipts on
accounts less than a year late in payment.

In the wake of our study, the City handed off the entirety of its late 
property holdings on August 15th, 2015. As demonstrated above, in the
absence of our treatments, the City would have been handing off more
properties and more debt (\textit{i.e.}, potential revenue).

We endeavor here to provide some estimates of the potential savings
reaped by the City on account of our letter treatments. The simpler test
entails comparing differential repayment rates in our Holdout group
(who received no letter) with respect to each of the treatment groups 
(including the Control letter). Specifically, we first get 
treatment-by-treatment point estimates of a participation effect,
$\Delta^T$ for each treatment group $T$, given by:

\[
\Delta^T = \frac{N^T_p}{N^T} - \frac{N^H_p}{N^H}
\]

where $N^t_p$ is the count of owners having paid by August 15th in
treatment $T$, and $N^t$ is the total count of owners in $T$. Then,
we repeatedly sample $\Delta^T * N^T$ owners for each treatment and find
6\% of the total repayment by July 23rd (the nearest date to August
15th for which we have accurate repayment totals).

The distribution of revenue effects from these simulations is given
in Figure \ref{fig:rev_prop}; the median revenue effect is about
\$33,000. For comparison, the total bill of contracting a third party
printing service to handle mailing of non-standard envelopes for 
our letter-size experiment was around \$15,000.

\begin{figure}[htpb]
\begin{center}
\caption{Proportional Revenue Effect Histogram}
\label{fig:rev_prop}
\bigskip
\includegraphics[width=6in]{hist_benefit_acceleration}
\end{center}
\end{figure}

A more sophisticated approach is to estimate impacts at the individual 
level. The classic causal framework associates with each individual
his potential outcomes $(Y_i(0), Y_i(1), \ldots, Y_i(k))$ in the
holdout group and the $k$ treatment groups. The causal impact of
treatment $\Delta_i$ is given by $Y_i(T_i) - Y_i(0)$, where $T_i$
is the actually assigned treatment. Since $Y_i(0)$ is unobserved
for all treated owners, we use the following model of outcomes
in each treatment:

\begin{eqnarray*}
Y_i(0) & = & X'_i \beta^0 + \varepsilon_i^0\\
Y_i(1) & = & X'_i \beta^1 + \varepsilon_i^1\\
    & \vdots & \\
Y_i(k) & = & X'_i \beta^k + \varepsilon_i^k\\
\end{eqnarray*}

Where the $X_i$ contain owner-level covariates: log assessed
property value
\footnote{
	This is missing for 50 properties; these are excluded
	given their rarity.
}
, log initial debt, log count of properties owned,
and indicators for residential properties and Philadelphia 
mailing addresses. 

Given randomization, there is on selection into treatment groups,
and we can use estimates of the regression for $Y_i(0)$,
$\hat{\beta}^0$, to construct predicted counterfactuals for 
treated owners, $X'_i \hat{\beta}^0$, leading to causal estimates

\[
\widehat{\Delta}_i = X'_i (\hat{\beta}^{T_i} - \hat{\beta}^0)
\]

The total effect of the experiment is then
$\sum\limits_{\mbox{\fontsize{5}{5}\selectfont Treated} \, i} \widehat{\Delta}_i$, which we 
estimate as roughly \$135,000.

\section{Conclusions}

We report the results of a field experiment testing different notification strategies 
to increase property tax compliance. In the context of this field experiment in municipal 
We find that simple notification strategies accelerate participation but are 
ineffective at reducing the amount of tax delinquency. This finding covers a 
wide array of theorized  ``nudges" including social norming, moral suasion, 
and tax morale. We also find that simply notifying delinquents more often only accelerated
payments and did not change overall property tax compliance. We also find null effects for
increasing the size of envelopes notifiying property owners they are deliquent in paying 
their taxes. Apparently, owners open propery tax bills even if they do not pay them.
In constrast, we find that credible threats to place lien on a property or put it up for 
sale by the city accelerate payments and increase the overall level of property tax compliance.  

For revenue collecting agenciee, acceleration of payment is a useful result 
in and of itself. Bills must be paid and debts must be serviced on regular 
schedules. Relatedly, the longer that tax bills remain unpaid, the more 
expensive it becomes to collect. Downstream collection practices leave 
diminished revenues. Early collection is, 
\textit{ceteris paribus}, better collection. This is not to say, 
however, that early collection is social welfare improving. If tardy 
but eventually-compliant tax delinquents forestall early payment to 
cover other expenses or invest in other assets which they eventually 
use to repay their tax bill with interest, then the welfare of the tax 
agency may not be synonymous with the welfare of society. This is 
especially true if tax payments are made weeks rather than months late, 
such that monthly payments can still be made based on expected monthly 
receipts. However, in the case of persistently-delayed but consistently-paid 
payments, it is less obvious that accelerating eventual or inevitable 
payment constitutes something of value. 

For this reason, the fact that credible threat notifications increase 
repayment rates and convert property tax defiers into compliers is particularly notable. 
It suggests that increasing repayment at low cost is possible. More research is 
needed on the reasons for different behavioral responses to credible threat messages. Perhaps, the difference 
between treatments that merely accelerate payment and those that increase 
payment lie in alternative theories of why tax delinquents exist in the 
first place. Non-consequentialist theories of non-payment implicitly rest 
on the assumption that non-compliers are not liquidity-constrained and are
merely unaware of the collective consequences of non-payment. Under this 
analysis, tax delinquency is due to discouragement, indifference, 
lack of appreciation, or unawareness. Delinquents merely need to be 
encouraged or reminded to participate. Our results, however, suggest 
that property tax delinquents need more than simple encouragement. Providing them with 
% -- Charles --
%Could add sentence here about the envelope experiment.
information about peer behavior, amenities, moral arguments, or civic 
duties does nothing to increase the overall repayment rates. This may 
reflect the fact that delinquents are indifferent to or already aware 
of their peers' positive and negative behavior. Likewise, the provision 
of information on public goods assumes that recipients have not 
incorporated consideration of public goods funded through property tax dollars 
into their payment behavior. If services are considered paid for by 
other taxes or the quality of the public services are considered to be 
subpar, then the rationale for funding these initiatives may not be 
particularly compelling.  Similarly, social contract theories of citizen- 
and state-shared responsibilities offer an idealized vision of civic 
responsibilities that ignores the reality that both sides are chronically 
dissatisfied with the performance of other parties.

At the same time, our results suggest that if property tax delinquents are 
seemingly unaware of the consequences of nonpayment, providing clear information 
about the collection process and that failure to pay will increase transaction costs 
can lead to notable improvements in behavioral compliance. These findings are consistent
with other work on deterrence suggesting that the consistent and timely provision of information on consequences 
from failures to abide by court orders can improve compliance in criminal cases\citep{hawken}. 
Perhaps, then, the puzzle of high non-payment 
rates despite perfect public information on noncompliance can be understood 
as a case of under-enforcement. This possibility is reinforced by the fact 
that conditional on any payment, converted defiers made full payments in almost all cases. 
This suggests that at least for the margin affected, liquidity-constraints are not the primary 
reason for initial non-payment of property taxes and that clear and credible threats matter.

\newpage

\begin{table}[ht]
\centering
\caption{Estimated Revenue Impacts (Ever Paid: Unique Owners vs. Holdout)} 
\label{rev_ep}
\begin{tabular}{rrr}
  \hline
Treatment & Impact Per Letter & Total Impact \\ 
  \hline
Control & \$10.49 & \$25,397 \\ 
  Amenities & -\$1.98 & -\$4,740 \\ 
  Moral & \$7.21 & \$17,588 \\ 
  Duty & \$17.43 & \$42,405 \\ 
  Peer & \$11.26 & \$27,225 \\ 
  Lien & \$31.01 & \$75,420 \\ 
  Sheriff & \$30.67 & \$74,201 \\ 
   \hline
\end{tabular}
\end{table}

\begin{table}[ht]
\centering
\caption{Estimated Revenue Impacts (Paid Full: Unique Owners vs. Holdout)} 
\label{rev_pf}
\begin{tabular}{rrr}
  \hline
Treatment & Impact Per Letter & Total Impact \\ 
  \hline
Control & \$12.13 & \$29,365 \\ 
  Amenities & -\$0.05 & -\$116 \\ 
  Moral & \$8.63 & \$21,055 \\ 
  Duty & \$8.01 & \$19,479 \\ 
  Peer & \$18.81 & \$45,471 \\ 
  Lien & \$39.18 & \$95,277 \\ 
  Sheriff & \$23.73 & \$57,401 \\ 
   \hline
\end{tabular}
\end{table}



\bibliography{references}

\end{document}

