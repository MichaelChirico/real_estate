\begin{document}

\part*{Treatment Compliance and Experiment Implementation/Fidelity}

We received in early June 2015 a file comprised of 27,264 OPA accounts
and some basic relevant information about each. After randomly excluding
3,000 properties at the behest of DoR, we matched properties sharing
an owner's name field, summed the total debt owed by each owner, and
block-randomized treatment to one of 14 groups by total debt owed.
Block randomization was done to assure roughly equal representation
of the debt distribution in each treatment group, and was carried
out logistically by sorting owners by debt and assigning a random
permutation of the 14 treatments to each subset of 14 sorted owners.
Randomly assigned properties were then split into small and large
envelope groups and distributed to our printing partners.

Printing and mailing were carried out in two parallel phases--one
for small and one for large envelopes (done, respectively, by the
DoR and Lawler Direct, Inc., a printing services business in Philadelphia).
Each group used their received samples and letter templates supplied
by the research team to perform mail merges on each of their 7 treatments;
the results of this mail merge process was returned to the research
team in .pdf form and double-checked for accuracy before being processed
into envelopes and delivered.

As a final check on fidelity, we leveraged the frequent typos found
in printed letter addresses which resulted in failed delivery of roughly
five percent of all intended letters; properties with faulty mailing
addresses (or other sources of failed delivery, such as property vacancy)
were returned to DoR. We were able to examine a (by-hand) random sample
of returned letters and triple-check that the assigned treatments
were found in each letter.
\end{document}