\documentclass[12pt,titlepage]{article}

\renewcommand\baselinestretch{1.5}
\setlength{\parskip}{0.08in}
\setlength{\medskipamount}{0.05in}
\textheight 8.0 in
\textwidth 6.0 in
\topmargin 0.25in
\tolerance=11000

\usepackage[utf8]{inputenc}
\usepackage{bbm}
\usepackage{rotating}
\usepackage[pdfencoding=auto,unicode=true]{hyperref}
\usepackage{graphicx}
\usepackage[outdir=./]{epstopdf}
\graphicspath{{images/analysis/act/}{images/}{images/balance/}}
\renewcommand{\thefootnote}{\fnsymbol{footnote}}

\begin{document}

\title{An Experimental Evaluation of Notification Strategies to
  Increase Property Tax Compliance: \\ Free-Riding in the City of
  Brotherly Love} \author{Michael Chirico, Robert Inman, Charles
  Loeffler, \\ John MacDonald, and Holger Sieg\thanks{We would like to
    thank Rob Dubow (Director of Finance), Clarena Tolson (Revenue
    Commissioner), and Marisa Waxman (Deputy Commissioner for
    Assessment of Properties) in the Department of Revenue of the City
    of Philadelphia for their help and support. We would also like to
    thank Reed Shuldiner and Chris Sanchirico as well as participants
    of numerous seminars for comments and suggestions. The views
    expressed here are those of the authors and do not necessarily
    represent or reflect the views of the City of Philadelphia.}
  \\ University of Pennsylvania \\ \\ Prepared for the 2015 NBER
  Conference on Tax Policy and the Economy} \date{\today}

\maketitle

\begin{abstract}

This study evaluates a set of notification strategies intended to
increase property tax collection. We develop a field experiment in
collaboration with the Philadelphia Department of Revenue.  The
different treatments emphasize aspects of behavior that play a large
role in the tax compliance: deterrence, the need to finance the
provision of public goods and services, and the desire for peer
conformity. Our empirical findings provide evidence that both moral
appeal to finance public goods and services and the desire for peer
conformity modestly improve tax compliance, while deterrence
notifications are no different from standard notifications.

\noindent KEYWORDS: Tax Compliance, Property Taxation, Field
Experiment, Deterrence, Moral Appeal, Peer Conformity.

\end{abstract}

\newpage

\renewcommand{\thefootnote}{\arabic{footnote}}

\renewcommand{\thefootnote}{\arabic{footnote}}

\section{Introduction}

The purpose of this study is to evaluate a set of notification
strategies intended to increase property tax revenue collection.  We
design several notification-based treatments motivated by the
theoretical literature on tax compliance.  Extrinsic motivations are
emphasized by Becker's deterrence model which explains tax
compliance as a combination of penalties and costly
monitoring; intrinsic motivations are captured by models that rely on
social norms or social contracts. 
To weigh the empirical importance of these competing narratives, the basic strategy behind our
controlled, randomized experiment is to vary the informational content
of notification letters sent to citizens late or delinquent on their property tax payment.
This field experiment was implemented in collaboration with the Philadelphia 
Department of Revenue (DoR).

While most of the previous tax compliance literature has focused on
collection of income taxes, we focus on property tax
collection.\footnote{The empirical literature on tax compliance is
  discuss in detail in Hallsworth, List, Metcalfe and Vlaev (2014).
  Other notable papers that use field experiments to study tax
  compliance are Blumenthal et al (2001), Kleven et al. (2011), Ariel
  (2012), and Pomeranz (2013).}  Property taxes in the United States play an integral role
in financing local municipalities and their school
districts\footnote{It is worth noting here that the U.S. provides more tax authority to local
  municipalities than do most other developed countries.} %Should cite something here, perhaps...
  In theory,
levying a property tax is straightforward.  It requires only
valuation of the property and a registery of the identities of
property holders. There is no need for tax payers to deal with
complicated forms or to give up personal information as with an income
tax.  Since property cannot be hidden, scuppered away to a tax haven, or
concealed in an electronic data system, the tax is difficult to evade.
Moreover, there are some compelling theoretical arguments for
taxing property instead of its alternatives (income, consumption, and
wealth).\footnote{Two common arguments for the property tax are that
  a) the burden of the tax is fairly and equally distributed among the
  residents and firms located in a municipality; and b) that the
  excess burden of the tax is likely to be low.}
  %The Economist mentioned some other reasons to favor LAND taxes in some articles:
  %http://www.economist.com/blogs/economist-explains/2014/11/economist-explains-0
  %http://www.economist.com/news/briefing/21647622-land-centre-pre-industrial-economy-has-returned-constraint-growth

The practice of property tax collection is a lot more
ambiguous.\footnote{The property tax base, which is the assessed value
  of the house, is established in a separate assessment process in
  most cities in the U.S. While there are frequent disagreements about
  correct property tax assessments, tax enforcement is dictated by the
  assessed value of the property.} In particular, property tax
compliance is a significant problem in many large
U.S. cities. Philadelphia is a prominent example.  A recent study by
the Pew Charitable Trust (2013) focused on a sample of 36 cities and
found that Philadelphia had the fifth-highest delinquency rate in
2011, the last year for which common statistics were available.  As of
April 2014, the city and school district were owed \$595 million in
delinquent taxes on 134,888 properties.

Enforcing property tax collection on
delinquents\footnote{In Philadelphia, tax payers are considered
delinquent if their debt is still outstanding at the close of the calendar year.}
is restricted by the limited set of sanctions
that are available to local municipalities.  In practice, a city
typically has three options. First, a tax lien may be imposed on a
property. Tax liens are an efficient tool to force delinquent tax payers to settle their
debts for properties that are sold in official, arms-length market transactions. The outstanding
debt can typically be recovered as a condition of sale when the title of the property needs
to be transferred to a new owner. Unfortunately, this punishment hardly
constrains any owner who has no intent of selling their property
on the open market--for example, if they intend to unofficially bequest
the title to their children or relatives--
since the owner does not need to transfer the title of
the property. Liens are also not effective for any owners that hold 
properties for purely speculative purposes, since such owners will only
flip the property if they expect sufficiently large returns. Tax
liens can also, in principle, be sold to collection agencies to generate
immediate revenues for a city. Many cities, however, are reluctant to
pursue this option, because there are some potentially large political costs
associated with outsourcing tax collection enforcement.

Second, a city can start a foreclosure, a legal process in
which the city as a lien holder attempts to recover the balance of the
debt of the property owner by forcing the sale of the property on
which a lien has been placed.  In a tax foreclosure, the local jurisdiction
petitions a court to award it the title based on an unpaid lien. 
Properties with court-ordered foreclosures can be
 sold in an auction or  at a sheriff's sale.\footnote{The Office of
the Philadelphia Sheriff completes about 60 Sheriff's Sales each month.}
Obtaining a court-ordered foreclosure is a difficult and costly process. While
it may be sensible for a bank to bear these costs 
when a lender defaults on a large mortgage, it
seems as a property tax collection tool to be rather crude.
Foreclosures can also be politically unpopular since they 
have the potential to generate the headlines which create
frictions between voters and local politicians.

Lastly, a city can try to engage a tax delinquent in a bargaining
process prior starting the foreclosure process. If a property is in
danger of foreclosure due to a tax delinquency, DoR will, upon
request of the debtor, attempt to work out a schedule for resolution
of the balance due over a period of several months, usually a
year\footnote{In fact, as of October 2013, the structure of such an
agreement is legally prescribed through Owner-Occupied Payment
Agreement (OOPA) plans. OOPA plans basically determine a monthly 
payment as one of four income-dependent percentages of monthly income}. While DoR has
had some success in negotiating tax payment plans in Philadelphia, it
should be emphasized that participation in these negotiations is
purely voluntary. In general, the bargaining power in these negotiations is
ultimately determined by the likelihood of enforcing liens or foreclosure.

Given the problems associated with the traditional methods of
enforcing property tax compliance, there is some interest in exploring
the efficacy of alternative, softer approaches. The main idea behind these softer
approaches is to use the power of persuasion and design an effective
notification strategy. The purpose of this paper is to provide new
insights into the effectiveness of different notification strategies
for increasing property tax collection. Our intervention focuses on
three treatments that are motivated by the recent literature on tax
compliance.

The first treatment is motivated by the deterrence model of tax
compliance, which emphasizes external motivations of behavior (Becker,
1968, Allingham and Sandmo, 1972). Key elements of deterrence are the
probability of detection (the monitoring probability) and the
seriousness of the penalty or fine that is levied on tax evaders who
are caught.  Given the uncertainty of tax enforcement, tax compliance
is primarily a function of subjective perceptions of the probability
of detection and the seriousness of the penalties for being caught
evading taxes. Our experiment does not explicitly vary the penalties
associated with non-compliance, i.e. we do not randomize based on
different fiscal incentives to pay taxes.  Neither do we expose tax
payers to different enforcement regimes.  Instead, we inform the
delinquent tax payers that the DoR will pursue their failure to
pay property taxes, and that evading property taxes has serious
repercussions.

Proceeding, the recent literature in public economics emphasizes that property
taxes are used to provide local public goods and services. We can
thus study the problem of tax compliance using insights of the literature
on voluntary public good provision (Bergstrom, Blume \& Varian,
1986). Paying property taxes may also be perceived by many households as the
``right thing to do", i.e. there may be some warm glow associated with tax
compliance (Andreoni, 1989).\footnote{ Parallels are drawn between the
  problem of raising taxes and the fund-raising problem faced by
  non-for-profit organizations (Andreoni, Erard and Feinstein,
  1998). For example, publicly funded radio stations typically need to
  raise funds from their listeners and cannot rely on threats of
  enforcement.}  The second notification strategy focuses on this
rationale for paying taxes. It reinforces the importance of public
goods and services that are provided by the city and points out that
these services cannot be provided without a fair distribution of the
tax burden; we interperet this as a moral appeal to pay taxes. The
treatment emphasizes the \textit{quid pro quo} of tax compliance in
the city--i.e., public goods and services (namely, education and safety
provision) funded by compliant taxpayers.

The third notification treatment builds on the insights of the
recent literature on social norms.\footnote{This literature also
  emphasizes concepts such as social fairness and reciprocity (Fehr
  and Gachter, 1998).} In particular, tax payers may comply with tax
laws due to peer pressure and the desire to conform to acceptable
standards of behavior (Bernheim, 1994). In principle, one could vary
the conformity costs within a randomized controlled experiment. For
example, one could threaten delinquent tax payers with public exposure
by printing names in newspapers or public web pages. In practice, such
interventions faces large legal hurdles and are, therefore,
onerous to implement; we instead focus on a simpler and softer
strategy.  The basic idea behind our peer conformity treatment is to
notify non-compliant individuals that their peers and neighbors are
paying taxes.

To implement our experiment we needed to integrate our three treatment
conditions into the standard operating procedures of the billing
system of the DoR. Fortunately, DoR assigns properties to billing
cycles using a pseudo-randomized mechanism, in which assignments are
primarily based on the last two digits of social security or employer
identification numbers.  We then assigned our three treatments
(deterrence, social contract, peer conformity) and the control condition
of basic payment notification to days in the billing cycles using randomly
drawn four-day sequences.

We conducted our experiment in November 2014 over a period
spanning 9 treatment days.  We assessed the fidelity of our
experiment by analyzing letters that were returned as undeliverable to DoR by the
U.S. Postal Service. Based on our analysis of the returned letters we
concluded that our treatment had been correctly assigned on these
days; our final sample consists of 4,927 properties. Based on a
variety of balance tests, we conclude that treatments were randomly
assigned within our sample. The DoR started to share data on property
tax payments immediately after the experiment. Our current sample is
based on all payments through January 6\textsuperscript{th}, 2015. We plan to expand our
sample until March 2015.

Our empirical analysis focuses on two sets of outcome measures. First,
we follow the recent empirical literature on tax compliance and
analyze discrete outcomes. Specifically, we would like to know whether each
owner paid any or all property taxes during the follow-up observation period.
Second, we examine the magnitude of payments received by DoR. We
normalize the received property tax payments by the assessed value of
the property.  Our findings show some evidence that both
moral appeal and peer conformity help to improve tax compliance. We
find that the deterrence notification performs no different than the
standard billing notification.\footnote{Other empirical studies that have
  failed to provide support for the deterrence model are Alm (1999)
  and Togler (2002).}

The rest of the paper is organized as follows. Section 2 discusses the
institutional background, focusing on property tax collection in
Philadelphia.  Section 3 provides a detailed discussion of our three
treatments and the control. Section 4 discusses the experimental
design and the fidelity of its implementation. Section 5
presents a descriptive analysis summarizing the main effects of our experiment. 
Section 6 provides a formal analysis and discusses the estimates of the treatment
effects.  Section 7 offers some conclusions and discusses future research. 

% Appendix A provides a more technical discussion of the key modeling 
% issues and motivates our treatments.


\section{Institutional Background}

Real estate taxes in Philadelphia are levied annually on a
property-level basis.  The Office of Property Assessment evaluates the
market value of each property, 1.34 \% of which must be paid to the
Philadelphia Department of Revenue. The city then splits property tax
revenue with the School District of Philadelphia, with the former
getting approximately 45\% revenues.  Tax bills are mailed by DoR in
batches throughout December and early January each year; owners
have until March 31\textsuperscript{st} to remit their balance to the City, after which
time the owners are considered late and the bill begins to accrue penalties and interest.

The DoR actively begins pursuing non-paying properties in the
September following nonpayment.  First, the city delegates roughly
$\frac{2}{3}$ of the debts to the authority of two designated external
law firms ($\frac{1}{3}$ each), which have contracts with the city for delinquent property
tax collection.\footnote{Currently, these law firms are Linebarger,
  Goggan, Blair \& Sampson and Goehring, Rutter \& Boehm.} These law
firms are free to pursue the collection of the debt as they see fit,
and are rewarded with a portion of any debts recovered for the City.

For those debts that remain targeted by DoR itself, the city
traditionally leverages one of the several legal options
described above as threat and punishment for nonpayment.
Beginning on March 31\textsuperscript{st}, the city regularly
sends plain bills to properties still in hock--roughly once every 10
weeks.\footnote{See the section on implementation below for exact
  details.} More substantive enforcement strategies begin when
the properties become officially delinquent on December 31\textsuperscript{st}.

Philadelphia operates with relatively wide latitude under the
Pennsylvania Municipal Claims and Tax Lien Law, enacted in 1923. The
city can place a lien on any property that has not yet paid real estate
taxes 9 months past the March 31\textsuperscript{st} deadline. A tax lien may be imposed
for delinquent taxes owed on real estate as a result of failure to
pay taxes.  A tax lien takes precedence over all other claims and
gives the holder of the lien basis for legal action, including
foreclosure.  Tax liens typically include the principal tax amount,
plus any interest and penalties.

Having obtained a lien the city can technically start a foreclosure
process. This is the legal process of seizing title of a property (or
the deed) and forcing its sale for the purpose of paying off a debt,
such as a tax lien. In a tax foreclosure, the local jurisdiction
petitions a court to award it the title based on an unpaid lien. The
jurisdiction may then sell the property in a tax-deed sale or auction,
hoping at least to recover the amount of the tax lien. The original
owner usually has the right to regain the property if she pays the
back taxes within a set time period after the sale, called the
redemption period. If nobody buys the deed, the tax lien remains
unpaid, and the jurisdiction keeps the title and responsibility for
the property. A Sheriff's Sale is basically a public auction of the
property, whereby all ownership rights to the property are stripped
from the former owner upon successful bid.\footnote{A list of
  properties currently up for Sheriff's Sale by the City of
  Philadelphia can be found at
  \it{http://www.officeofphiladelphiasheriff.com/en/real-estate/foreclosure-listings}}

For commercial properties, the city may also sequester income from the
business, become the property manager, or even shut down the
business. City employees may see their taxes deducted from their
salaries, and applicants for municipal jobs are vetted for
eligibility--new employees must either be fully repaid or have
started a payment plan. Philadelphia has recently begun to explore
several alternative enforcement options for property tax delinquents,
including levying liens on owners' home addresses outside Philadelphia
(targeting delinquent nonresident landlords) and impounding cars.

A recent study by the PCT (2013) concluded that ``compared to laws
governing delinquency collection in some other states and other
Pennsylvania counties, the state statutes governing Philadelphia give
city government a lot of discretion in setting policies on when to
initiate foreclosures or what kind of catch-up payment plans to
offer. In the past, Philadelphia has tended to use this discretion to
delay taking action, put up fewer properties for sale, or let
delinquents enroll and default on payment plans many times, all of
which has caused delinquencies to accumulate over the years." As a
consequence, threats of enforcement of property tax
collection may lack sufficient credibility in Philadelphia.

\section{Treatments}

To explore softer avenues for revenue augmentation, we determined
that the most logistically feasible approach was to include a
``message'' in the bills regularly mailed to non-payers, the language
of which was carefully chosen to target one of three aforementioned enforcement
strategies: deterrence (the ``Threat'' treatment), social normality
(the ``Moral'' treatment), and conformity (the ``Peer'' treatment). To
properly isolate the treatment effects of receiving a specific message
from the effects of receiving \textit{any} message given the status
quo of plain bills, we also randomly sent properties a nondescript message
(the ``Control'' treatment).\footnote{See Figure 1 for the original due letter and 
  Figures 2-5 for the exact style and full wording of each treatment letter.}
 % Appendix A provides a more technical discussion of models of tax compliance 
 % that motivate our experimental treatments.
 
The letters were designed carefully to differ only in the wording of
their middle paragraphs; for clarity, the crux of the idiosyncratic wording is
reiterated below. We also took care to minimize issues of communication
for those with limited literacy by simplifying the language of the
three different messages--shunning uncommon words and syntactic
complexity by churning the text through the latest linguistic software
for complexity analysis.

Further, in accordance with the DoR's general desire to reach out to
Philadelphia's substantial immigrant and refugee populations, we agreed to include Spanish
translations of each treatment's text on the reverse of our
stuffers.\footnote{As Puerto Ricans make up the plurality of
  Philadelphia's non-English-speaking population, we targeted the vernacular of
  the translation towards Caribbean Spanish speakers.}
  
\subsection{Treatment 1: Deterrence}

The goal of this treatment is to emphasize the repercussions
of not complying with property tax requirements with the city,
highlighting the policy tools available to the city. This message was
intended to educate owners who may have poorly-formed notions of the
extent of actions the city may be willing to take to recover taxes from
each property. The message contained the following sentences:

{\it Not paying your Real Estate Taxes is breaking the law. Failure to
  pay your Real Estate Taxes may result in seizure or sale of your
  property by the City. Do not make the mistake of assuming we are too
  busy to pursue your case.}

\subsection{Treatment 2: Moral Appeal \& Public Services}

The goal of this treatment is to emphasize the social contract
between tax payers and the city. The city can only provide
public goods if property taxes are paid punctually.  In particular, we
chose to highlight the correspondence between Real Estate Tax
compliance and the City's provision of public education and safety
services. The message contained the following sentences:

{\it We understand that paying your taxes can feel like a burden. We
  want to remind you of all the great services that you pay for with
  your Real Estate Tax Dollars. Your tax dollars pay for schools to
  teach our children.  They also pay for the police and firefighters
  who help keep our city safe.  Please pay your taxes as soon as you
  can to help us pay for these essential services.}

\subsection{Treatment 3: Social Pressure \& Conformity}

The goal of this treatment is to socially shame delinquents by
underlining their nonconformity compared to their neighbors.  The
message contained the following sentences:

{\it You have not paid your Real Estate Taxes. Almost all of your
  neighbors pay their fair share-- 9 out of 10 Philadelphians do
  so. Paying your taxes is your duty to the city you live in. By
  failing to pay, you are abusing the good will of your Philadelphia
  neighbors.}

\subsection{Control}

The control group message was designed to be as plain and detached as
possible, with the goal of the extra informational content being orthogonal
to each of the three treatment messages.  The control condition
contained the dispassionate billing notification:

{\it The enclosed bill details your outstanding Real Estate Taxes due
  to the City of Philadelphia.}

\section{Experimental Design}

\subsection{Randomization}

Our approach to randomization was constrained by the logistics of
DoR's enforcement faculties. We concluded after several discussions
with our collaborators at DoR that it would be in practice impossible
to assign individual properties at random to different treatments. Instead, we
chose to exploit the pseudo-random assignment of properties to billing
cycles and randomized treatments across them.  To understand
this decision it is useful to discuss the current practice of 
posting reminder letters by DoR.

Mailing of delinquent real estate tax bills works essentially as
follows.  Every property in the city is assigned to one of 50 mailing
cycles. Since it is cheaper and simpler to send at once all bills to
those owners owing taxes on multiple properties, assignment to cycles
is done at the owner level, so that each mailing cycle has roughly the
same number of owners.  Every morning, a printer at DoR taps the
in-house accounting system to find all properties that a) owe taxes to
the City and b) are in the current day's mailing cycle, with the
numbered cycles progressing in sequence day-by-day.
After identifying the bills to be printed for the day,
the printer merges into the bill several other pieces of information
stored with the delinquent balance such as the mailing address and an
in-house ID associated with the property. The 1200 or so bills that
are printed each day are then brought to the City's mailing room,
wherein they are stuffed into envelopes and delivered to the property
owners.

Given the volume of bills printed each day and the existing
infrastructure for processing them, especially the machine-automated
process of envelope stuffing, we determined the most practical
solution would be to randomize treatment at the mailing cycle level,
so that every bill printed on the same day would be paired with the
same message. Randomization of mailing days was handled by the
authors. We elected to randomize 4-day cycles--for each 4-day period,
we picked at random among the $4!=24$ possible arrangements of
treatments over the subsequent 4 days. Our experiment was conducted
on 9 days in November 2014, between the 4\textsuperscript{th} and
the 25\textsuperscript{th}.

While we are certain of the sanctity of our mailing cycle-level
randomization process, one may be concerned about the assignment of
properties to mailing cycles by the city. Fortunately, however, the
city uses a pseudo-random mechanism to assign owners to billing
cycles, which means that we achieve proper full-scale two-stage
randomization of the properties through our process of day-level
randomization.

In particular, the city assigned properties to cycles based on the
last two digits of an in-house ID number; those with final two digits
01 and 02 are mapped to cycle 1, those with final digits 03 and 04 are
mapped to cycle 2, and so on. The in-house ID itself is motley in
nature. For many properties, DoR has on file the owner's Social
Security Number (SSN); for many others, mainly commercial properties,
the DoR stores their Employer Identification Number (EIN); and for the
remainder of properties, DoR assigns its own in-house ID number. This
last is a 9-digit code which is assigned sequentially to property
owners who cannot be matched to either of the federal ID
numbers. While this assignment based on SSN or EIN is not purely
random, it is a pseudo-random assignment. It is hard to believe that
there would be any significant sorting or self-selection based on the
last two digits of SSN or EIN.

\subsection{Implementation Fidelity}

To assess the fidelity of the experimental design, we leveraged a
unique feature of the system. The Department of Revenue regularly
posts envelopes destined for addresses that are either unattended
(vacant) or do not exist in the first place due to typos. Either
before or after an attempted delivery to such an address, the postal
service flags down these missives and returns the bills to DoR, which
then processes them and attempts, if they can identify a suitable
alternative address, to re-deliver the tax bill. We took advantage of
the fact that a subset of bills made their way back to DoR to check
first-hand the extent of treatment fidelity. Our final sample consists
of the nine treatment days for which greater than 90\% fidelity was
achieved.

\subsection{Sample Size}

Our sample is derived from the database of 134,888 delinquent properties in the
City of Philadelphia. Figure \ref{fig:map_amt} plots the properties
using different colors to indicate the total outstanding tax
liabilities.  We find that the delinquent properties are fairly evenly
spaced throughout the city. As some confirmation of this, Figure \ref{fig:map_rit_amt}
plots the Rittenhouse Square neighborhood in Center City, one
of the most affluent neighborhoods in Philadelphia. Despite the 
uniformly high property values there, we find interspersed there a
fair number of delinquent properties.

Figure \ref{fig:map_time} plots the properties with different colors
indicating the duration of the tax liability. We find that
most liabilities are less than 2 years old. Nevertheless, there are also a
fair number of cases involving debt that has been outstanding for
more than a decade. These cases seem to be more prevalent in South Philadelphia.

From this original sample, we obtained the final sample 
of 4,927 properties that was used in our experiment by using
the following screening devices:
\begin{enumerate}
\item Payment agreement (23\%=31456)
\item Any tax abatement (5\% = 4706)
\item Not handled by DoR (62\%=61170)
\item Sheriff's Sale (11\%=4098)
\item Bankruptcy (3\%=948)
\item Sequestration (3\%=1130)
\item Returned mail flag (5\%=1429)
\item Not mailed during treatment period (83\%=24800)
\item Paid off all but \$0.61 of debt by mailing (4\%=224)
\end{enumerate}

Note in particular that this sample selection means
that our sample consists only of properties that are not
in the purview of the two law firms that DoR uses as collection
agencies. It is therefore useful to compare briefly the properties that are kept
in-house with those that are assigned to the law firms. We find that
properties kept in-house have lower balances, with a median of
\$1,000, as compared to \$1,700 overall. However, in-house properties have higher
market values--the DOR median is \$91,000 vs. \$66,100 overall. Properties
handled by DoR have younger debt--an average of 4 years vs. 7 and 11 for the
two law firms.  Even conditional on age of debt, in-house balances are
low.  DoR-managed accounts are more likely to be owner-occupied, less
likely to be in payment agreements, and more likely to result in a
sheriff's sales. In summary, it appears that the outside firms are
holding properties which, even given other characteristics, have
the highest potential returns.

\subsection{Sample Balance on Observables}

To confirm whether or not we indeed achieved randomization, we
performed a series of balance-on-observables tests. The null
hypotheses of these tests are that a given observable data moment is
identical across mailing cycles. We turn now to the results of those
tests.

Analysis of balance on observables is complicated by the random
assignment at the owner level.  Because there are some large holders
of property--thousands of properties owned by public entities like
the City of Philadelphia, the Philadelphia Housing Authority, and the
Redevelopment Authority of Philadelphia, hundreds owned by many others
such as the University of Pennsylvania and Drexel University--a
simple analysis of balance at the property level will likely be skewed
by these outliers. In addition, it is not clear how to aggregate many
of the property-level characteristics to the owner level meaningfully,
especially geographic variables, complicating the task of testing 
balance at the owner level. Our compromise was
to examine sample balance on the subset of properties for which a) the
owner is unique, and b) any tax exemption claimed by the property is
related to abatements for new construction.\footnote{We ran several
  other similar specifications, with the qualitative results remaining
  unchanged. We also ran tests on the subsample of properties for
  which we could obtain the secure ID used by the City, for which the
  putative mapping was violated; again, the results are qualitatively
  identical. See the Appendix for details.}

\begin{table}[htbp]
\centering
\caption{Tests of Sample Balance on Observables} \label{table:balanceI}
\begin{tabular}{rlllll}
\hline
  & Threat & Moral & Peer & Control & $p$-value \\ 
\hline
 Balance Due Quartiles &  &  &  &  &  \\ 
  $<$\$300 & 0.22 & 0.4 & 0.28 & 0.1 & 0.2 \\ 
  \lbrack\$300,\$1300) & 0.24 & 0.47 & 0.22 & 0.08 &  \\ 
  \lbrack\$1300,\$3300) & 0.23 & 0.45 & 0.2 & 0.11 &  \\ 
$>$  \$3300 & 0.18 & 0.48 & 0.23 & 0.11 &  \\ 
   \hline
Market Value Quartiles &  &  &  &  &  \\ 
  $<$\$46k & 0.24 & 0.43 & 0.21 & 0.12 & 0.2 \\ 
  \lbrack\$46k,\$82k) & 0.22 & 0.46 & 0.23 & 0.1 &  \\ 
  \lbrack\$82k,\$152k) & 0.21 & 0.45 & 0.25 & 0.09 &  \\ 
$>$  \$152k & 0.21 & 0.45 & 0.24 & 0.1 &  \\ 
   \hline
Land Area Quartiles &  &  &  &  &  \\ 
  $<$800 sq. ft. & 0.22 & 0.45 & 0.23 & 0.1 & 0.83 \\ 
  \lbrack800,1200) sq. ft. & 0.23 & 0.43 & 0.24 & 0.1 &  \\ 
  \lbrack1200,1800) sq. ft. & 0.21 & 0.47 & 0.22 & 0.1 &  \\ 
  $>$1800 sq. ft. & 0.21 & 0.44 & 0.24 & 0.1 &  \\ 
   \hline
Distribution of Properties & 0.22 & 0.45 & 0.23 & 0.1 & 0.08 \\ 
   \hline
Expected Distribution & 0.22 & 0.44 & 0.22 & 0.11 &  \\ 
   \hline
\end{tabular}
\end{table}

\begin{table}[htbp]
\caption{Tests of Sample Balance on Observables (cont.)}  \label{table:balanceII}
\centering
\begin{tabular}{rlllll}
\hline
  & Threat & Moral & Peer & Control & $p$-value \\ 
  \hline
 \# Rooms &  &  &  &  &  \\ 
  0-5 & 0.22 & 0.44 & 0.23 & 0.11 & 0.32 \\ 
  6 & 0.21 & 0.46 & 0.23 & 0.09 &  \\ 
  7+ & 0.22 & 0.44 & 0.24 & 0.1 &  \\ 
   \hline
Years of Debt &  &  &  &  &  \\ 
  1 Year & 0.23 & 0.43 & 0.24 & 0.09 & 0.32 \\ 
  2 Years & 0.22 & 0.44 & 0.24 & 0.1 &  \\ 
  3-5 Years & 0.2 & 0.48 & 0.22 & 0.1 &  \\ 
  6+ Years & 0.2 & 0.47 & 0.2 & 0.13 &  \\ 
   \hline
Category &  &  &  &  &  \\ 
  Residential & 0.22 & 0.45 & 0.23 & 0.09 & 0.07 \\ 
  Hotels\&Apts & 0.2 & 0.45 & 0.23 & 0.12 &  \\ 
  Store w. Dwell. & 0.21 & 0.48 & 0.22 & 0.09 &  \\ 
  Commercial & 0.15 & 0.5 & 0.24 & 0.11 &  \\ 
  Industrial & 0.27 & 0.42 & 0.2 & 0.11 &  \\ 
  Vacant Land & 0.25 & 0.39 & 0.23 & 0.13 &  \\ 
   \hline
Expected Distribution & 0.22 & 0.44 & 0.22 & 0.11 &  \\ 
   \hline
\end{tabular}
\end{table}


Most of the observed characteristics are categorical variables, so we can
test balance using standard $\chi^2$ tests. We had two days of threat
treatment, four days of moral treatment, two days of peer treatment
and one day of controls.\footnote{Initially, we had hope to have four
  days for each treatment, but the experimental design was compromised
  on 6 of the days in which we ran the experiment. We, therefore, had
  to exclude these observations from our sample.} Hence the expected
population frequencies should be approximately 0.22, 0.44, 0.22 and
0.11 respectively. Tables 1 and 2 report the observed
frequencies for each variable as well as $p$-values for the
categorical $\chi^2$ test. In addition, we have three continuous
variables: balance due, market value and land area. Since there are
some outliers in the data, we convert each of these continuous
variables into categorical variables using quartiles of the underlying
empirical distribution to mitigate their influence. We then used the same
categorical $\chi^2$ test to determine whether our sample is balanced
on these continuous measures.

As can be seen in Tables \ref{table:balanceI} and
\ref{table:balanceII}, randomization appears to have been successful.
The properties are strongly randomly distributed by location (their
political ward, of which there are 66 in Philadelphia), category (type
of property usage), property size (as measured by the number of rooms
or by the size of the tract), case assignment (this variable
captures, if applicable, to which outside law firm a property is
assigned), and whether the property is in sequestration or has entered a
payment agreement with the city. The number of properties assigned to
each treatment is further exactly as expected, given the unequal
number of mailing days in our treatment.

% Evidence of randomization is weaker for randomization on delinquent
% balance and randomization on market value; while the test on market
% value is not rejected, that on balance due at mailing is.  We suspect
% this is largely due to the influence of outliers and concentration
% among multiple-property owners-- as seen in Figure
% \ref{fig:balance_balance}, the distributions are visually similar.

\section{Empirical Results}


We consider results for three different subsamples.
The first sample (I) is the full sample
and consists of all 4927 observations; The second sample (II) eliminates
commercial property owners, which reduces the sample to 4749
observations; the third sample (III) eliminates owners of multiple
properties, resulting in a sample size of 3888.

Table \ref{table:summary} summarizes the impact of our experimental
intervention on revenue collection.  The table reports the total
balance owed, the amount generated, and the number of mailing days for
the three treatments and the control groups. It also reports the
percent of properties that paid the City anything and the percent
that paid off their full debt in our sample period.

% latex table generated in R 3.1.3 by xtable 1.7-4 package
% Fri Apr 10 13:12:40 2015
\begin{sidewaystable}[ht]
\centering
\begin{tabular}{|p{1.3cm}|p{1.3cm}|p{1.3cm}|p{1.3cm}|p{2cm}|p{1.4cm}|p{1.4cm}|p{1.4cm}|p{1.4cm}|p{1.4cm}|p{1.6cm}|}
  \hline
Sample & Group & Treated Days & No. Treated & Total Debt Owed & Percent Ever Paid & Percent Paid in Full & Dollars Received & Dollars Per Day Treated & Dollars above Control Per Day & Total Generated over All Days \\ 
  \hline
I & Threat & 1 & 499 & \$1,839,826 & 14 &  8 & \$71,176 & \$71,176 & \$10,883 & \$ 10,883 \\ 
  I & Moral & 4 & 2,211 & \$8,003,148 & 15 &  7 & \$447,728 & \$111,932 & \$51,639 & \$206,557 \\ 
  I & Peer & 2 & 1,142 & \$3,794,900 & 18 & 12 & \$152,217 & \$76,109 & \$15,816 & \$ 31,632 \\ 
  I & Control & 2 & 1,075 & \$3,294,516 & 16 & 10 & \$120,585 & \$60,293 & \$     0 & \$      0 \\ 
   \hline
II & Threat & 1 & 480 & \$1,657,379 & 15 &  8 & \$71,176 & \$71,176 & \$11,142 & \$11,142 \\ 
  II & Moral & 4 & 2,122 & \$7,024,458 & 15 &  7 & \$288,758 & \$72,189 & \$12,155 & \$48,621 \\ 
  II & Peer & 2 & 1,099 & \$3,350,147 & 19 & 12 & \$146,227 & \$73,114 & \$13,079 & \$26,158 \\ 
  II & Control & 2 & 1,048 & \$2,930,759 & 16 & 10 & \$120,069 & \$60,034 & \$     0 & \$     0 \\ 
   \hline
III & Threat & 1 & 406 & \$1,437,902 & 15 &  9 & \$51,309 & \$51,309 & \$18,011 & \$ 18,011 \\ 
  III & Moral & 4 & 1,754 & \$6,956,034 & 16 &  7 & \$418,767 & \$104,692 & \$71,393 & \$285,572 \\ 
  III & Peer & 2 & 891 & \$3,331,168 & 20 & 13 & \$130,016 & \$65,008 & \$31,710 & \$ 63,419 \\ 
  III & Control & 2 & 837 & \$3,007,232 & 16 &  9 & \$66,597 & \$33,299 & \$     0 & \$      0 \\ 
   \hline
\end{tabular}
\caption{Summary of Effectiveness of Treatment} 
\label{table:summary}
\end{sidewaystable}

We also report the dollars in revenue raised per day, which ranges
from \$60,292 in the control group to \$111,931 in the moral treatment
group. Note that the average payments per day is higher in all three
treatment group. A simple difference between the treatment and the
control group provides an estimate of the overall effectiveness of the
intervention. These estimates range from \$10,883 for the treat
treatment to \$51,639 for the moral treatment. Summing over all
treatment groups and days suggests that our experiment generated
approximately \$250,000 for the DoR in just nine days.

The results are qualitatively similar for the two other
samples. However, there are some important quantitative
differences. If we restrict attention to the subsample of
non-commercial properties, we find that the moral treatment raises a
much smaller amount. That suggests that these differences are driven
by a relatively small number of commercial property owners.  If we
restrict attention the properties of sole owners, all treatments 
appear in a more positive light as the intake of the control group
drops precipitously.



\section{Analysis of Treatment Effects}

To analyze the causal effects of our treatments, we first consider the
outcome that indicates whether the delinquent tax payer ever made a
payment.  We can measure this outcome over time starting with the date
of our experiment. Figure \ref{ever_paid_act} plots the time series
for the three treatment groups and the control group using the full
sample. To allow for postal delivery, we define time zero as five days
after our experiment was implemented for each cycle.  Note that the four time series look similar for
the first 15 days of our sample.  After that, we find that tax payers
in the peer treatment group outpace those in the other three
groups in participation rates. Moreover, tax payers in the moral group are less likely to pay
some of their debt than those in the control.

To formalize the analysis, control for observables and test for statistical significance, we
consider Logit regressions. Let $y_{i}=\mathbbm{1}\left[x_i>0\right]$,
where $\mathbbm{1}\left[\cdot\right]$ is an indicator taking the value
one when its argument is true and 0
otherwise. Let $x_i$ be the cumulative remittance to the city at the
conclusion of the sample period by property $i$. Given the random
assignment of treatments we can obtain a consistent estimator of the
causal impact of treatment on $y_i$ by using the following logistic
regression model:
\begin{equation}
y_{i}=X_i^T\beta +D_{T,i}\gamma_{T}+D_{M,i}\gamma_{M}+D_{P,i}\gamma_{P}
+\epsilon_{i},\hspace{1em}\epsilon \enskip\mbox{logistic}
\end{equation}
The $D_{k,i}$ are indicators for the three treatments, i.e.
$D_{k,i}=\mathbbm{1}\left[treatment_i=k\right]$, $k\in\left\{T,M,P\right\}$ for
Threat, Moral, and Peer, respectively. The coefficients $\gamma_{k}$,
then, measure the causal impacts of the treatments on the likelihood
of some degree of remittance to the city, relative to the control
treatment of a plain message.\footnote{To improve efficiency we also
  include some controls such as land area, maturity of debt (more or
  less than 5 years), geographic location (as approximated by City
  Council District), usage category, property exterior condition
  (whether or not the property was categorized as sealed/compromised
  by the city), whether the property took a homestead exemption
  (an indicator of owner occupancy),
  balance at mailing, and market value.} We report robust standard
errors that are clustered to deal with multiple ownership.
  
Table \ref{XX} summarizes the estimates and the estimated standard
errors for the three samples that we considered above.
As can be seen from Table \ref{XX}, the moral and the threat
treatments had no significant effect at the conclusion of the current
sample period. The Peer treatment is consistently positive and
significant when we restrict the sample to sole owners.
  
\begin{table}[htbp]
\caption{Logistic Regressions -- Ever Paid}\label{XX}
\begin{center}
\begin{tabular}{l c c c }
\hline
               & Full Sample & Non-Commercial & Sole Owner \\
\hline
Intercept      & $-1.69^{***}$ & $-1.67^{***}$ & $-1.68^{***}$ \\
               & $(0.08)$      & $(0.08)$      & $(0.10)$      \\
Moral          & $-0.07$       & $-0.10$       & $0.04$        \\
               & $(0.10)$      & $(0.10)$      & $(0.12)$      \\
Peer           & $0.21$        & $0.19$        & $0.30^{*}$    \\
               & $(0.11)$      & $(0.11)$      & $(0.13)$      \\
Threat         & $-0.09$       & $-0.06$       & $-0.03$       \\
               & $(0.15)$      & $(0.15)$      & $(0.17)$      \\
\hline
Log Likelihood & -2136.16      & -2068.89      & -1758.95      \\
Num. obs.      & 4927          & 4749          & 3888          \\
\hline
\multicolumn{4}{l}{\scriptsize{$^{***}p<0.001$, $^{**}p<0.01$, $^*p<0.05$}}
\end{tabular}
\end{center}
\end{table}

Next we investigate whether there is heterogeneity in response to the
treatment. It is plausible that tax payers who owe small amounts of
money behaved differently than those who owe larger amounts. To gain
some additional insights we define four quartiles of the distribution
of outstanding balance of the tax debt within each sample.
In the full sample, the first quartile consists of
tax payers that owe less than \$300, the second quartile consists of
tax payers that owe between \$300 and \$1400, the third quartile
consists of tax payers that owe between \$1400 and \$3600, and the highest
quartile consists of tax payers that owe more than \$3600.
These cutoffs do not change substantially in the other subsamples. 
Figure \ref{ever_paid_quar} plots the time series by quartile for the
three treatment groups and the control group for the full sample.

Figure \ref{ever_paid_quar} provides some important insights. First,
note that response rates in all four subsamples decline rather
dramatically across quartiles of the balance distribution. In the
first quartile the response rate is almost 30 percent by the end of
the sample. In the forth quartile, the response rate is typically
less than 10 percent.  Moreover, there are important differences by
treatment.  We find that the peer treatment is more effective for
delinquent tax payers in the low quartiles. In contrast, the moral
appeal appears to be more powerful for tax payers in the high
quartiles of the balance distribution.

To investigate these issues more formally we create dummy variables
that indicate whether a tax payer is in a given quartile of the balance
distribution or not. We then include these dummies into the Logit
regression to capture the declining participation rate by balance. We
also interact these dummies with the treatment indicators to capture
heterogeneity in treatment. Table \ref{YY} summarizes the estimates
and the estimated standard errors for our three subsamples.

\begin{table}[htbp]
\caption{Logistic Regressions -- Ever Paid} \label{YY}
{\footnotesize
\begin{center}
\begin{tabular}{l c c c }
\hline
                  & Full Sample & Non-Commercial & Sole Owner \\
\hline
Balance Q2        & $-0.46^{*}$   & $-0.52^{*}$   & $-0.33$       \\
                  & $(0.21)$      & $(0.22)$      & $(0.24)$      \\
Balance Q3        & $-1.03^{***}$ & $-0.97^{***}$ & $-1.54^{***}$ \\
                  & $(0.24)$      & $(0.24)$      & $(0.30)$      \\
Balance Q4        & $-1.25^{***}$ & $-1.15^{***}$ & $-1.36^{***}$ \\
                  & $(0.30)$      & $(0.30)$      & $(0.33)$      \\
Moral             & $-0.30$       & $-0.34$       & $-0.34$       \\
                  & $(0.18)$      & $(0.19)$      & $(0.20)$      \\
Moral*Balance Q2  & $0.00$        & $0.00$        & $0.00$        \\
                  & $(0.00)$      & $(0.00)$      & $(0.00)$      \\
Moral*Balance Q3  & $0.06$        & $0.08$        & $-0.11$       \\
                  & $(0.16)$      & $(0.16)$      & $(0.18)$      \\
Moral*Balance Q4  & $0.40^{*}$    & $0.42^{*}$    & $0.33$        \\
                  & $(0.20)$      & $(0.20)$      & $(0.21)$      \\
Peer              & $0.16$        & $0.13$        & $0.21$        \\
                  & $(0.19)$      & $(0.19)$      & $(0.21)$      \\
Peer*Balance Q2   & $0.54^{**}$   & $0.58^{***}$  & $0.58^{**}$   \\
                  & $(0.17)$      & $(0.17)$      & $(0.18)$      \\
Peer*Balance Q3   & $-0.08$       & $-0.07$       & $-0.19$       \\
                  & $(0.17)$      & $(0.17)$      & $(0.18)$      \\
Peer*Balance Q4   & $-0.49^{**}$  & $-0.45^{*}$   & $-0.61^{**}$  \\
                  & $(0.17)$      & $(0.17)$      & $(0.19)$      \\
Threat            & $-0.05$       & $-0.01$       & $-0.13$       \\
                  & $(0.26)$      & $(0.26)$      & $(0.29)$      \\
Threat*Balance Q2 & $0.07$        & $0.10$        & $0.11$        \\
                  & $(0.16)$      & $(0.16)$      & $(0.17)$      \\
Threat*Balance Q3 & $-0.50^{**}$  & $-0.49^{**}$  & $-0.64^{***}$ \\
                  & $(0.17)$      & $(0.18)$      & $(0.19)$      \\
Threat*Balance Q4 & $-0.04$       & $-0.03$       & $-0.20$       \\
                  & $(0.16)$      & $(0.16)$      & $(0.17)$      \\
\hline
Log Likelihood    & -2010.55      & -1948.32      & -1639.28      \\
Num. obs.         & 4927          & 4749          & 3888          \\
\hline
\multicolumn{4}{l}{\scriptsize{$^{***}p<0.001$, $^{**}p<0.01$, $^*p<0.05$. Control coefficients omitted for brevity; see Appendix.}}
\end{tabular}
\end{center}
}
\end{table}

Table \ref{YY} shows the quartile dummy variables are
significantly negatively correlated with participation. The more a delinquent
tax payer owes the city, the less likely he is to pay his tax
bills. Moreover, we find that the effect of the peer treatment is
large and significant for the second quartile. The effect of the moral
treatment is significantly different from zero and positive for the fourth
quartile in all three samples. We thus conclude that the peer
treatment seems to be most effective for tax payers that owe small
amounts, while the moral treatment is more effective for those who owe
larger amounts. The Threat treatment appears to have a negative effect
on tax payers that owe larger amounts.

\begin{table}[htbp]
\caption{Marginal  Predictions -- Ever Paid}  \label{ZZ}
\centering
\begin{tabular}{lcccc}
  \hline
 & Q1 & Q2 & Q3 & Q4 \\ 
  \hline
Control & 23.40 & 16.10 & 9.80 & 8.00 \\ 
  Moral & 18.50 & 12.70 & 12.10 & 11.40 \\ 
  Peer & 26.40 & 15.20 & 14.40 & 8.20 \\ 
  Threat & 22.40 & 12.20 & 13.40 & 7.10 \\ 
   \hline
\end{tabular}
\end{table}

Table \ref{ZZ} shows the marginal predictions for the probability that
properties in each treatment group enter the fold for each
quartile. These values represent odds ratios predicted near the 
center of the control variables--specifically, the sample average levels of
all dummy variables and the median of the continuous regressors 
(land area and market value). Considering the first quartile, the table shows us that
properties given the Peer treatment were about 3 percentage points
more likely to have contributed something to DoR prior to the sample's
conclusion than those who received the control treatment.

Next, we examine a more restrictive yes-no participation
outcome--namely, whether or not the property offered not just perhaps token
repayment in our sample period, but whether its debts were paid back
in full.\footnote{Due to some measurement issues, it is not possible
  to track on a day-to-day basis exactly the balance due for each
  property-- accrual of interest and other charges is hard to pinpoint
  exactly. In the main results below, we actually measure full
  repayment as submission of at least 95\% of the balance due.}  The
ever-paid outcome used above does not differentiate between tax payers that made
full payments and taxpayers that made partial payments. Figure
\ref{paid_full_act} shows the time series for this outcome for the
four treatment groups. The overall patterns are similar as in the
previous outcome. The peer conformity treatment outpaces all other
treatments after two weeks.

Again we can formalize the analysis using Logit models. Table \ref{VV}
summarizes the results from Logit regressions that use an indicator
for whether the Department of Revenue received full payment by the end
of the sample period as the outcome variable. We report the estimates
from the three samples.

\begin{table}[htbp]
\caption{Logistic Regressions -- Paid in Full}\label{VV}
\begin{center}
\begin{tabular}{l c c c }
\hline
               & Full Sample & Non-Commercial & Sole Owner \\
\hline
Intercept      & $-2.23^{***}$ & $-2.22^{***}$ & $-2.29^{***}$ \\
               & $(0.10)$      & $(0.10)$      & $(0.12)$      \\
Moral          & $-0.42^{**}$  & $-0.44^{**}$  & $-0.29$       \\
               & $(0.13)$      & $(0.14)$      & $(0.15)$      \\
Peer           & $0.24$        & $0.24$        & $0.41^{**}$   \\
               & $(0.14)$      & $(0.14)$      & $(0.16)$      \\
Threat         & $-0.21$       & $-0.18$       & $-0.04$       \\
               & $(0.19)$      & $(0.20)$      & $(0.21)$      \\
\hline
Log Likelihood & -1435.15      & -1395.06      & -1175.05      \\
Num. obs.      & 4927          & 4749          & 3888          \\
\hline
\multicolumn{4}{l}{\scriptsize{$^{***}p<0.001$, $^{**}p<0.01$, $^*p<0.05$}}
\end{tabular}
\end{center}
\end{table}
We find positive effects for the peer treatment in all three samples. The
effects are significantly different from zero in the sole owner sample. Moreover,
we find negative effects for the moral treatment.

Moving on, we investigate whether there is heterogeneity in response to the
treatment.  Figure \ref{full_paid_quar} plots the time series for the
three treatment groups and the control group by quartile of the
outstanding balance.  Figure \ref{full_paid_quar} reinforces our
finding that response rates in all four subsamples declines rather
dramatically across quartiles of the balance distribution. In the
first quartile the response rate is almost 30 percent by the end of
the sample. In the highest quartile, the response rate is typically
less than 2 percent.  The plots are also consistent with our previous
finding that the peer treatment is more effective for delinquent tax
payers in the first and second quartile.

Again we formalize the analysis by estimating Logit models with
interactions.  Table \ref{WW} summarizes the estimates and their
standard errors for the three samples that we considered
above.

\begin{table}[htbp]
\caption{Logistic Regressions -- Paid in Full}\label{WW}
{\footnotesize
\begin{center}
\begin{tabular}{l c c c }
\hline
                  & Full Sample & Non-Commercial & Sole Owner \\
\hline
Balance Q2        & $-1.28^{***}$ & $-1.42^{***}$ & $-1.35^{***}$ \\
                  & $(0.27)$      & $(0.28)$      & $(0.31)$      \\
Balance Q3        & $-2.32^{***}$ & $-2.18^{***}$ & $-3.18^{***}$ \\
                  & $(0.39)$      & $(0.37)$      & $(0.61)$      \\
Balance Q4        & $-3.27^{***}$ & $-2.85^{***}$ & $-3.83^{***}$ \\
                  & $(0.74)$      & $(0.61)$      & $(1.03)$      \\
Moral             & $-0.45^{*}$   & $-0.49^{*}$   & $-0.49^{*}$   \\
                  & $(0.19)$      & $(0.20)$      & $(0.22)$      \\
Moral*Balance Q2  & $0.00$        & $0.00$        & $0.00$        \\
                  & $(0.00)$      & $(0.00)$      & $(0.00)$      \\
Moral*Balance Q3  & $-0.27$       & $-0.23$       & $-0.54^{*}$   \\
                  & $(0.23)$      & $(0.24)$      & $(0.26)$      \\
Moral*Balance Q4  & $0.61^{*}$    & $0.65^{*}$    & $0.49$        \\
                  & $(0.26)$      & $(0.26)$      & $(0.27)$      \\
Peer              & $0.25$        & $0.22$        & $0.29$        \\
                  & $(0.19)$      & $(0.19)$      & $(0.22)$      \\
Peer*Balance Q2   & $1.01^{*}$    & $0.99^{*}$    & $1.06^{*}$    \\
                  & $(0.42)$      & $(0.42)$      & $(0.46)$      \\
Peer*Balance Q3   & $-0.18$       & $-0.13$       & $-0.33$       \\
                  & $(0.25)$      & $(0.25)$      & $(0.26)$      \\
Peer*Balance Q4   & $-0.39$       & $-0.32$       & $-0.54^{*}$   \\
                  & $(0.23)$      & $(0.23)$      & $(0.25)$      \\
Threat            & $-0.07$       & $-0.03$       & $-0.05$       \\
                  & $(0.27)$      & $(0.27)$      & $(0.29)$      \\
Threat*Balance Q2 & $0.21$        & $0.25$        & $0.13$        \\
                  & $(0.22)$      & $(0.22)$      & $(0.24)$      \\
Threat*Balance Q3 & $-0.64^{**}$  & $-0.65^{**}$  & $-0.80^{**}$  \\
                  & $(0.25)$      & $(0.25)$      & $(0.27)$      \\
Threat*Balance Q4 & $0.20$        & $0.22$        & $-0.03$       \\
                  & $(0.22)$      & $(0.22)$      & $(0.24)$      \\
\hline
Log Likelihood    & -1150.17      & -1120.45      & -919.68       \\
Num. obs.         & 4927          & 4749          & 3888          \\
\hline
\multicolumn{4}{l}{\scriptsize{$^{***}p<0.001$, $^{**}p<0.01$, $^*p<0.05$. Control coefficients omitted for brevity; see Appendix.}}
\end{tabular}
\end{center}
}
\end{table}

Table \ref{WW} shows a very steep decline in the willingness to pay
the tax bills as the amount owed increases.  These effects are even
larger than in the previous model, which suggests that many delinquent
tax payers in the higher quartiles only make partial payments when
they respond to the letters. The estimates of the treatment effects
are similar to the ever-paid outcome. Again we find some evidence that
the peer treatment works for tax payers in the lower quartiles while
the moral treatment appeals to those in the highest quartile. The
treatment seems to be counterproductive for delinquent tax payers in
the third quartile. The findings are reinforced by the marginal
predictions reported in Table \ref{TT}.

\begin{table}[htbp]
\caption{Marginal Predictions of Logistic Regressions -- Paid in Full}  \label{TT}
\centering
\begin{tabular}{lcccc}
  \hline
 & Q1 & Q2 & Q3 & Q4 \\ 
  \hline
Control & 19.90 & 6.40 & 2.40 & 0.90 \\ 
  Moral & 13.60 & 4.00 & 2.90 & 1.20 \\ 
  Peer & 24.10 & 6.90 & 3.10 & 1.80 \\ 
  Threat & 18.80 & 3.50 & 4.20 & 0.90 \\ 
   \hline
\end{tabular}
\end{table}

To evaluate the overall impact of the different notification
strategies, we would also like to know how much revenue can be raised
by the different approaches.  Figure \ref{fig:repay} depicts the
trajectories of repayments by each group, normalized by the
number of properties in the group.  Here we find
that all three treatments increased the cumulative average payment,
as initially noted in Table \ref{table:summary}.
Moreover, the moral appeal treatment caused the largest increase of
payment.  This result is confirmed in Figure \ref{fig:repay}, which
depicts the trajectories of the percentage of mailing day debt owed to
the city by each group. As discussed in detail above, the key
finding that the moral treatment raised more revenue than the other
treatments or the control is largely due to a small number of
delinquent tax payers that owed large amounts of money.  The sample
size is not large enough to find significant effects. This can be
shown by estimating a variety of regressions and Tobit models intended
to predict the magnitude of the received payment. We do not
report these results here since we did not find any significant
treatment effects.


\section{Conclusions}

This field experiment evaluated a set of different notification
strategies intended to increase property tax compliance. We tested
three of the most commonly suggested models of tax compliance:
deterrence, moral appeal, and peer conformity.  We have implemented
the experiment in collaboration with the Philadelphia Department of
Revenue (DoR).  Our findings provide some moderate evidence that
both moral appeal and peer conformity may improve tax compliance. We
find little evidence that supports the standard deterrence model
compared to a traditional simple bill notification.

Our study provides ample scope for future research.  Unlike several
recent papers (Kleven et al. 2011; Slemrod, Blumenthal, and Christian
2001), which have found large increases in compliance after providing
information about the threat of auditing, we find no evidence of a
deterrence effect.  It is probably not surprising that our deterrence
treatment was not effective. Philadelphia is city with a history of
high property tax delinquency. It is difficult to alter perceptions of
beliefs of punishment by sending one letter in a city with
already-high property tax delinquency.  Any threats may be considered
to be empty given previous attempts at collection and the lack of
enforcement penalties.  It would be more interesting to design an
intervention that is based on a more credible threat.  The peer
conformity treatment is also subject to the same potential
criticism. It may be more effective to randomly print a number of
delinquent tax payers in the local news paper or to send mail
informing delinquents' neighbors of their malfeasance. Of course, these types of intervention face
much larger legal hurdles and are therefore much harder to implement.
They may also backfire and increase tax noncompliance by encouraging
defiance of the law (Sherman 1993).

Consistent with other recent tax compliance experiments (Fellner,
Sausgruber, and Traxler 2013), we also find that the largest effect is
observed for the subset of taxpayers at the margin of compliance.
Beyond this traditional perspective on achieving higher rates of legal
compliance through modifications to legal threats, others have
examined how non-external factors contribute to tax
compliance. Similar to other studies we find that providing social
information about tax compliance provides some marginal increase in
collection (Wenzel and Taylor 2004; Wenzel 2005; Hallsworth et
al. 2014). While this finding is by no means universal (Fellner,
Sausgruber, and Traxler 2013), our results suggest that providing
social norm information is noticeably more effective than either
deterrent or moral persuasion messages.

\newpage

\section*{References}

Ariel, Barak (2012), ``Deterrence and moral persuasion effects on
corporate tax compliance: Findings from a randomized controlled
trial." Criminology, 50 (1), 27-69. \\
\\
Allingham, Michael G., and Agnar Sandmo (1972) ``Income Tax Evasion: A Theoretical
Analysis." Journal of Public Economics, 1: 323-38. \\
\\
Alm, James, Gary H. McClelland, and William D. Schulze (1992), ``Why Do People
Pay Taxes?" Journal of Public Economics 48: 21-38. \\
\\
Alm, James (1999), ``Tax compliance and administration." In: Hildreth, W. Bartley and James A. Richardson
(eds.) Handbook on Taxation. New York, USA, Marcel Dekker, Inc., pp. 741-768. \\
\\
Andreoni, James, Erard, Brian and Jonathan Feinstein (1998), ``Tax compliance." Journal of Economic
Literature, 36, 818-860. \\
\\
Becker, Gary S. (1968), ``Crime and Punishment: An Economic Approach."
Journal of Political Economy 76: 169-217.\\
\\
Bernheim, B. Douglas (1994), ``A Theory of Conformity." Journal of Political Economy, 102, 5, 841-877. \\
\\
Blumenthal, Marsha, Christian, Charles and Joel Slemrod (2001), ``Do normative appeals affect tax
compliance? Evidence from a controlled experiment in Minnesota." National Tax Journal, 54 (1),
125 - 138. \\
\\
Cowell, Frank A. and James P. F. Gordon (1988), `` Unwillingness to pay tax: tax evasion and public provision."
Journal of Public Economics, 36, 305-321.\\
\\
Fehr, Ernst and Simon Gachter (1998), ``Reciprocity and economics: The economic implications of homo
reciprocans." European Economic Review 42 (3-5), 845-59. \\
\\
Fellner, Gerlinde, Rupert Sausgruber, and Christian Traxler (2013), ``Testing Enforcement Strategies in the Field: Threat, Moral Appeal and Social Information." Journal of the European Economic Association 11, 3, 634-60.\\
\\
Frey, Bruno S., and Lars P. Feld (2002),  ``Deterrence and Morale in Taxation: An
Empirical Analysis." CESifo Working Paper no. 760, August 2002. \\
\\
Hallsworth, Michael., John List, Robert Metcalfe and Ivo Vlaev (2014), "The Behavioralist as Tax Collector,"
Using Natural Field Experiments to Enhance Tax Compliance." NBER Working Paper 20007. \\
\\
Harrison, Glenn W. and John A. List (2004), "Field Experiments." Journal of Economic Literature, 42 (4),
1009-1055.\\ 
\\
Kleven, Henrik J., Knudsen, Martin B., Kreiner, Claus T., Pedersen, Soren and Emmanuel Saez (2011),
``Unwilling or Unable to Cheat? Evidence From a Tax Audit Experiment in Denmark."
Econometrica, 79 (3), 651-692. \\
\\
Pew Charitable Trust (2013), ``Delinquent Property Tax in Philadelphia." Technical Report. \\
\\
Pomeranz, Dina (2013), ``No taxation without information: Deterrence and self-enforcement in the Value
Added Tax." Harvard Business School Working Paper. \\
\\
Reckers, Philip M. J., Sanders, Debra L. and Stephen J. Roark (1994), ``The influence of ethical attitudes on
taxpayer compliance." National Tax Journal, 47 (4), 825-836. \\
\\
Sherman, Lawrence (1993), ``Defiance, deterrence, and irrelevance: A theory of the criminal sanction."
Journal of Research in Crime and Delinquency, 30,  445-473. \\
\\ 
Slemrod, Joel (2007), ``Cheating ourselves: The economics of tax evasion." Journal of Economic
Perspectives, 21 (1), 25-48. \\
\\
Slemrod, Joel, Marsha Blumenthal, and Charles Christian (2001), ``Taxpayer Response to an Increased Probability of Audit: Evidence from a Controlled Experiment in Minnesota." Journal of Public Economics 79, 3, 455-83.\\
\\
Torgler, Benno (2002), ``Moral-suasion: An alternative tax policy strategy? Evidence from a controlled field
experiment in Switzerland." Economics of Governance 5 (3), 235-253. \\
\\
Torgler, Benno (2012),  ``A field experiment on moral-suasion and tax compliance focusing on under-declaration
and over-deduction." QUT School of Economics and Finance Working Paper no. 285. \\
\\
Wenzel, Michael (2005), ``Misperceptions of social Norms about Tax Compliance: From Theory to Intervention." Journal of Economic Psychology 26, 6, 862-83\\
\\
Wenzel, Michael and Natalie Taylor (2004),  ``An experimental evaluation of tax-reporting schedules: a case of
evidence-based tax administration." Journal of Public Economics, 88 (12), 2785-2799.

% \end{document}

\newpage

\begin{figure}[htpb]
\begin{center}
\caption{Standard Due Letter}
\bigskip
\includegraphics[width=6in]{PastDueLetter.pdf}
\end{center}
\end{figure}
\newpage
\begin{figure}[htpb]
\begin{center}
\caption{Treatment 1: Deterrence}
\bigskip
\includegraphics[width=6in]{flyer_options_141104_treat1.pdf}
\end{center}
\end{figure}
\newpage
\begin{figure}[htpb]
\begin{center}
\caption{Treatment 2: Moral Appeal}
\bigskip
\includegraphics[width=6in]{flyer_options_141104_treat2.pdf}
\end{center}
\end{figure}
\newpage
\begin{figure}[htpb]
\begin{center}
\caption{Treatment 3: Peer Conformity}
\bigskip
\includegraphics[width=6in]{flyer_options_141104_treat3.pdf}
\end{center}
\end{figure}
\newpage
\begin{figure}[htpb]
\begin{center}
\caption{Control}
\bigskip
\includegraphics[width=6in]{flyer_options_141104_treat4.pdf}
\end{center}
\end{figure}
\newpage

% \begin{figure}[htpb]
% \caption{}\label{fig:balance_balance}
% \begin{center}
% \includegraphics[width=4in]{total_balance_by_treatment}
% \par\end{center}
% \end{figure}
% \newpage

\begin{figure}[htpb]
\begin{center}
\caption{Delinquent Properties: Amount}\label{fig:map_amt}
\bigskip
\includegraphics[width=6in]{delinquents_as_of_nov_3_by_amt.png}
\end{center}
\end{figure}
\newpage
\begin{figure}[htpb]
\begin{center}
\caption{Rittenhouse Square}\label{fig:map_rit_amt}
\bigskip
\includegraphics[width=6in]{delinquents_as_of_nov_3_by_amt_ritt_fit.png}
\end{center}
\end{figure}
\newpage
\begin{figure}[htpb]
\begin{center}
\caption{Delinquent Properties: Time}\label{fig:map_time}
\bigskip
\includegraphics[width=6in]{delinquents_as_of_nov_3.png}
\end{center}
\end{figure}
\newpage
\begin{figure}[htbp]
\caption{}\label{ever_paid_act}
\begin{center}
\includegraphics[width=5in]{time_series_pct_ever_paid_act}
\par\end{center}
\end{figure}
\newpage
\begin{figure}[htbp]
\caption{}\label{ever_paid_quar}
\begin{center}
\includegraphics[width=5in]{time_series_pct_ever_paid_by_quartile_act.pdf}
\par\end{center}
\end{figure}
\newpage
\begin{figure}[htbp]
\begin{center}
\caption{} \label{paid_full_act}
\includegraphics[width=5in]{time_series_pct_paid_full_act}
\par\end{center}
\end{figure}
\newpage
\begin{figure}[htbp]
\caption{}\label{full_paid_quar}
\begin{center}
\includegraphics[width=5in]{time_series_pct_ever_paid_by_quartile_act.pdf}
\par\end{center}
\end{figure}
\newpage
\begin{figure}[htbp]
\begin{center}
\caption{}\label{fig:repay}
\includegraphics[width=5in]{time_series_average_payments_act}
\par\end{center}
\end{figure}
\newpage
\begin{figure}[htbp]
\begin{center}
\caption{}\label{fig:drawdown}
\includegraphics[width=5in]{time_series_debt_paydown_act}
\par\end{center}
\end{figure}

\end{document}

\begin{appendix}

\section{Modeling Property Tax Compliance}

\subsection{The Becker Model}

The starting model of our analysis is a model of tax compliance that
is based on ideas of the pioneering work by Becker (1968).  Consider a
tax payer's utility, denoted by $U(h,c)$, which is defined over
housing, $h$ and consumption, $c$.  Lifetime or permanent income is
denoted by $y$. The housing value or price is given by $v(h)$. The
city levies a property tax with rate equal to $t$. Thus, total tax
liabilities are given by $t \; v(h)$. Tax payers voluntarily comply
with the tax laws. If a tax payer decides not to comply with the tax
laws, the probability of detection of non-compliance is $p$. If
caught, the tax payer has to pay a fine which is equal to $f \;
v(h)$. The utility of complying with the tax laws is:
\begin{eqnarray}
U_c= U(h, y - (1+ t) \; v(h))
\end{eqnarray}
The (expected) utility of non-compliance is:
\begin{eqnarray}
U_n = (1-p) \; U(h, y - v(h)) + p \; U(h, y - (1+ t + f) \;  v(h)) 
\end{eqnarray}
Define a variable $d$ which indicates whether or not the agents pays
property taxes, i.e.  $d=1$ if the tax payer complies and $d=0$ if he
does not not comply. The decision problem of the tax payer is
therefore given by:
\begin{eqnarray}
\max_d  \left \{d \; U_c \; + \; (1-d) \; U_n \right \}
\end{eqnarray}
A tax payer will comply with the law ($d=1$) if and only if:
\begin{eqnarray}
U_c \ge U_n
\end{eqnarray}
The key implications of the Becker model are the following:
\begin{itemize}
\item Compliance is increasing in the fine.
\item Compliance is increasing in the detection probability.
\item Compliance is decreasing in the property tax rate.
\item Compliance is increasing in the degree of risk aversion of a tax payer.
\end{itemize}
Suppose we wanted to test the predictions of the Becker model using a
randomized controlled trial (RCT).  Ideally we would like to randomly
vary a) the fine (using, for example, an amnesty program); b) the
enforcement or detection probability; c) the property tax
rate. Unfortunately, none of these experiments can be easily
implemented.  A feasible strategy is to try to change the subjective
beliefs or the perceptions of tax payers regarding the probability of
detection and the consequences of non-compliance.

\subsection{The Becker-Bergstrom-Andreoni Model}

The recent literature in local public economics points to a number of
potential short-comings of the Becker model. Property taxes are used
to provide local public goods and services. We study the problem of
tax compliance using insights of the literature on voluntary public
good provision (Bergstrom, Blume \& Varian, 1986, Andreoni, 1989). To
capture these ideas, we need to recast the Becker model as a {\bf
  game} between tax payers since {\bf strategic interactions} among
tax payers matter.

We now consider a simultaneous move game with  complete information and
 two tax payers: $i=A,B$.  As before, let the variables $d_A$ and
$d_B$ indicate whether or not the agent pays his property taxes. Tax
revenues, $T$, are used to provide local public goods and services,
$G$. Utility is now given by $U(h,c) + V(G)$.

The expected tax payments of agent $i$ are given by:  
\begin{eqnarray}
T(d_i) = d_i \; t \; v(h) \; +  \; (1-d_i) \;  [p \; (t+f) \; v(h) \; + \; (1-p) \; 0] 
\end{eqnarray}
Total tax collection is:
\begin{eqnarray}
T (d_A,d_B) = T(d_A) + T(d_B)
\end{eqnarray}
Realistically, we want to assume that:
\begin{eqnarray}
 t \; v(h) &>&    p \; (t+f) \; v(h)
\end{eqnarray}
Tax revenues generated under compliance are larger than under
non-compliance. Hence, we have:
\begin{eqnarray}
T(1,1) > T(1,0) = T(0,1) > T(0,0)
\end{eqnarray}
Moreover, $T = G$ implies that:
\begin{eqnarray}
G(1,1) > G(1,0) = G(0,1) > G(0,0)
\end{eqnarray}
The pay-offs now depend on the strategies of both players:
\begin{eqnarray}
U_A(d_A, d_B) &=& d_A \; U_c \; + \; (1-d_A) \; U_n \; +  \; V(G(d_A,d_B)) \\
U_B(d_A, d_B) &=& d_B \; U_c \; + \; (1-d_B) \; U_n \; + \; V(G(d_A,d_B)) \nonumber
\end{eqnarray}
This game can have different equilibria in which zero, one or both
players comply with the tax laws.  Free riding can, therefore, occur
in equilibrium.

Note that tax payers now have stronger incentives to pay taxes than in
the basic Becker model since they will also benefit from the increase
in public good provision. 

More generally, we can assume that it is costly to transform tax
revenues into public goods. Not all cities are equally well rund and
hence equally efficient in providing public goods.  It is then
straight forward to show that an increase in the cost or a decrease in
the efficiency of providing public goods leads to lower tax
compliance. Again this prediction is difficult to test within a
randomized controlled experiment.

The simple voluntary tax compliance model with enforcement is not
likely to be consistent with observed behavior since compliance is too
high given the enforcement policies and the incentives to voluntarily
provide public good. Andreoni (1989) suggested to include a term in
the utility function that captures "warm glow," or "doing the right
thing." In this version of our model the pay-offs of both tax payers are:
\begin{eqnarray}
U_A(d_A, d_B) &=& d_A \; U_c \; + \; (1-d_A) \; U_n \; + \; V(G(d_A,d_B)) \; +  \; W(T_A(d_A)) \\
U_B(d_A, d_B) &=& d_B \; U_c \; + \; (1-d_B) \; U_n \; + \; V(G(d_A,d_B)) \; +  \; W(T_A(d_B)) \nonumber
\end{eqnarray}
The key implication of the Becker-Bergstrom-Andreoni Model is that
tax payers have much stronger incentives to pay taxes tan in the
simple Becker model.  We cannot randomly assign "warm glow" within a
controlled experiment. The basic idea behind our ``moral appeal"
treatment is to reinforce the warm glow aspects of tax compliance.

\subsection{The Becker-Bergstrom-Bernheim Model}

Tax payers may comply with tax laws because of deeply ingrained social
norms. In particular, tax payers may comply with the tax laws due to
peer pressure and the desire to conform to acceptable standards of
behavior (Bernheim, 1994).  Suppose that individuals bear some costs,
$C$, if they do not manage to coordinate their actions in equilibrium.
The pay-offs in this model are then given by:
\begin{eqnarray}
U_A(d_A, d_B) &=& d_A \; U_c \; + \; (1-d_A) \; U_n \; + \; V(G(d_A,d_B)) \; +  \; 1\{d_A \ne d_B \} \; C \\
U_B(d_A, d_B) &=& d_B \; U_c \; + \; (1-d_B) \; U_n \; + \; V(G(d_A,d_B)) \; +  \; 1\{d_A \ne d_B \} \; C \nonumber
\end{eqnarray}
where $1\{d_A \ne d_B \}$ denotes an indicator variable which is equal
to one if both players fail to coordinate their actions and zero
otherwise. Equilibria in which both tax payers take the same action
become more prevalent.  One could, in principle, vary the conformity
costs within a randomized controlled experiment. For example, one
could threaten delinquent tax payers with public exposure by printing
names in newspapers or online web pages.  Of course, these
interventions face larger legal hurdles and are, therefore, harder to
implement.  The basic idea behind our peer conformity treatment is to
reinforce the notion that your peers are paying taxes.

\end{appendix}




