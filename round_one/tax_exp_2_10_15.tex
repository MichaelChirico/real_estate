\documentstyle[12pt,titlepage]{article}

\renewcommand\baselinestretch{1.0}
\setlength{\parskip}{0.08in}
\setlength{\medskipamount}{0.05in}
\textheight 8.0 in
\textwidth 6.0 in
\topmargin 0.25in
\tolerance=11000

\begin{document}

\title{An Experimental Evaluation of Strategies to Increase
Property Tax Revenue Collection}
\author{Michael Chirico, Robert Inman, Charles Loeffler, \\
John MacDonald, and Holger Sieg \\
$$ \\
University of Pennsylvania}
\date{\today}  
\maketitle



\renewcommand{\thefootnote}{\arabic{footnote}}

\section{Introduction}

The purpose of this experimental study to evaluate alternative
strategies that are intended to increase property tax revenue
collection in the City of Philadelphia.  We design a number of
different treatments that are motivated by the theoretical literature
on tax compliance, which distinguished between extrinsic and intrinsic
motivations for paying taxes. Extrinsic motivations are primarily
based on the deterrence model which explains tax compliance by a
combination of penalties and costly monitoring. Intrinsic motivations
are captured in models that rely either on social norms or social
contracts.  We develop and implement a sequence of experiments to test
three of most commonly suggested hypothesis. The basic strategy behind
our randomized experiments is to vary the informational content of a
sequence of letters are sent to delinquent property tax payers.

The first experiment is based on the the deterrence model of tax
compliance which emphasizes extrinsic motivations of behavior. Key
elements of the model are the probability of detection (the monitoring
probability) and the size of the penalty or fine that is levied on tax
evaders who are caught. Given the uncertainty of law enforcement,
behavior is based on subjective probabilities of detection and
perceptions regarding the size of penalties.\footnote{The behavioral
  literature suggests that there may be behavioral biases since agents
  deal small detection probabilities} The key idea behind the first
experiment is to change the subjective beliefs regarding penalties and
detection probabilities in an experimental design. Note that our
experiment does involve a tax amnesties. We also do not change the
fiscal incentives to pay taxes in our proposed experiment. Instead, we
attempt to change subjective beliefs regarding detection and the
consequences of tax delinquencies.

Our second experiment is based on a more recent theory that emphasizes
intrinsic motivations of tax payers and focuses more broadly on tax
morale. The social contract approach emphasizes the promise of the
city to efficiently deliver public goods and services in exchange for
more or less voluntary tax payments.  Loyalties and emotional ties
play a large role in creating high tax moral. The problem of raising
taxes is viewed similarly as the fund-raising problem faced by
non-for-profit organizations.\footnote{For example, Public Radio
  stations need to raise funds from their listeners.} The basic idea
behind the second experiment is then to design an informational
treatment that reinforces this social contract. The treatment takes
the form of a moral appeal to the delinquent tax payer and emphasizes
progress in reductions in government waste and the need to raise tax
revenues to stay competitive and address serious problems.
 
Our third experiment is based on the recent literature on social norms
which emphasizes concepts of social fairness and reciprocity. This
literature suggests that social exposure of delinquent tax payers may
be an effective tool for penalizing tax evasion.  The idea behind our
third experiment is then to threaten delinquent property owners with 
public exposure.

We conduct these tax revenue collection experiments in collaboration
with the Revenue Department of the City of Philadelphia, which is
particularly useful city to study property tax evasions.  As of April
2012, the city and school district were owed \$292.3 million in
delinquent taxes on 102,789 properties, \$515.4 million when interest
and penalties are included.

The rest of the paper is organized as follows. Section 2 discusses the
institutional background.  Section 3 provides a
detailed discussion of our treatments. Section 4 discusses the experimental design and its
 implementation. Section 5 provides our current empirical findings. Section 6 offers some conclusions.


\section{Institutional Background}

The exact timing of payment and enforcement is as follows: 
\begin{enumerate}
\item Bills sent in December and early January, and payment is due on March 31.  
\item If taxes are played by February 28, the tax payer gets a 1 \% discount.  
\item After March 31 interests and penalties begin to accrue.  
\item If there is still no payment after by end of the summer, then beginning in September the �nudge� begins  using an outside collection agency.  About 2/3�s of the delinquent files go to the agency (and they do their thing) but 1/3 stay with the city.  
\end{enumerate}
The potential penalties for delinquent tax payers are the following
\begin{enumerate}
\item If no payment by December 31, then there is a lien on the property.  
\item Those properties are divided between 1/3 to law firm 1, 1/3 to law firm 2, and 1/3 stay with the city.  it is up to the law firms to decide what they can do, with the ultimate threat of the sheriff�s sale.  
\item If the tax payer is a city employee, the city  can deduct taxes from salary. 
\item If the delinquent is an applicant for city job, he will need to fully paid (or be in a payment plan) to be eligible for employment.  
\item For business properties, the city can also sequester the income from the property, become the property manager.  
\item It can also shut down the business.  
\end{enumerate}
Finally, the city is considering an intermediate penalty which is to go after private property e.g., boot the taxpayer�s car. 



\section{Treatments}

The basic research design is then straightforward. Delinquent property
owners in the treatment group receive a sequence of letters that are
discussed in detail below. We randomly assign the treatment status
conditional on predicted default probabilities.

\subsection{Experiment 1: Deterrence}

The owners in the control group receive the usual letter. The owners in the treatment group
receive an augmented letter that also contains a  threat of enforcing liens and foreclosure.
Here is an example that captures the key idea: \\

\noindent NEED TO UPDATE TO INCLUDE THE MESSAGES THAT WE ACTUALLY USED

"Our department is required by law to fully enforce the tax code. You are in violation of the law.
The City of Philadelphia will pursue all penalties available under statue, including, but not limited to,
obtaining a lien on your property and seeking eventual foreclosure. Please do not make the mistake of assuming 
that we are too busy to investigate your case. Ignoring this letter will only increase the jeopardy you may already 
be in. We strongly suggest that you contact the Department of Revenue and arrange a payment schedule
to avoid even more drastic sanctions."


\subsection{Experiment 2: Moral Appeal}

The owners in the control group receive the usual letter. The owners in the treatment group
receive an augmented letter that also contains a  strong moral appeal.
Here is an example: \\

"We know that you try to deal honestly with your taxes and may genuinely 
believe that you have good reasons for not filing your property tax returns. But there are many compelling reasons why 
 it is necessary for you  to file your outstanding returns. Without property taxes the City of
Philadelphia cannot afford essential services such as educating our children, providing public safety, and
maintaining our roads and parks. Your tax dollars are spent of goods and services that you and your
neighbors rely on every day. When taxpayers do not pay what they owe, the entire community suffers.
We would like to remind you to meet your obligations to the community and file your property tax return."

\subsection{Experiment 3: Social Pressure}

The objective is to evaluate an approach that is based on social pressure. The owners in the control group receive the usual letter. The owners in the treatment group receive an augmented letter that also contains a threat of public exposure. Here is an example:

" You are probably not aware of the fact that you belong to a small minority of residents that tries to take advantage of the community. On average, more than 90 percent of your neighbors have already paid their property taxes on time. You are currently free-riding on their willingness to support the community. Please do not make the mistake of assuming that you will be able to hide in anonymity forever. We strongly suggest that you contact the Department of Revenue and arrange a payment schedule to avoid drastic sanctions. Additional sanctions will be inevitable once your neighbors find out the truth about your unwillingness to pay taxes."

\section{Experimental Design and Fidelity of Implementation}

Discuss how the experiment was implemented and how we decided what sample to use based on the returned letters.


Show some tests that our sub-samples are balanced.

\section{Methods}

Describe four different regression models.


\section{Empirical Results}

Summarize the findings of the basic regression models (4 tables, one for each model) and plots that we have.

\section{Conclusions}

\end{document}  



\section{Property Tax Compliance in Philadelphia}

The following is a summary of the recent study by the Pew Charitable
Trust entitled "DELINQUENT PROPERTY TAX IN PHILADELPHIA," published in
2013.

Philadelphia operates with relatively wide latitude under the
Pennsylvania Municipal Claims and Tax Lien Law (MCTLL), enacted in
1923 and amended at least a dozen times. It applies mainly to the
state's first class and second class cities, namely Philadelphia and
Pittsburgh.

In Philadelphia, a property is considered tax delinquent nine months
after the city's March 31 payment deadline passes. Delinquent parcels
are assessed interest and fees, which over time can grow larger than
the principal. By law, properties can be sold at public auction as
quickly as nine months after they become delinquent. In practice, that
rarely happens, and many properties are allowed to linger on the
delinquency rolls for decades.

A tax lien may be imposed for delinquent taxes owed on real property
as a result of failure to pay taxes. A claim for payment that takes
precedence over all other claims and gives the holder of the
lien basis for legal action, including foreclosure. Tax liens are
imposed by a taxing jurisdiction after a property becomes delinquent
and typically includes the principal tax amount, plus any interest
and penalties. Some states allow local jurisdictions to sell or
transfer liens as certificates, akin to bonds, as a way of
recouping the lost revenue in the short term.

Foreclosure is the legal process of seizing title of a property (or
the deed) and forcing its sale for the purpose of paying off a debt,
such as a tax lien. In a tax foreclosure, the local jurisdiction
petitions a court to award it the title based on an unpaid
lien. The jurisdiction may then sell the property in a tax-deed sale
or auction, hoping at least for enough to cover the tax lien. The
original owner usually has the right to regain the property if she or
he pays the back taxes within a set time period after the sale,
called the redemption period. If nobody buys the deed, the tax lien
remains unpaid, and the jurisdiction keeps the title and
responsibility for the property.

Compared to laws governing delinquency collection in some other states
and other Pennsylvania counties, the state statutes governing
Philadelphia give city government a lot of discretion in setting
policies on when to initiate foreclosures or what kind of catch-up
payment plans to offer. In the past, Philadelphia has tended to use
this discretion to delay taking action, put up fewer properties for
sale, or let delinquents enroll and default on payment plans many
times, all of which has caused delinquencies to accumulate over the
years. (As of April 2012, owners of roughly one in six delinquent
properties were paying on installment plans; they owed \$57.6 million
in taxes and penalties.) In 2013, the city adopted new rules intended
to change most of those practices.

To reduce the number of new delinquencies, Philadelphia has begun to
adopt effective strategies used in some other places. The city has
significantly reduced the time it takes to notify delinquents about
their overdue bills, as well as property owners who are merely late
and in danger of becoming delinquent. (Non-payers are considered �late�
until Dec. 31 each tax year, at which point they are considered
delinquent). Philadelphia has begun centralizing all revenue
collections, leading to faster and better coordinated actions on
delinquencies. The city also is considering creation of a �land bank�
to help acquire and redevelop tax-foreclosed properties. The city has
not adopted other practices, including the selling of tax liens, which
generates immediate revenue.

Land banks are able to buy tax-foreclosed properties, often vacant
land, for the face value of their tax liens prior to a tax
auction. That enables the city to receive full payment and avoid the
expense of an auction, while helping to move property toward
development. In 2012, the Pennsylvania legislature enacted a law
authorizing localities to create land banks specifically for the
purpose of acquiring tax-delinquent and blighted property. The Nutter
administration, members of City Council and other organizations have
been weighing their options for creating a Philadelphia land bank.



\subsection{Some Stylized Facts}

As of April 2012, the city and school district were owed \$292.3
million in delinquent taxes on 102,789 properties, \$515.4 million
when interest and penalties are included. (The numbers have increased
since then.) About one-quarter of those properties had been delinquent
for more than a decade.

Of 36 cities studied in the PCT study, Philadelphia had the fifth
highest delinquency rate in 2011, the last year for which statistics
were available. Our study found that many of the cities with lower
delinquency rates than Philadelphia adhere to stricter timetables for
imposing enforcement measures against delinquent
property-owners timetables usually set by the state�and are more
willing to take properties away from owners who do not pay their
taxes. At the same time, a lot of these cities have lower percentages
of poor people, stronger real estate markets, and higher shares of
homeowners who pay their taxes automatically through mortgages.

In Philadelphia, 9 percent of 2011 property taxes went uncollected in
that year. The median delinquency rate in the 36 cities in this study
was 4.1 percent. It was closer to the 6 percent median for 14 cities
that, like Philadelphia, have poverty rates
above 25 percent.

Finally, our forth experiment focuses on procedural justice and, more
broadly speaking, interactions between a local government and tax
payers. Theory distinguishes between respectful and authoritarian
interactions. Here the idea is that hierarchical relationships that
treat tax payers as inferiors can create resentment and tax avoidance.
Transparency and clearness are other aspects of the procedural
approach. We could design a forth experiment that addresses these
concerns. We could focus on an elderly population that is likely to be
eligible for state aid.

