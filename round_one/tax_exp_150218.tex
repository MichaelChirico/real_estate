\documentclass[12pt,titlepage]{article}

\renewcommand\baselinestretch{1.0}
\setlength{\parskip}{0.08in}
\setlength{\medskipamount}{0.05in}
\textheight 8.0 in
\textwidth 6.0 in
\topmargin 0.25in
\tolerance=11000

\usepackage[utf8]{inputenc}
\usepackage{bbm}
\usepackage[pdfencoding=auto,unicode=true]{hyperref}
\usepackage{graphicx}
\usepackage{epstopdf}


\begin{document}

\title{An Experimental Evaluation of Strategies to Increase
Property Tax Revenue Collection}
\author{Michael Chirico%
\thanks{The research presented here was supported in part by the Institute
of Education Sciences, U.S. Department of Education, through Grant
\#R305B090015 to the University of Pennsylvania. The opinions expressed
are those of the presenter and do not represent the views of the Institute
or the U.S. Department of Education.%
}, Robert Inman, Charles Loeffler, \\
John MacDonald, and Holger Sieg \\
 \\
University of Pennsylvania}
\date{\today}  
\maketitle



\renewcommand{\thefootnote}{\arabic{footnote}}

\section{Introduction}
\label{sec:intro}

The purpose of this experimental study is to evaluate a set of
strategies that are intended to increase property tax revenue
collection in the City of Philadelphia. Philadelphia has one of the 
highest delinquency rates in the United States (Cite Pew
Charitable Trusts 2013); our data shows roughly 134,000 properties,
or 23\%, late or delinquent as of November 2014.  The total tax
bill due to the city from these properties is close to 
\$600,000,000.

We design a number of different treatments that are motivated by the theoretical literature
on tax compliance, which distinguishes between extrinsic and intrinsic
motivations for paying taxes. Extrinsic motivations are primarily
based on the deterrence model which explains tax compliance as a
combination of penalties and costly monitoring. Intrinsic motivations
are captured in models that rely either on social norms or social
contracts.  We develop and implement a sequence of experiments to test
three of most commonly suggested hypotheses. The basic strategy behind
our controlled randomized experiments is to vary the informational content of a
sequence of letters sent to real estate tax-delinquent property owners.

The first experiment is based on the the deterrence model of tax
compliance which emphasizes extrinsic motivations of behavior. Key
elements of the model are the probability of detection (the monitoring
probability) and the size of the penalty or fine that is levied on those tax
evaders who are caught. Given the uncertainty of law enforcement,
behavior is based on subjective probabilities of detection and
perceptions regarding the size of penalties.\footnote{The behavioral
  literature suggests that there may be behavioral biases since agents
  deal small detection probabilities} The key idea behind the first
experiment is to change these subjective beliefs over penalties and
detection probabilities in an experimental design. Note that our
experiment does not explicitly involve tax amnesties. Neither do we change the
fiscal incentives to pay taxes in our proposed experiment. We simply
attempt to change subjective beliefs regarding detection and the
consequences of tax delinquencies by merely threatening repercussions for nonpayment.

Our second experiment is based on a more recent theory that emphasizes
the intrinsic motivations of tax payers and focuses more broadly on tax
morale. The social contract approach emphasizes the promise of the
city to efficiently deliver public goods and services in exchange for
more-or-less voluntary tax payments.  Loyalties and emotional ties
play a large role in creating high tax moral. Parallels are drawn betwen
the problem of raising taxes and the fund-raising problem faced by
non-for-profit organizations.\footnote{For example, publicly funded radio
  stations typically need to raise funds from their listeners.} The basic idea
behind the second experiment is then to design an informational
treatment that reinforces this social contract. The treatment takes
the form of a moral appeal to the delinquent tax payer and emphasizes
the \textit{quid pro quo} of tax compliance in the city--public goods
and services (namely, education and safety provision) funded by compliant taxpayers.
 
Our third experiment is based on the recent literature on social norms
which emphasizes concepts of social fairness and reciprocity. This
literature suggests that social exposure of delinquent tax payers may
be an effective tool for penalizing tax evasion.  The idea behind our
third experiment is then to appeal to delinquents' proclivities for peer conformity.

We conduct these tax revenue collection experiments in collaboration
with the Revenue Department of the City of Philadelphia, which is
an eminently useful city in which to study property tax evasion.  As of April
2012, the city and school district were owed \$292.3 million in
delinquent taxes on 102,789 properties, a figure which balloons to 
\$515.4 million when interest and penalties are included.

The rest of the paper is organized as follows. Section \ref{sec:background} discusses the
institutional background.  Section \ref{sec:treatments} provides a
detailed discussion of our treatments. Section \ref{sec:design_fidelity} discusses the experimental design
and the fidelity of its implementation. Section \ref{sec:methods} details the statistical approach
to measuring treatment response. Section \ref{sec:results} provides our current empirical findings.
Section \ref{sec:conclusions} offers some conclusions.


\section{Institutional Background}
\label{sec:background}

Real Estate Taxes in Philadelphia are levied annually on a property-level basis. 
The Office of Property Assessment evaluates the market value of each property, 
1.34 \% of which must be paid to the Philadelphia Department of Revenue (DOR). The City
then splits this take between the City's coffers and the School District of
Philadelphia (with the former getting roughly 45\% of Real Estate Tax revenues).

Tax bills are mailed by DOR in batches throughout December and early January each year; 
the owners have until March 31\textsuperscript{st} to remit their balance to the City\footnote{Those 
enterprising owners who file their taxes prior to March 1st are credited a 1 \% 
discount.}, after which time their bill begins to accrue penalties and interest.

The City begins actively pursuing non-paying properties in the September following nonpayment.
First, the city delegates roughly $\frac{2}{3}$ of the debts to the authority of one of
the two designated external law firms which have contracts with the City for this purpose,
with each being assigned half of the externalized nonpayers\footnote{Currently, these 
law firms are Linebarger, Goggan, Blair \& Sampson and Goehring, Rutter \& Boehm.}. These law firms
are free to pursue the collection of the debt as they see fit, and are rewarded with a portion
of any debts recovered for the City.

For those debts that remain targeted by DOR itself, the City traditionally leverages one of
several legal options as threat and punishment for nonpayment.
Beginning on March 31\textsuperscript{st}, the city 
regularly sends plain bills to properties still in hock--roughly once every 10
weeks\footnote{See the section on implementation below for exact details.}. More substantive
enforcement strategies begin upon expiry of the tax year on
December 31\textsuperscript{st}--since these
properties are then officially delinquent, the City has the
right to put a lien on any property which has not yet paid Real Estate Taxes 9 months
past the March 31\textsuperscript{st} deadline.
Having a lien on a property entitles the city to several powerful
recovery mechanisms. If they see fit, the City can send a property to Sheriff's Sale, which is
basically a public auction of the property, whereby all ownership rights to the property are
stripped from the former owner upon successful bid\footnote{A list of properties currently
up for Sheriff's Sale by the City of Philadelphia can be found at 
\href{http://www.officeofphiladelphiasheriff.com/en/real-estate/foreclosure-listings}
{http://www.officeofphiladelphiasheriff.com/en/real-estate/foreclosure-listings}}. Next, 
for commercial properties, the City may sequester income from the business, become the 
property manager, or even shut down the business. City employees may see their taxes 
deducted from their salaries, and applicants for City jobs are vetted for eligibility--
new employees must either be fully repaid or have started a payment plan\footnote{The City 
has also begun recently to explore several alternative enforcement options, such as levying
liens on owners' home addresses outside Philadelphia (targeting delinquent nonresident landlords)
and intermediate penalties such as impounding of delinquents' cars, etc.}.

\section{Treatments}
\label{sec:treatments}

To explore softer avenues for revenue take augmentation, we determined that the
most logistically feasible approach was to include a ``stuffer'' in the bills regularly
mailed to nonpayers, the language of which was carefully chosen to target one of three
enforcement strategies: deterrence (the ``Threat'' treatment), social normality
(the ``Moral'' treatment), and conformity (the ``Peer'' treatment). To properly isolate
the treatment effects of receiving a specific stuffer from the effects of receiving
\textit{any} stuffer (given the status quo of plain bills), we also randomly sent
properties a plain stuffer (the ``Control'' treatment).

The letters were designed carefully to differ only in the wording of their second 
paragraph; for clarity, the idiosyncratic wording is reiterated here. We also took care
to minimize communication issues by vigorously simplifying the language of our
chosen messages--shunning uncommon words and syntactic complexity by churning the
stuffers text through the latest linguistic tools for complexity analysis to be sure
our vocabulary was accessible to any nonpayers who may be of limited literacy.

Further, in accordance with the City's general desire to reach out to its substantial
immigrant populations, we agreed with the City to include Spanish translations of each
treatment's text on the reverse of the stuffers, given that Puerto Ricans make up
the plurality of Philadelphia's non-English-speaking population\footnote{See Appendix
for the exact style and full wording of each treatment, including the Spanish 
translations we ultimately chose with input from several native speakers}. 

\subsection{Stuffer 1: Deterrence}

The goal of the Threat treatment was to emphasize the repurcussions of 
noncooperation with the city and highlight the policy tools available to the city,
with the idea that owners may have poorly-formed notions of the extent of action
the city may be willing to take to recover taxes from each property.

\subsubsection*{Stuffer Text}

\begin{quotation}Not paying your Real Estate Taxes is breaking the law.\end{quotation}

\begin{quotation}Failure to pay your Real Estate Taxes may result in
seizure or sale of your property by the City.\end{quotation}

\begin{quotation}Do not make the mistake of assuming we are too busy
to pursue your case.\end{quotation}

\subsection{Stuffer 2: Moral Appeal}

The goal of the Moral treatment was to emphasize the social contract between
tax payers and the City--that, in exchange for compliance and timely submission
of taxes due to the City, the City provides a level of public goods
commensurate with the level of tax intake. In particular, we chose to 
highlight the correspondence between Real Estate Tax compliance and
the City's provision of public education and safety services.

\subsubsection*{Stuffer Text}

\begin{quotation}We understand that paying your taxes
can feel like a burden.\end{quotation}

\begin{quotation}We want to remind you of all the great services that
you pay for with your Real Estate Tax Dollars.\end{quotation}

\begin{quotation}Your tax dollars pay for schools to teach our children.
They also pay for the police and firefighters who help keep our city safe.
Please pay your taxes as soon as you can to help us pay for these
essential services.\end{quotation}

\subsection{Stuffer 3: Social Pressure}

The goal of the Peer treatment was to socially penalize delinquents
by underlining their nonconformity among their fellow Philadelphians.

\subsubsection*{Stuffer Text}

\begin{quotation}You have not paid your Real Estate Taxes.\end{quotation}

\begin{quotation}Almost all of your neighbors pay their fair share--
9 out of 10 Philadelphians do so. Paying your taxes is your duty to the city
you live in. By failing to pay, you are abusing the good will of your
Philadelphia neighbors.\end{quotation}

\subsection{Stuffer 4: Control}

The Control stuffer was designed to be as plain and unpersuasive as possible, 
with the goal of the informational content being orthogonal to each of the
three treatment stuffers. 

\subsubsection*{Stuffer Text}

\begin{quotation}The enclosed bill details your outstanding 
Real Estate Taxes due to the City of Philadelphia.\end{quotation}

\section{Experimental Design and Fidelity of Implementation}
\label{sec:design_fidelity}

Our approach to treatment was constrained by the logistics of DOR's
enforcement faculties. Ideally, we would have randomly assigned one
of the four treatment stuffers at the property level; we concluded
after several discussions with our correspondents at DOR that this 
would be logistically nigh-impossible.

Mailing of delinquent Real Estate Tax Bills by DOR works roughly as follows.
Every property in the city is assigned to one of 50 mailing cycles; since it is
cheaper and simpler to send at once all bills to those owners owing taxes on
multiple properties, assignment to cycles is done at the owner level, so that
each mailing cycle has roughly the same number of owners.
Every morning, a printer at DOR taps the in-house accounting system to find
all properties that a) owe taxes to the City and b) are in the current day's 
mailing cycle, which increases by one each day (as well as several other 
restrictions discussed in more detail below). After identifying the bills to be 
printed for the day, the printer merges several other pieces of information 
stored with the delinquent balance (e.g., the mailing address and an in-house 
ID associated with the property). The 1200 or so bills that are printed each 
day are then brought to the City's mailing room, wherein they are stuffed 
into envelopes and delivered to the property owners.

Given the volume of bills printed each day and the existing infrastructure
for processing them, especially the machine-automated process of envelope
stuffing, we determined the most practical solution would be to randomize
treatment at the mailing cycle level, so that every bill printed on the same
day would be paired with the same stuffer.

Randomization of mailing days was handled by the authors (code and
random seed available upon request). At the beginning of treatment,
we were unsure when our treatment period would end for the year. As
such, we could not easily take a simple random sample of days (i.e.,
we could not take one of the $4^T$ possible treatment schedules 
because we did not know $T$ \textit{ex ante}). We therefore elected to
take a sort of \textit{ad hoc} approach that would balance this
uncertainty with a desire to see a balance of treatment group
representation in our final sample, which we did by randomizing in
4-day cycles--for each 4-day period, we picked at random among the $4!=24$
possible arrangements of treatments over the subsequent 4 days.

We were optimistic that we could continue mailing through the Thanksgiving
holiday and even into the beginning of December, but this was snuffed
by the constraint that DOR began mailing new bills for the 2015 Tax Year
starting from early December, which is a logistically intensive process
given the roughly 400,000 bills to be sent in a few weeks' period. Ultimately,
we succeeded in stuffing bills for 15 days in November, from the 
4\textsuperscript{th} through the 25\textsuperscript{th}. The Threat
treatment group drew the short straw, as we had 4 days of treatment for the
Control, Moral, and Peer groups but only 3 days of treatment for
the Threat group. 

\begin{center}
\begin{tabular}{|c||c|c|c|c|}
\hline 
Treatment & THREAT & MORAL & PEER & CONTROL\tabularnewline
\hline 
\hline 
Count & 3 & 4 & 4 & 4\tabularnewline
\hline 
\end{tabular}
\par\end{center}

While we are certain of the sanctity of our mailing cycle-level
randomization process, a latent concern, is that assignment of properties to mailing cycles
by the city was done via some outcome-correlated mechanism. Luckily, however,
the city claims to have done so randomly, which would mean that we could
achieve proper full-scale two-stage randomization of the properties through our 
process of day-level randomization. 

In particular, the city assigned properties to cycles based on the last two
digits of an in-house ID number; those with final two digits 01 and 02 are
mapped to cycle 1, those with final digits 03 and 04 are mapped to cycle 2,
and so on. The in-house ID itself is motley in nature. For many properties,
DOR has on file the owner's Social Security Number (SSN); for many others,
mainly commercial properties, the DOR stores their Employer Identification
Number (EIN); and for the remainder of properties, DOR assigns its own
in-house ID number. This last is a 9-digit code which is assigned 
sequentially to property owners who cannot be matched to either of the 
federal ID numbers.

The assignment of SSNs has not historically been itself random (cite SSA 2015)
\footnote{This has been updated as of June 25, 2011; see 
\href{http://www.ssa.gov/employer/randomization.html}
{http://www.ssa.gov/employer/randomization.html} for details on the
newly implemented randomization scheme. We expect the lion's share of 
SSNs in our sample to have been assigned prior to this regime shift.}. In
particular, the last 4 digits within each Area Number-Group Number (first
5 digits) combination are assigned sequentially. One can imagine most property
owners were born in the Philadelphia area, and it is possible a spate of
births in a particular neighborhood engendered some degree of spatial
correlation in the assignment process, but it would be, practically speaking,
farfetched to believe this was substantially undermining randomization in the scheme
outlined above.

The assignment of EINs is less clear\footnote{The most complete source
we could find in this respect is
\href{http://www.irs.gov/Businesses/Small-Businesses-\&-Self-Employed/How-EINs-are-Assigned-and-Valid-EIN-Prefixes}
{http://www.irs.gov/Businesses/Small-Businesses-\&-Self-Employed/How-EINs-are-Assigned-and-Valid-EIN-Prefixes}.},
but for similar reasons we are satisfied with the likelihood that assignment
based on EIN digits is effectively random. Lastly, given the mapping from 
final digits to mailing cycle, because the DOR-assigned
ID is assigned sequentially, all but pairs of numbers assigned to a given 
mailing cycle will be randomly separated by the second stage of 
randomization which we implemented ourselves.

\begin{center}
\includegraphics[width=2.9in]{cycle_intended_vs_actual}
\label{fig:leakage}
\par\end{center}

Nevertheless, the skeptical reader is correct to be suspicious of this 
somewhat nontraditional randomization mechanism. This is compounded by the fact that
we never saw the code used to implement the assignment (its engineer has
since retired), and for the subsample of properties for which we we able
to obtain the ID used by DOR, as depicted in Figure \ref{fig:leakage} 
there appears to have been infrequent yet noteworthy
leakage from the asserted mapping assignment (i.e., there are properties
whose actual assignment does not conform to the officially designated mapping).

\subsection{Sample Balance on Observables}

To confirm whether or not we indeed achieved randomization, we performed
a series of balance-on-observables tests, the null hypotheses of which are that
a given observable data moment is identical across mailing cycles. We turn now to
the results of those tests.

Analysis of balance on observables is complicated by the random assignment at the owner level. 
Because there is some strong clustering by some massive holders of property (thousands of 
properties owned by public entities like the City of Philadelphia, the Philadelphia Housing 
Authority, and the Redevelopment Authority of Philadelphia, hundreds owned by many others such as
the University of Pennsylvania and Drexel University), a simple analysis of balance at the
property level is strongly skewed by these outliers; further, it's not clear how to aggregate
many of the property-level characteristics to the owner level meaningfully (especially 
geographic variables).

Our approach to solve this issue was to examine sample balance on the subset of properties for which
a) the owner is unique (i.e., owns no other properties) and b) any tax exemption claimed by the 
property is related to abatements for new construction.\footnote{A full list of exemptions can be found in the
Appendix, but the most important non-abatement restrictions cover religious institutions, institutions
of learning, and medical/health facilities.}\textsuperscript{,}
\footnote{We ran several other similar specifications, with the 
qualitative results remaining unchanged. We also ran tests on the subsample of properties for which
we could obtain the secure ID used by the City, for which the putative mapping was violated; again, the
results are qualitatively identical. See the Appendix for details.}.

% latex table generated in R 3.0.2 by xtable 1.7-4 package
% Sun Feb 15 13:28:45 2015
\begin{table}[ht]
\centering
\begin{tabular}{rllllll}
  \hline
% latex table generated in R 3.0.2 by xtable 1.7-4 package
% Sun Feb 15 17:23:13 2015
 & Threat & Moral & Peer & Control & Test & $p$-value \\ 
  \hline
Postive balance due & 4864.72 & 4614.98 & 4504.19 & 4574.43 & ANOVA & 0.36 \\ 
  Market value ('000)$^{*}$ & 185 & 205 & 195 & 196 & $\chi^2$ & 0.66 \\ 
  Land Area (ft\textsuperscript{2})$^{*}$ & 1710.88 & 1705.57 & 1702.84 & 1721.64 & $\chi^2$ & 0.87 \\ 
  \# Rooms$^{**}$ & 6.11 & 6.11 & 6.11 & 6.11 & ANOVA & 0.84 \\ 
  Case Assignment$^{***}$ &  &  &  &  & $\chi^2$ & 0.87 \\ 
  Political Ward$^{***}$ &  &  &  &  & $\chi^2$ & 0.81 \\ 
   Category: \\ 
 \hline
Residential & 0.2 & 0.27 & 0.27 & 0.27 & $\chi^2$ & 0.87 \\ 
  Hotels\&Apts & 0.2 & 0.26 & 0.27 & 0.27 &  &  \\ 
  Store w. Dwell. & 0.19 & 0.27 & 0.26 & 0.28 &  &  \\ 
  Commercial & 0.2 & 0.26 & 0.27 & 0.27 &  &  \\ 
  Industrial & 0.22 & 0.29 & 0.23 & 0.26 &  &  \\ 
  Vacant Land & 0.2 & 0.27 & 0.27 & 0.26 &  &  \\ 
   \hline
Distribution of Properties & 0.2 & 0.27 & 0.27 & 0.27 & $\chi^2$ & 0.99 \\ 
   \hline
Expected Distribution & 0.2 & 0.27 & 0.27 & 0.27 &  &  \\ 
   \hline
%End piece imported from R-xtable
\multicolumn{7}{l}
{\scriptsize{$^{*}$Tested as a two-way $\chi^2$ test of quartile vs. treatment, due to outliers.}} \\
\multicolumn{7}{l}
{\scriptsize{$^{**}$Properties below 2 or above 12 rooms were trimmed to reduce the influence of outliers.}} \\
\multicolumn{7}{l}
{\scriptsize{$^{***}$See Appendix for full two-way tables of these variables.}}
\end{tabular}
\caption{Tests of Sample Balance on Observables}
\label{table:balance}
\end{table}

As can be seen in Table \ref{table:balance}, randomization appears to have been successful. 
The properties are strongly randomly distributed by location (their political ward, of
which there are 66 in Philadelphia), category (type of property usage),
property size (as measured by the number of rooms or by the size of the tract),
and case assignment (this variable captures, if applicable, to which outside
law firm a property is assigned, whether the property is in sequestration, 
or has entered a payment agreement with the city). The number of properties
assigned to each treatment is further exactly as expected, given the unequal number of 
mailing days in our treatment.

\begin{center}
\includegraphics[width=2.9in]{total_balance_single_property_nonexempt}
\label{fig:balance_balance}
\par\end{center}

Evidence of randomization is slightly weaker for randomization on delinquent balance and
randomization on market value, though tests of both are far from rejecting the null
hypothesis of equal group means. We suspect this is largely due to the influence of outliers--
as seen in Figure \ref{fig:balance_balance}, the distributions are highly similar visually
\footnote{see the Appendix for more robust examination of these variables.}.

\subsection{Implementation Fidelity}

Given the logistic devils in the details of implemenation, we were apprehensive
about the experiment's fidelity. In addition to the foreseeable difficulty of
coordinating among many actors in order for our experiment to fit in with
the DOR's normal billing process, there was a particular hiccup in the normal
flow of work towards the beginning of our treatment period when the
DOR printer failed to print its daily allotment of bills. This reverberated
and was compounded by the fact that this happened just prior to Veterans' Day,
a City holiday, which led to a substantial backup of bill processing.

To help assuage our meticulous trepidation, we leveraged a unique secondary check on 
implementation to help update our beliefs about the extent to which our beliefs about 
who got which stuffer were correct. The Department of Revenue regularly posts envelopes
destined for addresses that are either unattended (e.g., vacant) or do not exist in the
first place (e.g., typos). Either before or after an attempted delivery to such an 
address, the postal service flags down these bills and returns the missives to DOR,
which then process them and attempts, if they can identify a suitable alternative
address, to re-deliver the tax bill. We took advantage of the fact that a (presumably
random, or at least orthogonal to assignment) subset of bills made their way back to DOR
to check firsthand the extent of treatment fidelity.

\begin{table}[ht]
\centering
\begin{tabular}{rlllllllllllllll}
  \hline
 Day: & 1 & 2 & 3 & 4 & 5 & 6 & 7 & 8 & 9 & 10 & 11 & 12 & 13 & 14 & 15 \\ 
  \hline
Intended Stuffer & M & P & T & C & T & P & C & M & M & C & P & T & C & P & M \\ 
\% Inaccurate & 0 & 35 & 55 & 21 & 70 & 35 & 0 & 0 & 0 & 38 & 1 & 0 & 9 & 7 & 2 \\ 
Actual Stuffer & M & P & T & T & P & P & C & M & M & M & P & T & C & P & M \\
   \hline
\multicolumn{16}{l}
{\scriptsize{T: Threat, M: Moral, P: Peer, C: Control}}
\end{tabular}
\caption{Measured Fidelity of Implementation}
\label{table:fidelity}
\end{table}

As can be seen from Table \ref{table:fidelity}, actual implementation was less than
perfect. We retrieved 929 returned letters, 122 (13\%) of which were found to have
a different stuffer from what we expected. On six days--the second through sixth 
and the tenth--more than ten percent of returned letters were mismatched
\footnote{See Appendix for more details about the sample of returned letters}.
A small number (8) of the letters were found to be completely empty--no
stuffer whatsoever was packaged with the posted bill. 

Given this substantial deviation from intended treatment, we present main results
only for the subsample of treatment days for which greater than 90\% fidelity
was achieved--that is, days 1, 7-9, and 12-15. For robustness purposes, we 
provide intent-to-treat (ITT) and instrumented analyses in the Appendix.

\begin{center}
\begin{tabular}{|c||c|c|c|c|}
\hline 
Treatment & THREAT & MORAL & PEER & CONTROL\tabularnewline
\hline 
\hline 
Count & 1 & 4 & 2 & 2\tabularnewline
\hline 
\end{tabular}
\par\end{center}

\section{Methods}
\label{sec:methods}

To analyze the statistical effect of our experiments, we implement four tests
consisting of two pairs of related tests. One pair examines the effect of
treatment on two binary outcomes--namely, one indicator for whether a given
account submitted some payment to DOR in the sample period and another
for whether the account at a property was cleared completely. The other pair
turns instead to the intensity of activity. Specifically, we attempt to 
detect if treatment causally impacted the dollar value of payments submitted
to the city or the relative value of payments submitted to the city.

We turn next to a more formal presentation of the models outlined above.

\subsection{Model I: Logistic Regression of 
\texorpdfstring{$\mathbbm{1}\left[Ever Paid\right]$}{\unichar{"1D7D9}[Ever Paid]}}

Let $y_{I,i}=\mathbbm{1}\left[x_i>0\right]$, where $\mathbbm{1}\left[\cdot\right]$
is an indicator taking the value one when its argument $\cdot$ is true and
0 otherwise and $x_i$ is the cumulative remittance to the city at the conclusion
of the sample period by property $i$.

Given the randomized nature of our analysis, we get consistent estimates of
the causal impact of treatment on $y_I$ by evaluating the following logistic
regression model:

\begin{equation}
y_{I,i}=\beta_{I}+D_{T,i}\gamma_{I,T}+D_{M,i}\gamma_{I,M}+D_{P,i}\gamma_{I,P}
+\epsilon_{I,i},\hspace{1em}\epsilon_I\enskip\mbox{logistic}
\end{equation}

The $D_{k,i}$ are indicators for the three treatments, i.e.
$D_{k,i}=\mathbbm{1}\left[treatment_i=k\right]$, $k\in{T,M,P}$ for Threat, Moral, and Peer,
respectively. The $\gamma_{I,k}$, then, are the causal impacts of the treatments on the
likelihood of some degree of remittance to the city, relative to the control treatment of
a plain stuffer.

These estimates are consistent, but for precision's sake, it is prudent to include
some controls which, though asymptotically orthogonal to the treatments, retain
some correlations in small samples and thus are a source of standard error inefficiency.
As such, we also estimate logistic regressions with controls of the form:

\begin{equation}
y_{I,i}=X_i^T\beta_{I}+D_{T,i}\gamma_{I,T}+D_{M,i}\gamma_{I,M}+D_{P,i}\gamma_{I,P}
+\epsilon_{I,i},\hspace{1em}\epsilon_I\enskip\mbox{logistic}
\end{equation}

We include such variables as (log) land area, maturity of debt (more or less than 5 years),
geographic location (as proxied by City Council District), usage category, 
property exterior condition (whether or not the property was categorized as sealed/compromised
by the city), whether the property took a homestead exemption\footnote{Philadelphia
offers a tax discount of \$30,000 off the taxable value of the property for those
residents who are verifiably owner-occupants of a property--thus not taking a homestead
exemption is a signal that a property is operated by a remote landlord.}, (log) balance at mailing,
and (log) market value.

\subsection{Model II: Logistic Regression of
\texorpdfstring{$\mathbbm{1}\left[PaidFull\right]$}{\unichar{"1D7D9}[Paid Full]}}

Next, we examine a more restrictive yes-no participation outcome--namely, whether
or not the property offered not just token repayment in our sample period, but 
whether its debts were paid back in full\footnote{Due to some measurement issues, it is
not possible to track on a day-to-day basis exactly the balance due for each property--
accrual of interest and other charges is hard to pinpoint exactly. In the main results 
below, we actually measure full repayment as submission of at least 95\% of the 
balance due; see the Appendix for some exploration of the robustness of this threshold,
but suffice it so say here that results are qualitatively identical.}.

To this end, let $y_{II,i}=\mathbbm{1}\left[x_i=m_i\right]$ be an indicator for 
full repayment of debt in the sample timeframe ($m_i$ is the amount owed by 
property $i$ at the start of treatment). Then causal estimates of the impacts of
treatment are recovered through estimation of the logistic model:

\begin{equation}
y_{II,i}=\beta_{II}+D_{T,i}\gamma_{I,T}+D_{M,i}\gamma_{I,M}+D_{P,i}\gamma_{I,P}
+\epsilon_{II,i},\hspace{1em}\epsilon_{II}\enskip\mbox{logistic}
\end{equation}

Again for precision's sake, we estimate the controlled analogue of this regression:

\begin{equation}
y_{II,i}=X_i^T\beta_{II}+D_{T,i}\gamma_{I,T}+D_{M,i}\gamma_{I,M}+D_{P,i}\gamma_{I,P}
+\epsilon_{II,i},\hspace{1em}\epsilon_{II}\enskip\mbox{logistic}
\end{equation}

\subsection{Model III: Tobit Regression of Amount Paid vs. Market Value}

Next, we turn our attention to measuring effects of treatment on the magnitude of
debt drawdown--that is, instead of answering questions to the effect of
``Did our treatment(s) significantly induce repayment?'', we hope to address
the intermediate question of ``Did our treatment(s) induce significantly more 
repayment?''.

In particular, we hope to isolate the relative magnitude of repayment by
essentially normalizing the dollar amount repaid by the ``size'' of the property,
so that bigger payments by ``bigger'' properties in some group don't mechanically skew our
results in favor of that group.

In our first Tobit Model, our notion of ``size'' is the market value of 
the property. The estimates of this model will highlight any significant differences
in the amount repaid relative to market value.

Specifically, we estimate the following Tobit model:

\begin{eqnarray}
\log x_{i}^{*} & = & \beta_{III}+\delta_{III}\log v_{i}+
\sum_{k\in\{T,M,P\}}D_{k,i}\left(\gamma_{III,k}+\log v_{i}\eta_{III,k}\right)+u_{III,i}\\
x_{i} & = & \mathbbm{1}\left[x_{i}^{*}>=0\right]x_{i}^{*}
\end{eqnarray}

$x_{III,i}^{*}$ is the latent repayment amount, whose observed counterpart $x_i$ only takes
nonzero value when $x_i^{*}$ is strictly positive. By controlling for (log) market value $v_i$,
the $\eta_{III,k}$ represent the group-specific additional repayment for each (log) 
market value. Thus a positive coefficient on $D_{k,i}\times\log v_i$ has the interpretation
that higher-value properties were more likely to pay their taxes.

The controlled analogue to Model III is as follows:

\begin{eqnarray}
\log x_{III,i}^{*} & = & X^T_i \beta_{III}+\delta_{III}\log v_{i}+
\sum_{k\in\{T,M,P\}}D_{k,i}\left(\gamma_{III,k}+\log v_{i}\eta_{III,k}\right)+u_{III,i}\\
x_{i} & = & \mathbbm{1}\left[x_{III,i}^{*}>=0\right]x_{III,i}^{*}
\end{eqnarray}

\subsection{Model IV: Tobit Regression of Amount Paid vs. Amount Owed}

Our second pair of Tobit regressions entails a different (but correlated) measure
of the ``size'' of a property--namely, the amount owed by the property at the 
onset of treatment\footnote{Though, on each year's initial tax bill, the market value
is exactly proportional to the property's tax debt (modulo any exemptions claimed,
but these properties are excluded from our analysis), the history of payments 
in general stands to create a wedge in this proportional relationship--low-value
properties that have a long history of nonpayment or high value properties that
have already partially repaid their debts drive dissonance in either direction.}.
Here, we are primarily concerned with whether those properties that owed more
to begin with ended up remitting more to DOR in our sample timeframe. 

Specifically, Model IV is the Tobit regression:

\begin{eqnarray}
\log x_{IV,i}^{*} & = & \beta_{IV}+\delta_{IV}\log m_{i}+
\sum_{k\in\{T,M,P\}}D_{k,i}\left(\gamma_{IV,k}+\log m_{i}\eta_{IV,k}\right)+u_{IV,i}\\
x_{i} & = & \mathbbm{1}\left[x_{IV,i}^{*}>=0\right]x_{IV,i}^{*}
\end{eqnarray}

Again, $x_{IV,i}^{*}$ is the latent repayment amount. The controlled version is:

\begin{eqnarray}
\log x_{IV,i}^{*} & = & X_i^T \beta_{IV}+\delta_{IV}\log m_{i}+
\sum_{k\in\{T,M,P\}}D_{k,i}\left(\gamma_{IV,k}+\log m_{i}\eta_{IV,k}\right)+u_{IV,i}\\
x_{i} & = & \mathbbm{1}\left[x_{IV,i}^{*}>=0\right]x_{IV,i}^{*}
\end{eqnarray}

\section{Empirical Results}
\label{sec:results}

\subsection{Data and Sample Selection}

\begin{enumerate}
\item {\footnotesize{Delinquency Snapshot, November 3rd}}{\footnotesize \par}

\begin{enumerate}
\item {\scriptsize{Covariates on every property with positive balance due
as of 11/3}}{\scriptsize \par}
\item {\scriptsize{Owner name, stock debt breakdown (principal/interest/penalty/other/total),
\# years owed \& rough period, taxable value, tax abatements, homestead,
building usage code, property address \& $\left(X,Y\right)$ coordinates,
mailing address, City Council District}}{\scriptsize \par}
\item {\scriptsize{Also used for sample restrictions (next): Payment agreements,
case status, sheriff's sale status, bankruptcy status, sequestration
status, returned mail flag}}{\scriptsize \par}
\end{enumerate}
\item {\footnotesize{Payments File (October 6th - January 6th)}}{\footnotesize \par}

\begin{enumerate}
\item {\scriptsize{Every payment (credit/debit) from every account over
the period}}{\scriptsize \par}
\item {\scriptsize{Valid \& posting dates, period and breakdown (principal/interest/penalty/other)
of debited debt}}{\scriptsize \par}
\end{enumerate}
\item {\footnotesize{Office of Property Assessment Data}}{\footnotesize \par}

\begin{enumerate}
\item {\scriptsize{Covariates maintained by the OPA for assessment--every
property in Philadelphia}}{\scriptsize \par}
\item {\scriptsize{market value, property usage category, land area, exterior
condition}}{\scriptsize \par}
\end{enumerate}
\item {\footnotesize{Mailing Cycle info}}{\footnotesize \par}

\begin{enumerate}
\item {\scriptsize{mainly, the mailing cycle associated with each property}}{\scriptsize \par}
\end{enumerate}
\end{enumerate}

Sample restrictions (original sample: 134,888 delinquent properties)
\begin{enumerate}
\item Payment agreement (23\%=31456)
\item Any tax abatement (5\% = 4706)
\item Not handled by DOR (62\%=61170)
\item Sheriff's Sale (11\%=4098)
\item Bankruptcy (3\%=948)
\item Sequestration (3\%=1130)
\item Returned mail flag (5\%=1429)
\item Not mailed during treatment between Nov. 4 \& Nov. 24 (83\%=24800)
\end{enumerate}
Final sample: 5151 properties

Timing restrictions
\begin{itemize}
\item Want same measurement timeframe for all cycles (and hence all properties)

\begin{itemize}
\item Prior to mailing day, data limited by first mailing day (40 days between
start of sample \& mailing day 1)
\item Subsequent to mailing day, data limited by final mailing day (currently
36 days between end of sample \& final mailing day)
\end{itemize}
\end{itemize}
So in the end, we currently have 77 relatively matched days of payment
behavior for each account

(for a total of 77{*}5151=393085 day-account observations)

Our choice of mailing day is bolstered by the fact that we found the most
locally common day of payment for each cycle (see Figure \ref{fig:mailing_day})

\begin{center}
\includegraphics[width=2.9in]{fidelity_check_mailing_day_act}
\label{fig:mailing_day}
\par\end{center}

\subsection{Descriptive Statistics}

\begin{table}[ht] \centering \begin{tabular}{rrrrrr}
\hline
                 & Full & Threat & Moral & Peer & Control \\
\hline
Years: [0,2]          & 0.66 & 0.63 & 0.64 & 0.69 & 0.69 \\
Years: (2,5]          & 0.17 & 0.16 & 0.18 & 0.16 & 0.16 \\
Years: (5,10]         & 0.09 & 0.12 & 0.09 & 0.08 & 0.08 \\
Years: (10,20]        & 0.06 & 0.06 & 0.06 & 0.05 & 0.05 \\ 
Years: (20,40]        & 0.02 & 0.03 & 0.02 & 0.02 & 0.02 \\   
Cat: Commercial       & 0.04 & 0.04 & 0.04 & 0.04 & 0.02 \\    
Cat: Hotels/Apts      & 0.08 & 0.10 & 0.09 & 0.08 & 0.08 \\   
Cat: Industrial       & 0.01 & 0.02 & 0.01 & 0.01 & 0.02 \\    
Cat: Residential      & 0.69 & 0.65 & 0.70 & 0.70 & 0.70 \\   
Cat: Store+Resid      & 0.07 & 0.06 & 0.07 & 0.07 & 0.06 \\    
Cat: Vacant           & 0.10 & 0.14 & 0.09 & 0.11 & 0.12 \\    
\% Sealed/Compromised & 0.03 & 0.02 & 0.04 & 0.02 & 0.03 \\    
\% Homestead          & 0.23 & 0.19 & 0.23 & 0.24 & 0.22 \\     
\hline 
\end{tabular} 
\end{table}

We present results as the outcomes of estimation of the aforementioned models,
followed by some time series demonstrating the evolution of the results 
throughout the sample period and some subsample analysis to isolate some sources
of the overall effects as well as to illustrate differential effects of 
treatment beyond our main results.

\begin{table}
\begin{center}
\begin{tabular}{l c c }
\hline
                       & Ever Paid & Ever Paid (with Controls) \\
\hline
Intercept              & $-1.74^{***}$ & $-3.35^{***}$ \\
                       & $(0.08)$      & $(0.74)$      \\
Treatment Group        &               &               \\
                       &               &               \
\\quad Moral            & $-0.06$       & $-0.07$       \\
                       & $(0.10)$      & $(0.10)$      \\
\quad Peer             & $0.22$        & $0.18$        \\
                       & $(0.11)$      & $(0.12)$      \\
\quad Threat           & $-0.10$       & $-0.07$       \\
                       & $(0.15)$      & $(0.15)$      \\
Log Balance at Mailing &               & $-0.00$       \\
                       &               & $(0.01)$      \\
Log Market Value       &               & $0.09$        \\
                       &               & $(0.06)$      \\
\hline
AIC                    & 4343.10       & 4225.54       \\
BIC                    & 4369.25       & 4382.46       \\
Log Likelihood         & -2167.55      & -2088.77      \\
Deviance               & 4335.10       & 4177.54       \\
Num. obs.              & 5105          & 5105          \\
\hline
\multicolumn{3}{l}{\scriptsize{$^{***}p<0.001$, $^{**}p<0.01$, $^*p<0.05$. Control coefficients omitted for brevity; see Appendix}}
\end{tabular}
\caption{Model I: Logistic on $\mathbbm{1}[EverPaid]$}
\label{table:modelI}
\end{center}
\end{table}

As can be seen from Table \ref{table:modelI}, none of the treatments had
a strongly significant effect at the conclusion of the sample period; the 
Peer treatment was marginally signficant ($p=.056$), but this
significance disappears upon inclusion of control variables ($p=.124$).

\noindent NEED TO DO FORMAL TESTING OF TIME SERIES

\begin{center}
\includegraphics[width=2.9in]{time_series_pct_ever_paid_act}
\label{fig:ever_paid_act}
\par\end{center}

The end-of-sample insignificance contrasts with a sustained outpacing of the 
rise in repayment participation since about two weeks subsequent to mailing.
This can be seen in Figure \ref{fig:ever_paid_act}.

\begin{table}
\begin{center}
\begin{tabular}{l c c }
\hline
                       & Paid in Full & Paid in Full (with Controls) \\
\hline
Intercept              & $-2.28^{***}$ & $-3.04^{**}$  \\
                       & $(0.10)$      & $(1.03)$      \\
Treatment Group        &               &               \\
                       &               &               \\
\quad Moral            & $-0.41^{**}$  & $-0.40^{**}$  \\
                       & $(0.13)$      & $(0.14)$      \\
\quad Peer             & $0.25$        & $0.21$        \\
                       & $(0.14)$      & $(0.14)$      \\
\quad Threat           & $-0.22$       & $-0.19$       \\
                       & $(0.19)$      & $(0.20)$      \\
Log Balance at Mailing &               & $-0.11^{***}$ \\
                       &               & $(0.02)$      \\
Log Market Value       &               & $-0.02$       \\
                       &               & $(0.08)$      \\
\hline
AIC                    & 2915.19       & 2748.35       \\
BIC                    & 2941.34       & 2905.26       \\
Log Likelihood         & -1453.60      & -1350.17      \\
Deviance               & 2907.19       & 2700.35       \\
Num. obs.              & 5105          & 5105          \\
\hline
\multicolumn{3}{l}{\scriptsize{$^{***}p<0.001$, $^{**}p<0.01$, $^*p<0.05$. Control coefficients omitted for brevity; see Appendix}}
\end{tabular}
\caption{Model II: Logistic on $\mathbbm{1}[PaidFull]$}
\label{table:modelII}
\end{center}
\end{table}

Table \ref{table:modelII} presents the results of the second model,
which examined the likelihood of full repayment of debts by
treatment group. As with Model I, the Peer treatment shows signs from
about two weeks past treatment onset of outpacing the control group;
as seen in Figure \ref{fig:paid_full_act}, the trajectory of full
repayment tracks that of partial repayment for the Peer group quite well.

\begin{center}
\includegraphics[width=2.9in]{time_series_pct_paid_full_act}
\label{fig:paid_full_act}
\par\end{center}

Further, the Moral treatment caused a significant reduction in
full repayment. Figure \ref{fig:paid_full_act} suggests full
repayment started flagging about three weeks after treatment after
tracking the control group quite well thereto. The Threat
treatment is all but indistinguishable from the control.

\begin{table}
\begin{center}
\begin{tabular}{l c c }
\hline
                       & Paid vs. Market Value & Paid vs. Market Value (with Controls) \\
\hline
Intercept              & $-12348.07^{***}$ & $-11096.96^{***}$ \\
                       & $(2318.61)$       & $(2883.17)$       \\
Treatment Group        &                   &                   \\
                       &                   &                   \\
\quad Moral            & $-7128.53^{*}$    & $-7441.16^{*}$    \\
                       & $(2893.30)$       & $(2985.93)$       \\
\quad Peer             & $1398.56$         & $806.87$          \\
                       & $(3150.31)$       & $(3223.92)$       \\
\quad Threat           & $-1364.18$        & $-1456.06$        \\
                       & $(3998.04)$       & $(4110.60)$       \\
Log Market Value       & $563.40^{**}$     & $404.39$          \\
                       & $(200.81)$        & $(253.71)$        \\
Moral*MV               & $624.16^{*}$      & $653.08^{*}$      \\
                       & $(251.37)$        & $(259.30)$        \\
Peer*MV                & $-79.53$          & $-32.17$          \\
                       & $(274.75)$        & $(281.09)$        \\
Threat*MV              & $106.23$          & $114.47$          \\
                       & $(348.01)$        & $(357.51)$        \\
Log Balance at Mailing &                   & $61.90$           \\
                       &                   & $(45.97)$         \\
\hline
AIC                    & 17746.27          & 17729.78          \\
BIC                    & 17805.12          & 17912.85          \\
Log Likelihood         & -8864.14          & -8836.89          \\
Deviance               & 3088.60           & 3027.81           \\
Total                  & 5105              & 5105              \\
Left-censored          & 4331              & 4331              \\
Uncensored             & 774               & 774               \\
Right-censored         & 0                 & 0                 \\
Wald Test              & 83.35             & 125.02            \\
\hline
\multicolumn{3}{l}{\scriptsize{$^{***}p<0.001$, $^{**}p<0.01$, $^*p<0.05$. Control coefficients omitted for brevity; see Appendix}}
\end{tabular}
\caption{Model III: Tobit of Repayment vs. Market Value}
\label{table:modelIII}
\end{center}
\end{table}

The resuls of the first Tobit model are displayed in Table \ref{table:modelIII}.
Only about 15\% of observations were uncensored. Results of Model III suggest that
those owing the most in the Moral treatment were significantly more likely
to pay more than their control counterparts. This is confirmed in Figure
\ref{fig:repay_quart_act}, which depicts the trajectories of (normalized) repayments
by each group, for each quartile of debt owed at mailing day. Though the 
paths are largely similar for the first three quartiles, those in the highest
quartile of property value ultimately renumerated much more.

\begin{center}
\includegraphics[width=2.9in]{time_series_average_payments_by_quartile_act}
\label{fig:repay_quart_act}
\par\end{center}


\begin{table}
\begin{center}
\begin{tabular}{l c c }
\hline
                       & Paid vs. Total Debt & Paid vs. Total Debt (with Controls) \\
\hline
Intercept              & $-5478.88^{***}$ & $-13650.77^{***}$ \\
                       & $(598.31)$       & $(2074.35)$       \\
Treatment Group        &                  &                   \\
                       &                  &                   \\
\quad Moral            & $-1505.26^{*}$   & $-1669.69^{*}$    \\
                       & $(752.51)$       & $(748.86)$        \\
\quad Peer             & $779.26$         & $773.16$          \\
                       & $(785.66)$       & $(778.14)$        \\
\quad Threat           & $-629.38$        & $-579.61$         \\
                       & $(1031.67)$      & $(1029.50)$       \\
Log Balance at Mailing & $-107.27$        & $-37.85$          \\
                       & $(87.24)$        & $(87.50)$         \\
Moral*Balance          & $249.08^{*}$     & $272.72^{*}$      \\
                       & $(110.60)$       & $(110.23)$        \\
Peer*Balance           & $-44.45$         & $-58.25$          \\
                       & $(119.07)$       & $(118.16)$        \\
Threat*Balance         & $79.00$          & $70.76$           \\
                       & $(151.02)$       & $(151.20)$        \\
Log Market Value       &                  & $687.78^{***}$    \\
                       &                  & $(175.00)$        \\
\hline
AIC                    & 17820.05         & 17728.50          \\
BIC                    & 17878.89         & 17911.57          \\
Log Likelihood         & -8901.02         & -8836.25          \\
Deviance               & 3119.44          & 3022.06           \\
Total                  & 5105             & 5105              \\
Left-censored          & 4331             & 4331              \\
Uncensored             & 774              & 774               \\
Right-censored         & 0                & 0                 \\
Wald Test              & 13.48            & 126.19            \\
\hline
\multicolumn{3}{l}{\scriptsize{$^{***}p<0.001$, $^{**}p<0.01$, $^*p<0.05$. Control coefficients omitted for brevity; see Appendix}}
\end{tabular}
\caption{Model IV: Tobit of Repayment vs. Total Debt}
\label{table:modelIV}
\end{center}
\end{table}

We move finally to the results from the final model, which focuses not on 
payments relative to property value, but instead on payments relative
to the balanced owed by each account at mailing day. The results of this
model are qualitatively identical to those of Model III, which is further
mirrored in the time series results of Figure \ref{fig:drawdown_quart_act}.

\begin{center}
\includegraphics[width=2.9in]{time_series_debt_paydown_by_quartile_act}
\label{fig:drawdown_quart_act}
\par\end{center}


\subsection{Robustness}

\subsubsection{Sensitivity to multiple ownership}
\begin{center}
\includegraphics[width=2.9in]{time_series_pct_ever_paid_act_single_owners}
\label{fig:ever_paid_act_so}
\par\end{center}

\begin{center}
\includegraphics[width=2.9in]{time_series_pct_paid_full_act_single_owners}
\label{fig:paid_full_act_so}
\par\end{center}

\begin{center}
\includegraphics[width=2.9in]{time_series_average_payments_int_single_owners}
\label{fig:repay_act_so}
\par\end{center}

\begin{center}
\includegraphics[width=2.9in]{time_series_debt_paydown_int_single_owners}
\label{fig:drawdown_act_so}
\par\end{center}

\subsubsection{Sensitivity to Intended vs. Actual Treatment}
\begin{center}
\includegraphics[width=2.9in]{time_series_pct_ever_paid_int_single_owners}
\label{fig:ever_paid_int_so}
\par\end{center}

\begin{center}
\includegraphics[width=2.9in]{time_series_pct_paid_full_int_single_owners}
\label{fig:paid_full_int_so}
\par\end{center}

\begin{center}
\includegraphics[width=2.9in]{time_series_average_payments_int_single_owners}
\label{fig:repay_int_so}
\par\end{center}

\begin{center}
\includegraphics[width=2.9in]{time_series_debt_paydown_int_single_owners}
\label{fig:drawdown_int_so}
\par\end{center}

\section{Conclusions}
\label{sec:conclusions}

\end{document}  



\section{Property Tax Compliance in Philadelphia}

The following is a summary of the recent study by the Pew Charitable
Trust entitled ``DELINQUENT PROPERTY TAX IN PHILADELPHIA'', published in
2013.

Philadelphia operates with relatively wide latitude under the
Pennsylvania Municipal Claims and Tax Lien Law (MCTLL), enacted in
1923 and amended at least a dozen times. It applies mainly to the
state's first class and second class cities, namely Philadelphia and
Pittsburgh.

In Philadelphia, a property is considered tax delinquent nine months
after the city's March 31 payment deadline passes. Delinquent parcels
are assessed interest and fees, which over time can grow larger than
the principal. By law, properties can be sold at public auction as
quickly as nine months after they become delinquent. In practice, that
rarely happens, and many properties are allowed to linger on the
delinquency rolls for decades.

A tax lien may be imposed for delinquent taxes owed on real property
as a result of failure to pay taxes. A claim for payment that takes
precedence over all other claims and gives the holder of the
lien basis for legal action, including foreclosure. Tax liens are
imposed by a taxing jurisdiction after a property becomes delinquent
and typically includes the principal tax amount, plus any interest
and penalties. Some states allow local jurisdictions to sell or
transfer liens as certificates, akin to bonds, as a way of
recouping the lost revenue in the short term.

Foreclosure is the legal process of seizing title of a property (or
the deed) and forcing its sale for the purpose of paying off a debt,
such as a tax lien. In a tax foreclosure, the local jurisdiction
petitions a court to award it the title based on an unpaid
lien. The jurisdiction may then sell the property in a tax-deed sale
or auction, hoping at least for enough to cover the tax lien. The
original owner usually has the right to regain the property if she or
he pays the back taxes within a set time period after the sale,
called the redemption period. If nobody buys the deed, the tax lien
remains unpaid, and the jurisdiction keeps the title and
responsibility for the property.

Compared to laws governing delinquency collection in some other states
and other Pennsylvania counties, the state statutes governing
Philadelphia give city government a lot of discretion in setting
policies on when to initiate foreclosures or what kind of catch-up
payment plans to offer. In the past, Philadelphia has tended to use
this discretion to delay taking action, put up fewer properties for
sale, or let delinquents enroll and default on payment plans many
times, all of which has caused delinquencies to accumulate over the
years. (As of April 2012, owners of roughly one in six delinquent
properties were paying on installment plans; they owed \$57.6 million
in taxes and penalties.) In 2013, the city adopted new rules intended
to change most of those practices.

To reduce the number of new delinquencies, Philadelphia has begun to
adopt effective strategies used in some other places. The city has
significantly reduced the time it takes to notify delinquents about
their overdue bills, as well as property owners who are merely late
and in danger of becoming delinquent. (Non-payers are considered �late�
until Dec. 31 each tax year, at which point they are considered
delinquent). Philadelphia has begun centralizing all revenue
collections, leading to faster and better coordinated actions on
delinquencies. The city also is considering creation of a �land bank�
to help acquire and redevelop tax-foreclosed properties. The city has
not adopted other practices, including the selling of tax liens, which
generates immediate revenue.

Land banks are able to buy tax-foreclosed properties, often vacant
land, for the face value of their tax liens prior to a tax
auction. That enables the city to receive full payment and avoid the
expense of an auction, while helping to move property toward
development. In 2012, the Pennsylvania legislature enacted a law
authorizing localities to create land banks specifically for the
purpose of acquiring tax-delinquent and blighted property. The Nutter
administration, members of City Council and other organizations have
been weighing their options for creating a Philadelphia land bank.



\subsection{Some Stylized Facts}

As of April 2012, the city and school district were owed \$292.3
million in delinquent taxes on 102,789 properties, \$515.4 million
when interest and penalties are included. (The numbers have increased
since then.) About one-quarter of those properties had been delinquent
for more than a decade.

Of 36 cities studied in the PCT study, Philadelphia had the fifth
highest delinquency rate in 2011, the last year for which statistics
were available. Our study found that many of the cities with lower
delinquency rates than Philadelphia adhere to stricter timetables for
imposing enforcement measures against delinquent
property-owners timetables usually set by the state�and are more
willing to take properties away from owners who do not pay their
taxes. At the same time, a lot of these cities have lower percentages
of poor people, stronger real estate markets, and higher shares of
homeowners who pay their taxes automatically through mortgages.

In Philadelphia, 9 percent of 2011 property taxes went uncollected in
that year. The median delinquency rate in the 36 cities in this study
was 4.1 percent. It was closer to the 6 percent median for 14 cities
that, like Philadelphia, have poverty rates
above 25 percent.

Finally, our forth experiment focuses on procedural justice and, more
broadly speaking, interactions between a local government and tax
payers. Theory distinguishes between respectful and authoritarian
interactions. Here the idea is that hierarchical relationships that
treat tax payers as inferiors can create resentment and tax avoidance.
Transparency and clearness are other aspects of the procedural
approach. We could design a forth experiment that addresses these
concerns. We could focus on an elderly population that is likely to be
eligible for state aid.

