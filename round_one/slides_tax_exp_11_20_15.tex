\documentclass{beamer}
\renewcommand{\thefootnote}{\fnsymbol{footnote}}
\usepackage{graphics,pdflscape}
\usepackage[utf8]{inputenc}
\usepackage{longtable}
\usepackage{bbm}
\usepackage{rotating}

\title{An Experimental Evaluation of Strategies to Increase Property Tax Compliance: \\
Free-Riding in the City of Brotherly Love}
\author{Michael Chirico, Robert Inman, Charles Loeffler, \\
John MacDonald, and Holger Sieg \\
University of Pennsylvania}


\begin{document}

\frame{\titlepage}


\frame{
\frametitle{Tax Compliance}
\begin{itemize}
\item The lack of tax compliance has become a policy issue of central
importance to all levels of government in developed and developing
economies.
\item In 2009, the developed economies of the OECD reported an average tax
non-compliance rate of 14.2.
\item The rates range from a low of 2-3 percent in
Austria, Denmark, Germany, Korea, and Norway to 25 percent or more in
Belgium, Iceland, Portugal, the Slovak Republic, to a high of 73
percent in Greece. 
\item In developing economies with significant cash economies, tax non-compliance likely much higher. 
\item The OECD estimates an average rate of tax non-compliance in non-OECD
countries of 37 percent.
\end{itemize}
}

\frame{
\frametitle{Why is Noncompliance a Significant Concern?}
\begin{enumerate}
\item Governments are denied the revenues needed to provide basic
  public services essential for ensuring the safety, health, and
  minimal well-being of all citizens.
\item If there is significant non-compliance and basic services are to
  be provided, then tax rates will need to rise on those who pay
  taxes.
\item Non-compliance undermines the principle that everyone has to pay
  their ``fair share" of taxes.
  \item Significant non-compliance may threaten the stability of democratic governance. 
\end{enumerate}
}


\frame{
\frametitle{What do I owe?}
\begin{itemize}
\item Taxpayers have the ability to influence what is owed on any tax
  that requires self-reporting of income or assets, such as
  self-reported consulting or business income.
\item The need for self-reporting
is reduced as the formal economy and the use of audited business
records expands.  
\item Taxes which can take advantage of those records
maximize tax compliance and are preferred for just this reason. 
\item Examples are the value added tax and the property tax.
\end{itemize}
}

\frame{
\frametitle{What happens if I don't pay?}
\begin{itemize}
\item The taxing jurisdiction can also control compliance by influencing the
decision to evade, once the tax liability has been assessed.  
\item The most common strategy is the economic stick - {\bf fines and penalties.}  
\item Liens and sheriff sales are more extreme and costly measures to
  enforce property tax compliance.
\end{itemize}
}


\frame{
\frametitle{Are there Alternatives to Fines and Penalties?}
\begin{itemize}
\item Researchers have turned attention to stressing other 
motivations for increasing tax compliance.
\item Such motives are grounded in the value taxpayers place in
  helping to achieve positive outcomes such as the {\bf provision of
    public services}.
\item Individuals may also derive satisfaction from knowing, or from
  having others know, that they have done their {\bf civic duty}.
\item Researchers have started to explore the effectiveness of softer,
  nudge approaches to reinforce the different motivations of tax compliance.
\end{itemize}
}



\frame{
\frametitle{An Experimental Evaluation of Nudge Strategies}
\begin{itemize}
\item In the  fall of 2014, we assisted the City of
Philadelphia's Department of Revenue  in designing an experiment to evaluate
potential nudge strategies to improve local property tax collection.
\item The City's historical performance in tax
compliance has not been good, collecting only 90 percent of assessed
property tax revenues compared to an average compliance rate among
large U.S. cities of nearly 95 percent. 
\item We conducted a controlled experiment redrafting letters
  reminding taxpayers that their 2014 property tax payments were
  overdue.
\end{itemize}
}

\frame{

\begin{table}
\caption{Property Tax Compliance: 2005-2014} 
\begin{center}
{\footnotesize
\begin{tabular}{| l | c |  c|}
\hline 
\multicolumn{1}{|c|}{\textbf{City}} & \multicolumn{1}{c|}{\textbf{Percent Compliance}} & \multicolumn{1}{c|}{\textbf{Delinquent Tax Collected }} \\ 
\multicolumn{1}{|c|}{\textbf{}} & \multicolumn{1}{c|}{\textbf{Current Yr; 10 Year Average}} & \multicolumn{1}{c|}{\textbf{Five Year, Yearly Average}} \\ 
\hline
Large City Average & .946; .945	& .112 \\
Atlanta*	  & .982 ;  .960 & .182 \\
Baltimore*	 & .960 ;  .950	 & .128 \\
Boston	         & .996;  .992	 & - \\
Chicago*	 & .962;  .930	 & - \\
Cleveland*	 & .841;  .850	 & .090 \\
Detroit*	 & .683;  .891	 & - \\
Flint*	         & .654;  .785	 & .151 \\
New York City	 & .915;  .925	 & .041 \\
Philadelphia*	 & .940;  .880	 & .125 \\
Pittsburgh*	 & .849;  .860	 & .048 \\
San Francisco	 & .988;  .980	 & - \\
\hline
\end{tabular}
}
\end{center}
\end{table}
}


\frame{
\frametitle{Standard Letter}
\begin{center}
\includegraphics[width=4in, height=4in]{PastDueLetter.pdf}
\end{center}
}

\frame{
\frametitle{Our Notification Strategies}
\begin{itemize}
\item We proposed three additional formats for DoR's reminder
letter.  (See the appendix of the paper.)
\item In addition to the listing of tax payments, interest, and
penalties, the alternative letters contained a sentence that either
\begin{enumerate}
\item threatened the potential loss of the taxpayer's home or property
if taxes were not paid, or 
\item  appealed to the positive community
benefits in provided public services that the taxpayer's dollars
provide, or 
\item appealed to the positive benefits of fulfilling
your civic duty to yourself and others by paying your taxes.
\end{enumerate}
\item For legal reasons we were not allowed to modify the reminder
  letter directly. Hence, we used a second letter that was added to
  the standard letter in the mailing cycles.
\end{itemize}
}


\frame{
\frametitle{Threat}
\begin{center}
\includegraphics[width=3in, height=3in]{flyer_options_141104_treat1.pdf}
\end{center}
}


\frame{
\frametitle{Service Appeal}
\begin{center}
\includegraphics[width=3in, height=3in]{flyer_options_141104_treat2.pdf}
\end{center}
}

\frame{
\frametitle{Civic Duty Appeal}
\begin{center}
\includegraphics[width=3in, height=3in]{flyer_options_141104_treat3.pdf}
\end{center}
}

\frame{
\frametitle{The Timing of Property Tax Enforcement in Philadelphia}
\begin{itemize}
\item In Philadelphia, each year's property tax payments are mailed to
property owners by mid-January and are due in full by March 31st of
that year.
\item Beginning in May of the tax year, the DoR sends a common
reminder letter to each late taxpayer, usually once every two months
until payment is received.  
\item The common reminder letter states the
taxpayer's liability and accrued interest and penalties.
\item If payment
has not been received by September of the tax year, the taxpayer is
randomly assigned with a 1/3 probability to either of two law firms
for collection or to the DoR for continued efforts at collection. 
\end{itemize}
}


\frame{
\frametitle{The Experimental Design}
\begin{itemize}
\item The approach to randomization was constrained by the logistics of
DoR's enforcement faculties. 
\item We concluded after several discussions with our collaborators at
  DoR that it would be logistically impossible to assign properties at
  random to different treatments.
\item Instead, we chose to exploit the pseudo-random assignment of
  properties to billing cycles and randomized treatments across these
  cycles.
\item We evaluated the fidelity of the experiment by analyzing returned
mail. We concluded that fidelity had been compromised on 6 of the 15 
treatment days.
\end{itemize}
}

\frame{
\frametitle{Descriptive Statistics of Our Sample}

\begin{center}
\begin{tabular}{|c|c|c|}
  \hline
 & Full Sample &  Analysis Sample \\ 
  \hline
 Number Observations & 134887 & 4927 \\ 
Amount Due & 4409 & 3465 \\ 
  Assessed Property Value & 138867 & 186691 \\ 
  Value of Tax & 1586 & 2405 \\ 
  Length of Debt & 6 & 4 \\ 
  \% Residential & 80 & 80 \\ 
  \% w/ Philadelphia Mailing Address & 88  & 83 \\ 
  \% Owner-Occupied & 24 & 22 \\ 
   \hline
\end{tabular}
\end{center}
Differences between the full sample and our analysis sample are largely due to the fact that 2/3 of tax payers are
assigned to collection agencies in September of each year.
}

\frame{

\begin{table}
\caption{Tests of Sample Balance on Observables} \label{table:balance} 
\begin{center}
{\footnotesize
\begin{tabular}{| l | c |  c| c| c| c|}
 \hline 
Variable & Control & Threat & Pub Service & Civic Duty & $p$-value \\ 
\hline 
  Taxes Due Quartiles & & & & & \\ 
$<$ \$300 & 0.22 & 0.10 & 0.40 & 0.28 & 0.00 \\ 
  \lbrack\$300, \$1300) & 0.24 & 0.08 & 0.46 & 0.22 &  \\ 
  \lbrack\$1300, \$3300) & 0.23 & 0.11 & 0.45 & 0.20 &  \\ 
  $>$ \$3300 & 0.18 & 0.11 & 0.48 & 0.23 &  \\ 
   \hline 
Market Value Quartiles & & & & & \\ 
$<$ \$46k & 0.24 & 0.12 & 0.43 & 0.21 & 0.33 \\ 
  \lbrack\$46k, \$82k) & 0.22 & 0.09 & 0.46 & 0.23 &  \\ 
  \lbrack\$82k, \$151k) & 0.21 & 0.09 & 0.45 & 0.25 &  \\ 
  $>$ \$151k & 0.21 & 0.10 & 0.45 & 0.24 &  \\ 
   \hline 
Land Area Quartiles & & & & & \\ 
$<$ 800 sq. ft & 0.22 & 0.10 & 0.45 & 0.23 & 0.92 \\ 
  \lbrack800, 1200) sq. ft & 0.23 & 0.10 & 0.43 & 0.24 &  \\ 
  \lbrack1200, 1800) sq. ft & 0.21 & 0.10 & 0.47 & 0.22 &  \\ 
  $>$ 1800 sq. ft & 0.21 & 0.10 & 0.44 & 0.24 &  \\ 
   \hline 
Empirical Distribution  & 0.22 & 0.10 & 0.45 & 0.23 & 0.62 \\ 
   \hline
Expected Distribution & 0.22 & 0.11 & 0.44 & 0.22 &  \\ 
   \hline
\end{tabular}
}
\end{center}
\end{table}
}


\frame{
\begin{table}[htbp]
\caption{Estimated Average Treatment Effects: Revenues}\label{dif_mean}
\begin{center}
{\footnotesize
\begin{tabular}{|l|c|c|c|}
\hline
               & Main Sample & Non-Commercial Sample & Unique Owner Sample \\
\hline
Threat         & $10.78$   & $23.00$   & $40.55$       \\
               & $(42.34)$ & $(41.65)$ & $(31.37)$     \\
Public Service & $76.04$   & $14.03$   & $152.38^{**}$ \\
               & $(55.53)$ & $(31.27)$ & $(66.57)$     \\
Civic Duty     & $28.95$   & $22.96$   & $82.31^{**}$  \\
               & $(38.15)$ & $(38.62)$ & $(40.58)$     \\
\hline
\multicolumn{4}{l}{\scriptsize{$^{***}p<0.01$, $^{**}p<0.05$, $^*p<0.1$}}
\end{tabular}
}
\end{center}
\end{table}
}

\frame{
\frametitle{Treatment Effects: Compliance}
\begin{itemize}
\item To gain more insights into the causal impact of the different treatment letters on taxpayer compliance we turn the statistical analysis of discrete outcomes.
\item The compliance outcome 
is one if the tardy taxpayer makes any payment at all  and zero if not, i.e. 
$y = 1$, if the tax payer made a positive payment, and $y = 0$, if not.
\item The outcome is of interest because even small 
additional payments help, but perhaps more importantly, a tax 
contribution represents a willingness by the taxpayer to be engaged 
with city governance. 
\item We estimate logistic regressions.
\end{itemize}
}

\frame{
\begin{table}[htbp]
\caption{Logistic Regressions for Ever Paid: Compliance}\label{table:ep_log_I}
\begin{center}
\begin{tabular}{| l | c | c | c |}
\hline
               & Full Sample & Non-Commercial & Sole Owner \\
\hline
Threat         & $-0.07$  & $-0.04$  & $-0.03$     \\
               & $(0.17)$ & $(0.17)$ & $(0.17)$    \\
Public Service & $-0.06$  & $-0.08$  & $0.05$      \\
               & $(0.13)$ & $(0.13)$ & $(0.11)$    \\
Civic Duty     & $0.22$   & $0.20$   & $0.30^{**}$ \\
               & $(0.13)$ & $(0.14)$ & $(0.12)$    \\
\hline
Log Likelihood & -2127.64 & -2054.97 & -1749.98    \\
Num. obs.      & 4927     & 4749     & 3888        \\
\hline
\multicolumn{4}{l}{\scriptsize{$^{***}p<0.01$, $^{**}p<0.05$, $^*p<0.1$}} \\
\multicolumn{4}{l}{NOTE: This table reports the parameter estimates from the} \\
\multicolumn{4}{l}{basic Logit Model that uses ``ever paid" as outcome.}
\end{tabular}
\end{center}
\end{table}
}

\frame{
\frametitle{Heterogeneity of Treatment}
\begin{itemize}
\item We investigate whether there is heterogeneity in response to the
treatment. It is plausible that very tardy taxpayers who owe small
amounts of money behave differently than those who owe larger amounts.
\item We find some statistical evidence that the appeal to
  civic duty is an effective strategy for encouraging at least some
  tax payment and often payment in full.
\item Its most significant impact are on residents with relatively low levels of
tax debt (\$0 to \$300).
\item  Stressing the benefits of payment for the provision of city services may also have improved tax compliance.
\item Its most significant impact appears to be on taxpayers with the
highest levels of outstanding tax debts (owing more than \$3600).  
\end{itemize}
}

\frame{
\frametitle{Conclusions}
\begin{itemize}
\item We have shown that different notification strategies have the
  potential to improve Philadelphia tax compliance and collections
  among tardy city taxpayers.
\item The findings imply that it  may be beneficial to  target a message
  to specific group in the population.
\end{itemize}
}
\end{document}

