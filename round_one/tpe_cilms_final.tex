\documentclass[12pt,titlepage]{article}

\renewcommand\baselinestretch{1.5}
\setlength{\parskip}{0.08in}
\setlength{\medskipamount}{0.05in}
\textheight 8.1 in
\textwidth 6.0 in
\topmargin 0.25in
\tolerance=11000

\usepackage[utf8]{inputenc}
\usepackage{longtable}
\usepackage{bbm}
\usepackage{rotating}
\usepackage[pdfencoding=auto,unicode=true]{hyperref}
\usepackage{graphicx}
\usepackage[outdir=./]{epstopdf}
\graphicspath{{images/analysis/act/}{images/}{images/balance/}}
\renewcommand{\thefootnote}{\fnsymbol{footnote}}

\begin{document}

\title{An Experimental Evaluation of Notification Strategies to
  Increase Property Tax Compliance: \\ Free-Riding in the City of
  Brotherly Love} \author{Michael Chirico, Robert Inman, Charles
  Loeffler, \\ John MacDonald, and Holger Sieg\thanks{We would like to
    thank Rob Dubow (Director of Finance), Clarena Tolson (Revenue
    Commissioner), and Marisa Waxman (Deputy Commissioner for
    Assessment of Properties) in the Department of Revenue of the City
    of Philadelphia for their help and support. We would also like to
    thank Jeff Brown, Chris Sanchirico, Wolfgang Sch\"on, Reed
    Shuldiner and participants of numerous seminars for comments and
    suggestions. The views expressed here are those of the authors and
    do not necessarily represent or reflect the views of the City of
    Philadelphia.}  \\ University of Pennsylvania \\ \\ Prepared for
  the 2015 NBER Conference on Tax Policy and the Economy}
  
  \date{\today}

\maketitle

\begin{abstract}

This study evaluates a set of notification strategies intended to
increase property tax collection. To test these strategies, we develop
a field experiment in collaboration with the Philadelphia Department
of Revenue.  The resulting notification strategies draw on core
rationales for tax compliance: deterrence, the need to finance the
provision of public goods and services, as well as the appeal to civic
duty. Our empirical findings provide evidence that carefully designed
and targeted notification strategies can modestly improve tax
compliance.

\noindent KEYWORDS: Tax Compliance, Property Taxation, Field
Experiment, Deterrence, Public Service Appeal, Appeal to Civic Duty.

\end{abstract}

\newpage

\renewcommand{\thefootnote}{\arabic{footnote}}

\renewcommand{\thefootnote}{\arabic{footnote}}

\section{Introduction}

The lack of tax compliance has become a policy issue of central
importance to all levels of government in developed and developing
economies. In 2009, the developed economies of the OECD reported a tax
non-compliance rate of 14.2, ranging from a low of 2-3 percent in
Austria, Denmark, Germany, Korea, and Norway to 25 percent or more in
Belgium, Iceland, Portugal, the Slovak Republic, to a high of 73
percent in Greece (OECD, 2011, Table 36). In developing economies with
significant cash economies, tax non-compliance is likely much higher. The
OECD estimates an average rate of tax non-compliance in non-OECD
countries of 37 percent (OECD, 2011, Chart 7).

Noncompliance is a significant concern for at least four reasons.
First, governments are denied the revenues needed to provide basic
public services essential for ensuring the safety, health, and minimal
well-being of all citizens.  Second, if there is significant
non-compliance and basic services are to be provided, then tax rates
will need to rise on those who pay taxes. Rising tax rates for honest
payers will discourage their use and desire of public services,
potentially encouraging their exit from the formal economy. The
negative consequences for overall economic performance can be sizable;
Greece today serves as an unfortunate reminder.  Third, non-compliance
undermines the principle that everyone has to pay their ``fair share"
of taxes.  The evidence suggests upper income taxpayers are more
likely to be non-compliers.  Finally, significant non-compliance may
threaten the stability of democratic governance.  When democratic
governments fail to deliver essential services, impose large tax
burdens on the legitimate private economy, and are viewed as
capricious or actively unfair, then dictatorial alternatives may
become attractive. In an important sense, tax compliance is a first
order of business for efficient, fair, and democratic governance.

Tax compliance requires government to manage the taxpayer's decision
to pay taxes.  Taxpayers may ask: What do I owe and what happens if I
don't pay?  Taxpayers have the ability to influence what is owed on
any tax that requires self-reporting of income or assets, such as
self-reported consulting or business income.  Taxing jurisdictions can
in principle increase compliance by requiring less self reporting and
directly assessing the tax base.  Since property cannot be hidden,
scuppered away to a tax haven, or concealed in an electronic data
system, self- or non-reporting is less of an issue in a property tax
system.  Privately assessed wages, dividends, and interest income by
individuals and businesses are easier to conceal to the tax
authorities.  However, private assessors have an incentive to report
incomes truthfully as those payments are typically deductible expenses
for their own taxes.  The need for self-reporting is also reduced as
the formal economy and the use of audited business records expands.
Taxes which can take advantage of those records maximize tax
compliance and are preferred for just this reason.  The increased
popularity of the value-added tax (VAT) over the past twenty years in
economies with developing formal consumption sectors is a case on
point (see Keen and Lockwood, 2010; Pomeranz, 2015).  Self-reporting
matters for tax compliance in developed economies as well.  Kleven
et. al. (2011), for example, show that tax non-compliance among Danish
taxpayers is significantly higher for individuals with self-reported
income.

The taxing jurisdiction can also control compliance by influencing the
decision to evade, once the tax liability has been assessed.  The most
common strategy is the economic stick - fines and penalties.  Failure
to pay in time leads to interest penalties sufficiently large that
there is no arbitrage advantage to waiting, and perhaps to a
significant late fine as well.  For long-time non-payers, fines may
include the garnishment of wages, seizure of property, or jail.  Early
empirical studies found little impact of such penalties on aggregate
tax compliance, however; see Slemrod (2007).  But more recent, nuanced
studies, have found an impact of fines on both the level and speed of
tax payments.  Fellner, et. al. (2013) find that a reminder letter for
payment of the Austrian TV license fee that explicitly threatens legal
action if the resident does not provide the required information for
assessment performed significantly better than the standard reminder
letter informing residents that they had not yet returned the required
forms.  Wenzel and Taylor (2004) find that including a letter
reminding taxpayers that their statement of rental income can be
audited and that faulty reporting may lead to fines significantly
reduced deductions when compared to forms submitted by taxpayers who
did not receive the threatening letter.  Hallsworth, et. al. (2014)
find the speed with which taxpayers pay their liabilities can also be
improved with increased fines.

But fines only work if taxpayers believe they will be enforced.  Large
fines may be seen by taxpayers as a signal of a desperate and
ineffective tax collector, as politically inviable and thus as empty
threats, or in the extreme, as a breakdown of cooperative democratic
governance.  If so, an increase in fines may even reduce tax
compliance, as indeed happened in Israel with the payment of corporate
taxes; see Ariel (2012).  On balance, the estimated effects of fines
on tax payments have been positive, but modest in magnitude.

Given evidence as to the limited ability of economic sanctions to
impact aggregate collections, attention has turned to other behavioral
motivations for increasing tax compliance.  Such motives are grounded
in the value taxpayers place upon their role and position within the
democratic community.  The role may be instrumental leading to
outcomes valued by the taxpayer, such as additional public services,
or of value in its own right.  Both may provide an incentive for tax
compliance.

Instrumentally, for example, if each taxpayer thought of himself as
simply a single citizen within the democracy, there would be no
incentive, apart from a fine, to pay for services.  If fines are too
low or unenforced, then free-riding is the preferred private strategy.
If all taxpayers think of themselves as private citizens only, then
all might free ride and no services would be provided.  Alternatively,
citizens might think of themselves as representative of a community of
like-minded residents, where cooperation by any one citizen is
reflective of how all citizens may think; see Feddersen and Sandroni
(2006).  If so, a cooperative, tax-paying outcome might occur.  As
examples, we vote, we tip in restaurants we will never visit again,
and we put our litter in waste cans.

Or the citizen may view their role as a cooperating member of the
community, not in an instrumental or strategic way, but rather as
having value in its own right, quite apart from any impact playing
such a role may have on valued social outcomes.  Individuals may
derive satisfaction from knowing, or from having others know, that
they have done their ``civic duty."  Duty can extend far beyond tax
compliance to all forms of law abiding behavior; see Posner (2000).
Consistent with theories of social norms, the more people conform to
law-abiding behavior, the more likely it may be that the ``marginal"
citizen will conform as well; see Benabou and Tirole (2011).

Both the instrumental motive and the motive born from civic duty have
been used to stimulate tax compliance.  The evidence is mixed.  The
most careful study of the two motives was done Blumenthal et
al. (2001), where two different letters were sent to Minnesota state
taxpayers reminding the taxpayer when taxes were due and to report
their income accurately.  Hoping to elicit cooperative behavior from
an instrumental perspective, one letter stressed that taxes pay for
important state services.  Hoping to tap a personal sense of civic
duty, the second letter emphasized that most state taxpayers correctly
report their taxable income on time.  There were 15,000 taxpayers in
each group, and their reported taxable incomes were compared to a
control group of 15,000 taxpayers who received no letter.  We should
expect the largest effect on self-reported incomes; see Kleven,
et. al. (2011) and Pomeranz (2015).  For both letters, there were
statistically significant positive and negative effects on the various
categories of self-reported incomes, with no statistically significant
change in aggregate taxable income over that reported by the control
group.  The one strong effect was a relatively large negative effect
on reported income by the richest taxpayers from having received the
civic duty letter.

Three more recent studies have been more encouraging as to the impact
of behavioral appeals. In an effort to improve the speed of tax
compliance for British income taxpayers, Hallsworth et al. (2014) sent
either of two letters to taxpayers both encouraging them to pay their
taxes on time.  Again, appealing to the strategic advantages of
cooperative behavior in the decision to pay taxes, one letter stressed
that payment ensures important national services will be provided.  A
second letter appealed to a citizen's personal sense of community and
stressed that ``nine out of ten" taxpayers pay their taxes on time.
Both sets of letters had a statistically significant effect in
encouraging sooner tax payments, and the effects were greatest for the
appeal to ``civic duty" when the letter explicitly mentioned the 
taxpayer's most likely reference group of fellow citizens.

Perez-Truglia and Troiano (2015) explored the impact of what they call
a ``shaming penalty" administered through a letter to a subset of
delinquent state taxpayers reminding them that the state has placed
their name on a publicly available list of tax delinquents and that
only payment in full or acceptance of a payment schedule can remove
their name from the delinquent list.  The reminder letter made a
significant positive difference to eventual tax compliance, with the
greatest effects observed for taxpayers with the lowest level of taxes
owed.  In addition, reminding tax delinquents that there is a growing
financial penalty to late payments also had a positive impact on
compliance and particularly so for wage-only taxpayers whose income
could be most easily attached for payment and penalties.

Finally, Besley, Jensen, and Persson (2014) estimate a dynamic model
of tax compliance to explore whether more complying taxpayers
encourages further compliance as implied by social norm behaviors.
The theory is tested for British local government tax compliance
following the tax revolt of 1990 in response to the replacement of the
wealth-based property tax by a regressive poll or ``head" tax.  Local
compliance fell from an average rate of 97 percent to 82 percent
within two years.  The poll tax was removed and wealth taxation
restored in 1993 but it took more than ten years for tax compliance
for the wealth tax to only gradually return to its original levels,
suggesting a dynamic process for learning of the cooperative tax
behavior of other citizens.

Our agenda here is to extend our understanding of tax compliance to
include the payment of local property taxation.  We do so by
implementing a pilot tax compliance experiment in one large US city:
Philadelphia.  The City's historical performance in tax compliance has
not been good.  Compared to a national average tax collection rate of
95 percent, Philadelphia has collected only 88 percent of assessed
property tax revenues on average over a recent ten year period.  The
City's performance has improved in recent years.  Hoping to improve
performance still further, the City asked for our assistance in an
evaluation of possible alternative formats for their letter reminding
tardy taxpayers that their payments are due. Each year, property a 
property tax bill is mailed to property owners by mid-January and 
payment is due in full by March 31st of that year.  If payment has 
not been received by the end of April, a reminder letters are mailed 
beginning in May, usually once every two months until payment is 
received.  The current reminder letter states the taxpayer’s liability 
and accrued interest and penalties.  If payment has not been received 
by September of the tax year, then two out of three the tardy taxpayers 
are assigned to either of two law firms for collection; the remaining 
third of the tardy taxpayers remain with DoR for continued efforts at 
internal collection.  We assisted DoR with collection from their 
share of these tardy taxpayers. 

We proposed three additional
formats for DoR's reminder letter.  In addition to the listing of tax
payments, interest, and penalties, the alternative letters contained a
sentence that either (i) threatened the potential loss of the
taxpayer's home or property if taxes were not paid, or (ii) appealed
to the positive community benefits in provided public services that
the taxpayer's dollars provide, or (iii), appealed to the positive
benefits of fulfilling your civic duty to yourself and others by
paying your taxes.

We find evidence that the letter that appealed to the benefits of
fulfilling your civic duty had a positive effect on tax compliance
above that of the City's standard reminder letter.  This appeal was an
effective strategy for encouraging at least some tax payment and often
payment in full, and had its biggest differential and most significant
impact on those residents with relatively low levels of tax debt.  We
also find evidence that stressing the benefits of payment for the
provision of city services may also improve tax compliance for the
taxpayer.  This effect is most significant for taxpayers that owe
larger amounts of taxes. The letter that threatened the possible loss
of the taxpayer's property did not significantly improve tax
collection.  As a group, the three letters encouraged an economically
significant improvement in aggregate revenues
collected over that available from the City's current reminder letter.
For the full sample of property owners, the average treatment effect
from receiving one of our three re-drafted reminder letters was to
improve collections above that from the City's current letter by
\$62 per letter.  For the sample of owners of single properties, the
average treatment effect from the receipt of one of our reminder
letters was \$117 per letter above that of the City's current letter.
Finally, the fact that our sample of ``most tardy" taxpayers responded
differentially to the three reminder letters strongly suggests that a
strategy that targets different reminders to specific cohorts can
improve collection performance.  A uniform reminder to all late or
non-compliant taxpayers is not likely to be revenue maximizing.

The rest of the paper is organized as follows.  Section 2 provides a
brief overview of tax compliance in U.S. cities. Section 3 provides
a detailed discussion of our three treatments and the control. It also
discusses the experimental design and the fidelity of its
implementation. Section 4 presents a descriptive analysis summarizing
the main effects of our experiment on city revenues.  Section 5 provides 
some additional analysis of discrete outcomes focusing on whether 
taxpayers made payments at all or paid the debt back in full.
Section 6 offers some conclusions and discusses future research.


\section{Property Tax Compliance in U.S. Cities: An Overview}


The property tax is one of the most important taxes for the financing
of local government services in the United States.  For the country as
a whole, approximately 21 percent of all state and local government
revenues were raised using the property tax in 2011 (Gruber, 2014).
For the largest cities that percentage is much higher.  The potential
economic advantages of the property tax are well known.\footnote{A
  well-administered property tax has two potential economic
  advantages, one relating to economic efficiency and the other to
  economic fairness.  First, if households and businesses are mobile
  across local political jurisdictions and if local jurisdictions use
  their zoning powers to ``sort" taxpayers by the value of their
  properties, then the property tax becomes the economic equivalent of
  a benefit tax relating taxes paid directly to the costs of the
  services provided (see Hamilton, 1975).  This will lead to the
  efficient provision of local government services.  The two
  efficiency assumptions are likely to hold in suburbs, but not in
  central cities.  In the case of the central city, efficiency will
  require the tax be close to a tax on existing structures and ideally
  land, rather than on new investment.  The tax will be least
  efficient in those cities with very elastic demand and supply for
  new construction.  In declining cities with no new construction, the
  supply curve is inelastic, at the level of existing structures.  In
  successful, growing cities demand for location is likely to be
  inelastic and new supply constrained by available land.  In these
  two cases, therefore, the property tax remains a relatively
  efficient local tax.  With regard to economic fairness, if the
  property tax is based on market value assessments, then the tax
  becomes a proportional tax on property wealth (see Aaron, 1975;
  Mieszkowski, 1972).  Since property wealth increases at least in
  proportion to increases in income, the tax will be proportional or
  perhaps progressive.}  But so too does the tax have significant
administrative advantages.  With modern techniques for assessment,
properties can be accurately assessed at their market values, and
assessments can be easily updated at the time of each ``arms-length"
transaction.  Thus, there is no need for taxpayer reporting of the tax
base, in contrast to income, profits, sales, or VAT taxation.  Property
values, based as they are on long-run economic returns, are usually
less volatile than tax bases dependent on current economic activity,
such as income or sales. Stable tax bases allow for stable revenue
flows and thus less volatile service flows or, alternatively, tax
rates.\footnote{Any remaining volatility in revenues can be managed
  with rainy day funds.} Finally, when the tax base is determined by
market-based assessments, the taxpayer's tax bases will have been
objectively set and easily understood.  There is no need for
complicated tax forms or contentious and subjective appeals.  This too saves on
administrative costs, and, one hopes, increases citizen confidence in
the fairness of their tax payments.

\begin{center}
\begin{longtable}{| l | c |  c|}
\caption{Property Tax Compliance: 2005-2014} \label{table:comp} \\
\hline 
\multicolumn{1}{|c|}{\textbf{City}} & \multicolumn{1}{c|}{\textbf{Proportion Compliance}} & \multicolumn{1}{c|}{\textbf{Delinquent Tax Collected }} \\ 
\multicolumn{1}{|c|}{\textbf{}} & \multicolumn{1}{c|}{\textbf{Current Yr; 10-Year Average}} & \multicolumn{1}{c|}{\textbf{Five-Year Yearly Average}} \\ 
\hline 
\endfirsthead
\multicolumn{3}{c}%
{{\bfseries \tablename\ \thetable{} -- continued from previous page}} \\
\hline \multicolumn{1}{|c|}{\textbf{City}} &
\multicolumn{1}{c|}{\textbf{Proportion Compliance}} &
\multicolumn{1}{c|}{\textbf{Delinquent Tax Collected}} \\ 
\multicolumn{1}{|c|}{\textbf{}} & \multicolumn{1}{c|}{\textbf{Current Yr; 10-Year Average}} & \multicolumn{1}{c|}{\textbf{Five-Year Yearly Average}} \\ 
\hline 
\endhead
\hline \multicolumn{3}{|l|}{{Continued on next page}} \\ \hline
\endfoot
\hline 
\multicolumn{3}{l}{* City Poverty Rate was greater than or equal to 20 percent in 2009-2013.}  \\
\multicolumn{3}{l}{{\it Annals of Statistics}: Each city's Comprehensive Annual Financial Report,} \\
\multicolumn{3}{l}{annually over the years 2005 to 2014.} \\
\multicolumn{3}{l}{ \textit{Proportion Compliance}: Computed as the proportion of property taxes levied in} \\
\multicolumn{3}{l}{each fiscal year that are actually paid during the fiscal (or collection) year.} \\ 
\multicolumn{3}{l}{\textit{Delinquent Taxes Collected}: Delinquent taxes not paid in the year due may} \\
\multicolumn{3}{l}{be paid in subsequent years.  The annual rate is computed as the average collection} \\
\multicolumn{3}{l}{ rate over a five year period following the year after the tax is first due. The aggregate} \\
\multicolumn{3}{l}{ percent of the delinquent taxes paid after five years, the typical horizon over which} \\
\multicolumn{3}{l}{no further payments can be expected, can be computed as 5 x [yearly average].  } \\
\multicolumn{3}{l}{The (-) indicates that data were not available to compute the rate of delinquent tax} \\
\multicolumn{3}{l}{collection for that city.}  \\
\endlastfoot
Large City Average & .946; .945	& .112 \\
Atlanta*	  & .982 ;  .960 & .182 \\
Baltimore*	 & .960 ;  .950	 & .128 \\
Birmingham*	 & .983;  .955	& - \\
Boston	         & .996;  .992	 & - \\
Buffalo*	 & .947;  .945	 & .175 \\
Charlotte	 & .984;  .980	 & - \\
Chicago*	 & .962;  .930	 & - \\
Cincinnati*	 & .940;  .925	 & .120 \\
Cleveland*	 & .841;  .850	 & .090 \\
Columbus*	 & .938;  .920	 & .075 \\
Dallas*	         & .989;  .985	 & .085 \\
Washington, DC	 & .985;  .980	 & - \\
Denver	         & .990;  .989	 & - \\
Detroit*	 & .683;  .891	 & - \\
Flint, MI*	         & .654;  .785	 & .151 \\
Houston*	 & .983;  .975	 & .171 \\
Kansas City	 & .943; 938	 & - \\
Los Angeles	 & .992;  .940	 & - \\
Memphis*	 & .984;  .945	 & .085 \\
Miami*	         & .975;  .970	 & .045 \\
Milwaukee*	 & .865;  .875	 & .191 \\
Minneapolis*	 & .985;  .972	 & .102 \\
Nashville	 & .984;  .986	 & - \\
New Orleans*	 & .948;  .921	 & .172 \\
New York City	 & .915;  .925	 & .041 \\
Oklahoma City	 & .958;  .949	 & .161 \\
Orlando	         & .991;  .988	 & .072 \\
Philadelphia*	 & .940;  .880	 & .125 \\
Phoenix*	 & .977;  .965	 & .130 \\
Pittsburgh*	 & .849;  .860	 & .048 \\
Portland	 & .942;  .934	 & .109 \\
Richmond*	 & .924;  .955	 & .171 \\
Riverside	 & .990;  .982	 & - \\
Sacramento	 & .996;  .980	 & - \\
Salt Lake City	 & .985;  .980	 & .140 \\
San Antonio	 & .989;  .985	 & .134 \\
San Diego	 & .980;  .950	 & - \\
San Francisco	 & .988;  .980	 & - \\
San Jose	 & .999;  .990	 & - \\
Seattle	         & .985;  .983	 & .170 \\
St.  Louis*	 & .921;  .890	 & .123 \\
Tampa	         & .959;  .957	 & .032 \\
\end{longtable}
\end{center}

Once market-based assessments are in place, the administrative issue
that remains is this: Will property owners pay their taxes?  Table
\ref{table:comp} summarizes the record for property tax compliance for
forty of the largest U.S. cities, plus Flint, Michigan, a poster child
for weak compliance.  Tax compliance is defined as the percent of
taxes levied in the collection year that are paid in the year due.
Taxes not paid in the collection year are then considered
delinquent.\footnote{A city's collection year need not correspond to
  the city's fiscal year.  For example, in Philadelphia the collection
  year is the calendar year while the fiscal year runs from July 1 to
  June 30 of the next year.  Tax bills are mailed in January of each
  collection year\textemdash the middle of the fiscal year\textemdash and are due on
  March 31 of that year.  Payments received after March 31 are
  considered late payments and entail accrued interest and late payment
  penalties.  All payments received by December 31 of the collection
  year are then recorded as taxes paid during the collection year.
  Taxes that are not paid by December 31 are then classified as
  delinquent for that collection year.  Since property tax payments
  arrive in the last half of each fiscal year, Philadelphia will use
  some its tax receipts to repay the short-term ``cash-flow" loans of
  that fiscal year and then save a significant fraction of the
  remaining revenues for spending in the first half of the next fiscal
  year.}

Property tax compliance in these large cities over the past ten years,
years that included the deepest recession in decades, has been very
good.  On average, these large cities collected nearly 95 percent of
their property taxes in the tax year due, and overall the recession years did
not lower collection rates substantially.  Still, the average
amount of uncollected, delinquent revenues is noteworthy, and
particularly so for the seven poorest performing cities: Cleveland
(.85), Detroit (.89), Flint, MI (.79), Milwaukee (.88), Philadelphia
(.88), Pittsburgh (.86), and St. Louis (.89).

Taxes that have not been paid in the tax year become delinquent
balances, and cities seek to collect those taxes through various
enforcement mechanisms.  The most common strategy is to send a
reminder letter to the taxpayer stressing that unpaid taxes accumulate
interest and penalties and need to be paid.  If still unpaid, the tax
bill can be given to a private collection agent with revenues shared
between the agent and city or perhaps sold to the agent for immediate
revenues.  Alternatively the wages of, or payments to, the tax delinquent can be
garnished.  Philadelphia does so for public employees and for private
contractors working for the city.  Finally, a tax lien can be imposed
on the property to be paid when the home is sold.  As a last resort,
the city can seize the property and require a Sheriff's Sale to
collect back taxes.  The end result is the collection of some portion
of delinquent taxes.  Table \label{table:comp} reports each city's five-year yearly
average for the collection of delinquent taxes.\footnote{The average
  annual collection rate for delinquent taxes was estimated from data
  provided by the sample cities in each city's Comprehensive Annual
  Financial Report.  The required data was reported either as the
  amount finally collected from a given year's delinquent taxes\textemdash
  reported as ``Collections in Subsequent Years" - or as all
  delinquent taxes collected in a year from all previous years\textemdash
  reported as ``Delinquent Tax Collections."  For cities reporting
  ``Collections in Subsequent Years" the average annual rate was
  computed as ratio [Collections/(Tax Year Taxes Levied - Tax Year
    Taxes Collected)] then divided by 5.  The assumption is that all
  taxes levied but not collected in the tax year are classified as
  delinquent and that no significant amount of delinquent taxes are
  collected after five years.  For cities reporting ``Delinquent Tax
  Collections" the average annual rate was computed as ratio
  [Collections/ $\sum$ (Tax Year Taxes Levied - Tax Year Taxes
    Collected)], summed over the previous five tax years.  In both
  cases the average annual rate is an average of the actual
  collections in each of the five years following tax delinquency,
  where typically the first year rate of collection is the highest
  with a declining rate in years two to five.  Included in
  ``Collections" in both cases will be taxes plus interest plus
  penalties collected, the proceeds from the sale of tax liens to
  private collection agents, and the proceeds from the sheriff's sales
  of delinquent properties.} The typical pattern of collection for
delinquent taxes shows a relatively high success rate in the first
year of delinquency followed by a very sharp decline in payments
thereafter.\footnote{Atlanta is one of the better-performing cities in
  its collection of delinquent taxes and the pattern of its collection
  success is typical.  We estimate that in the first year of
  delinquency for its 2005 tax collection year, the city collected 56
  percent of delinquent taxes owed. That was in 2006.  In 2007, the
  second year of delinquency for 2005 taxes, an additional 8 percent
  was collected.  In 2008, an additional 1 percent was collected.  In
  2009, an additional 7 percent was collected.  And in the 2010, an
  additional 12 percent was collected.  After five years, the final
  amount collected of the 2005 delinquent tax owed was 84 percent.
  The five year annual average for 2005 was therefore .168.  In
  subsequent years, Atlanta has done a bit better.  Its annual average
  collection rate has been .182 for an aggregate average collection
  rate of delinquent taxes of .91.} Most cities view tax bills that
have been delinquent for more than five years as uncollectible.
Multiplying the five-year average rate reported in Table 1 by five
yields the average aggregate collection rate of any one year's
delinquent taxes.  For the average city in our sample, this aggregate
collection rate is .560 ( = .112 x 5).  The better-performing cities,
such as Atlanta, may eventually collect more than 90 percent of their
delinquent taxes; the poorer-performing cities, perhaps not much more
than 30 percent.

Table 1 also indicates with an asterisk those cities with poverty rates greater than
.20 for the period 2009-2013.  The expectation is that high poverty
cities should have lower rates of initial tax compliance and possibly
more difficulty in collecting delinquent taxes.  A comparison of the
(unweighted) mean rates of tax compliance shows this to be the case for initial
collection efforts: .92 for the 22 high poverty rate cities (.94
excluding Detroit and Flint) and .98 for the 20 cities with relatively
low poverty rates.  The average annual ability to collect delinquent
taxes in the two sets of cities is about the same (= .11), however,
perhaps because the pool of delinquent taxpayers is very poor in all
cities.  Importantly, however, some cities with high poverty rates are
very successful in collecting property taxes on time and in collecting
delinquent taxes.  Among the poorer cities, Atlanta, Baltimore,
Houston, New Orleans, and Phoenix perform as well, and often better,
than the average low-poverty city.  The fact that property tax
compliance can be well managed in the face of difficult economic
realities emphasizes the value of looking at the administrative
strategies of successful cities and of searching for new strategies as
well.  It is the latter agenda we pursue here, using taxpayer
compliance in Philadelphia as a laboratory to experimentally evaluate
three alternative collection strategies to encourage payment by very tardy,
soon-to-be-delinquent city taxpayers.

\section{The Philadelphia Tax Experiment}

\subsection{Treatments}

In Philadelphia, each year's property tax payments are mailed to
property owners by mid-January and are due in full by March 31st of
that year.  Beginning in May of the tax year, DoR sends a common
reminder letter to each late taxpayer, usually once every two months
until payment is received.  The common reminder letter is impersonal
and simply states the taxpayer's liability and accrued interest and
penalties; see Appendix Figure 1.  The only means for responding to the letter
is to either send a check with the detached portion of the letter to
DoR or to call a phone number given (without instructions) at the head of the letter.
If payment has not been received by September
of the tax year, the taxpayer is assigned
to either of two law firms for collection or to DoR
for continued efforts at collection.  The law firms are free to pursue
the collection of the debt as they see fit.  Proceeds from their
efforts are shared with the City.  In the past, DoR's efforts at
collection from these very tardy taxpayers have been largely limited to simply
re-mailing the usual reminder letter.

In collaboration with the staff of DoR, we proposed two changes to
their usual tax collection efforts.  First, a generic reminder letter,
which included a Spanish translation of the letter on the reverse,
was included with the traditional tax
bill.\footnote{The Spanish translation was targeted at the substantial
  Latino population and is available upon request. Phone and e-mail contacts were
  also included.}  This revised letter serves as our ``control"
treatment.  Second, we offered three alternative letters to the
control letter which might encourage additional tax compliance: one
that threatened the potential loss of the taxpayer's home or property
if taxes were not paid, a second that appealed to the positive
aggregate benefits of public services provided by a cooperating
taxpayer's remittances, and a third letter that appealed to the
positive benefits the taxpayer alone may feel from fulfilling their
civic duty to themselves and to others by paying their
taxes. Specifically, the letters included the following phrasing:

{\it Treatment Letter 1: Threat: } {\bf Not paying your Real Estate
  Taxes is breaking the law.} Failure to pay your Real Estate Taxes
may result in seizure or sale of your property by the City. Do not
make the mistake of assuming we are too busy to pursue your case.

{\it Treatment Letter 2: Public Service Appeal: } {\bf We understand that
  paying your taxes can feel like a burden.} We want to remind you of
all the great services that you pay for with your Real Estate Tax
dollars. Your tax dollars pay for schools to teach our children.  They
also pay for the police and firefighters who help keep our city safe.
Please pay your taxes as soon as you can to help us pay for these
essential services.
  
{ \it Treatment Letter 3: Civic Duty Appeal: } {\bf You have not paid your
  Real Estate Taxes.}  Almost all of your neighbors pay their fair
share-- 9 out of 10 Philadelphians do so. Paying your taxes is your
duty to the city you live in. By failing to pay, you are abusing the
good will of your Philadelphia neighbors.

The formats of the three letters were constructed to differ only in
the wording of their middle paragraph; see Appendix Figures 2-5 for replicas.
Care was taken in ensuring that each letter was intelligible to those with a
5th grade education to minimize issues of communication for those with limited English
literacy.  Like the revised control letter, all treatment
letters also included a Spanish translation on the reverse as well phone
and e-mail contacts.  Letters were mailed in the November mailing cycle to the
taxpayers who had by then not yet paid, those we and the City consider to be
the ``most tardy" of the tardy taxpayers.  The receipt of tax payments,
or lack thereof, were recorded for 36 days, beginning five days after the
mailing date.


\subsection{Experimental Design}

To ensure that the results of the experiment allow for a causal
interpretation from the receipt of the letter to increased revenues and tax compliance,
great care was taken to establish a random assignment of all four
letters across the pool of DoR's ``most tardy'' taxpayers.
Unfortunately, DoR's administration for mailing the letters did not
allow for a purely random assignment of tardy taxpayers to each
letter.

Our approach to randomization was constrained by the logistics of
DoR's enforcement capabilities. We concluded after several discussions
with the staff of DoR that it would be difficult in practice
to assign individual properties at random to different
treatments. Instead, we chose to exploit the pseudo-random assignment
of properties by billing cycles and randomized treatments across them.
To understand assignment it is useful to discuss the current
practice of posting reminder letters by DoR.

Mailing of tardy real estate tax bills proceeded as follows.  Since it is
cheaper and simpler to send all bills at once to those owners owing
taxes on multiple properties, assignment to mailing cycles is done at the
owner level, so that each mailing cycle has roughly the same number of
owners.  Every morning the DoR accounting and records system
identifies all properties in the current day's mailing cycle that owe taxes to the City.
The mailing cycles progress in sequence,
day by day.  After identifying the bills to be printed for the day,
the DoR printer then merges the tax bill with the mailing address of
the owner and an in-house identifier associated with the property.
The printed bills for each day are then brought to the City's mailing
room, wherein they are mechanically inserted into envelopes and mailed
to property owners.
	
Given the volume of bills printed each day and the fact that the 
envelopes are stuffed largely by machine, the most practical means 
for randomization was to use DoR's mailing cycle as a randomization block.  As a result, 
every bill printed on the same day was randomly paired with one of the four 
reminder letters. Our experiment was conducted 
over 15 days from  November, 4, 2014 to November 25, 2014; given the limited timeframe
of our experiment, we also blocked on four-day periods. That is, for each 
four-day period, we randomized among the 4! = 24 possible 
arrangements of treatment letters.  

While we are certain of the sanctity of our mailing cycle-level
randomization process, one may be concerned about the assignment of
properties to mailing cycles by the City. Fortunately, however, the city uses a
pseudo-random mechanism to assign owners to billing cycles; in
particular, the city assigns properties to cycles based on the last
two digits of the property owner's in-house ID, which is one of a social 
security number, an Employer Identification Number, or (lacking these) a DoR-assigned 
nine-digit identification number. This assignment procedure means that we in principal achieve proper 
full-scale two-stage randomization of the properties through our process 
of day-level randomization. We believe that this quasi-random
assignment process should remove any significant sorting or self-selection bias in
the assignment of treatment letters.


\subsection{Implementation Fidelity}

To assess whether the final implementation of our mailing of treatment
letters is as intended, we leveraged a unique feature of the DoR's
mailing process.  The Department of Revenue regularly posts envelopes
destined for addresses that are either unattended (vacant) or due to typos do not
exist in the USPS database in the first place. Either before or after an
attempted delivery to such an address, the postal service identifies
these letters and returns them to DoR, which then processes the
letters and attempts, if they can identify a suitable alternative
address, to re-deliver the tax bill. We took advantage of the fact
that a subset of bills made their way back to DoR to check first-hand
the extent of treatment fidelity. Our final sample consists of the
nine treatment days for which greater than 90\% fidelity was
achieved\footnote{From letters that were returned to the 
mailroom because they could not be delivered, we discovered that six 
of our fifteen mailing cycle days involved possible errors to the 
insertion of our reminder letters with the tax bill.  As a result, 
the experiment's fidelity for those six treatment days was 
potentially compromised.  Since we could not be certain which 
reminder letters had been mailed on those mailing days, we removed 
all responses for those days from our sample.  This left a final 
sample of 4,927 letters mailed that could be identified with 
treatments.  Because the nine remaining days were not distributed 
evenly among treatments, there is a sample size imbalance across treatments.}.

\subsection{Sample Size}

From the original full sample of 134,888 ``most tardy'' properties, we
select a final sample of 4,927 properties for our experiment.  This
final sample removes all properties no longer handled by DoR (=
61,170), for which payment agreements have been reached (= 31,456),
were not part of our nine-day treatment period (= 24,800), which
qualified for a tax abatement (= 4706), or which, because of one or more
years of tax delinquency, qualified for a sheriff's sale (= 4098),
sequestration (= 1130), or bankruptcy ( = 948).  In addition,
there were properties for which DoR had no working address (= 1429) or
had an outstanding bill of less than \$.61 (= 224).\footnote{The city
  operates 50 billing cycles. Each cycles has approximately 2,500
  observations.  Once we apply the sample selection criteria discussed
  above we obtain between 493 and 633 observations per day.}


Table \ref{table:descriptives} provides descriptive statistics for the full sample 
of all tardy and delinquent properties as of November, 2014, prior to 
imposition of the aforementioned restrictions for not sheriff sale, 
sequestration, bankruptcy, etc.   Also shown 
are the descriptive statistics for properties assigned to third-party 
law firms, properties in the restricted sample, and finally for the 
4,927 properties that qualify for our experiment.


% latex table generated in R 3.2.2 by xtable 1.7-4 package
% Tue Oct  6 16:26:32 2015
\begin{table}[ht]
\centering
\caption{Descriptive Statistics} 
\label{table:descriptives}
\begin{tabular}{|r|r|r|r|r|}
  \hline
 & All Delinquent & Law Firm & Restricted & Analysis \\ 
  \hline
Average Amount Due & \$4,409 & \$4,608 & \$3,761 & \$3,465 \\ 
  Median Amount Due & \$1,695 & \$2,216 & \$1,285 & \$1,311 \\ 
  Average Assessed Property Value & \$138,867 & \$76,478 & \$242,604 & \$186,691 \\ 
  Median Assessed Property Value & \$66,700 & \$50,100 & \$82,900 & \$81,900 \\ 
  Tax Due & \$1,586 & \$925 & \$3,123 & \$2,405 \\ 
  Avg. Years of Debt & 6 & 8 & 4 & 4 \\ 
  Med. Years of Debt & 2 & 4 & 1 & 1 \\ 
  \% Residential & 80 & 74 & 81 & 80 \\ 
  \% with Philadelphia Mailing Address & 88 & 87 & 82 & 83 \\ 
  \% Owner-Occupied & 24 & 14 & 21 & 22 \\ 
  Number Observations & 134,887 & 70,031 & 29,951 & 4,927 \\ 
   \hline
\end{tabular}
NOTE: This table provides some descriptive statistics for all
properties in Philadelphia, all properties that satisfy our sampling
restrictions, and the sample used in the analysis.
\end{table}


Note in particular that this sample selection means that our sample
consists only of properties that are not in the purview of the two law
firms that DoR uses as collection agencies. It is therefore useful to
compare briefly the properties that are kept in-house with those that
are assigned to the law firms. We find that properties kept in-house
have less extreme but higher balances\textemdash while in-house and
law firm properties have similar average balances (\$4,195 vs. \$4,607),
the law firm properties have a much higher median balance (\$2216 vs.
\$1023 in-house). However, in-house properties have higher market values--the
DoR median is \$91,000 vs. \$50,100 in the law firm sample. Properties handled by DoR
have younger debt--an average (median) of 4 years vs. 8 (1 year vs. 4) for the law
firms.  Even conditional on age of debt, in-house balances are low.
DoR-managed accounts are more likely to be owner-occupied, less likely
to be in payment agreements, and more likely to result in a sheriff's
sales. In summary, it appears that the outside firms are holding
properties which, even given other characteristics, have the highest
potential returns.

\subsection{Sample Balance on Observables}

To confirm whether or not we indeed achieved randomization, we
performed a series of balance-on-observables tests. The null
hypotheses of these tests are that a given observable's sample means are
identical across mailing cycles. We turn now to the results of those
tests.

Analysis of balance on observables is complicated by the prominent
presence of concentrated ownership in our sample. Given that many mailing 
addresses are used by an owner with multiple land holdings, e.g. the University
of Pennsylvania, our analysis of sample balance for each
treatment letter may be skewed by correlated characteristics among such owners
with multiple properties.  Further, if there are multiple properties
associated with one mailing address, it is not obvious how best to
aggregate property-level characteristics for the owner's multiple
holdings.  For these reasons, the hypothesis test results reported in
Table \ref{table:balance} were computed with owner-level permutation
tests\footnote{In a permutation test, a statistic of interest is computed
repeatedly on pseudosamples constructed by shuffling (permuting) the treatment
assignments, in our case across owners, which picks up the nature of the
variability introduced by the clustering among owners. In our sample, lacking
a biometric identifier, we elected to denote as an owner a legal name-mailing
address pair.}.
  
Specifically, on each permutation, we calculate the Pearson's $\chi^2$ test 
statistic for testing independence of treatment and a series of categorical variables.
The full sample
consists of letters mailed over nine days, one day of which was sent the
Threat treatment letter, four of which were sent the Public Service
treatment letter, two of which were sent the Civic Duty treatment letter,
and two of which were mailed the control letter.  This meant that of
the 4,927 letters mailed, 11 percent (1/9) where Threat letters, 44
percent (4/9) were Public Service letters, 22 percent (2/9) were
Civic Duty letters, and 22 percent (2/9) were Control letters.  If our
treatment letters are randomly allocated across observable
characteristics of properties owing taxes, then we should observe the
same distribution of letters by each observable characteristic.  Table
\ref{table:balance} shows these distributions and the resulting $p$
values for the test of the null hypothesis that the individual letter
distribution by characteristic matches the characteristics in the
overall distribution of letters.

In most cases, we cannot reject the null hypothesis that the letters
have been randomly distributed by the observable characteristics shown
in Table \ref{table:balance}: market values, land areas, number
of rooms, years of that debt is owed, and property usage. Finally, the number of properties
assigned to each treatment is exactly as it should be based upon the
number of days used for the mailing of each treatment.


% latex table generated in R 3.2.2 by xtable 1.7-4 package
% Mon Oct  5 00:58:37 2015
\begin{table}[ht]
\centering
\caption{Tests of Sample Balance on Observables} 
\label{table:balance}
\begin{tabular}{|l|c|c|c|c|c|}
  \hline
Variable & Control & Threat & Public Service & Civic Duty & $p$-value \\ 
  \hline 
Taxes Due Quartiles & & & & & \\ 
$<$ \$300 & 0.22 & 0.10 & 0.40 & 0.28 & 0.00 \\ 
  \lbrack\$300, \$1300) & 0.24 & 0.08 & 0.46 & 0.22 &  \\ 
  \lbrack\$1300, \$3300) & 0.23 & 0.11 & 0.45 & 0.20 &  \\ 
  $>$ \$3300 & 0.18 & 0.11 & 0.48 & 0.23 &  \\ 
   \hline 
Market Value Quartiles & & & & & \\ 
$<$ \$46k & 0.24 & 0.12 & 0.43 & 0.21 & 0.56 \\ 
  \lbrack\$46k, \$82k) & 0.22 & 0.10 & 0.46 & 0.23 &  \\ 
  \lbrack\$82k, \$152k) & 0.21 & 0.09 & 0.45 & 0.25 &  \\ 
  $>$ \$152k & 0.21 & 0.10 & 0.45 & 0.24 &  \\ 
   \hline 
Land Area Quartiles & & & & & \\ 
$<$ 800 sq. ft & 0.22 & 0.10 & 0.45 & 0.23 & 0.92 \\ 
  \lbrack800, 1200) sq. ft & 0.23 & 0.10 & 0.43 & 0.24 &  \\ 
  \lbrack1200, 1800) sq. ft & 0.21 & 0.10 & 0.47 & 0.22 &  \\ 
  $>$ 1800 sq. ft & 0.21 & 0.10 & 0.44 & 0.24 &  \\ 
   \hline 
\# Rooms & & & & & \\ 
0-5 & 0.22 & 0.11 & 0.44 & 0.23 & 0.52 \\ 
  6 & 0.21 & 0.09 & 0.46 & 0.23 &  \\ 
  7+ & 0.22 & 0.10 & 0.44 & 0.24 &  \\ 
   \hline 
Years of Debt & & & & & \\ 
1 Years & 0.23 & 0.09 & 0.43 & 0.24 & 0.11 \\ 
  2 Years & 0.22 & 0.10 & 0.44 & 0.24 &  \\ 
  3-5 Years & 0.20 & 0.10 & 0.48 & 0.22 &  \\ 
  6+ Years & 0.20 & 0.13 & 0.47 & 0.20 &  \\ 
   \hline 
Category & & & & & \\ 
Residential & 0.22 & 0.09 & 0.45 & 0.23 & 0.28 \\ 
  Hotels\&Apts & 0.20 & 0.12 & 0.45 & 0.23 &  \\ 
  Store w/ Dwelling & 0.21 & 0.09 & 0.48 & 0.22 &  \\ 
  Commercial & 0.15 & 0.11 & 0.50 & 0.24 &  \\ 
  Industrial & 0.27 & 0.11 & 0.42 & 0.20 &  \\ 
  Vacant & 0.25 & 0.13 & 0.39 & 0.23 &  \\ 
   \hline
Distribution of Properties & 0.22 & 0.10 & 0.45 & 0.23 & 0.60 \\ 
   \hline
Expected Distribution & 0.22 & 0.11 & 0.44 & 0.22 &  \\ 
   \hline
\multicolumn{6}{|l|}{NOTE: This table shows that there are no significant differences in the} \\
\multicolumn{6}{|l|}{distribution of observed variables among the treatment and control samples.} \\ 
\hline
\end{tabular}
\end{table}


\section{Average Treatment Effects: Revenues}


We consider results for three different subsamples.  The first sample
(I) is the full sample and consists of all 4927 observations. The
second sample (II) eliminates commercial property owners, which
reduces the sample to 4749 observations. The third sample (III)
eliminates owners of multiple properties, resulting in a sample size
of 3888.

Table \ref{table:summary} summarizes the impact of our experimental
intervention on total revenue collection.  The table reports the total
taxes owed, the amount generated, and the number of mailing days for
the three treatments and the control groups. It also reports the
percent of properties that paid the City anything and the percent that
paid off their tax debt in full in our sample period.


% latex table generated in R 3.2.2 by xtable 1.7-4 package
% Mon Oct  5 00:58:37 2015
\begin{sidewaystable}[htbp]
\centering
\caption{Estimated Average Treatment Effects: Revenues} 
\label{table:summary}
\begin{tabular}{|p{2.2cm}|p{1.4cm}|p{1.8cm}|p{1.2cm}|p{1.2cm}|p{1.4cm}|p{1.4cm}|p{2.2cm}|p{1.8cm}|}
  \hline
Sample & Treatment (Properties) & Total Debt Owed & Percent Ever Paid & Percent Paid in Full & Dollars Received & Percent Debt Received & Dollars above Control Per Property & Total Surplus over All Properties \\ 
  \hline
Full & Control (n=1,075) & \$3,294,516 & 16 & 10 & \$120,585 & 3.7 & \$0 & \$0 \\ 
  (N=4,927) & Threat (n=499) & \$1,839,826 & 14 & 8 & \$71,176 & 3.9 & \$30 & \$15,202 \\ 
   & Public Service (n=2,211) & \$8,003,148 & 15 & 7 & \$447,728 & 5.6 & \$90 & \$199,714 \\ 
   & Civic Duty (n=1,142) & \$3,794,900 & 18 & 12 & \$152,217 & 4.0 & \$21 & \$24,116 \\ 
   \hline
Non-Commercial & Control (n=1,048) & \$2,930,759 & 16 & 10 & \$120,069 & 4.1 & \$0 & \$0 \\ 
  (N=4,749) & Threat (n=480) & \$1,657,379 & 15 & 8 & \$71,176 & 4.3 & \$34 & \$16,183 \\ 
   & Public Service (n=2,122) & \$7,024,458 & 15 & 7 & \$288,758 & 4.1 & \$22 & \$45,642 \\ 
   & Civic Duty (n=1,099) & \$3,350,147 & 19 & 12 & \$146,227 & 4.4 & \$18 & \$20,315 \\ 
   \hline
Unique Owner & Control (n=837) & \$3,007,232 & 16 & 9 & \$66,597 & 2.2 & \$0 & \$0 \\ 
  (N=3,888) & Threat (n=406) & \$1,437,902 & 15 & 9 & \$51,309 & 3.6 & \$47 & \$19,005 \\ 
   & Public Service (n=1,754) & \$6,956,034 & 16 & 7 & \$418,767 & 6.0 & \$159 & \$279,207 \\ 
   & Civic Duty (n=891) & \$3,331,168 & 20 & 13 & \$130,016 & 3.9 & \$66 & \$59,123 \\ 
   \hline
\multicolumn{9}{l}{NOTE: The table shows how much additional revenues were generated by the different treatments.} \\
\end{tabular}
\end{sidewaystable}


We also report the dollars in revenue raised per day, which ranges
from \$60,292 in the control group to \$111,931 in the Public Service
treatment group. Note that the average payments per day is higher in
all three treatment group. A simple difference between the treatment
and the control group provides an estimate of the overall
effectiveness of the intervention. These estimates range from \$10,883
for the threat treatment to \$51,639 for the Public Service
treatment. Summing over all treatment groups and days shows 
our experiment generated approximately \$250,000 for the DoR in just
nine treatment days.  


\begin{table}
\caption{Estimated Average Treatment Effects: Revenues}
\begin{center}
\begin{tabular}{|l|c|c|c|}
\hline
               & Main Sample & Non-Commercial Sample & Unique Owner Sample \\
\hline
Intercept      & $112.17^{***}$ & $114.57^{***}$ & $79.57^{***}$ \\
               & $(17.52)$      & $(17.96)$      & $(12.73)$     \\
Threat         & $30.46$        & $33.71$        & $46.81$       \\
               & $(36.18)$      & $(37.47)$      & $(29.86)$     \\
Public Service & $90.33$        & $21.51$        & $159.18^{**}$ \\
               & $(57.85)$      & $(26.97)$      & $(70.50)$     \\
Civic Duty     & $21.12$        & $18.49$        & $66.36^{*}$   \\
               & $(33.90)$      & $(34.97)$      & $(38.75)$     \\
\hline
\multicolumn{4}{l}{\scriptsize{$^{***}p<0.01$, $^{**}p<0.05$, $^*p<0.1$}}
\end{tabular}
\label{dif_mean}
\end{center}
Note that the intercept captures the baseline effectiveness of the
control group. The coefficients of the treatments measure the
difference in the mean effectiveness relative to the control group.
\end{table}

Table \ref{dif_mean} provides estimates of the average treatment effect 
of each reminder letter on the revenues collected per letter from 
our sample of these most tardy taxpayers.   Estimates are provided for 
each of three samples:  the full sample of all most tardy 
properties, the sample of properties excluding commercial 
properties, and finally the sample of properties having a single 
owner.  Estimates are obtained from a simple OLS regression of 
revenues collected per letter mailed against each treatment 
letter, Threat, Public Service, or Civic Duty, and an intercept 
which measures revenues collected from the control letter.  
The estimated regression coefficients for each treatment letter 
measure the additional revenues raised by the treatment above 
those revenues raised by the city's standard reminder letter 
(the control).  Because Table \ref{table:balance} showed the receipt of each 
treatment letter is uncorrelated with any important property 
attributes or taxes due, that is, the treatments are ``balanced,"
no additional controls are included in the three OLS regressions.   
Robust standard errors are also reported.   	
	
Table \ref{dif_mean} shows there is significant heterogeneity in the impacts
of the three treatments across our three samples.  For the full
sample, the average treatment effect is \$90 for the Public Service
treatment, \$21 for the Civic Duty treatment, and \$30 for Threat
treatment.  If we restrict our analysis to the subsample of properties
that are non-commercial, we find that impact of the Public Service
treatment is now much smaller, only \$21.51.  This difference implies
that those making the largest contribution in response to the Public
Service letter are commercial properties.  Finally, when we restrict
attention to the subsample of singe property owners we find all
letter's average treatment effects are much larger and more
significant statistically.  This single owner sample includes resident
homeowners, single owners of rental properties, or single owners of
city businesses and represents 79 percent of the properties receiving
treatment letters.  This sample is most representative of the average
voting taxpayer in the city.  For this subsample of tardy taxpayers,
the Threat letter raises an additional \$47/letter, the Public Service
letter raises an additional \$159/letter, and the Civic Duty letter
raises an additional \$66/letter. The Public Service and Civic Duty
treatment effects are significant at the 5 percent and 10 percent
confidence levels, respectively.

We have also estimated the average treatment effect on revenues for simply 
receiving one of our three treatment letters (results not shown).   
These estimated impacts measure the revenue advantage of the 
experiment itself over simply continuing with business as usual 
using the City's current reminder letter.  The estimated average 
treatment effect averaged over all three treatment letters is an 
additional \$62/letter for the full sample (standard error = \$37, 
implying treatment significance at the 10 percent level of confidence), 
\$22/letter for the non-commercial subsample (standard error = \$23, 
implying not significant at the 10 percent level of confidence), 
and finally, \$117/letter for the sole owner subsample (standard 
error = \$43, implying treatment significance at the 1 percent 
level of confidence).  While there remains uncertainty as to 
exactly which of the three treatment letters is most effective, 
it is clear that doing something different from the City's current 
reminder letter holds potential for raising significantly more 
revenues from this group of ``most tardy" taxpayers.  

\section{Average Treatment Effects: Compliance}

\subsection{Specification}

While collecting additional revenues from this ``most tardy" group 
of taxpayers is of interest, it is also of value to know how we 
might encourage taxpayer compliance, and particularly so, if 
there is a dynamic to tax payments where one is more likely to 
pay, if one's fellow citizens are also paying their taxes.   
Greece, for example, has experienced a significant downward spiral 
in compliance as more and more citizens avoided tax payments.   
To gain insight into the nature of tax compliance in Philadelphia, 
we consider two compliance outcomes.   The first compliance outcome 
is if the tardy taxpayer makes any payment at all (called ``ever paid"), 
and denoted as $y = 1$, and if not, $y = 0$.   The second compliance 
outcome is if the taxpayer makes a full payment of taxes owed (called 
``paid in full") and denoted as $y = 1$, and if not, $y = 0$.  Both outcomes 
are of interest.  “Paid in full” because this represents a potentially 
significant increase in City revenues.   “Ever paid” because even small 
additional payments help, but perhaps more importantly, a tax 
contribution represents a willingness by the taxpayer to be engaged 
with city governance.  It is worth stressing again that our sample is 
for the ``most tardy" of the City's taxpayers, perhaps those least 
likely because of inclination or because of resources to feel a stake 
in the City's fiscal fortunes.    

We specify and estimate compliance
as a logistic function of the control and three treatments, with each
estimated effect measuring the treatment's impact on tax payment
relative to that available from receipt of the control letter.
Generally, for $y = 1$ if the individual pays their taxes, and 0
otherwise, the probability of paying taxes can be specified as:
\begin{eqnarray*}
Pr \{ y=1 \} \; = \; \frac{exp(X' \beta)}{1 + exp(X' \beta)}
\end{eqnarray*}
where $X$ is a vector of explanatory variables and $\beta$ a vector of
coefficients to be estimated.

The benefits of the logistic specification, over the more familiar
linear specification, is that once estimated the computed
probabilities of payment are bounded between 0 and 1, and the partial
effect of any of the independent variables on the probability of
payment can vary according to the overall value of $X'\beta$.  For our
analysis, the vector of explanatory variables $X$ will include three
(1,0) indicator variables for whether the taxpayer received one of the
three treatment letters (Threat, Public Service, or Civic Duty), the
level of taxes owed represented by one of four categories of debt as
LOW (less than \$300), MODerate (\$301 to \$1300), HIGH (\$1301 to
\$3300), or Very HIGH (greater than \$3301), and the interaction of
treatment letters with the level of taxes owed.  The taxpayer is
assigned a value of 1 if the property's tax bill falls within a debt
category, and 0 otherwise.  The omitted debt level for comparison is
LOW.  The interactions of the debt levels with treatments will explore
the possible advantages of targeting treatment letters to taxpayers of
varying debt levels.

Finally, while care has been taken to randomly assign the treatment
and control letters across taxpayers, and our initial balance tests
reported above suggest that we have been successful along the broad
categories of taxes owed and years of tax debt, property values and
property type, and property size and land area, the question remains
of whether taxpayer compliance behaviors might vary along other
attributes of the property or the taxpayer.  If so, and if compliance
behavior is correlated with these excluded variables, then the
estimated effects on payment behavior of the treatment letters may be
biased.  To control for this possibility, we also include in our basic
logistic regression as elements of X measures of the location of the
property within one of ten city neighborhoods (each a City Council
District), the exterior condition of the property (classified as a
``sealed/compromised," i.e. dilapidated and dangerous), and whether
the property qualified for a low income homestead exemption.  



\subsection{Results}
  
Table \ref{table:ep_log_I} summarizes the estimates and the estimated standard
errors for the three samples that we considered above. We report
robust standard errors that are clustered to deal with multiple
ownership. As can be seen from Table \ref{table:ep_log_I}, the Public Service appeal and
the threat treatments had no significant effect on ``ever paid" at the
conclusion of the 30 day payment period.  The Civic Duty treatment is
consistently positive and statistically significant at least the 10
percent level of confidence in the full sample and at the 5
percent level for the sample of sole owners.

\begin{table}[htbp]
\caption{Logistic Regressions for Ever Paid: Compliance}
\label{table:ep_log_I}
\begin{center}
\begin{tabular}{l c c c }
\hline
               & Full Sample & Non-Commercial & Sole Owner \\
\hline
Intercept      & $-1.69^{***}$ & $-1.67^{***}$ & $-1.68^{***}$ \\
               & $(0.11)$      & $(0.11)$      & $(0.09)$      \\
Threat         & $-0.09$       & $-0.06$       & $-0.03$       \\
               & $(0.17)$      & $(0.17)$      & $(0.17)$      \\
Public Service & $-0.07$       & $-0.10$       & $0.04$        \\
               & $(0.13)$      & $(0.13)$      & $(0.12)$      \\
Civic Duty     & $0.21$        & $0.19$        & $0.30^{**}$   \\
               & $(0.13)$      & $(0.14)$      & $(0.13)$      \\
\hline
Log Likelihood & -2136.16      & -2068.89      & -1758.95      \\
Num. obs.      & 4927          & 4749          & 3888          \\
\hline
\multicolumn{4}{l}{\scriptsize{$^{***}p<0.01$, $^{**}p<0.05$, $^*p<0.1$}} \\
\multicolumn{4}{l}{NOTE: This table reports the parameter estimates from the} \\
\multicolumn{4}{l}{basic Logit Model that uses ``ever paid" as outcome.}
\end{tabular}
\end{center}
\end{table}

Next we investigate whether there is heterogeneity in response to the
treatment. It is plausible that very tardy taxpayers who owe small amounts of
money behave differently than those who owe larger amounts.  To gain
insight into this possibility we include in our regression for ``ever
paid" the indicator variables for the levels of taxes owed - LOW, MOD,
HIGH, and VHIGH - and the interaction of those variables with our
three treatments.  The variable LOW is omitted from the regression so
all results provide comparisons to the behavior of those in the higher
debt levels to taxpayers in the lowest level of taxes owed.  Table
\ref{table:ep_log_II} summarizes the estimates and the estimated standard errors
for the full sample and the two subsamples.


\begin{table}[htbp]
\caption{Logistic Regressions for Ever Paid with Interactions: Compliance}
\begin{center}
\begin{tabular}{|l|c|c|c|}
\hline
                             & Full Sample & Non-Commercial & Sole Owner \\
\hline
Balance MOD                  & $-0.46^{*}$   & $-0.52^{**}$  & $-0.33$       \\
                             & $(0.24)$      & $(0.25)$      & $(0.24)$      \\
Balance HIGH                 & $-1.03^{***}$ & $-0.97^{***}$ & $-1.54^{***}$ \\
                             & $(0.27)$      & $(0.27)$      & $(0.31)$      \\
Balance VHIGH                & $-1.25^{***}$ & $-1.15^{***}$ & $-1.36^{***}$ \\
                             & $(0.34)$      & $(0.32)$      & $(0.34)$      \\
Threat                       & $-0.05$       & $-0.01$       & $-0.13$       \\
                             & $(0.29)$      & $(0.29)$      & $(0.30)$      \\
Threat*Balance MOD           & $-0.27$       & $-0.27$       & $-0.22$       \\
                             & $(0.45)$      & $(0.46)$      & $(0.46)$      \\
Threat*Balance HIGH          & $0.40$        & $0.38$        & $0.94^{*}$    \\
                             & $(0.44)$      & $(0.44)$      & $(0.48)$      \\
Threat*Balance VHIGH         & $-0.07$       & $-0.20$       & $-0.19$       \\
                             & $(0.62)$      & $(0.56)$      & $(1.26)$      \\
Public Service               & $-0.30$       & $-0.34^{*}$   & $-0.34$       \\
                             & $(0.20)$      & $(0.21)$      & $(0.21)$      \\
Public Service*Balance MOD   & $0.02$        & $0.10$        & $0.10$        \\
                             & $(0.31)$      & $(0.32)$      & $(0.30)$      \\
Public Service*Balance HIGH  & $0.53^{*}$    & $0.53$        & $1.07^{***}$  \\
                             & $(0.32)$      & $(0.33)$      & $(0.36)$      \\
Public Service*Balance VHIGH & $0.70^{*}$    & $0.62^{*}$    & $0.92^{**}$   \\
                             & $(0.38)$      & $(0.37)$      & $(0.39)$      \\
Civic Duty                   & $0.16$        & $0.13$        & $0.21$        \\
                             & $(0.21)$      & $(0.22)$      & $(0.22)$      \\
Civic Duty*Balance MOD       & $-0.23$       & $-0.09$       & $-0.30$       \\
                             & $(0.33)$      & $(0.33)$      & $(0.32)$      \\
Civic Duty*Balance HIGH      & $0.27$        & $0.15$        & $0.59$        \\
                             & $(0.35)$      & $(0.37)$      & $(0.40)$      \\
Civic Duty*Balance VHIGH     & $-0.13$       & $-0.14$       & $-0.06$       \\
                             & $(0.43)$      & $(0.42)$      & $(0.44)$      \\
\hline
Log Likelihood               & -2010.63      & -1948.32      & -1639.28      \\
Num. obs.                    & 4927          & 4749          & 3888          \\
\hline
\multicolumn{4}{l}{\scriptsize{$^{***}p<0.01$, $^{**}p<0.05$, $^*p<0.1$}} \\
\multicolumn{4}{l}{NOTE: This table reports the parameter estimates from the} \\
\multicolumn{4}{l}{Logit Model with interactions that uses ``ever paid" as outcome.} \\
\end{tabular}
\label{table:ep_log_II}
\end{center}
\end{table}


Table \ref{table:ep_log_II} shows the indicator variables for taxes owed by
quartile are significantly negative - that is, the more a taxpayer
owes the less likely he or she is to pay their taxes.  The Civic Duty
treatment helps to moderate this growing negative effect of tax debt,
but only for those who owe a moderate amount of taxes.  For taxpayers
with a very large tax debt, the Civic Duty letter discourages payment.
Exactly the opposite responses are observed for those who receive the
Public Service letter.  Those with low or moderate levels of taxes
owed react negatively or not at all to the Public Service letter,
while those with high levels of taxes owed are more likely to make a
contribution when they receive the Public Service letter.  The Threat
letter never helps tax compliance and significantly discourages payment
by the most tardy taxpayers with high levels of taxes owed.  One can
speculate as to why motives for payment are tied to the levels of
taxes owed - civic duty is ``price elastic" and free riding falls with
larger property holdings and greater payments. The important
conclusion here is that a targeted treatment strategy is the most
effective approach for encouraging tax participation for this sample
of tardy Philadelphia taxpayers.


% latex table generated in R 3.2.2 by xtable 1.7-4 package
% Mon Oct  5 01:46:07 2015
\begin{table}[htbp]
\caption{Marginal Predictions - Ever Paid} 
\label{table:modelI_marg}
\centering
\begin{tabular}{|l|c|c|c|c|}
  \hline
 & LOW & MOD & HIGH & VHIGH \\ 
  \hline
Control & 23.35 & 16.13 & 9.82 & 7.99 \\ 
  Threat & 22.42 & 12.23 & 13.41 & 7.12 \\ 
  Public Service & 18.45 & 12.69 & 12.10 & 11.45 \\ 
  Civic Duty & 26.37 & 15.23 & 14.38 & 8.25 \\ 
   \hline
\multicolumn{5}{l}{NOTE: This table reports the marginal effects} \\
\multicolumn{5}{l}{from the Logit Model with interactions that uses} \\
\multicolumn{5}{l}{``ever paid" as outcome.} \\
\end{tabular}
\end{table}


Table \ref{table:modelI_marg} shows the marginal predictions for the probability that
a very tardy taxpayer in each treatment group and for each quartile of
taxes owed will make some payment (``ever paid").  The values here
represent the predicted probability of payment, computed for the
``average taxpayer" as represented by the sample average level of all
indicator control variables and the median values of the continuous
control variables.  Here we observe the final impacts of the
treatments as they apply to the most tardy taxpayers with different
levels of taxes owed.  The Civic Duty letter increases the chance of
payment over the control letter for taxpayers with low debt by about 3
percentage points and for taxpayers with relatively high payments by
as much 4 percentage points.  The Public Service letter is most
effective for tardy taxpayers with very high levels of taxes owed.
Because of this heterogeneity in response to different treatments, a
preferred overall strategy may be to target different treatment
letters to different cohorts of tardy taxpayers by taxes owed.

Next we examine whether our treatment strategies might also impact the
larger matter: When do tardy taxpayers pay their full amount of taxes
owed?  The ever-paid outcome does not differentiate between taxpayers
that made full payment and those who made only a partial contribution.

Table \ref{table:pf_log_I} presents the results for the Logit specification for
the outcome ``paid in full" by the end of our 30 day payment period.
Again the analysis is separated into the full sample and the
two subsamples.  The results are similar to those for ``ever paid."
The Threat letter is never effective.  The Public Service letter
discourages full payment while the Civic Duty letter encourages full
payment.

\begin{table}[htbp]
\caption{Logistic Regressions for Paid in Full: Compliance}
\begin{center}
\begin{tabular}{l c c c }
\hline
               & Full Sample & Non-Commercial & Sole Owner \\
\hline
Intercept      & $-2.23^{***}$ & $-2.22^{***}$ & $-2.29^{***}$ \\
               & $(0.14)$      & $(0.14)$      & $(0.12)$      \\
Threat         & $-0.21$       & $-0.18$       & $-0.04$       \\
               & $(0.22)$      & $(0.23)$      & $(0.21)$      \\
Public Service & $-0.42^{**}$  & $-0.44^{***}$ & $-0.29^{*}$   \\
               & $(0.17)$      & $(0.17)$      & $(0.15)$      \\
Civic Duty     & $0.24$        & $0.24$        & $0.41^{***}$  \\
               & $(0.17)$      & $(0.17)$      & $(0.15)$      \\
\hline
Log Likelihood & -1435.15      & -1395.06      & -1175.05      \\
Num. obs.      & 4927          & 4749          & 3888          \\
\hline
\multicolumn{4}{l}{$^{***}p<0.001$, $^{**}p<0.01$, $^*p<0.05$.} \\
\multicolumn{4}{l}{NOTE: This table reports the parameter estimates from the} \\
\multicolumn{4}{l}{basic Logit Model that uses ``paid in full" as outcome.} \\
\end{tabular}
\label{table:pf_log_I}
\end{center}
\end{table}


Table \ref{table:pf_log_II} presents the results that allow for the influence of
taxes owed - LOW, MOD, HIGH, and VHIGH - on ``paid in full."  Owing more
taxes reduces the likelihood of paying in full and the negative effect
increases with the level of taxes owed.  These effects are even larger
than those in the ``ever paid" analysis suggesting that many tardy
taxpayers in the higher quartiles of taxes owed make only partial
payments when the respond (if at all) to the control and treatment
letters.  We continue to see the negative impact of the Public Service
letter on full payment, but again, as for ever paid, the strong
negative effect disappears for those with the greatest tax debts.  The
letter that has the greatest positive impact on encouraging full tax
payment is the Civic Duty letter, and this is particularly so for
those with low and moderate levels of taxes owed.


\begin{table}[htbp]
\caption{Logistic Regressions for Paid in Full with Interactions: Compliance}
\begin{center}
\begin{tabular}{|l|c|c|c|}
\hline
                             & Full Sample & Non-Commercial & Sole Owner \\
\hline
Balance MOD                  & $-1.25^{***}$ & $-1.38^{***}$ & $-1.30^{***}$ \\
                             & $(0.31)$      & $(0.32)$      & $(0.30)$      \\
Balance HIGH                 & $-1.96^{***}$ & $-1.86^{***}$ & $-2.54^{***}$ \\
                             & $(0.55)$      & $(0.46)$      & $(0.87)$      \\
Balance VHIGH                & $-2.65$       & $-2.42$       & $-3.90$       \\
                             & $(2.76)$      & $(2.00)$      & $(6.39)$      \\
Threat                       & $-0.10$       & $-0.07$       & $-0.10$       \\
                             & $(0.30)$      & $(0.30)$      & $(0.30)$      \\
Threat*Balance MOD           & $-0.39$       & $-0.31$       & $-0.32$       \\
                             & $(1.19)$      & $(1.19)$      & $(1.62)$      \\
Threat*Balance HIGH          & $0.27$        & $0.20$        & $0.79$        \\
                             & $(0.87)$      & $(0.74)$      & $(1.40)$      \\
Threat*Balance VHIGH         & $-0.00$       & $-0.33$       & $0.99$        \\
                             & $(5.26)$      & $(5.56)$      & $(9.11)$      \\
Public Service               & $-0.45^{**}$  & $-0.48^{**}$  & $-0.48^{**}$  \\
                             & $(0.21)$      & $(0.22)$      & $(0.22)$      \\
Public Service*Balance MOD   & $-0.09$       & $0.10$        & $0.07$        \\
                             & $(0.39)$      & $(0.41)$      & $(0.39)$      \\
Public Service*Balance HIGH  & $0.32$        & $0.29$        & $0.80$        \\
                             & $(0.61)$      & $(0.53)$      & $(0.91)$      \\
Public Service*Balance VHIGH & $0.13$        & $-0.18$       & $1.69$        \\
                             & $(2.81)$      & $(2.12)$      & $(6.40)$      \\
Civic Duty                   & $0.23$        & $0.21$        & $0.28$        \\
                             & $(0.22)$      & $(0.23)$      & $(0.22)$      \\
Civic Duty*Balance MOD       & $-0.25$       & $-0.11$       & $-0.12$       \\
                             & $(0.42)$      & $(0.43)$      & $(0.41)$      \\
Civic Duty*Balance HIGH      & $-0.38$       & $-0.43$       & $-0.13$       \\
                             & $(0.68)$      & $(0.61)$      & $(1.27)$      \\
Civic Duty*Balance VHIGH     & $-0.25$       & $-0.48$       & $1.01$        \\
                             & $(3.08)$      & $(2.42)$      & $(6.67)$      \\
\hline
Log Likelihood               & -1210.73      & -1179.75      & -967.31       \\
Num. obs.                    & 4927          & 4749          & 3888          \\
\hline
\multicolumn{4}{l}{\scriptsize{$^{***}p<0.01$, $^{**}p<0.05$, $^*p<0.1$}} \\
\multicolumn{4}{l}{NOTE: This table reports the parameter estimates from the} \\
\multicolumn{4}{l}{Logit Model with interactions that uses ``paid in full" as outcome.}
\end{tabular}
\label{table:pf_log_II}
\end{center}
\end{table}


As for Table \ref{table:modelI_marg}, the impact of tax debt owed on full compliance is
seen most clearly by computing the marginal propensity for payment in
full by each treatment for each level of debt owed; see Table \ref{table:modelI_marg}.
Again the Civic Duty letter has the strongest effect on the decision
of these very tardy taxpayers to pay their taxes, here to meet their
full obligations.


% latex table generated in R 3.2.2 by xtable 1.7-4 package
% Mon Oct  5 01:46:07 2015
\begin{table}[htbp]
\caption{Marginal Predictions - Paid in Full} 
\label{table:modelI_marg}
\centering
\begin{tabular}{|l|c|c|c|c|}
  \hline
 & LOW & MOD & HIGH & VHIGH \\ 
  \hline
Control & 21.36 & 7.22 & 3.68 & 1.87 \\ 
  Threat & 19.68 & 4.55 & 4.31 & 1.69 \\ 
  Public Service & 14.80 & 4.34 & 3.26 & 1.37 \\ 
  Civic Duty & 25.55 & 7.10 & 3.18 & 1.84 \\ 
   \hline
\multicolumn{5}{l}{NOTE: This table reports the marginal effects} \\
\multicolumn{5}{l}{from the Logit Model with interactions that uses} \\
\multicolumn{5}{l}{``paid in full" as outcome.} \\
\end{tabular}
\end{table}


Our study reveals there is heterogeneity in response to different
treatments.  A preferred overall strategy might take advantage of
these differential responses of taxpayers to the treatment
letters. More research is clearly needed to assess the efficiency of
targeted reminding strategies.


\section{Conclusions}

This field experiment evaluated three alternative notification
strategies intended to increase property tax compliance.  We
implemented our experiment in collaboration with Philadelphia's
Department of Revenue (DoR).  This initial study of property tax
compliance in Philadelphia has value for at least three reasons.
First, it is the first study that systematically examines alternative
tax compliance strategies for taxation in a large city.  Second, the
study of property tax compliance for which there is a known tax
liability has allowed us to focus directly on motives for paying
taxes.  Third, great care was given to separately specify, identify,
and directly compare the three common motives for tax payment that
play a prominent role in the tax compliance literature.

Our findings provide tentative support for the conclusion that appeals
to public service provision and to a citizen's sense of civic duty can
improve property tax collections and taxpayer compliance, particularly
for single owner properties.  The estimated impact of each appeal
varies by the degree of tax debt owed.  The Public Service appeal is
effective for those taxpayers who owe very large tax debts (greater
than \$3300) while the Civic Duty appeal has its greatest impact on
taxpayers who have very small debts owed (less than \$300).  The
average treatment effects on city revenues are largest, and most
significant statistically, for the sample of single owner properties.
For this sample of very tardy taxpayers, the Public Service appeal
adds \$159/letter and the Civic Duty appeal adds \$66/letter over
revenues raised by the City's current reminder letter.  Our results
for taxpayer compliance show why.  The Public Service appeal is
particularly effective in encouraging those with large debts to make
at least some payment towards their taxes owned.  The Civic Duty
letter differentially encourages at least some payments from those
tardy taxpayers with small levels of debt owed.  The fact that
taxpayers responded differentially to the three reminder letters
strongly suggests that a strategy that targets different reminders to
specific cohorts can improve collection performance.  A uniform
reminder to all late or non-compliant taxpayers is unlikely to be
revenue maximizing.
	
In contrast, we find no statistically significant support for the use
of our Threat letter as a means to increase revenues collected or
taxpayer compliance.  This result deserves further analysis, for at
least three reasons.  First, our sample of taxpayers receiving the
threat letter is relatively small.  Second, our sample may be unique
in its taxpaying motivations.  These are the very tardiest of the
City's taxpayers.  By the time they have received our Threat letter in
mid to late November, they had already received at least one and maybe
as many as three previous reminders to pay their taxes, and each
reminder had included a summary of the fines and penalties that were
accruing with continued delays.  Further appeals to an economic
motivation may be irrelevant for this cohort of very tardy taxpayers.
Third, the wording of our Threat letter is very blunt and therefore
may seem too draconian or too remote to be credible.  Alternative
specifications for the threat of penalty should be considered to make
it seem more relevant and thus enforceable.
	
If there is a single, strong lesson to be learned from our analysis,
it is that tax compliance experiments that explore a range of
motivations for city taxpayer behavior are well worth doing.  For two
reasons.  First, moving beyond the usual practice of simply mailing
the tax bill and assuming payment and towards mailing tax bills and
reminder letters that invoke a reason to pay can be profitable.  In
our study we find that just receiving one of our three motivational
treatment letters increased average tax payment per letter by
\$62/letter for the full sample and by \$117/letter for the most
important subsample of taxpayers owning just a single property.
Second, for this sample of very tardy taxpayers, the Civic Duty and
Public Service letters encouraged increased taxpayer compliance.
Quite apart from any additional revenues, citizen participation
through tax payments can be an important motivation for additional
citizen engagement with the fiscal management of their city, and
particularly so if there are dynamic effects onto tax compliance and
citizen engagement by other citizens.

There are limitations to our study, of course.  Strictly speaking, our
conclusions apply only to Philadelphia taxpayers, and among those
citizens, only those who are most tardy in paying their taxes. Second, our
sample of taxpayers is small, only 4927 taxpayers in total.  And
finally, while our focus on property tax compliance has the advantage
of allowing us to more cleanly identify motives for tax payments,
Philadelphia and other cities raise significant revenues from wage
taxes, income and profits taxes, sales taxes, and fees.  Payment
compliance for cities for these other revenue instruments deserves
careful analysis too.  All said, however, we feel our work here is an
encouraging first step towards introducing the new behavioral methodologies of
tax compliance into the practice of city government finances.

\newpage

\section*{References}

Aaron, Henry (1975), Who Pays the Property Tax? Washington, DC:
Brookings Institution. \\
\\
Ariel, Barak (2012), ``Deterrence and moral persuasion effects on
corporate tax compliance: Findings from a randomized controlled
trial." Criminology, 50 (1), 27-69. \\
\\
Allingham, Michael G., and Agnar Sandmo (1972) ``Income Tax Evasion: A
Theoretical Analysis." Journal of Public Economics, 1: 323-38. \\
\\
Alm, James, Gary H. McClelland, and William D. Schulze (1992), ``Why Do People
Pay Taxes?" Journal of Public Economics 48: 21-38. \\
\\
Alm, James (1999), ``Tax compliance and administration." In: Hildreth,
W. Bartley and James A. Richardson (eds.) Handbook on Taxation. New
York, USA, Marcel Dekker, Inc., pp. 741-768. \\ 
\\
Andreoni, James, Erard, Brian and Jonathan Feinstein (1998), ``Tax
compliance." Journal of Economic Literature, 36, 818-860. \\ 
\\
Becker, Gary S. (1968), ``Crime and Punishment: An Economic Approach."
Journal of Political Economy 76: 169-217.\\
\\
Benabou, Roland and Jean Tirole (2011), ``Laws and Norms," NBER
Working Paper, No. 17579. \\
\\
Bernheim, B. Douglas (1994), ``A Theory of Conformity." Journal of
Political Economy, 102, 5, 841-877. \\
\\
Besley, Timothy, Anders Jensen, and Torsten Persson (2014), ``Norm,
Enforcement, and Tax Evasion," London School of Economics.  \\ 
\\
Blumenthal, Marsha, Christian, Charles and Joel Slemrod (2001), ``Do
normative appeals affect tax compliance? Evidence from a controlled
experiment in Minnesota." National Tax Journal, 54 (1), 125 -
138. \\ 
\\ 
Cowell, Frank A. and James P. F. Gordon (1988), `` Unwillingness to
pay tax: tax evasion and public provision."  Journal of Public
Economics, 36, 305-321.\\ 
\\ 
Feddersen, Timothy and Alvaro Sandroni (2006), ``A Theory of
Participation in Elections," American Economic Review, Vol. 96
(September), 1271-1282.  \\ 
\\ 
Fehr, Ernst and Simon Gachter (1998), ``Reciprocity and economics: The
economic implications of homo reciprocans." European Economic Review
42 (3-5), 845-59. \\ 
\\ 
Fellner, Gerlinde, Rupert Sausgruber, and Christian Traxler (2013),
``Testing Enforcement Strategies in the Field: Threat, Moral Appeal
and Social Information." Journal of the European Economic Association
11, 3, 634-60.\\ 
\\ 
Frey, Bruno S., and Lars P. Feld (2002), ``Deterrence and Morale in
Taxation: An Empirical Analysis."  CESifo Working Paper no. 760,
August 2002. \\ 
\\ 
Hallsworth, Michael., John List, Robert Metcalfe and Ivo Vlaev (2014),
``The Behavioralist as Tax Collector," Using Natural Field Experiments
to Enhance Tax Compliance." NBER Working Paper 20007. \\ 
\\ 
Hamilton, Bruce (1975), ``Zoning and Property Taxation in a System of
Local Governments," Urban Studies, Vol. 12 (June),
205-211. \\ 
\\ 
Harrison, Glenn W. and John A. List (2004), ``Field Experiments."
Journal of Economic Literature, 42 (4), 1009-1055.\\ 
\\ 
Keen, Michael and Ben Lockwood (2010), ``The Value-Added Tax: Its
Causes and Consequences," Journal of Development Economics, Vol. 92
(July), 138-151. \\ 
\\ 
Kleven, Henrik J., Knudsen, Martin B., Kreiner, Claus T., Pedersen,
Soren and Emmanuel Saez (2011), ``Unwilling or Unable to Cheat?
Evidence From a Tax Audit Experiment in Denmark."  Econometrica, 79
(3), 651-692. \\ 
\\ 
Mieszkowski, Peter (1972), ``The Property Tax: an Excise or a Profits
Tax?" Journal of Public Economics, Vol. 1 (April),
73-96. \\ 
\\ 
Organization for Economic Cooperation and Development 2011, Tax
Administration in OECD and Selected Non-OECD Countries: Comparative
Information Series (2011), Center for Tax Policy and Administration.
\\ 
\\ 
Perez-Truglia, Ricardo and Ugo Troiano (2015), ``Tax Debt Enforcement:
Theory and Evidence from a Field Experiment in the United States,"
University of Michigan.  \\ 
\\ 
Pew Charitable Trust (2013), ``Delinquent Property Tax in
Philadelphia." Technical Report. \\ 
\\ 
Pomeranz, Dina (2015), ``No taxation without information: Deterrence
and self-enforcement in the Value Added Tax."  American Economic Review, 
105(8): 2539-69. \\ 
\\ 
Posner, Eric (2000), ``Law and Social Norms: the Case of Tax
Compliance," Virginia Law Review, Vol. 86, (No. 8)
1781-1819. \\ 
\\ 
Reckers, Philip M. J., Sanders, Debra L. and Stephen J. Roark (1994),
``The influence of ethical attitudes on taxpayer compliance." National
Tax Journal, 47 (4), 825-836. \\ 
\\ 
Sherman, Lawrence (1993), ``Defiance, deterrence, and irrelevance: A
theory of the criminal sanction."  Journal of Research in Crime and
Delinquency, 30, 445-473. \\ 
\\ 
Slemrod, Joel (2007), ``Cheating ourselves: The economics of tax
evasion." Journal of Economic Perspectives, 21 (1),
25-48. \\ 
\\ 
Slemrod, Joel, Marsha Blumenthal, and Charles Christian (2001),
``Taxpayer Response to an Increased Probability of Audit: Evidence
from a Controlled Experiment in Minnesota." Journal of Public
Economics 79, 3, 455-83.\\ 
\\ 
Torgler, Benno (2002), ``Moral-suasion: An alternative tax policy
strategy?  Evidence from a controlled field experiment in
Switzerland." Economics of Governance 5 (3), 235-253. \\ 
\\ 
Torgler, Benno (2012), ``A field experiment on moral-suasion and tax
compliance focusing on under-declaration and over-deduction." QUT
School of Economics and Finance Working Paper no. 285. \\ 
\\ 
Wenzel, Michael (2005), ``Misperceptions of social Norms about Tax
Compliance: From Theory to Intervention." Journal of Economic
Psychology 26, 6, 862-83\\ 
\\ 
Wenzel, Michael and Natalie Taylor (2004), ``An experimental
evaluation of tax-reporting schedules: a case of evidence-based tax
administration." Journal of Public Economics, 88 (12), 2785-2799.

\bigskip

\bigskip

\section*{Appendix: Figures 1 through 5}
 \newpage
 
\begin{figure}[htpb]
\begin{center}
\caption{Standard Due Letter}
\bigskip
\includegraphics[width=6in]{PastDueLetter.pdf}
\end{center}
\end{figure}
\newpage
\begin{figure}[htpb]
\begin{center}
\caption{Treatment 1: Deterrence}
\bigskip
\includegraphics[width=6in]{flyer_options_141104_treat1.pdf}
\end{center}
\end{figure}
\newpage
\begin{figure}[htpb]
\begin{center}
\caption{Treatment 2: Public Service Appeal}
\bigskip
\includegraphics[width=6in]{flyer_options_141104_treat2.pdf}
\end{center}
\end{figure}
\newpage
\begin{figure}[htpb]
\begin{center}
\caption{Treatment 3: Civic Duty}
\bigskip
\includegraphics[width=6in]{flyer_options_141104_treat3.pdf}
\end{center}
\end{figure}
\newpage
\begin{figure}[htpb]
\begin{center}
\caption{Control}
\bigskip
\includegraphics[width=6in]{flyer_options_141104_treat4.pdf}
\end{center}
\end{figure}

\end{document}


