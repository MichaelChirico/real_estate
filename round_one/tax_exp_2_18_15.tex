\documentclass[12pt,titlepage]{article}

\renewcommand\baselinestretch{1.5}
\setlength{\parskip}{0.08in}
\setlength{\medskipamount}{0.05in}
\textheight 8.0 in
\textwidth 6.0 in
\topmargin 0.25in
\tolerance=11000

\usepackage[utf8]{inputenc}
\usepackage{bbm}
\usepackage[pdfencoding=auto,unicode=true]{hyperref}
\usepackage{graphicx}
\usepackage{epstopdf}
\renewcommand{\thefootnote}{\fnsymbol{footnote}}

\begin{document}

\title{An Experimental Evaluation of Strategies to Increase Property Tax Compliance}
\author{Michael Chirico, Robert Inman, Charles Loeffler, \\
John MacDonald, and Holger Sieg\thanks{We would like to thank  Rob Dubow, Clarena Tolson  and Marisa Waxman  in the 
Department of Revenue of City of Philadelphia for their help and suggestions.}
\\
University of Pennsylvania \\
\\
Preliminary Draft}
\date{\today}  
\maketitle

\begin{abstract}
The purpose of this experimental study is to evaluate a set of
strategies intended to increase property tax revenue
collection. We develop  a field experiment to test three of the most commonly
suggested hypotheses: deterrence, moral appeal, and peer conformity.
We have  implemented the experiment in collaboration with the the Department of Revenue (DoR) of the City
of Philadelphia. Our preliminary findings provide some evidence that both moral appeal
and peer conformity may help to help to improve tax compliance. We find
little evidence that supports the standard deterrence model.

\noindent KEYWORDS:  Tax Compliance, Property Taxation, Field Experiment, Deterrence, Moral Appeal, Peer Conformity.
\end{abstract}

\newpage

\renewcommand{\thefootnote}{\arabic{footnote}}

\renewcommand{\thefootnote}{\arabic{footnote}}

\section{Introduction}

The purpose of this experimental study is to evaluate a set of
strategies intended to increase property tax revenue
collection.  We design a number of different treatments that are
motivated by the theoretical literature on tax compliance, which
distinguishes between extrinsic and intrinsic motivations for paying
taxes. Extrinsic motivations are emphasized by the deterrence
model (Becker, 1968) which explains tax compliance as a combination of penalties and
costly monitoring. Intrinsic motivations are captured in models that
rely either on social norms or social contracts.  We develop and
implement a field experiment to test three of the most commonly
suggested hypotheses: deterrence, moral appeal, and peer conformity.
The basic strategy behind our controlled randomized experiments is to
vary the informational content of letters sent to real
estate tax-delinquent property owners.\footnote{The empirical literature on tax compliance 
is discuss in detail in Hallsworth, List, Metcalfe and Vlaev (2014). 
Other  notable papers that use field experiments to study tax compliance are 
Blumenthal et al (2001), Kleven et al. (2011), Ariel (2012), and Pomeranz (2013).}


While most of the previous literature has focused on income tax
collection, we focus on property tax collection which has not gained
as much attention. Nevertheless, there are some compelling reasons to
focus on property taxes.  First, property tax collection should be,
in principle, much easier than income tax collection. The property tax
base, which is the assessed value of the house, is easily observed and
verified. Establishing the tax base for an income tax (taxable income)
is much more difficult. Moreover, the property tax base is established
in a separate assessment process in most cities in the U.S. While
there are frequent disagreements about correct property tax
assessments, these issues are not relevant from the perceptive of tax
collection enforcement. Second, local revenues play a unique role in
the financing of local municipalities and school districts in the U.S. They are
an important part of the U.S. revenue system, which in contrast to
many other developed countries provides much greater autonomy to
local governments to levy taxes. The ability of local
governments to perform this job efficiently and fairly is thus crucial
for the functioning of local governments in the U.S.

Property tax compliance is a significant problem in the many
large U.S. cities. Philadelphia is a leading example.  A recent study by the
Pew Charitable Trust (2013) focused on a sample of 36 cities and found
that Philadelphia had the fifth highest delinquency rate in 2011, the
last year for which common statistics were available. In Philadelphia,
9 percent of 2011 property taxes went uncollected in that year. The
median delinquency rate in the 36 cities in that study was 4.1
percent.  As of April 2014, the city and school district were owed
\$595 million in delinquent taxes on 134,888 properties.  As a
consequence, Philadelphia provides an ideal laboratory to study
property tax compliance issues.

We implemented our field experiment in collaboration with the the Philadelphia Department of Revenue (DoR)
to provide new insights into promising paths to increase revenue collection. Our intervention
focuses on three treatments that are motivated by the recent literature on tax
compliance.  The first treatment is based on the deterrence model of
tax compliance which emphasizes extrinsic motivations of behavior
(Becker, 1968, Allingham and Sandmo, 1972). Key elements of the model are the
probability of detection (the monitoring probability) and the size of
the penalty or fine that is levied on those tax evaders who are
caught. Given the uncertainty of law enforcement, behavior is based on
subjective probabilities of detection and perceptions regarding the
size of penalties. The basic idea behind the first treatment is to change
these subjective beliefs over penalties and detection probabilities. Our 
experiment does not explicitly involve tax amnesties. Neither do we change the fiscal incentives to
pay taxes in our proposed experiment. Instead, we attempt to change
subjective beliefs regarding detection and the consequences of tax
delinquencies by informing agents and threatening repercussions for nonpayment.

Our second treatment is based on a more recent theory that emphasizes
the intrinsic motivations of tax payers and focuses more broadly on
tax morale (Alm, McClelland and Schulze, 1992). The social contract
approach emphasizes the promise of the city to efficiently deliver
public goods and services in exchange for more-or-less voluntary tax
payments.  Loyalties and emotional ties play a large role in creating
high tax moral. Parallels are drawn between the problem of raising
taxes and the fund-raising problem faced by non-for-profit
organizations (Andreoni, Erard and Feinstein, 1998).\footnote{For example, publicly funded radio stations
  typically need to raise funds from their listeners.} The basic idea
behind the second experiment is then to design an informational
treatment that reinforces this social contract. The treatment takes
the form of a moral appeal to the delinquent tax payer and emphasizes
the \textit{quid pro quo} of tax compliance in the city--public goods
and services (namely, education and safety provision) funded by
compliant taxpayers. Finally, our third treatment is based on the recent literature on
social norms which emphasizes concepts of social fairness and
reciprocity (Fehr and Gachter, 1998). The idea behind our third experiment is then to appeal to
delinquents' proclivities for peer conformity.

To implement our experiments we needed to integrate our experimental
design with the standard operating procedures of the billing system of
the DoR. Fortunately, DoR assigns properties to billing cycles
using a pseudo-randomized mechanism, in which assignments are primarily
based on the last two digits of social security or  employer
identification numbers.  As a consequence assignment of a property to a day in the billing cycle can be treated as random.
Moreover, we assigned treatments to
days in the billing cycles by using randomized four-day sequences.

We conducted our experiment in November 2014 spanning a period 
consisting of 15 treatment days.  We assessed the fidelity of our
experiment by analyzing letters that were returned to the DoR by the
U.S. Postal Service. Based on our analysis of the returned letters we
concluded that our treatment had been correctly assigned on 9 of the
15 days.\footnote{Problems resulted due to technical difficulties in
  the printing office as well as a public holiday that interfered with
  the intended treatments.}  Our final sample consists of 5151
properties. Based on a variety of randomization tests, we conclude
that treatments were randomly assigned within our sample. The DoR
started to share data on property tax payments immediately after the
experiment. Our current sample is based on all payments up to January,
2015. We plan to expand our sample until the March 2015.

Our empirical analysis focuses on two sets of outcome measures. First,
we follow the recent literature and analyze discrete outcomes. We
would like to know whether the property owner ever or fully payed something
during our sample period.  Second, we focus on the actual payments received
by DoR.  Due to the heterogeneity in assessed values, we normalize the
received property tax payments by the assessed value of the property.
Our preliminary findings provide some evidence that both moral appeal
and peer conformity may help to improve tax compliance. We find
little evidence that supports the standard deterrence model.\footnote{other
empirical evidence have failed to provide support for the deterrence
modle. See for example Alm (1999) or Togler (2002).}

The rest of the paper is organized as follows. Section 2 discusses the
institutional background focusing on property tax collection in
Philadelphia.  Section 3 provides a detailed discussion of our
three treatments and the control. Section 4 discusses the experimental design and the
fidelity of its implementation. Section 5 details the statistical
approach to measuring treatment response. Section 6 discusses our sample. Section 7 
presents our current empirical findings.  Section 8 offers some conclusions.


\section{Institutional Background}

We briefly describe the institutional setting focusing
on property tax collection in Philadelphia.  Real Estate Taxes in
Philadelphia are levied annually on a property-level basis.  The
Office of Property Assessment evaluates the market value of each
property, 1.34 \% of which must be paid to the Philadelphia Department
of Revenue. The City then splits this take between the City's coffers
and the School District of Philadelphia, with the former getting
approximately 45\% of Real Estate Tax revenues. Tax bills are mailed by DoR in batches throughout December and early
January each year; the owners have until March 31st to remit their
balance to the City, after which
time their bill begins to accrue penalties and interest.

The City begins actively pursuing non-paying properties in the
September following nonpayment.  First, the city delegates roughly
$\frac{2}{3}$ of the debts to the authority of one of the two
designated external law firms which have contracts with the City for
this purpose, with each being assigned half of the externalized
non-payers.\footnote{Currently, these law firms are Linebarger, Goggan,
  Blair \& Sampson and Goehring, Rutter \& Boehm.} These law firms
are free to pursue the collection of the debt as they see fit, and are
rewarded with a portion of any debts recovered for the City.

For those debts that remain targeted by DoR itself, the City
traditionally leverages one of several legal options as threat and
punishment for nonpayment. Beginning on March 31st, the city regularly
sends plain bills to properties still in hock--roughly once every 10
weeks.\footnote{See the section on implementation below for exact
  details.} More substantive enforcement strategies begin upon expiry
of the tax year on December 31st, when the properties are officially
delinquent.

Philadelphia operates with relatively wide latitude under the
Pennsylvania Municipal Claims and Tax Lien Law (MCTLL), enacted in
1923. The City has the right to put a lien on any property which has not yet
paid Real Estate Taxes 9 months past the March 31st deadline. A tax
lien may be imposed for delinquent taxes owed on real property as a
result of failure to pay taxes.  It is a claim for payment that takes
precedence over all other claims and gives the holder of the lien
basis for legal action, including foreclosure. Tax liens are imposed
by a taxing jurisdiction after a property becomes delinquent and
typically includes the principal tax amount, plus any interest and
penalties.

Having obtained a lien the city can technically start a foreclosure process although it may chose not to pursue foreclosure. This
is the legal process of seizing title of a property (or the deed) and
forcing its sale for the purpose of paying off a debt, such as a tax
lien. In a tax foreclosure, the local jurisdiction petitions a court
to award it the title based on an unpaid lien. The jurisdiction may
then sell the property in a tax-deed sale or auction, hoping at least
for enough to cover the tax lien. The original owner usually has the
right to regain the property if she  pays the back taxes within a
set time period after the sale, called the redemption period. If
nobody buys the deed, the tax lien remains unpaid, and the
jurisdiction keeps the title and responsibility for the
property. A Sheriff's Sale is basically a public auction
of the property, whereby all ownership rights to the property are
stripped from the former owner upon successful bid.\footnote{A list of
  properties currently up for Sheriff's Sale by the City of
  Philadelphia can be found at
  \it{http://www.officeofphiladelphiasheriff.com/en/real-estate/foreclosure-listings}}

For commercial properties, the City may also sequester income from the
business, become the property manager, or even shut down the
business. City employees may see their taxes deducted from their
salaries, and applicants for City jobs are vetted for eligibility--
new employees must either be fully repaid or have started a payment
plan. The City has also begun recently to explore several
  alternative enforcement options, such as levying liens on owners'
  home addresses outside Philadelphia (targeting delinquent
  nonresident landlords) and intermediate penalties such as impounding
  of delinquents' cars.

A recent study by the PCT (2013) concluded that ``compared to laws
governing delinquency collection in some other states and other
Pennsylvania counties, the state statutes governing Philadelphia give
city government a lot of discretion in setting policies on when to
initiate foreclosures or what kind of catch-up payment plans to
offer. In the past, Philadelphia has tended to use this discretion to
delay taking action, put up fewer properties for sale, or let
delinquents enroll and default on payment plans many times, all of
which has caused delinquencies to accumulate over the years." As a
consequence, threats of enforcement of property tax collection may
lack credibility in Philadelphia.

\section{Treatments}

To explore softer avenues for revenue take augmentation, we determined
that the most logistically feasible approach was to include a
``stuffer'' in the bills regularly mailed to non-payers, the language
of which was carefully chosen to target one of three enforcement
strategies: deterrence (the ``Threat'' treatment), social normality
(the ``Moral'' treatment), and conformity (the ``Peer'' treatment). To
properly isolate the treatment effects of receiving a specific stuffer
from the effects of receiving \textit{any} stuffer given the status
quo of plain bills, we also randomly sent properties a plain stuffer
(the ``Control'' treatment).

The letters were designed carefully to differ only in the wording of
their second paragraph; for clarity, the idiosyncratic wording is
reiterated here. We also took care to minimize communication issues by
vigorously simplifying the language of our chosen messages--shunning
uncommon words and syntactic complexity by churning the stuffers text
through the latest linguistic tools for complexity analysis to be sure
our vocabulary was accessible to any non-payers who may be of limited
literacy.

Further, in accordance with the City's general desire to reach out to
its substantial immigrant populations, we agreed with the City to
include Spanish translations of each treatment's text on the reverse
of the stuffers, given that Puerto Ricans make up the plurality of
Philadelphia's non-English-speaking population.\footnote{See Appendix
  for the exact style and full wording of each treatment, including
  the Spanish translations we ultimately chose with input from several
  native speakers}

\subsection{Treatment 1: Deterrence}

The goal of the Threat treatment is to emphasize the repercussions of
noncooperation with the city and highlight the policy tools available
to the city, with the idea that owners may have poorly-formed notions
of the extent of action the city may be willing to take to recover
taxes from each property. The stuffer contained the following sentences:

{\it Not paying your Real Estate Taxes is breaking the law. Failure to pay your Real Estate Taxes may result in
seizure or sale of your property by the City. Do not make the mistake of assuming we are too busy
to pursue your case.}

\subsection{Treatment 2: Moral Appeal}

The goal of the Moral treatment is to emphasize the social contract
between tax payers and the City--that, in exchange for compliance and
timely submission of taxes due to the City, the City provides a level
of public goods commensurate with the level of tax intake. In
particular, we chose to highlight the correspondence between Real
Estate Tax compliance and the City's provision of public education and
safety services.The stuffer contained the following sentences:

{\it We understand that paying your taxes
can feel like a burden. We want to remind you of all the great services that
you pay for with your Real Estate Tax Dollars. Your tax dollars pay for schools to teach our children.
They also pay for the police and firefighters who help keep our city safe.
Please pay your taxes as soon as you can to help us pay for these
essential services.}

\subsection{Treatment 3: Social Pressure \& Conformity}

The goal of the Peer treatment is to socially penalize delinquents
by underlining their nonconformity among their fellow Philadelphians.
The stuffer contained the following sentences:

{\it You have not paid your Real Estate Taxes. Almost all of your neighbors pay their fair share--
9 out of 10 Philadelphians do so. Paying your taxes is your duty to the city
you live in. By failing to pay, you are abusing the good will of your
Philadelphia neighbors.}

\subsection{Control}

The Control stuffer was designed to be as plain and unpersuasive as possible, 
with the goal of the informational content being orthogonal to each of the
three treatment stuffers. 

{\it The enclosed bill details your outstanding 
Real Estate Taxes due to the City of Philadelphia.}

\section{Experimental Design}

\subsection{Randomization}

Our approach to randomization was constrained by the logistics of
DoR's enforcement faculties. Ideally, we would have randomly assigned
one of the four treatment stuffers at the property level; we concluded
after several discussions with our correspondents at DoR that this
would be logistically impossible. Instead, we chose to exploit the
pseudo-random assignment of properties to billing cycles and randomized
treatments across billing cycles.  To understand this decision it is
useful to discuss the current practice of sending out letters by DoR.

Mailing of delinquent Real Estate Tax Bills by DoR works roughly as
follows.  Every property in the city is assigned to one of 50 mailing
cycles; since it is cheaper and simpler to send at once all bills to
those owners owing taxes on multiple properties, assignment to cycles
is done at the owner level, so that each mailing cycle has roughly the
same number of owners.  Every morning, a printer at DoR taps the
in-house accounting system to find all properties that a) owe taxes to
the City and b) are in the current day's mailing cycle, which
increases by one each day. After identifying the bills to be
printed for the day, the printer merges several other pieces of
information stored with the delinquent balance such as the mailing
address and an in-house ID associated with the property. The 1200 or
so bills that are printed each day are then brought to the City's
mailing room, wherein they are stuffed into envelopes and delivered to
the property owners.

Given the volume of bills printed each day and the existing
infrastructure for processing them, especially the machine-automated
process of envelope stuffing, we determined the most practical
solution would be to randomize treatment at the mailing cycle level,
so that every bill printed on the same day would be paired with the
same stuffer. Randomization of mailing days was handled by the
authors. We elected to randomize 4-day cycles--for each 4-day period,
we picked at random among the $4!=24$ possible arrangements of
treatments over the subsequent 4 days. Our experiment was conducted
during 15 days in November 2014, from the 4\textsuperscript{th}
through the 25\textsuperscript{th}. We had 4 days of treatment for the
Control, Moral, and Peer groups but only 3 days of treatment for the
Threat group.

While we are certain of the sanctity of our mailing cycle-level
randomization process, one may be concerned about the assignment of
properties to mailing cycles by the city. Fortunately, however, the city uses a
pseudo-random mechanism to assign owners to billing cycles, which means
that we achieve proper full-scale two-stage randomization of the
properties through our process of day-level randomization.

In particular, the city assigned properties to cycles based on the
last two digits of an in-house ID number; those with final two digits
01 and 02 are mapped to cycle 1, those with final digits 03 and 04 are
mapped to cycle 2, and so on. The in-house ID itself is motley in
nature. For many properties, DoR has on file the owner's Social
Security Number (SSN); for many others, mainly commercial properties,
the DoR stores their Employer Identification Number (EIN); and for the
remainder of properties, DoR assigns its own in-house ID number. This
last is a 9-digit code which is assigned sequentially to property
owners who cannot be matched to either of the federal ID
numbers. While this assignment based on SSN or EIN is not purely
random, it is a pseudo-random assignment. It is hard to believe
that there would be any significant sorting or self-selection based on
the last two digits of SSN or EIN.

\subsection{Sample Balance on Observables}

To confirm whether or not we indeed achieved randomization, we
performed a series of balance-on-observables tests. The null
hypotheses of these tests are that a given observable data moment is
identical across mailing cycles. We turn now to the results of those
tests.

Analysis of balance on observables is complicated by the random
assignment at the owner level.  Because there are some large holders of property  -- thousands of
properties owned by public entities like the City of Philadelphia, the
Philadelphia Housing Authority, and the Redevelopment Authority of
Philadelphia, hundreds owned by many others such as the University of
Pennsylvania and Drexel University -- a simple analysis of balance at
the property level is potentially skewed by these outliers. In addition,
it is not clear how to aggregate many of the property-level
characteristics to the owner level meaningfully, especially geographic
variables.

Our approach to solve this issue was to examine sample balance on the
subset of properties for which a) the owner is unique, and b) any tax exemption claimed by the property is
related to abatements for new construction.\footnote{A full list of
  exemptions can be found in the Appendix, but the most important
  non-abatement restrictions cover religious institutions,
  institutions of learning, and medical/health
  facilities.}\textsuperscript{,}
\footnote{We ran several other similar specifications, with the 
qualitative results remaining unchanged. We also ran tests on the subsample of properties for which
we could obtain the secure ID used by the City, for which the putative mapping was violated; again, the
results are qualitatively identical. See the Appendix for details.}


\begin{table}[htbp]
\caption{Tests of Sample Balance on Observables} \label{table:balance}
\centering
\begin{tabular}{rllllll}
  \hline
% latex table generated in R 3.0.2 by xtable 1.7-4 package
% Thu Feb 19 15:34:31 2015
 & Threat & Moral & Peer & Control & Test & $p$-value \\ 
  \hline
Postive balance due & 3332.12 & 3745.42 & 3532.9 & 3277.44 & ANOVA & 0.08 \\ 
  Market value ('000)$^{*}$ & 192 & 170 & 188 & 294 & $\chi^2$ & 0.2 \\ 
  Land Area (ft\textsuperscript{2})$^{*}$ & 3949.91 & 3475.54 & 3408.11 & 3732.49 & $\chi^2$ & 0.64 \\ 
  \# Rooms$^{*}$ & 0.34 & 0.36 & 0.36 & 0.34 & $\chi^2$ & 0.18 \\ 
  \# Rooms &  &  &  &  & $\chi^2$ & 0.34 \\ 
  City Council District &  &  &  &  & $\chi^2$ & 0.56 \\ 
   Category: \\ 
 \hline
Residential & 0.15 & 0.48 & 0.25 & 0.12 & $\chi^2$ & 0.47 \\ 
  Hotels\&Apts & 0.21 & 0.46 & 0.21 & 0.13 &  &  \\ 
  Store w. Dwell. & 0.25 & 0.42 & 0.21 & 0.12 &  &  \\ 
  Commercial & 0.21 & 0.45 & 0.24 & 0.1 &  &  \\ 
  Industrial & 0.21 & 0.47 & 0.23 & 0.09 &  &  \\ 
  Vacant Land & 0.25 & 0.4 & 0.22 & 0.13 &  &  \\ 
   \hline
Distribution of Properties & 0.21 & 0.45 & 0.23 & 0.11 & $\chi^2$ & 0.22 \\ 
   \hline
Expected Distribution & 0.22 & 0.44 & 0.22 & 0.11 &  &  \\ 
   \hline
%End piece imported from R-xtable
\multicolumn{7}{l}
{\scriptsize{$^{*}$Tested as a two-way $\chi^2$ test of quartile vs. treatment, due to outliers.}} \\
\multicolumn{7}{l}
{\scriptsize{$^{**}$Properties below 2 or above 12 rooms were trimmed to reduce the influence of outliers.}} \\
\multicolumn{7}{l}
{\scriptsize{$^{***}$See Appendix for full two-way tables of these variables.}}
\end{tabular}
\end{table}

As can be seen in Table \ref{table:balance}, randomization appears to
have been successful.  The properties are strongly randomly
distributed by location (their political ward, of which there are 66
in Philadelphia), category (type of property usage), property size (as
measured by the number of rooms or by the size of the tract), and case
assignment (this variable captures, if applicable, to which outside
law firm a property is assigned, whether the property is in
sequestration, or has entered a payment agreement with the city). The
number of properties assigned to each treatment is further exactly as
expected, given the unequal number of mailing days in our treatment.

\begin{figure}
\caption{}\label{fig:balance_balance}
\begin{center}
\includegraphics[width=4in]{total_balance_single_property_nonexempt}
\par\end{center}
\end{figure}

Evidence of randomization is only slightly weaker for randomization on
delinquent balance and randomization on market value, though tests of
both are far from rejecting the null hypothesis of equal group
means. We suspect this is largely due to the influence of outliers--
as seen in Figure \ref{fig:balance_balance}, the distributions are
highly similar visually
\footnote{see the Appendix for more robust examination of these variables.}.

\subsection{Implementation Fidelity}

Given the logistic difficulties in the details of implementation, we were
concerned about the experiment's fidelity. In addition to the
foreseeable difficulty of coordinating among many actors in order for
our experiment to fit in with the DoR's normal billing process, the printing office experienced some
technical failures of some of their equipment. This problem was compounded by
the fact that this happened just prior to Veterans' Day, a City
holiday, which led to a substantial backup of bill processing.

To assess the fidelity of the experimental design, we leveraged a unique
feature of the environment. The Department of Revenue regularly posts envelopes destined
for addresses that are either unattended (vacant) or do not
exist in the first place due to typos. Either before or after an
attempted delivery to such an address, the postal service flags down
these bills and returns the missives to DoR, which then process them
and attempts, if they can identify a suitable alternative address, to
re-deliver the tax bill. We took advantage of the fact that a
subset of bills made their way back to DoR to check firsthand the extent of
treatment fidelity.

\begin{table}[htbp]
\centering
\caption{Measured Fidelity of Implementation}\label{table:fidelity}
\begin{tabular}{rlllllllllllllll}
  \hline
 Day: & 1 & 2 & 3 & 4 & 5 & 6 & 7 & 8 & 9 & 10 & 11 & 12 & 13 & 14 & 15 \\ 
  \hline
Intended Stuffer & M & P & T & C & T & P & C & M & M & C & P & T & C & P & M \\ 
\% Inaccurate & 0 & 35 & 55 & 21 & 70 & 35 & 0 & 0 & 0 & 38 & 1 & 0 & 9 & 7 & 2 \\ 
Actual Stuffer & M & P & T & T & P & P & C & M & M & M & P & T & C & P & M \\
   \hline
\multicolumn{16}{l}
{\scriptsize{T: Threat, M: Moral, P: Peer, C: Control}}
\end{tabular}
\end{table}

As can be seen from Table \ref{table:fidelity}, actual implementation
was less than perfect. We retrieved 929 returned letters, 122 (13\%)
of which were found to have a different stuffer from what we
expected. On six days--the second through sixth and the tenth--more
than ten percent of returned letters were mismatched.
\footnote{See Appendix for more details about the sample of returned letters}
A small number (8) of the letters were found to be completely empty--no
stuffer whatsoever was packaged with the posted bill. 

Given this substantial deviation from intended treatment, we present
our main empirical results only for the subsample of treatment days
for which greater than 90\% fidelity was achieved--that is, days 1,
7-9, and 12-15. Table 3 summarizes the number of days for each
treatment after we eliminate those observations from the sample for
which fidelity could not be established.\footnote{For robustness
  purposes, we provide intent-to-treat (ITT) and instrumented analyses
  in the Appendix.}

\begin{table}[htbp]
\caption{Number of Days of Different Treatments}
\begin{center}
\begin{tabular}{|c|c|c|c|c|}
\hline 
Treatment & THREAT & MORAL & PEER & CONTROL\tabularnewline
\hline 
Count & 1 & 4 & 2 & 2\tabularnewline
\hline 
\end{tabular}
\par\end{center}
\end{table}

\section{Methods}

To analyze the causal effects of our treatments, we consider two types of outcomes.
First, we examine the
effect of treatment on two binary outcomes--namely, one indicator for
whether a given account submitted some payment to DoR in the sample
period and another for whether the account at a property was cleared
completely. Second, we attempt to detect if treatment causally
impacted the dollar value of payments submitted to the city or the
relative value of payments submitted to the city.  

\subsection{Logistic Regressions}

To analyze discrete outcomes we use logistic regressions.
Let $y_{i}=\mathbbm{1}\left[x_i>0\right]$, where
$\mathbbm{1}\left[\cdot\right]$ is an indicator taking the value one
when its argument inside the brackets is true and 0 otherwise. $x_i$ is the
cumulative remittance to the city at the conclusion of the sample
period by property $i$. Given the random assignment of treatments
we can obtain a consistent estimator of the causal impact of treatment on
$y_I$ by using the following logistic regression model:
\begin{equation}
y_{i}=X_i^T\beta +D_{T,i}\gamma_{T}+D_{M,i}\gamma_{M}+D_{P,i}\gamma_{P}
+\epsilon_{i},\hspace{1em}\epsilon \enskip\mbox{logistic}
\end{equation}
The $D_{k,i}$ are indicators for the three treatments, i.e.
$D_{k,i}=\mathbbm{1}\left[treatment_i=k\right]$, $k\in{T,M,P}$ for
Threat, Moral, and Peer, respectively. The coefficients $\gamma_{k}$, then, measure
the causal impacts of the treatments on the likelihood of some degree
of remittance to the city, relative to the control treatment of a
plain stuffer. 

To improve efficiency we also include some controls such as (log) land area, maturity of debt (more
or less than 5 years), geographic location (as proxied by City Council
District), usage category, property exterior condition (whether or not
the property was categorized as sealed/compromised by the city),
whether the property took a homestead exemption\footnote{Philadelphia
  offers a tax discount of \$30,000 off the taxable value of the
  property for those residents who are verifiably owner-occupants of a
  property--thus not taking a homestead exemption is a signal that a
  property is operated by a remote landlord.}, (log) balance at
mailing, and (log) market value.

Similarly, we examine a more restrictive yes-no participation
outcome--namely, whether or not the property offered not just token
repayment in our sample period, but whether its debts were paid back
in full.\footnote{Due to some measurement issues, it is not possible to
  track on a day-to-day basis exactly the balance due for each
  property-- accrual of interest and other charges is hard to pinpoint
  exactly. In the main results below, we actually measure full
  repayment as submission of at least 95\% of the balance due; see the
  Appendix for some exploration of the robustness of this threshold,
  but suffice it so say here that results are qualitatively
  identical.} To this end, let $y_{i}=\mathbbm{1}\left[x_i=m_i\right]$ be an
indicator for full repayment of debt in the sample timeframe where $m_i$ is
the amount owed by property $i$ at the start of treatment. 

\subsection{Tobit Regressions}

Next, we turn our attention to measuring effects of treatment on the
magnitude of debt drawdown--that is, instead of answering questions to
the effect of ``Did our treatment(s) significantly induce
repayment?'', we address the intermediate question of ``Did
our treatment(s) induce significantly more repayment?''. In particular, we try to isolate the relative magnitude of repayment
by essentially normalizing the dollar amount repaid by the ``size'' of
the property, so that bigger payments by ``bigger'' properties in some
group don't mechanically skew our results in favor of that group.

In our first Tobit Model, our notion of ``size'' is the market value
of the property. The estimates of this model will highlight any
significant differences in the amount repaid relative to market value.

Specifically, we estimate the following Tobit model:
\begin{eqnarray}
\log x_{i}^{*} & = & X^T_i \beta +\delta \log v_{i}+
\sum_{k\in\{T,M,P\}}D_{k,i}\left(\gamma_{k}+\log v_{i}\eta_{k}\right)+u_{i}\\
x_{i} & = & \mathbbm{1}\left[x_{i}^{*}>=0\right]x_{i}^{*} \nonumber
\end{eqnarray}
where  $x_{i}^{*}$ is the latent repayment amount, whose observed
counterpart $x_i$ only takes nonzero value when $x_i^{*}$ is strictly
positive. By controlling for (log) market value $v_i$, the
$\eta_{k}$ represent the group-specific additional repayment for
each (log) market value. Thus a positive coefficient on
$D_{k,i}\times\log v_i$ has the interpretation that higher-value
properties were more likely to pay their taxes.

Our second pair of Tobit regressions entails a different (but
correlated) measure of the ``size'' of a property--namely, the amount
owed by the property at the onset of treatment\footnote{Though, on
  each year's initial tax bill, the market value is exactly
  proportional to the property's tax debt (modulo any exemptions
  claimed, but these properties are excluded from our analysis), the
  history of payments in general stands to create a wedge in this
  proportional relationship--low-value properties that have a long
  history of nonpayment or high value properties that have already
  partially repaid their debts drive dissonance in either direction.}.
Here, we are primarily concerned with whether those properties that
owed more to begin with ended up remitting more to DoR in our sample
timeframe.

\section{Data}

Starting with the original sample of 134,888 delinquent properties, we
obtained our final sample by using the following screening devices:
\begin{enumerate}
\item Payment agreement (23\%=31456)
\item Any tax abatement (5\% = 4706)
\item Not handled by DoR (62\%=61170)
\item Sheriff's Sale (11\%=4098)
\item Bankruptcy (3\%=948)
\item Sequestration (3\%=1130)
\item Returned mail flag (5\%=1429)
\item Not mailed during treatment between Nov. 4 \& Nov. 24 (83\%=24800)
\end{enumerate}
Our final sample thus consisted of 5151 properties. Table 4 provides
some descriptive statistics for our sample by treatment and control
groups.


\begin{table}[htbp] 
\caption{Descriptive Statistics of Housing Characteristics}
\centering 
\begin{tabular}{rrrrrr}
\hline
                 & Full & Threat & Moral & Peer & Control \\
\hline
Years: [0,2]          & 0.66 & 0.63 & 0.64 & 0.69 & 0.69 \\
Years: (2,5]          & 0.17 & 0.16 & 0.18 & 0.16 & 0.16 \\
Years: (5,10]         & 0.09 & 0.12 & 0.09 & 0.08 & 0.08 \\
Years: (10,20]        & 0.06 & 0.06 & 0.06 & 0.05 & 0.05 \\ 
Years: (20,40]        & 0.02 & 0.03 & 0.02 & 0.02 & 0.02 \\   
Cat: Commercial       & 0.04 & 0.04 & 0.04 & 0.04 & 0.02 \\    
Cat: Hotels/Apts      & 0.08 & 0.10 & 0.09 & 0.08 & 0.08 \\   
Cat: Industrial       & 0.01 & 0.02 & 0.01 & 0.01 & 0.02 \\    
Cat: Residential      & 0.69 & 0.65 & 0.70 & 0.70 & 0.70 \\   
Cat: Store+Resid      & 0.07 & 0.06 & 0.07 & 0.07 & 0.06 \\    
Cat: Vacant           & 0.10 & 0.14 & 0.09 & 0.11 & 0.12 \\    
\% Sealed/Compromised & 0.03 & 0.02 & 0.04 & 0.02 & 0.03 \\    
\% Homestead          & 0.23 & 0.19 & 0.23 & 0.24 & 0.22 \\     
\hline 
\end{tabular} 
\end{table}

\newpage


\section{Empirical Results (Preliminary) }

We present some preliminary empirical results as the outcomes of
estimation of the aforementioned models. We also report some plots of time series
demonstrating the evolution of key outcomes by treatment throughout the sample
period.


\begin{table}[htbp]
\caption{Model I: Logistic Regressions -- Ever Paid} \label{table:modelI}
\begin{center}
\begin{tabular}{l c c }
\hline
                       & Ever Paid & Ever Paid (with Controls) \\
\hline
Intercept              & $-1.74^{***}$ & $-3.35^{***}$ \\
                       & $(0.08)$      & $(0.74)$      \\
Treatment Group        &               &               \\
                       &               &               \\
\quad Moral            & $-0.06$       & $-0.07$       \\
                       & $(0.10)$      & $(0.10)$      \\
\quad Peer             & $0.22^{*}$        & $0.18$        \\
                       & $(0.11)$      & $(0.12)$      \\
\quad Threat           & $-0.10$       & $-0.07$       \\
                       & $(0.15)$      & $(0.15)$      \\
Log Balance at Mailing &               & $-0.00$       \\
                       &               & $(0.01)$      \\
Log Market Value       &               & $0.09$        \\
                       &               & $(0.06)$      \\
\hline
% AIC                    & 4343.10       & 4225.54       \\
% BIC                    & 4369.25       & 4382.46       \\
Log Likelihood         & -2167.55      & -2088.77      \\
% Deviance               & 4335.10       & 4177.54       \\
Num. obs.              & 5105          & 5105          \\
\hline
\multicolumn{3}{l}{\scriptsize{$^{***}p<0.001$, $^{**}p<0.01$, $^*p<0.05$. Control coefficients omitted for brevity; see Appendix}}
\end{tabular}
\end{center}
\end{table}

Table \ref{table:modelI} reports the results from logit regressions using
ever paid as the outcome variable. As can be seen from Table \ref{table:modelI}, the moral and the threat
treatments had a strongly significant effect at the conclusion of the sample
period. The Peer treatment is significant ($p=.056$) in the baseline model, but
this significance disappears upon inclusion of control variables
($p=.124$).

\begin{figure}[htbp]
\caption{}\label{ever_paid_act}
\begin{center}
\includegraphics[width=4in]{time_series_pct_ever_paid_act}
\par\end{center}
\end{figure}

The end-of-sample insignificance contrasts with a sustained outpacing
of the rise in repayment participation since about two weeks
subsequent to mailing.  This can be seen in Figure
\ref{ever_paid_act}.

\begin{table}[htbp]
\caption{Model II: Logistic Regressions -- Paid Full} \label{table:modelII}
\begin{center}
\begin{tabular}{l c c }
\hline
                       & Paid in Full & Paid in Full (with Controls) \\
\hline
Intercept              & $-2.28^{***}$ & $-3.04^{**}$  \\
                       & $(0.10)$      & $(1.03)$      \\
Treatment Group        &               &               \\
                       &               &               \\
\quad Moral            & $-0.41^{**}$  & $-0.40^{**}$  \\
                       & $(0.13)$      & $(0.14)$      \\
\quad Peer             & $0.25$        & $0.21$        \\
                       & $(0.14)$      & $(0.14)$      \\
\quad Threat           & $-0.22$       & $-0.19$       \\
                       & $(0.19)$      & $(0.20)$      \\
Log Balance at Mailing &               & $-0.11^{***}$ \\
                       &               & $(0.02)$      \\
Log Market Value       &               & $-0.02$       \\
                       &               & $(0.08)$      \\
\hline
% AIC                    & 2915.19       & 2748.35       \\
% BIC                    & 2941.34       & 2905.26       \\
Log Likelihood         & -1453.60      & -1350.17      \\
% Deviance               & 2907.19       & 2700.35       \\
Num. obs.              & 5105          & 5105          \\
\hline
\multicolumn{3}{l}{\scriptsize{$^{***}p<0.001$, $^{**}p<0.01$, $^*p<0.05$. Control coefficients omitted for brevity; see Appendix}}
\end{tabular}
\end{center}
\end{table}

Table \ref{table:modelII} presents the results of the second model,
which examined the likelihood of full repayment of debts by treatment
group. As with Model I, the Peer treatment shows signs from about two
weeks past treatment onset of outpacing the control group; as seen in
Figure \ref{paid_full_act}, the trajectory of full repayment tracks
that of partial repayment for the Peer group quite well.

\begin{figure}[htbp]
\begin{center}
\caption{} \label{paid_full_act}
\includegraphics[width=4in]{time_series_pct_paid_full_act}
\par\end{center}
\end{figure}

Further, the Moral treatment caused a significant reduction in full
repayment. Figure \ref{paid_full_act} suggests full repayment started
flagging about three weeks after treatment after tracking the control
group quite well thereto. The Threat treatment is all but
indistinguishable from the control.

\begin{table}[htbp]
\begin{center}
\caption{Model III: Tobit of Repayment vs. Market Value} \label{table:modelIII}
\begin{tabular}{l c c }
\hline
                       & Paid vs. Market Value & Paid vs. Market Value (with Controls) \\
\hline
Intercept              & $-12348.07^{***}$ & $-11096.96^{***}$ \\
                       & $(2318.61)$       & $(2883.17)$       \\
Treatment Group        &                   &                   \\
                       &                   &                   \\
\quad Moral            & $-7128.53^{*}$    & $-7441.16^{*}$    \\
                       & $(2893.30)$       & $(2985.93)$       \\
\quad Peer             & $1398.56$         & $806.87$          \\
                       & $(3150.31)$       & $(3223.92)$       \\
\quad Threat           & $-1364.18$        & $-1456.06$        \\
                       & $(3998.04)$       & $(4110.60)$       \\
Log Market Value       & $563.40^{**}$     & $404.39$          \\
                       & $(200.81)$        & $(253.71)$        \\
Moral*MV               & $624.16^{*}$      & $653.08^{*}$      \\
                       & $(251.37)$        & $(259.30)$        \\
Peer*MV                & $-79.53$          & $-32.17$          \\
                       & $(274.75)$        & $(281.09)$        \\
Threat*MV              & $106.23$          & $114.47$          \\
                       & $(348.01)$        & $(357.51)$        \\
Log Balance at Mailing &                   & $61.90$           \\
                       &                   & $(45.97)$         \\
\hline
% AIC                    & 17746.27          & 17729.78          \\
% BIC                    & 17805.12          & 17912.85          \\
Log Likelihood         & -8864.14          & -8836.89          \\
% Deviance               & 3088.60           & 3027.81           \\
% Total                  & 5105              & 5105              \\
Left-censored          & 4331              & 4331              \\
Uncensored             & 774               & 774               \\
% Right-censored         & 0                 & 0                 \\
% Wald Test              & 83.35             & 125.02            \\
\hline
\multicolumn{3}{l}{\scriptsize{$^{***}p<0.001$, $^{**}p<0.01$, $^*p<0.05$. Control coefficients omitted for brevity}}
\end{tabular}
\end{center}
\end{table}

The results of the first Tobit model are displayed in Table
\ref{table:modelIII}.  Only about 15\% of observations were
uncensored. Results of Model III suggest that those owing the most in
the Moral treatment were significantly more likely to pay more than
their control counterparts. This is confirmed in Figure
\ref{repay_quart_act}, which depicts the trajectories of (normalized)
repayments by each group.

\begin{figure}[htbp]
\begin{center}
\caption{}\label{repay_quart_act}
\includegraphics[width=4in]{time_series_average_payments_act}
\par\end{center}
\end{figure}

\begin{table}
\begin{center}
\caption{Model IV: Tobit of Repayment vs. Total Debt}
\label{table:modelIV}
\begin{tabular}{l c c }
\hline
                       & Paid vs. Total Debt & Paid vs. Total Debt (with Controls) \\
\hline
Intercept              & $-5478.88^{***}$ & $-13650.77^{***}$ \\
                       & $(598.31)$       & $(2074.35)$       \\
Treatment Group        &                  &                   \\
                       &                  &                   \\
\quad Moral            & $-1505.26^{*}$   & $-1669.69^{*}$    \\
                       & $(752.51)$       & $(748.86)$        \\
\quad Peer             & $779.26$         & $773.16$          \\
                       & $(785.66)$       & $(778.14)$        \\
\quad Threat           & $-629.38$        & $-579.61$         \\
                       & $(1031.67)$      & $(1029.50)$       \\
Log Balance at Mailing & $-107.27$        & $-37.85$          \\
                       & $(87.24)$        & $(87.50)$         \\
Moral*Balance          & $249.08^{*}$     & $272.72^{*}$      \\
                       & $(110.60)$       & $(110.23)$        \\
Peer*Balance           & $-44.45$         & $-58.25$          \\
                       & $(119.07)$       & $(118.16)$        \\
Threat*Balance         & $79.00$          & $70.76$           \\
                       & $(151.02)$       & $(151.20)$        \\
Log Market Value       &                  & $687.78^{***}$    \\
                       &                  & $(175.00)$        \\
\hline
% AIC                    & 17820.05         & 17728.50          \\
% BIC                    & 17878.89         & 17911.57          \\
Log Likelihood         & -8901.02         & -8836.25          \\
% Deviance               & 3119.44          & 3022.06           \\
% Total                  & 5105             & 5105              \\
Left-censored          & 4331             & 4331              \\
Uncensored             & 774              & 774               \\
% Right-censored         & 0                & 0                 \\
% Wald Test              & 13.48            & 126.19            \\
\hline
\multicolumn{3}{l}{\scriptsize{$^{***}p<0.001$, $^{**}p<0.01$, $^*p<0.05$. Control coefficients omitted for brevity; see Appendix}}
\end{tabular}
\end{center}
\end{table}

We move finally to the results from the final model, which focuses not
on payments relative to property value, but instead on payments
relative to the balanced owed by each account at mailing day. The
results of this model are qualitatively identical to those of Model
III, which is further mirrored in the time series results of Figure
\ref{drawdown_quart_act}.

\begin{figure}[htbp]
\begin{center}
\caption{}\label{drawdown_quart_act}
\includegraphics[width=4in]{time_series_debt_paydown_act}
\par\end{center}
\end{figure}

\newpage

\section{Conclusions}

This experimental study has evaluated a set of
strategies intended to increase property tax compliance. We have developed  a field experiment to test three of the most commonly
suggested hypotheses that may explain tax compliance: deterrence, moral appeal, and peer conformity.
We have implemented the experiment in collaboration with the the Department of Revenue (DoR) of the City
of Philadelphia. Our preliminary findings provide some evidence that both moral appeal
and peer conformity may help to help to improve tax compliance. We find
little evidence that supports the standard deterrence model.

Our study provides ample scope for future research.  It is probably not surprising that our deterrence treatment
was not effective. Philadelphia is city with a history of low tax moral. it is hard to belief that one could significantly alter subjective
beliefs of punishment by sending one letter. Any threats may be considered to be empty given previous attempts of 
the law enforcement. It would be more interesting to design an intervention that is based on a more credible threat.
The peer conformity treatment is also subject to the same potential criticism. It my be more effective to randomly print 
a number of delinquent tax payers in the local news paper or to send mail informing the neighbors. Of course, these type of intervention face much larger legal hurdles and, therefore, much harder to implement.




\newpage

\section*{References}

Ariel, Barak (2012) Deterrence and moral persuasion effects on corporate tax compliance: Findings from a
randomized controlled trial. Criminology, 50 (1), 27-69. \\
\\
Allingham, Michael G., and Agnar Sandmo (1972) ``Income Tax Evasion: A Theoretical
Analysis," Journal of Public Economics, 1: 323-38. \\
\\
Alm, James, Gary H. McClelland, and William D. Schulze (1992) ``Why Do People
Pay Taxes?" Journal of Public Economics 48: 21-38. \\
\\
Alm, James (1999) Tax compliance and administration. In: Hildreth, W. Bartley and James A. Richardson
(eds.) Handbook on Taxation. New York, USA, Marcel Dekker, Inc., pp. 741-768. \\
\\
Andreoni, James, Erard, Brian and Jonathan Feinstein (1998) Tax compliance. Journal of Economic
Literature, 36, 818-860. \\
\\
Becker, Gary S. (1968) ``Crime and Punishment: An Economic Approach,"
Journal of Political Economy 76: 169-217.\\
\\
Blumenthal, Marsha, Christian, Charles and Joel Slemrod (2001) Do normative appeals affect tax
compliance? Evidence from a controlled experiment in Minnesota. National Tax Journal, 54 (1),
125 - 138. \\
\\
Cowell, Frank A. and James P. F. Gordon (1988) Unwillingness to pay tax: tax evasion and public provision.
Journal of Public Economics, 36, 305-321.\\
\\
Fehr, Ernst and Simon Gachter (1998) Reciprocity and economics: The economic implications of homo
reciprocans. European Economic Review 42 (3-5), 845-59. \\
\\
Frey, Bruno S., and Lars P. Feld (2002),  ``Deterrence and Morale in Taxation: An
Empirical Analysis." CESifo Working Paper no. 760, August 2002. \\
\\
Hallsworth, M., J. List, R. Metcalfe and I Vlaev (2014), "The Behavioralist as Tax Collector,"
Using Natural Field Experiments to Enhance Tax Compliance, " NBER Working Paper 20007. \\
\\
Harrison, Glenn W. and John A. List (2004) Field Experiments. Journal of Economic Literature, 42 (4),
1009-1055. \\
\\
Kleven, Henrik J., Knudsen, Martin B., Kreiner, Claus T., Pedersen, Soren and Emmanuel Saez (2011)
Unwilling or Unable to Cheat? Evidence From a Tax Audit Experiment in Denmark.
Econometrica, 79 (3), 651-692. \\
\\
Pew Charitable Trust (2013), ``Delinquen Property Tax in Philadelphia." Technical Report. \\
\\
Pomeranz, Dina (2013) No taxation without information: Deterrence and self-enforcement in the Value
Added Tax. Harvard Business School Working Paper. \\
\\
Reckers, Philip M. J., Sanders, Debra L. and Stephen J. Roark (1994) The influence of ethical attitudes on
taxpayer compliance. National Tax Journal, 47 (4), 825-836. \\
\\
Slemrod, Joel (2007) Cheating ourselves: The economics of tax evasion. Journal of Economic
Perspectives, 21 (1), 25-48. \\
\\
Torgler, Benno (2002) Moral-suasion: An alternative tax policy strategy? Evidence from a controlled field
experiment in Switzerland. Economics of Governance 5 (3), 235-253. \\
\\
Torgler, Benno (2012) A field experiment on moral-suasion and tax compliance focusing on under-declaration
and over-deduction. QUT School of Economics and Finance Working Paper no. 285. \\
\\
Wenzel, Michael and Natalie Taylor (2004) An experimental evaluation of tax-reporting schedules: a case of
evidence-based tax administration. Journal of Public Economics, 88 (12), 2785-2799.

\newpage

\includegraphics[width=6in]{flyer_options_141104_treat1.pdf}
\newpage
\includegraphics[width=6in]{flyer_options_141104_treat2.pdf}
\newpage
\includegraphics[width=6in]{flyer_options_141104_treat3.pdf}
\newpage
\includegraphics[width=6in]{flyer_options_141104_treat4.pdf}

\end{document}


