% Preview source code

%% LyX 2.0.6 created this file.  For more info, see http://www.lyx.org/.
%% Do not edit unless you really know what you are doing.
\documentclass[english]{article}
\usepackage[T1]{fontenc}
\usepackage[latin9]{inputenc}
\usepackage{geometry}
\geometry{verbose,lmargin=2.54cm,rmargin=2.54cm}
\usepackage{babel}
\usepackage{amsmath}
\usepackage{graphicx}
\usepackage[unicode=true]
 {hyperref}

\makeatletter
%%%%%%%%%%%%%%%%%%%%%%%%%%%%%% Textclass specific LaTeX commands.
\numberwithin{figure}{section}

%%%%%%%%%%%%%%%%%%%%%%%%%%%%%% User specified LaTeX commands.
\usepackage[T1]{fontenc}
\usepackage[latin9]{inputenc}
\usepackage{color}
\usepackage{amsmath}
\usepackage{amssymb}
\usepackage{babel}
\usepackage{bbm}
\usepackage{dsfont}
\usepackage[]{natbib}
%\bibliographystyle{econometrica}
%\bibliographystyle{/home/michael/Dropbox/Papers/chicago}
%\bibliographystyle{/home/michael/Dropbox/Papers/ecta}
\bibliographystyle{/home/michael/Dropbox/Papers/aer}
\renewcommand{\cite}{\citet}
\renewcommand\[{\begin{equation}}
\renewcommand\]{\end{equation}}
\DeclareMathOperator*{\argmin}{arg\,min}
\DeclareMathOperator*{\argmax}{arg\,max}
\DeclareMathOperator*{\argMax}{arg\,Max}
\DeclareMathOperator*{\Max}{\,Max}

\makeatother

\begin{document}

\title{Real Estate Tax Collection in Philadelphia\\
Preliminary Analysis}


\author{Michael Chirico, Charles Loeffler, Robert Inman, John MacDonald,
Holger Sieg}

\maketitle

\part*{Preliminary Highlights}
\begin{itemize}
\item Roughly \$36 million in property taxes is owed by properties which,
\textit{prima facie}, should be out of hock%
\footnote{By this, we mean that the predicted probability of delinquency is
quite low--8.14\% or less (only 1\% of properties had a predicted
probability of delinquency so far off)--but the property is behind
on its taxes nonetheless%
}--this is roughly 8 percent of the total delinquency bill in the sample
we have thus far.
\item The strongest independent predictors of delinquency are, perhaps not
surprisingly, property condition and assessed value. Other things
(including property vlaue) being equal, vacant properties are 15\%
and sealed properties 24\% more likely to owe taxes than properties
with no exterior (e.g. apartments); even simply having the exterior
rated as ``below average'' is associated with an 8\% increase in
delinquency incidence. As to property value, keeping other things
equal, we would predict a 9\% increase in delinquency likelihood when
changing a property from the first quartile (\$60,500) to the third
quartile (\$172,600).
\item The city GMA owing the most in taxes is West Philadelphia (GMA-A)--over
\$83 million--but, relative to the value of property, North Philadelphia
West (north of Poplar, south of Lincoln Highway west of Broad, GMA-H)
owes the most (3 cents of tax owed for every dollars' worth of real
estate). This is compared to the angels--GMA-D (Rhawnhurst-Fox Chase-Burholme)
owes only \$34 million and GMA-P (center city) owes only .07 cents
(7/10000 of a dollar) per dollar of real estate (understandable given
housing prices). Certain city council districts also seem to contain
disproportionate likelihood of tax delinquency.
\item Over \$76 million is owed by taxpayers claiming the homestead exemption,
\textit{but} owner-occupied tracts are predicted to be delinquent
less often (by about 7\%).
\end{itemize}

\part*{Data}

We pulled the data offered publicly by the blog \href{http://philadelinquency.com}{philadelinquency.com}
(found \href{http://www.philadelinquency.com/FULL\%20OPA\%20AND\%20REVENUE\%20DUMP\%20APR2014\%20(REVISED).zip}{here}),
which contains a point-in-time snapshot of the city's records from
three sources: the OPA, Revenue, and Valuations. The data appears
to be current to around March 6th, 2014 (this is the most recent date
of sale for any property in the OPA database).


\subsection*{Sample Restrictions}
\begin{enumerate}
\item Use residential properties only. Specifically, we use properties with
an ``R'' zoning code OR listed as mixed commercial-residential or
mixed industrial-residential.
\item Use properties with positive market value only (some 27 properties
had an assessed value of \$0; 22 of these had had positive assessments
in prior years, so these numbers were used instead).
\item Use properties identified with their GMA--not all properties had their
GMA listed in the OPA records, which is essential for identifying
neighborhood level effects.
\end{enumerate}

\subsection*{Descriptive Statistics}

\begin{table}[ht]
\centering
\begin{tabular}{rrrr}
  \hline
 & All Properties & Delinquent Properties & Nondelinquent Properties \\ 
  \hline
  Delinquent & 0.163 & 1.000 & 0.000 \\ 
  Owe $>$1 Year & 0.145 & 0.891 & 0.000 \\ 
  Exterior: N/A & 0.070 & 0.197 & 0.045 \\ 
  Exterior: New/Remodeled & 0.032 & 0.010 & 0.036 \\ 
  Exterior: Above Average & 0.032 & 0.009 & 0.036 \\ 
  Exterior: Average & 0.803 & 0.611 & 0.840 \\ 
  Exterior: Below Average & 0.041 & 0.094 & 0.030 \\ 
  Exterior: Vacant & 0.007 & 0.019 & 0.005 \\ 
  Exterior: Sealed & 0.016 & 0.060 & 0.007 \\ 
  Owner-Occupied & 0.419 & 0.192 & 0.463 \\ 
  Philly Mailing Address & 0.892 & 0.894 & 0.891 \\ 
   \hline
\end{tabular}
\end{table}

\begin{center}
\includegraphics[scale=0.3]{gmas_by_population}\includegraphics[scale=0.4]{council_districts_by_n_properties}
\par\end{center}

\begin{center}
\includegraphics[scale=0.5]{log_market_values}\includegraphics[scale=0.5]{log_delinquency_levels}
\par\end{center}


\section*{Regression Tables from Chosen Model}

For simplicity, for now we've used a linear probability model of the
binary delinquency variable (defined to be 1 when the OPA records
show a nonzero value in the delinquent field). This was regressed
on log market value, dummy variables for GMA zone (lettered A through
P, excluding I and O), exterior condition (coded 0 and 2-7), owner
occupancy, having a philadelphia mailing address, and city council
district (districts 1-10). The omitted GMA was A, the omitted exterior
code 0, and the omitted council district was 1 (which is roughly Old
City, Fish Town, and Northern Liberties).

Standard errors are sloppy--standard linear regression errors were
assumed.

\begin{table}
\begin{center}
\begin{tabular}{l c c c }
\hline
                             & LPM & LPM (Robust SE) & Probit \\
\hline
Intercept                    & $1.152 \; (0.009)^{***}$  & $1.152 \; (0.009)^{***}$  & $3.060 \; (0.045)^{***}$  \\
Log Market Value (\$)        & $-0.083 \; (0.001)^{***}$ & $-0.083 \; (0.001)^{***}$ & $-0.364 \; (0.004)^{***}$ \\
GMA B                        & $-0.013 \; (0.003)^{***}$ & $-0.013 \; (0.004)^{**}$  & $-0.049 \; (0.014)^{***}$ \\
GMA C                        & $-0.044 \; (0.005)^{***}$ & $-0.044 \; (0.005)^{***}$ & $-0.456 \; (0.031)^{***}$ \\
GMA D                        & $-0.052 \; (0.005)^{***}$ & $-0.052 \; (0.005)^{***}$ & $-0.469 \; (0.032)^{***}$ \\
GMA E                        & $-0.070 \; (0.004)^{***}$ & $-0.070 \; (0.004)^{***}$ & $-0.477 \; (0.021)^{***}$ \\
GMA F                        & $-0.035 \; (0.004)^{***}$ & $-0.035 \; (0.004)^{***}$ & $-0.234 \; (0.021)^{***}$ \\
GMA G                        & $0.049 \; (0.004)^{***}$  & $0.049 \; (0.005)^{***}$  & $0.025 \; (0.020)$        \\
GMA H                        & $0.056 \; (0.004)^{***}$  & $0.056 \; (0.004)^{***}$  & $0.020 \; (0.017)$        \\
GMA J                        & $-0.039 \; (0.005)^{***}$ & $-0.039 \; (0.005)^{***}$ & $-0.330 \; (0.024)^{***}$ \\
GMA K                        & $-0.009 \; (0.004)^{*}$   & $-0.009 \; (0.005)$       & $-0.017 \; (0.019)$       \\
GMA L                        & $-0.026 \; (0.005)^{***}$ & $-0.026 \; (0.005)^{***}$ & $-0.168 \; (0.022)^{***}$ \\
GMA M                        & $-0.009 \; (0.004)^{*}$   & $-0.009 \; (0.004)^{*}$   & $-0.063 \; (0.019)^{**}$  \\
GMA N                        & $-0.070 \; (0.003)^{***}$ & $-0.070 \; (0.003)^{***}$ & $-0.530 \; (0.022)^{***}$ \\
GMA P                        & $-0.024 \; (0.004)^{***}$ & $-0.024 \; (0.004)^{***}$ & $-0.444 \; (0.023)^{***}$ \\
Exterior: New/Remodeled      & $-0.076 \; (0.004)^{***}$ & $-0.076 \; (0.004)^{***}$ & $-0.151 \; (0.022)^{***}$ \\
Exterior: Above Average      & $-0.059 \; (0.004)^{***}$ & $-0.059 \; (0.004)^{***}$ & $-0.083 \; (0.023)^{***}$ \\
Exterior: Average            & $-0.059 \; (0.003)^{***}$ & $-0.059 \; (0.003)^{***}$ & $0.107 \; (0.012)^{***}$  \\
Exterior: Below Average      & $0.077 \; (0.003)^{***}$  & $0.077 \; (0.004)^{***}$  & $0.506 \; (0.014)^{***}$  \\
Exterior: Vacant             & $0.147 \; (0.006)^{***}$  & $0.147 \; (0.009)^{***}$  & $0.661 \; (0.024)^{***}$  \\
Exterior: Sealed             & $0.242 \; (0.004)^{***}$  & $0.242 \; (0.006)^{***}$  & $0.832 \; (0.017)^{***}$  \\
Owner Occupied               & $-0.072 \; (0.001)^{***}$ & $-0.072 \; (0.001)^{***}$ & $-0.398 \; (0.006)^{***}$ \\
Philadelphia Mailing Address & $0.046 \; (0.002)^{***}$  & $0.046 \; (0.002)^{***}$  & $0.188 \; (0.008)^{***}$  \\
City Council District 2      & $0.006 \; (0.004)$        & $0.006 \; (0.003)$        & $-0.081 \; (0.024)^{***}$ \\
City Council District 3      & $0.068 \; (0.004)^{***}$  & $0.068 \; (0.004)^{***}$  & $0.144 \; (0.023)^{***}$  \\
City Council District 4      & $0.029 \; (0.004)^{***}$  & $0.029 \; (0.004)^{***}$  & $0.047 \; (0.021)^{*}$    \\
City Council District 5      & $0.009 \; (0.003)^{**}$   & $0.009 \; (0.003)^{**}$   & $0.028 \; (0.017)$        \\
City Council District 6      & $-0.014 \; (0.004)^{***}$ & $-0.014 \; (0.003)^{***}$ & $-0.147 \; (0.020)^{***}$ \\
City Council District 7      & $-0.006 \; (0.003)$       & $-0.006 \; (0.003)$       & $0.001 \; (0.015)$        \\
City Council District 8      & $0.076 \; (0.004)^{***}$  & $0.076 \; (0.004)^{***}$  & $0.252 \; (0.019)^{***}$  \\
City Council District 9      & $0.003 \; (0.004)$        & $0.003 \; (0.003)$        & $-0.018 \; (0.020)$       \\
City Council District 10     & $-0.006 \; (0.005)$       & $-0.006 \; (0.004)$       & $-0.207 \; (0.029)^{***}$ \\
\hline
R$^2$                        & 0.182                     & 0.182                     &                           \\
Adj. R$^2$                   & 0.182                     & 0.182                     &                           \\
Num. obs.                    & 511513                    & 511513                    & 511513                    \\
AIC                          &                           &                           & 366504.212                \\
BIC                          &                           &                           & 366860.856                \\
Log Likelihood               &                           &                           & -183220.106               \\
Deviance                     &                           &                           & 366440.212                \\
\hline
\multicolumn{4}{l}{\scriptsize{$^{***}p<0.001$, $^{**}p<0.01$, $^*p<0.05$}}
\end{tabular}
\caption{Modelling (Binary) Delinquency vs. Property Characteristics}
\label{table:coefficients}
\end{center}
\end{table}

\newpage{}


\subsection*{Good Candidates for Collection}

These plots depict histograms of the predicted probabilities of delinquency
for houses that are actually delinquent. That there are actually delinquent
houses with low predicted probabilities of delinquency means that
there are many properties which are delinquent but don't ``fit the
mold'' of the other delinquent properties--as such, there should
be a good chance of collecting taxes due from these properties.

\begin{center}
\includegraphics[scale=0.5]{delinquency_probability_fitted_for_delinquents_lpm}\includegraphics[scale=0.5]{delinquency_probability_fitted_for_delinquents_probit}
\par\end{center}
\end{document}
