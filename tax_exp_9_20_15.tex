\documentclass[12pt,titlepage]{article}

\renewcommand\baselinestretch{1.5}
\setlength{\parskip}{0.08in}
\setlength{\medskipamount}{0.05in}
\textheight 8.0 in
\textwidth 6.0 in
\topmargin 0.25in
\tolerance=11000

\usepackage[utf8]{inputenc}
\usepackage{longtable}
\usepackage{bbm}
\usepackage{rotating}
\usepackage[pdfencoding=auto,unicode=true]{hyperref}
\usepackage{graphicx}
\usepackage[outdir=./]{epstopdf}
\graphicspath{{images/analysis/act/}{images/}{images/balance/}}
\renewcommand{\thefootnote}{\fnsymbol{footnote}}

\begin{document}

\title{An Experimental Evaluation of Notification Strategies to
  Increase Property Tax Compliance: \\ Free-Riding in the City of
  Brotherly Love} \author{Michael Chirico, Robert Inman, Charles
  Loeffler, \\ John MacDonald, and Holger Sieg\thanks{We would like to
    thank Rob Dubow (Director of Finance), Clarena Tolson (Revenue
    Commissioner), and Marisa Waxman (Deputy Commissioner for
    Assessment of Properties) in the Department of Revenue of the City
    of Philadelphia for their help and support. We would also like to
    thank Jeff Brown, Chris Sanchirico, Wolfgang Sch\"on, Reed
    Shuldiner and participants of numerous seminars for comments and
    suggestions. The views expressed here are those of the authors and
    do not necessarily represent or reflect the views of the City of
    Philadelphia.}  \\ University of Pennsylvania \\ \\ Prepared for
  the 2015 NBER Conference on Tax Policy and the Economy}
  
  \date{\today}

\maketitle

\begin{abstract}

This study evaluates a set of notification strategies intended to
increase property tax collection. To test these strategies, we develop
a field experiment in collaboration with the Philadelphia Department
of Revenue.  The resulting notification strategies draw on core
rationales for tax compliance: deterrence, the need to finance the
provision of public goods and services, as well as the appeal to civic
duty. Our empirical findings provide evidence that carefully designed
and targeted notification strategies can modestly improve tax
compliance.

\noindent KEYWORDS: Tax Compliance, Property Taxation, Field
Experiment, Deterrence, Public Service Appeal, Appeal to Civic Duty.

\end{abstract}

\newpage

\renewcommand{\thefootnote}{\arabic{footnote}}

\renewcommand{\thefootnote}{\arabic{footnote}}

\section{Introduction}

The lack of tax compliance has become a policy issue of central
importance to all levels of government in developed and developing
economies. In 2009, the developed economies of the OECD reported a tax
non-compliance rate of 14.2, ranging from a low of 2-3 percent in
Austria, Denmark, Germany, Korea, and Norway to 25 percent or more in
Belgium, Iceland, Portugal, the Slovak Republic, to a high of 73
percent in Greece (OECD, 2011, Table 36). In developing economies with
significant cash economies, tax non-compliance is likely much higher. The
OECD estimates an average rate of tax non-compliance in non-OECD
countries of 37 percent (OECD, 2011, Chart 7).

Noncompliance is a significant concern for at least four reasons.
First, governments are denied the revenues needed to provide basic
public services essential for ensuring the safety, health, and minimal
well-being of all citizens.  Second, if there is significant
non-compliance and basic services are to be provided, then tax rates
will need to rise on those who pay taxes. Rising tax rates for honest
payers will discourage their use and desire of public services,
potentially encouraging their exit from the formal economy. The
negative consequences for overall economic performance can be sizable;
Greece today serves as an unfortunate reminder.  Third, non-compliance
undermines the principle that everyone has to pay their ``fair share"
of taxes.  The evidence suggests upper income taxpayers are more
likely to be non-compliers.  Finally, significant non-compliance may
threaten the stability of democratic governance.  When democratic
governments fail to deliver essential services, impose large tax
burdens on the legitimate private economy, and are viewed as
capricious or actively unfair, then dictatorial alternatives may
become attractive. In an important sense, tax compliance is a first
order of business for efficient, fair, and democratic governance.

Tax compliance requires government to manage the taxpayer's decision
to pay taxes.  Taxpayers may ask: What do I owe and what happens if I
don't pay?  Taxpayers have the ability to influence what is owed on
any tax that requires self-reporting of income or assets, such as
self-reported consulting or business income.  Taxing jurisdictions can
in principle increase compliance by requiring less self reporting and
directly assessing the tax base.  Since property cannot be hidden,
scuppered away to a tax haven, or concealed in an electronic data
system, self- or non-reporting is less of an issue in a property tax
system.  Privately assessed wages, dividends, and interest income by
individuals and businesses are easier to conceal to the tax
authorities.  However, private assessors have an incentive to report
incomes truthfully as those payments are typically deductible expenses
for their own taxes.  The need for self-reporting is also reduced as
the formal economy and the use of audited business records expands.
Taxes which can take advantage of those records maximize tax
compliance and are preferred for just this reason.  The increased
popularity of the value-added tax (VAT) over the past twenty years in
economies with developing formal consumption sectors is a case on
point (see Keen and Lockwood, 2010; Pomeranz, 2013).  Self-reporting
matters for tax compliance in developed economies as well.  Kleven
et. al. (2011), for example, show that tax non-compliance among Danish
taxpayers is significantly higher for individuals with self-reported
income.

The taxing jurisdiction can also control compliance by influencing the
decision to evade, once the tax liability has been assessed.  The most
common strategy is the economic stick - fines and penalties.  Failure
to pay in time leads to interest penalties sufficiently large that
there is no arbitrage advantage to waiting, and perhaps to a
significant late fine as well.  For long-time non-payers, fines may
include the garnishment of wages, seizure of property, or jail.  Early
empirical studies found little impact of such penalties on aggregate
tax compliance, however; see Slemrod (2007).  But more recent, nuanced
studies, did find an impact of fines on both the level and speed of
tax payments.  Increasing fines have increased self-assessment of tax
liabilities for those facing possible audits of self-reported incomes;
see Kleven, et. al. (2011 and Pomeranz (2013).  Hallsworth, et. al.
(2014) find the speed with which taxpayers pay their liabilities can
also be improved with increased fines.  But fines only work if
taxpayers believe they will be enforced.  Large fines may be seen by
taxpayers as a signal of a desperate and ineffective tax collector, as
politically not viable and thus as empty threats, or in the extreme,
as a breakdown of cooperative democratic governance.  If so, an
increase in fines may even reduce tax compliance, as indeed happened
in Israel with the payment of corporate taxes; see Ariel (2012).  On
balance, the estimated effects of fines on tax payments have been
positive, but modest in magnitude.

Given evidence as to the limited ability of economic sanctions to
impact aggregate collections, attention has turned to employing
non-economic or ``behavioral" motivations for increasing tax
compliance.  Such motives are grounded in the value taxpayers place
upon their role and position within the democratic community.  The
role may be instrumental leading to outcomes valued by the taxpayer,
such as additional public services, or of value in its own right.
Both provide an incentive for tax compliance.

Instrumentally, for example, if each taxpayer thought of himself as
simply a single citizen within the democracy, there would be no
incentive, apart from a fine, to pay for services.  If fines are too
low or unenforced, then free-riding is the preferred private strategy.
But if each taxpayer views himself as part of a community of taxpayers
and assumes all other taxpayers also think in a private way, then no
services would be provided.  Facing this possibility, there may arise
a community, or cooperative, equilibrium in which all citizens agree
to pay their taxes, as long as all others agree to pay as well.  But
citizens must think of themselves as representatives of their
community, not as a citizen alone; see Feddersen and Sandroni (2006).
If so, the cooperative outcome can emerge.  As examples, we vote, we
tip in restaurants we will never visit again, and we put our litter in
waste cans.

Or the citizen's role in the community may be valued in its own right,
quite apart from any impact playing such a role may have on valued
social outcomes.  Individuals may derive satisfaction from knowing, or
from having others know, that they have done their ``civic duty."
Duty can extend far beyond tax compliance to all forms of law abiding
behavior; see Posner (2000).  Consistent with theories of social
norms, the more people conform to law-abiding behavior, the more
likely it may be that the ``marginal" citizen will conform as well;
see Benabou and Tirole (2011).

Both the instrumental motive and the motive born from civic duty have
been used to stimulate tax compliance.  The evidence is mixed.  The
most careful study of the two motives was done Blumenthal et
al. (2001), where two different letters were sent to Minnesota state
taxpayers reminding the taxpayer when taxes were due and to report
their income accurately.  One letter stressed that taxes pay for
important state services.  The other letter emphasized that most state
taxpayers correctly report their taxable income on time.  There were
15,000 taxpayers in each group, and their reported taxable incomes
were compared to a control group of 15,000 taxpayers who received no
letter.  From the work of Kleven et al. (2011) and Pomeranz (2013) we
should expect the largest effect on self-reported incomes.  For both
letters, there were statistically significant positive and negative
effects on the various categories of self-reported incomes, with no
statistically significant change in aggregate taxable income over that
reported by the control group.  The one strong effect was a relatively
large negative effect on reported income by the richest taxpayers from
having received the civic duty letter.

Three more recent studies have been more encouraging as to impact of
behavioral appeals. In an effort to improve the speed of tax
compliance for British income taxpayers, Hallsworth et al. (2014)
sent either of two letters to taxpayers both encouraging them to pay
their taxes on time, with one letter stressing that payment ensures
important national services will be provided and a second stressing
that ``nine out of ten" taxpayers pay their taxes on time.  Both sets
of letters had a statistically significant effect in encouraging
sooner tax payments, and the effects were greatest for the appeal to
``civic duty" when explicitly mentioned the taxpayer's most likely
reference group of fellow citizens.

Perez-Truglia and Troiano (2015) explored the impact of what they call
a ``shaming penalty" administered through a letter to a subset of
delinquent state taxpayers reminding them that the state has placed
their name on a publicly available list of tax delinquents and that
only payment in full or acceptance of a payment schedule can remove
their name from the delinquent list.  The reminder letter made a
significant positive difference to eventual tax compliance, with the
greatest effects observed for taxpayers with the lowest level of taxes
owed.  In addition, reminding tax delinquents that there is a growing
financial penalty to late payments also had a positive impact on
compliance and particularly so for wage-only taxpayers whose income
can be most easily attached for payment and penalties.

Finally, Besley, Jensen, and Persson (2014) estimate a dynamic model
of tax compliance to explore whether more complying taxpayers
encourages further compliance as implied by social norm behaviors.
The theory is tested for British local government tax compliance
following the tax revolt of 1990 in response to the replacement of the
wealth-based property tax by a regressive poll or ``head" tax.  Local
compliance fell from an average rate of 97 percent to 82 percent
within two years.  The poll tax was removed and wealth taxation
restored in 1993 but it took more than ten years for tax compliance
for the wealth tax to return to its original levels.

Our agenda here is to extend our understanding of tax compliance to
include the payment of local property taxation.  We do so by
implementing a tax compliance experiment in one large U.S. city,
Philadelphia.  In the late fall of 2014, we assisted the City of
Philadelphia's Department of Revenue (DoR) in an evaluation
of their efforts to improve local property tax collection through
redrafting letters reminding taxpayers that their 2014 property tax
payments were overdue.  The City's historical performance in tax
compliance has not been good, collecting only 90 percent of assessed
property tax revenues compared to an average compliance rate among
large U.S. cities of nearly 95 percent.

In Philadelphia, each year's property tax payments are mailed to
property owners by mid-January and are due in full by March 31st of
that year.  Beginning in May of the tax year, the DoR sends a common
reminder letter to each late taxpayer, usually once every two months
until payment is received.  The common reminder letter states the
taxpayer's liability and accrued interest and penalties.  If payment
has not been received by September of the tax year, 2 out of 3 taxpayers
are assigned  to either of two law firms for collection; the remaining third 
stays with DoR for continued efforts at internal collection.  We
assisted DoR with collection from their share of these ``tardy"
taxpayers.  We proposed three additional formats for DoR's reminder
letter.  In addition to the listing of tax payments, interest, and
penalties, the alternative letters contained a sentence that either
(i) threatened the potential loss of the taxpayer's home or property
if taxes were not paid, or (ii) appealed to the positive community
benefits in provided public services that the taxpayer's dollars
provide, or (iii), appealed to the positive benefits of fulfilling
your civic duty to yourself and others by paying your taxes.

We find evidence that the letter that appealed to the benefits of
fulfilling your civic duty had a positive 
effect on tax compliance above that of the City's standard reminder
letter.  This appeal was an effective strategy for encouraging
at least some tax payment and often payment in full, and had its
biggest differential and most significant impact on those residents
with relatively low levels of tax debt.  We also find some evidence
that stressing the benefits of payment for the provision of city
services may also improve tax compliance for the tax payer. The effects
are most significant for tax payers that owe larger
amounts of taxes. The letter that threatened the possible loss of the
taxpayer's property did not significantly improve tax
collection. Finally, our results suggest that a preferred overall
strategy may take advantage of the differential responses of taxpayers
to the treatment letters. A uniform message to all late or
non-compliant tax payers is not likely to be desirable.

The rest of the paper is organized as follows.  Section 2 provides a
brief overview of tax compliance in U.S. cities. Section 3 provides
a detailed discussion of our three treatments and the control. It also
discusses the experimental design and the fidelity of its
implementation. Section 4 presents a descriptive analysis summarizing
the main effects of our experiment.  Section 5 provides some additional
analysis of discrete outcomes focusing on whether tax payers made 
payments at all or paid the debt back in full.
Section 6 offers some conclusions and discusses future research.


\section{Property Tax Compliance in U.S. Cities: An Overview}


The property tax is one of the most important taxes for the financing
of local government services in the United States.  For the country as
a whole, approximately 21 percent of all state and local government
revenues were raised using the property tax in 2011 (Gruber, 2014).
For the largest cities that percentage is much higher.  The potential
economic advantages of the property tax are well known.\footnote{A
  well administered property tax is has two potential economic
  advantages, one relating to economic efficiency and the other to
  economic fairness.  First, if households and businesses are mobile
  across local political jurisdiction and if local jurisdictions use
  their zoning powers to ``sort" taxpayers by the value of their
  properties, then the property tax becomes the economic equivalent of
  a benefit tax relating taxes paid directly to the costs of the
  services provided (see Hamilton, 1975).  This will lead to the
  efficient provision of local government services.  The two
  efficiency assumptions are likely to hold in suburbs, but not in
  central cities.  In the case of the central city, efficiency will
  require the tax be close to a tax on existing structures and ideally
  land, rather than on new investment.  The tax will be least
  efficient in those cities with very elastic demand and supply for
  new construction.  In declining cities with no new construction, the
  supply curve is inelastic, at the level of existing structures.  In
  successful, growing cities demand for location is likely to be
  inelastic and new supply constrained by available land.  In these
  two cases, therefore, the property tax remains a relatively
  efficient local tax.  With regard to economic fairness, if the
  property tax is based on market value assessments, then the tax
  becomes a proportional tax on property wealth (see Aaron, 1975;
  Mieszkowski, 1972).  Since property wealth increases at least in
  proportion to increases in income, the tax will be proportional or
  perhaps progressive.}  But so too does the tax have significant
administrative advantages.  With modern techniques for assessment,
properties can be accurately assessed at their market values, and
assessments can be easily updated at the time of each ``arms-length"
transaction.  Thus, there is no need for taxpayer reporting of the tax
base, as with income, profits, sales, or VAT taxation.  Property
values, based as they are on long-run economic returns, are usually
less volatile than tax bases dependent on current economic activity,
such as income or sales. Stable tax bases allow for stable revenue
flows and thus less volatile service flows or, alternatively, tax
rates.\footnote{Any remaining volatility in revenues can be managed
  with rainy day funds.} Finally, when the tax base is determined by
market-based assessments, the taxpayer's tax bases will have been
objectively set and easily understood.  There is no need for
complicated tax forms or contentious appeals.  This too saves on
administrative costs, and one hopes, increases citizen confidence in
the fairness of their tax payments.

\begin{center}
\begin{longtable}{| l | c |  c|}
\caption{Property Tax Compliance: 2005-2014} \label{table:comp} \\
\hline 
\multicolumn{1}{|c|}{\textbf{City}} & \multicolumn{1}{c|}{\textbf{Percent Compliance}} & \multicolumn{1}{c|}{\textbf{Delinquent Tax Collected }} \\ 
\multicolumn{1}{|c|}{\textbf{}} & \multicolumn{1}{c|}{\textbf{Current Yr; 10 Year Average}} & \multicolumn{1}{c|}{\textbf{Five Year, Yearly Average}} \\ 
\hline 
\endfirsthead
\multicolumn{3}{c}%
{{\bfseries \tablename\ \thetable{} -- continued from previous page}} \\
\hline \multicolumn{1}{|c|}{\textbf{City}} &
\multicolumn{1}{c|}{\textbf{Percent Compliance}} &
\multicolumn{1}{c|}{\textbf{Delinquent Tax Collected}} \\ 
\multicolumn{1}{|c|}{\textbf{}} & \multicolumn{1}{c|}{\textbf{Current Yr; 10 Year Average}} & \multicolumn{1}{c|}{\textbf{Five Year, Yearly Average}} \\ 
\hline 
\endhead
\hline \multicolumn{3}{|l|}{{Continued on next page}} \\ \hline
\endfoot
\hline 
\multicolumn{3}{l}{* City Poverty Rate is greater than or equal to .20 in 2009-2013.}  \\
\multicolumn{3}{l}{{\it Annals of Statistics}: Each city's Comprehensive Annual Financial Report,} \\
\multicolumn{3}{l}{annually over the years, 2005 to 2014.} \\
\multicolumn{3}{l}{ \textit{Percent Compliance}: Computed as the percent of property taxes levied in} \\
\multicolumn{3}{l}{each fiscal year that are actually paid during the fiscal (or collection) year.} \\ 
\multicolumn{3}{l}{\textit{Delinquent Taxes Collected}: Delinquent taxes not paid in the year due may} \\
\multicolumn{3}{l}{be paid in subsequent years.  The annual rate is computed as the average collection} \\
\multicolumn{3}{l}{ rate over a five year period following the year after the tax is first due. The aggregate} \\
\multicolumn{3}{l}{ percent of the delinquent taxes paid after five years, the typical horizon over which} \\
\multicolumn{3}{l}{no further payments can be expected, can be computed as 5 x [yearly average].  } \\
\multicolumn{3}{l}{The (-) indicates that data were not available to compute the rate of delinquent tax} \\
\multicolumn{3}{l}{collection for that city.}  \\
\endlastfoot
Large City Average & .946; .945	& .112 \\
Atlanta*	  & .982 ;  .960 & .182 \\
Baltimore*	 & .960 ;  .950	 & .128 \\
Birmingham*	 & .983;  .955	& - \\
Boston	         & .996;  .992	 & - \\
Buffalo*	 & .947;  .945	 & .175 \\
Charlotte	 & .984;  .980	 & - \\
Chicago*	 & .962;  .930	 & - \\
Cincinnati*	 & .940;  .925	 & .120 \\
Cleveland*	 & .841;  .850	 & .090 \\
Columbus*	 & .938;  .920	 & .075 \\
Dallas*	         & .989;  .985	 & .085 \\
Washington, DC	 & .985;  .980	 & - \\
Denver	         & .990;  .989	 & - \\
Detroit*	 & .683;  .891	 & - \\
Flint*	         & .654;  .785	 & .151 \\
Houston*	 & .983;  .975	 & .171 \\
Kansas City	 & .943; 938	 & - \\
Los Angeles	 & .992;  .940	 & - \\
Memphis*	 & .984;  .945	 & .085 \\
Miami*	         & .975;  .970	 & .045 \\
Milwaukee*	 & .865;  .875	 & .191 \\
Minneapolis*	 & .985;  .972	 & .102 \\
Nashville	 & .984;  .986	 & - \\
New Orleans*	 & .948;  .921	 & .172 \\
New York City	 & .915;  .925	 & .041 \\
Oklahoma City	 & .958;  .949	 & .161 \\
Orlando	         & .991;  .988	 & .072 \\
Philadelphia*	 & .940;  .880	 & .125 \\
Phoenix*	 & .977;  .965	 & .130 \\
Pittsburgh*	 & .849;  .860	 & .048 \\
Portland	 & .942;  .934	 & .109 \\
Richmond*	 & .924;  .955	 & .171 \\
Riverside	 & .990;  .982	 & - \\
Sacramento	 & .996;  .980	 & - \\
Salt Lake City	 & .985;  .980	 & .140 \\
San Antonio	 & .989;  .985	 & .134 \\
San Diego	 & .980;  .950	 & - \\
San Francisco	 & .988;  .980	 & - \\
San Jose	 & .999;  .990	 & - \\
Seattle	         & .985;  .983	 & .170 \\
St.  Louis*	 & .921;  .890	 & .123 \\
Tampa	         & .959;  .957	 & .032 \\
\end{longtable}
\end{center}

Once market-based assessments are in place, the administrative issue
that remains is this: Will property owners pay their taxes?  Table
\ref{table:comp} summarizes the record for property tax compliance for
forty of the largest U.S. cities, plus Flint, Michigan, a poster child
for weak compliance.  Tax compliance is defined as the percent of
taxes levied in the collection year that are paid in the year due.
Taxes not paid in the collection year are then considered
delinquent.\footnote{A city's collection year need not correspond to
  the city's fiscal year.  For example, in Philadelphia the collection
  year is the calendar year while the fiscal year runs from July 1 to
  June 30 of the next year.  Tax bills are mailed in January of each
  collection year - the middle of the fiscal year - and are due on
  March 31 of that year.  Payments received after March 31 are
  considered late payments and will incur interest and late payment
  penalties.  All payments received by December 31 of the collection
  year are then recorded as taxes paid during the collection year.
  Payments that are not paid by December 31 are then classified as
  delinquent for that collection year.  Since property tax payments
  arrive in the last half of each fiscal year, Philadelphia will use
  some its tax receipts to repay the short-term ``cash-flow" loans of
  that fiscal year and then save a significant fraction of the
  remaining revenues for spending in the first half of the next fiscal
  year.}

Property tax compliance in these large cities over the past ten years,
years that included the deepest recession in decades, has been very
good.  On average, these large cities collected nearly 95 percent of
their property taxes in the tax year due, and the recession years did
not lower collection rates at all significantly.  Still the average
amount of uncollected, delinquent revenues is significant too, and
particularly so for the seven poorest performing cities: Cleveland
(.85), Detroit (.89), Flint (.79), Milwaukee (.88), Philadelphia
(.88), Pittsburgh (.86), and St. Louis (.89).

Taxes that have not been paid in the tax year become delinquent
payments, and cities seek to collect those taxes through various
enforcement mechanisms.  The most common strategy is to send a
reminder letter to the taxpayer stressing that unpaid taxes accumulate
interest and penalties and need to be paid.  If still unpaid, the tax
bill can be given to a private collection agent with revenues shared
between the agent and city or perhaps sold to the agent for immediate
revenues.  Or the wages of, or payment to, the tax delinquent can be
garnished.  Philadelphia does so for public employees and for private
contractors working for the city.  Finally, a tax lien can be imposed
on the property to be paid when the home is sold.  As a last resort,
the city can seize the property and require a sheriff's sale to
collect back taxes.  The end result is the collection of some portion
of delinquent taxes.  Table 1 reports each city's five year, yearly
average for the collection of delinquent taxes.\footnote{The average
  annual collection rate for delinquent taxes was estimated from data
  provided by the sample cities in each city's Comprehensive Annual
  Financial Report.  The required data was reported either as the
  amount finally collected from a given year's delinquent taxes -
  reported as ``Collections in Subsequent Years" - or as all
  delinquent taxes collected in a year from all previous years -
  reported as ``Delinquent Tax Collections."  For cities reporting
  ``Collections in Subsequent Years" the average annual rate was
  computed as ratio [Collections/(Tax Year Taxes Levied - Tax Year
    Taxes Collected)] then divided by 5.  The assumption is that all
  taxes levied but not collected in the tax year are classified as
  delinquent and that no significant amount of delinquent taxes are
  collected after five years.  For cities reporting ``Delinquent Tax
  Collections" the average annual rate was computed as ratio
  [Collections/ $\sum$ (Tax Year Taxes Levied - Tax Year Taxes
    Collected)], summed over the previous five tax years.  In both
  cases the average annual rate is an average of the actual
  collections in each of the five years following tax delinquency,
  where typically the first year rate of collection is the highest
  with a declining rate in years two to five.  Included in
  ``Collections" in both cases will be taxes plus interest plus
  penalties collected, the proceeds from the sale of tax liens to
  private collection agents, and the proceeds from the sheriff's sales
  of delinquent properties.} The typical pattern of collection for
delinquent taxes shows a relatively high success rate in the first
year of delinquency, and then a very sharp decline in payments
thereafter.\footnote{Atlanta is one of the better performing cities in
  its collection of delinquent taxes and the pattern of its collection
  success is typical.  We estimate that in the first year of
  delinquency for its 2005 tax collection year, the city collected 56
  percent of delinquent taxes owed. That was in 2006.  In 2007, the
  second year of delinquency for 2005 taxes, an additional 8 percent
  was collected.  In 2008, an additional 1 percent was collected.  In
  2009, an additional 7 percent was collected.  And in the 2010, an
  additional 12 percent was collected.  After five years, the final
  amount collected of the 2005 delinquent tax owed was 84 percent.
  The five year annual average for 2005 was therefore .168.  In
  subsequent years, Atlanta has done a bit better.  Its annual average
  collection rate has been .182 for an aggregate average collection
  rate of delinquent taxes of .91.} Most cities view tax bills that
have been delinquent for more than five years as uncollectible.
Multiplying the five year average rate reported in Table 1 by five
yields the average aggregate collection rate of any one year's
delinquent taxes.  For the average city in our sample, this aggregate
collection rate is .560 ( = .112 x 5).  The better performing cities,
such as Atlanta, may eventually collect more than 90 percent of their
delinquent taxes, the poorer performing cities perhaps not much more
than 30 percent.

Table 1 also indicates those cities with poverty rates greater than
.20 for the period 2009-2013.  The expectation is that high poverty
cities should have lower rates of initial tax compliance and possibly
more difficulty in collecting delinquent taxes.  A comparison of the
mean rates of tax compliance shows this to be case for initial
collection efforts: .92 for the 22 high poverty rate cities (.94
excluding Detroit and Flint) and .98 for the 20 cities with relatively
low poverty rates.  The average annual ability to collect delinquent
taxes in the two sets of cities is about the same (= .11), however,
perhaps because the pool of delinquent taxpayers is very poor in all
cities.  Importantly, however, some cities with high poverty rates are
very successful in collecting property taxes on time and in collecting
delinquent taxes.  Among the poorer cities, Atlanta, Baltimore,
Houston, New Orleans, and Phoenix perform as well, and often better,
than the average low poverty city.  The fact that property tax
compliance can be well managed in the face of difficult economic
realities suggests the value of looking at the administrative
strategies of successful cities and searching for new strategies as
well.  It is the latter agenda we pursue here, using taxpayer
compliance in Philadelphia as a laboratory to experimentally evaluate
four alternative collection strategies to encourage payment by tardy,
soon to be delinquent, city taxpayers.

\section{The Philadelphia Tax Experiment}

\subsection{Treatments}

In Philadelphia, each year's property tax payments are mailed to
property owners by mid-January and are due in full by March 31st of
that year.  Beginning in May of the tax year, the DoR sends a common
reminder letter to each late taxpayer, usually once every two months
until payment is received.  The common reminder letter is impersonal
and simply states the taxpayer's liability and accrued interest and
penalties; see Figure 1.  The only means for responding to the letter
is to either send a check with the detached portion of the letter to
DoR or to call a phone number given at the top of the letter, but
without instructions.  If payment has not been received by September
of the tax year, the taxpayer is assigned
to either of two law firms for collection or to the DoR
for continued efforts at collection.  The law firms are free to pursue
the collection of the debt as they see fit.  Proceeds from their
efforts are shared with the City.  In the past, DoR's efforts at
collection from these tardy taxpayers have been limited to simply
re-mailing the usual reminder letter.

In collaboration with the staff of DoR, we proposed two changes to
their usual tax collection efforts.  First, a generic reminder letter,
that included a Spanish translation of the letter on the reverse side
and also provided a list of contact numbers for taxpayers whose
native language is not English, was included with the traditional tax
bill.\footnote{The Spanish translation was targeted at the substantial Latino
  population and is available upon request. 
   Phone contacts were also included.}  This revised
letter serves as our ``control" treatment.  Second, we offered three
alternative letters to the control letter which might encourage
additional tax compliance: one that threatened the potential loss of
the taxpayer's home or property if taxes were not paid, a second that
appealed to the positive community benefits in provided public
services that the taxpayer's dollars provides, and third that appealed
to the positive benefits the taxpayer may feel from fulfilling their
civic duty to themselves and to others by paying their taxes.
Specifically:

{\it Treatment Letter 1: Threat: } {\bf Not paying your Real Estate
  Taxes is breaking the law.} Failure to pay your Real Estate Taxes
may result in seizure or sale of your property by the City. Do not
make the mistake of assuming we are too busy to pursue your case.

{\it Treatment Letter 2: Service Appeal: } {\bf We understand that
  paying your taxes can feel like a burden.} We want to remind you of
all the great services that you pay for with your Real Estate Tax
Dollars. Your tax dollars pay for schools to teach our children.  They
also pay for the police and firefighters who help keep our city safe.
Please pay your taxes as soon as you can to help us pay for these
essential services.
  
{ \it Treatment Letter 3: Civic Appeal: } {\bf You have not paid your
  Real Estate Taxes.}  Almost all of your neighbors pay their fair
share-- 9 out of 10 Philadelphians do so. Paying your taxes is your
duty to the city you live in. By failing to pay, you are abusing the
good will of your Philadelphia neighbors.

The formats of the three letters were constructed to only differ in
their wording of the middle paragraph; see Appendix A.  Care was taken
to minimize issues of communication for those with limited English
literacy, ensuring that each letter was intelligible to those with a
5th grade education.  Like the revised control letter, all treatment
letters also included a Spanish translation as well as a list of phone
lines for different language translations on the reverse side of the
letter.  Letters were mailed in the November mailing cycle to the
still tardy taxpayers.  The receipt of tax payments, or not, were
recorded for 30 days, beginning five days after the mailing date.


\subsection{Experimental Design}

To ensure that the results of the experiment allow for a causal
interpretation from the receipt of the letter to increased payment,
great care was taken to establish a random assignment of all four
letters across the pool of DoR's tardy taxpayers.  Unfortunately,
DoR's administration for mailing the letters did not allow for a
purely random assignment of tardy taxpayers to each letter.  

Our approach to randomization was constrained by the logistics of
DoR's enforcement capabilities. We concluded after several discussions
with our collaborators at DoR that it would be  impossible in practice
to assign individual properties at random to different
treatments. Instead, we chose to exploit the pseudo-random assignment
of properties to billing cycles and randomized treatments across them.
To understand this decision it is useful to discuss the current
practice of posting reminder letters by DoR.

Mailing of tardy real estate tax bills is as follows.
Since it is cheaper and simpler to send all bills at once to those
owners owing taxes on multiple properties, assignment to cycles is
done at the owner level, so that each mailing cycle has roughly the
same number of owners.  Every morning, a printer at DoR taps the
in-house accounting system to find all properties that a) owe taxes to
the City and b) are in the current day's mailing cycle, with the
numbered cycles progressing in sequence day-by-day.  After identifying
the bills to be printed for the day, the printer merges into the bill
several other pieces of information stored with the tardy balance such
as the mailing address and an in-house ID associated with the
property. The bills printed each day are then
brought to the City's mailing room, wherein they are stuffed into
envelopes and delivered to the property owners.

Given the volume of bills printed each day and the existing
infrastructure for processing them, especially the machine-automated
process of envelope stuffing, the most practical solution was to
randomize treatment at the mailing cycle level, so that every bill
printed on the same day would be paired with the same message.  We
elected to randomize 4-day cycles--for each 4-day period, we picked at
random among the $4!=24$ possible arrangements of treatments over the
subsequent 4 days. Our experiment was conducted on 9 days in November
2014, between the 4\textsuperscript{th} and the
25\textsuperscript{th}.

While we are certain of the sanctity of our mailing cycle-level
randomization process, one may be concerned about the assignment of
properties to mailing cycles. Fortunately, however, the city uses a
pseudo-random mechanism to assign owners to billing cycles, which
means that we achieve proper full-scale two-stage randomization of the
properties through our process of day-level randomization. In
particular, the city assigns properties to cycles based on the last
two digits of the property owner's social security number, or Employer
Identification Number, or (lacking those identifiers) to a DoR nine
digit identification number.  We believe that this quasi-random
assignment removes any significant sorting or self-selection bias in
the assignment of treatment letters.


\subsection{Implementation Fidelity}

To assess whether the final implementation of our mailing of treatment
letters is as intended, we leveraged a unique feature of the DoR's
mailing process.  The Department of Revenue regularly posts envelopes
destined for addresses that are either unattended (vacant) or do not
exist in the first place due to typos. Either before or after an
attempted delivery to such an address, the postal service identifies
these letters and returns them to the DoR, which then processes the
letters and attempts, if they can identify a suitable alternative
address, to re-deliver the tax bill. We took advantage of the fact
that a subset of bills made their way back to DoR to check first-hand
the extent of treatment fidelity. Our final sample consists of the
nine treatment days for which greater than 90\% fidelity was achieved.

\subsection{Sample Size}

From this original sample of 134,888 tardy tax payments we select a
final sample of 4,927 properties for our experiment.  This final
sample removes all properties no longer handled by DoR (= 61,170), or
for which payment agreements have been reached (= 31,456), or were not
part of our nine day mailing cycle (= 24,800), or which qualified for
a tax abatement (= 4706), or in sheriff's sale (=4098), sequestration
(= 1130) or bankruptcy (= 948), or for which we had no working address
(= 1429), or had a tax bill remaining of less than \$.61 (=
224).\footnote{The city operates 50 billing cycles. Each cycles has
  approximately 2,500 observations.  Once we apply the sample
  selection criteria discussed above we obtain between 493 and 633
  observations per day.}


Table \ref{table:ds} provides some descriptive statistics of the full
sample of all tardy and delinquent properties in the city -- i.e.,
without any sample restrictions (Sheriff's Sale, etc.). It also
includes full restricted sample and the sample used in our analysis.

\begin{table}[htb]
\begin{center}
\caption{Descriptive Statistics}\label{table:ds}
\begin{tabular}{|r|r|r|r|}
  \hline
 & All Properties & Restricted Properties & Analysis Sample \\ 
  \hline
Amount Due & 4409 & 3761 & 3465 \\ 
  Assessed Property Value & 138867 & 242604 & 186691 \\ 
  Value of Tax & 1586 & 3123 & 2405 \\ 
  Length of Debt & 6 & 4 & 4 \\ 
  \% Residential & 80 & 81 & 80 \\ 
  \% w/ Phila. Mailing Address & 88 & 82 & 83 \\ 
  \% Owner-Occupied & 24 & 21 & 22 \\ 
  Number Observations & 134887 & 29951 & 4927 \\ 
   \hline
\end{tabular}
\end{center}
NOTE: This table provides some descriptive statistics for all
properties in Philadelphia, all properties that satisfy our sampling
restrictions, and the sample used in the analysis.
\end{table}

Note in particular that this sample selection means that our sample
consists only of properties that are not in the purview of the two law
firms that DoR uses as collection agencies. It is therefore useful to
compare briefly the properties that are kept in-house with those that
are assigned to the law firms. We find that properties kept in-house
have lower balances, with a median of \$1,000, as compared to \$1,700
overall. However, in-house properties have higher market values--the
DoR median is \$91,000 vs. \$66,100 overall. Properties handled by DoR
have younger debt--an average of 4 years vs. 7 and 11 for the two law
firms.  Even conditional on age of debt, in-house balances are low.
DoR-managed accounts are more likely to be owner-occupied, less likely
to be in payment agreements, and more likely to result in a sheriff's
sales. In summary, it appears that the outside firms are holding
properties which, even given other characteristics, have the highest
potential returns.

\subsection{Sample Balance on Observables}

To confirm whether or not we indeed achieved randomization, we
performed a series of balance-on-observables tests. The null
hypotheses of these tests are that a given observable data moment is
identical across mailing cycles. We turn now to the results of those
tests.

Analysis of balance on observables is complicated by the random
assignment at the owner level.  Because there are some large holders
of property -- e.g., the City of Philadelphia, the Philadelphia
Housing Authority, the Redevelopment Authority of Philadelphia, The
University of Pennsylvania and Drexel University -- a simple analysis
of balance at the property level will likely be skewed by these
outliers. In addition, it is not clear how to aggregate many of the
property-level characteristics to the owner level meaningfully,
especially geographic variables, complicating the task of testing
balance at the owner level. Our compromise was to examine sample
balance on the subset of properties for which a) the owner is unique,
and b) any tax exemption claimed by the property is related to
abatements for new construction.
  
Most of the observed characteristics are categorical variables, so we
can test balance using standard $\chi^2$ tests. The full sample
consists of letters mailed over nine days, two of which sent the
Threat treatment letter, four of which sent the Public Service
treatment letter, two of which sent the Civic Duty treatment letter,
and one final day which mailed the control letter.  This meant that of
the 4,297 letters mailed, 22 percent (2/9's) where threat letters, 44
percent (4/9's) were public service letters, 22 percent (2/9's) were
civic duty letters, and 11 percent (1/9) were control letters.  If our
treatment letters are randomly allocated across observable
characteristics of properties owing taxes, then we should observe the
same distribution of letters by each observable characteristic.  Table
\ref{table:balance} shows these distributions and the resulting p
values for the test of the null hypothesis that the letter
distribution by characteristic matches the overall distribution of
letters.  In each case, we cannot reject the null hypothesis that the
letters have been randomly distributed by the observable
characteristics shown in Table \ref{table:balance}.


\begin{center}
\begin{longtable}{| l | c |  c| c| c| c|}
\caption{Tests of Sample Balance on Observables} \label{table:balance} \\
\hline 
  & Threat & Service & Civic Duty & Control & $p$-value \\ 
\hline 
\endfirsthead
\multicolumn{3}{c}%
{{ \tablename\ \thetable{} -- continued from previous page}} \\
\hline 
  & Threat & Service & Civic Duty & Control & $p$-value \\ 
  \hline 
\endhead
\hline 
\multicolumn{6}{|l|}{{Continued on next page}} \\ 
\hline
\endfoot
\multicolumn{6}{|l|}{NOTE: This table shows that there are no significant differences in the} \\
\multicolumn{6}{|l|}{distribution of observed variables among the treatment and control samples.} \\ 
\hline 
\endlastfoot

 Taxes Due Quartiles &  &  &  &  &  \\ 
  $<$\$300 & 0.22 & 0.4 & 0.28 & 0.1 & 0.2 \\ 
  \lbrack\$300,\$1300) & 0.24 & 0.47 & 0.22 & 0.08 &  \\ 
  \lbrack\$1300,\$3300) & 0.23 & 0.45 & 0.2 & 0.11 &  \\ 
$>$  \$3300 & 0.18 & 0.48 & 0.23 & 0.11 &  \\ 
   \hline
Market Value Quartiles &  &  &  &  &  \\ 
  $<$\$46k & 0.24 & 0.43 & 0.21 & 0.12 & 0.2 \\ 
  \lbrack\$46k,\$82k) & 0.22 & 0.46 & 0.23 & 0.1 &  \\ 
  \lbrack\$82k,\$152k) & 0.21 & 0.45 & 0.25 & 0.09 &  \\ 
$>$  \$152k & 0.21 & 0.45 & 0.24 & 0.1 &  \\ 
   \hline
Land Area Quartiles &  &  &  &  &  \\ 
  $<$800 sq. ft. & 0.22 & 0.45 & 0.23 & 0.1 & 0.83 \\ 
  \lbrack800,1200) sq. ft. & 0.23 & 0.43 & 0.24 & 0.1 &  \\ 
  \lbrack1200,1800) sq. ft. & 0.21 & 0.47 & 0.22 & 0.1 &  \\ 
  $>$1800 sq. ft. & 0.21 & 0.44 & 0.24 & 0.1 &  \\ 
   \hline
Distribution of Properties & 0.22 & 0.45 & 0.23 & 0.1 & 0.08 \\ 
   \hline
Expected Distribution & 0.22 & 0.44 & 0.22 & 0.11 &  \\ 
   \hline
 \# Rooms &  &  &  &  &  \\ 
  0-5 & 0.22 & 0.44 & 0.23 & 0.11 & 0.32 \\ 
  6 & 0.21 & 0.46 & 0.23 & 0.09 &  \\ 
  7+ & 0.22 & 0.44 & 0.24 & 0.1 &  \\ 
   \hline
Years of Debt &  &  &  &  &  \\ 
  1 Year & 0.23 & 0.43 & 0.24 & 0.09 & 0.32 \\ 
  2 Years & 0.22 & 0.44 & 0.24 & 0.1 &  \\ 
  3-5 Years & 0.2 & 0.48 & 0.22 & 0.1 &  \\ 
  6+ Years & 0.2 & 0.47 & 0.2 & 0.13 &  \\ 
   \hline
Category &  &  &  &  &  \\ 
  Residential & 0.22 & 0.45 & 0.23 & 0.09 & 0.07 \\ 
  Hotels\&Apts & 0.2 & 0.45 & 0.23 & 0.12 &  \\ 
  Store w. Dwell. & 0.21 & 0.48 & 0.22 & 0.09 &  \\ 
  Commercial & 0.15 & 0.5 & 0.24 & 0.11 &  \\ 
  Industrial & 0.27 & 0.42 & 0.2 & 0.11 &  \\ 
  Vacant Land & 0.25 & 0.39 & 0.23 & 0.13 &  \\ 
   \hline
Expected Distribution & 0.22 & 0.44 & 0.22 & 0.11 &  \\ 
   \hline
\end{longtable}
\end{center}

As can be seen from Table \ref{table:balance} randomization appears to
have been successful.  The properties are strongly randomly
distributed by location (their political ward, of which there are 66
in Philadelphia), category (type of property usage), property size (as
measured by the number of rooms or by the size of the tract), case
assignment (this variable captures, if applicable, to which outside
law firm a property is assigned), and whether the property is in
sequestration or has entered a payment agreement with the city. The
number of properties assigned to each treatment is further exactly as
expected, given the unequal number of mailing days in our treatment.


\section{Average Treatment Effects}


We consider results for three different subsamples.  The first sample
(I) is the full sample and consists of all 4927 observations; The
second sample (II) eliminates commercial property owners, which
reduces the sample to 4749 observations; the third sample (III)
eliminates owners of multiple properties, resulting in a sample size
of 3888.

Table \ref{table:summary} summarizes the impact of our experimental
intervention on revenue collection.  The table reports the total
taxes owed, the amount generated, and the number of mailing days for
the three treatments and the control groups. It also reports the
percent of properties that paid the City anything and the percent that
paid off their full debt in our sample period.


\begin{sidewaystable}[htbp]
\caption{Summary of Effectiveness of Treatment}  \label{table:summary}
\bigskip
\centering
\begin{tabular}{|p{1.3cm}|p{1.3cm}|p{1.3cm}|p{1.3cm}|p{2cm}|p{1.4cm}|p{1.4cm}|p{1.4cm}|p{1.4cm}|p{1.4cm}|p{1.6cm}|}
  \hline
Sample & Group & Treated Days & No. Treated & Total Taxes Owed & Percent Ever Paid & Percent Paid in Full & Dollars Received & Dollars Per Day Treated & Dollars above Control Per Day & Total Generated over All Days \\ 
  \hline
 & Threat & 1 & 499 & \$1,839,826 & 14 &  8 & \$71,176 & \$71,176 & \$10,883 & \$ 10,883 \\ 
   & Service & 4 & 2,211 & \$8,003,148 & 15 &  7 & \$447,728 & \$111,932 & \$51,639 & \$206,557 \\ 
  I & Civic Duty & 2 & 1,142 & \$3,794,900 & 18 & 12 & \$152,217 & \$76,109 & \$15,816 & \$ 31,632 \\ 
  & Control & 2 & 1,075 & \$3,294,516 & 16 & 10 & \$120,585 & \$60,293 & \$     0 & \$      0 \\ 
   \hline
 & Threat & 1 & 480 & \$1,657,379 & 15 &  8 & \$71,176 & \$71,176 & \$11,142 & \$11,142 \\ 
 & Service & 4 & 2,122 & \$7,024,458 & 15 &  7 & \$288,758 & \$72,189 & \$12,155 & \$48,621 \\ 
II & Civic Duty & 2 & 1,099 & \$3,350,147 & 19 & 12 & \$146,227 & \$73,114 & \$13,079 & \$26,158 \\ 
 & Control & 2 & 1,048 & \$2,930,759 & 16 & 10 & \$120,069 & \$60,034 & \$     0 & \$     0 \\ 
   \hline
 & Threat & 1 & 406 & \$1,437,902 & 15 &  9 & \$51,309 & \$51,309 & \$18,011 & \$ 18,011 \\ 
 & Service & 4 & 1,754 & \$6,956,034 & 16 &  7 & \$418,767 & \$104,692 & \$71,393 & \$285,572 \\ 
  III & Civic Duty & 2 & 891 & \$3,331,168 & 20 & 13 & \$130,016 & \$65,008 & \$31,710 & \$ 63,419 \\ 
 & Control & 2 & 837 & \$3,007,232 & 16 &  9 & \$66,597 & \$33,299 & \$     0 & \$      0 \\ 
   \hline
\multicolumn{11}{l}{NOTE: The table shows how much additional revenues were generated by the different treatments.} \\
\end{tabular}
\end{sidewaystable}

We also report the dollars in revenue raised per day, which ranges
from \$60,292 in the control group to \$111,931 in the service
treatment group. Note that the average payments per day is higher in
all three treatment group. A simple difference between the treatment
and the control group provides an impression of the overall
effectiveness of the intervention. These estimates range from \$10,883
for the threat treatment to \$51,639 for the service
treatment. Summing over all treatment groups and days suggests that
our experiment generated approximately \$250,000 for the DoR in just
nine days.  

To conduct a formal statistical analysis and determine whether the treatments 
had a significant positive effect on tax collections, we estimate a regression model 
for each of the three samples. The basic idea is to  regress the tax receipts on an intercept 
and the three treatment dummies. The intercept is the mean receipt in the control group and the 
treatment coefficients measure the change in tax collection induced by each.
Table 5 reports our parameter estimates and corresponding robust standard errors.

\begin{table}
\caption{Difference in Mean Tests}\label{dif_mean}
\begin{center}
\begin{tabular}{|c|c|c|}
\hline
\multicolumn{3}{|c|}{Main Sample} \\
\hline
parameter & estimate & std error \\
\hline
Intercept&  112.17 & 17.52 \\
Threat & 30.47  & 36.18 \\
Service & 90.33 & 57.85 \\ 
Civic & 21.12 & 33.90 \\
\hline
\multicolumn{3}{|c|}{Non-Commercial Sample} \\
\hline
parameter & estimate & std error \\
\hline
Intercept & 114.56 & 17.95 \\
Threat & 33.71 &  37.46 \\
Service & 21.51 &  26.97 \\
Civic & 18.48 & 34.97 \\
\hline
\multicolumn{3}{|c|}{Unique Owner Sample} \\
\hline
parameter & estimate & std error \\
\hline
Intercept &  79.56 &  12.73  \\
Threat & 46.81 &  29.86 \\ 
Service & $159.18^{*}$ & 70.50 \\ 
Civic  & 66.35 &  38.74 \\
\hline
\multicolumn{3}{|l|}{* indicates $p<0.05$.} \\
\hline
\end{tabular}
\end{center}
\end{table}

Table \ref{dif_mean} shows there is a fair bit of heterogeneity among treatments. The differences in means 
is \$90 for the Service treatment, \$21 for the Civic Duty treatment, and \$30 for the 
Threat treatment. The results are qualitatively similar for the two other samples, but
there are some important quantitative differences. If we restrict attention to the subsample of
non-commercial properties, we find that the Service treatment raises a
much smaller amount than in the full sample. This suggests that excluding a relatively small number of 
commercial property owners affects the magnitude of the overall effects.  If we restrict
attention the properties of sole owners, all treatments appear in a
more positive light as the intake of the control group drops
precipitously. While the differences in means are suggestive of
an average positive impact of the treatment letters on taxpayer compliance, only one of the nine key coefficients is significantly different from zero at conventional significance levels.

\section{Analysis of Discrete Compliance Outcomes}

\subsection{Specification}

To gain some additional insights into the nature of compliance, we 
consider two discrete compliance outcomes, denoted by $y \in \{0,1\}$. 
In the first case, $y= 1$, if the taxpayer made any payment at all (``ever paid"), 0
otherwise.  In the second case  $y = 1$, if the taxpayer made a full payment of their
taxes owed (``paid in full"), 0 otherwise.  Both responses are of
potential interest.  Paid in full is of obvious interest to DoR as
this is a measure of how well the city does in collecting taxes owed.
Even the small payments measured in ever paid are of interest,
however.  First, every dollar helps.  Second, making even a small
payment, rather than no payment at all, stands as recognition that the
citizen still values, however modestly, their relationship to the city
government and its governance; that is, they have not ``dropped out."

We specify and estimate compliance
as a logistic function of the control and three treatments, with each
estimated effect measuring the treatment's impact on tax payment
relative to that available from receipt of the control letter.
Generally, for $y = 1$ if the individual pays their taxes, and 0
otherwise, the probability of paying taxes can be specified as:
\begin{eqnarray*}
Pr \{ y=1 \} \; = \; \frac{exp(X' \beta)}{1 + exp(X' \beta)}
\end{eqnarray*}
where $X$ is a vector of explanatory variables and $\beta$ a vector of
coefficients to be estimated.

The benefits of the logistic specification, over the more familiar
linear specification, is that once estimated the computed
probabilities of payment are bounded between 0 and 1, and the partial
effect of any of the independent variables on the probability of
payment can vary according to the overall value of $X'\beta$.  For our
analysis, the vector of explanatory variables $X$ will include three
(1,0) indicator variables for whether the taxpayer received a Threat
Letter, a Service Letter, or a Civic Letter, respectively.  The
omitted category is receipt of the reminder control letter.

The estimated coefficient for each treatment letter will allow us to
compute the marginal impact on payment of that letter over that of the
usual (revised) reminder letter.  A positive and statistically
significant value of the coefficient for a treatment letter indicates
that this letter encouraged more tax payments than did the control
letter; conversely so, for a negative and statistically significant
value of the coefficient.  

We also explore how payment behaviors may vary across the level of
taxes owed.  Each taxpayer is assigned to one of four categories of
tax debt: LOW (Less than \$300), MODerate (\$300 to \$1300), HIGH
(\$1301 to \$3300), and Very HIGH (Greater than \$3300).  The taxpayer
is assigned a value of 1 if they tax bill falls within a debt level,
and 0 otherwise.  The omitted debt level for comparison is LOW.  In
addition, we allow for an interaction of each debt level with each
treatment letter to explore the possible advantages of targeting
treatment letters to taxpayers of varying debt levels. 

Finally, while care has been taken to randomly assign the treatment
and control letters across taxpayers, and our initial balance tests
reported above suggest that we have been successful along the broad
categories of taxes owed and years of tax debt, property values and
property type, and property size and land area, the question remains
of whether taxpayer compliance behaviors might vary along other
attributes of the property or the taxpayer.  If so, and if compliance
behavior is correlated with these excluded variables, then the
estimated effects on payment behavior of the treatment letters may be
biased.  To control for this possibility, we also include in our basic
logistic regression as elements of X measures of the location of the
property within one of ten city neighborhoods (each a City Council
District), the exterior condition of the property (classified as a
``sealed/compromised," i.e. dilapidated and dangerous), and whether
the property qualified for a low income homestead exemption.  



\subsection{Results}
  
Table \ref{XX} summarizes the estimates and the estimated standard
errors for the three samples that we considered above. We report
robust standard errors that are clustered to deal with multiple
ownership. As can be seen from Table \ref{XX}, the service appeal and
the threat treatments had no significant effect on ``ever paid" at the
conclusion of the 30 day payment period.  The Civic Duty treatment is
consistently positive and statistically significant at least the 10
percent level of confidence and at the 5 percent level for the sample
of sole owners.

\begin{table}[htbp]
\caption{Logistic Regressions -- Ever Paid}\label{XX}
\begin{center}
\begin{tabular}{| l |  c | c | c |}
\hline
               & Full Sample & Non-Commercial & Sole Owner \\
\hline
Intercept      & $-1.69^{***}$ & $-1.67^{***}$ & $-1.68^{***}$ \\
               & $(0.08)$      & $(0.08)$      & $(0.10)$      \\
Service          & $-0.07$       & $-0.10$       & $0.04$        \\
               & $(0.10)$      & $(0.10)$      & $(0.12)$      \\
Civic Duty           & $0.21$        & $0.19$        & $0.30^{*}$    \\
               & $(0.11)$      & $(0.11)$      & $(0.13)$      \\
Threat         & $-0.09$       & $-0.06$       & $-0.03$       \\
               & $(0.15)$      & $(0.15)$      & $(0.17)$      \\
\hline
Log Likelihood & -2136.16      & -2068.89      & -1758.95      \\
Num. obs.      & 4927          & 4749          & 3888          \\
\hline
\multicolumn{4}{l}{$^{***}p<0.001$, $^{**}p<0.01$, $^*p<0.05$.} \\
\multicolumn{4}{l}{NOTE: This table reports the parameter estimates from the} \\
\multicolumn{4}{l}{basic Logit Model that uses ``ever paid" as outcome.}
\end{tabular}
\end{center}
\end{table}

Next we investigate whether there is heterogeneity in response to the
treatment. It is plausible that tax payers who owe small amounts of
money behave differently than those who owe larger amounts.  To gain
insight into this possibility we include in our regression for ``ever
paid" the indicator variables for the levels of taxes owed - LOW, MOD,
HIGH, and VHIGH - and the interaction of those variables with our
three treatments.  The variable LOW is omitted from the regression so
all results provide comparisons to the behavior of those in the higher
debt levels to taxpayers in the lowest level of taxes owed.  Table
\ref{YY} summarizes the estimates and the estimated standard errors
for the full sample and the two subsamples.

\begin{center}
\begin{longtable}{| l | c |  c| c|}
\caption{Logistic Regressions -- Ever Paid} \label{YY} \\
\hline 
 & Full Sample & Non-Commercial & Sole Owner \\
\hline 
\endfirsthead
\multicolumn{4}{c}%
{{\tablename\ \thetable{} -- continued from previous page}} \\
\hline
& Full Sample & Non-Commercial & Sole Owner \\
\hline 
\endhead
\hline \multicolumn{4}{|l|}{{Continued on next page}} \\ \hline
\endfoot
\hline 
\multicolumn{4}{l}{$^{***}p<0.001$, $^{**}p<0.01$, $^*p<0.05$. Control coefficients omitted.} \\
\multicolumn{4}{l}{NOTE: This table reports the parameter estimates from the} \\
\multicolumn{4}{l}{Logit Model with interactions that uses ``ever paid" as outcome.}
\endlastfoot
Balance MOD        & $-0.46^{*}$   & $-0.52^{*}$   & $-0.33$       \\
                  & $(0.21)$      & $(0.22)$      & $(0.24)$      \\
Balance HIGH        & $-1.03^{***}$ & $-0.97^{***}$ & $-1.54^{***}$ \\
                  & $(0.24)$      & $(0.24)$      & $(0.30)$      \\
Balance VHIGH        & $-1.25^{***}$ & $-1.15^{***}$ & $-1.36^{***}$ \\
                  & $(0.30)$      & $(0.30)$      & $(0.33)$      \\
Service             & $-0.30$       & $-0.34$       & $-0.34$       \\
                  & $(0.18)$      & $(0.19)$      & $(0.20)$      \\
Service*Balance MOD  & $0.00$        & $0.00$        & $0.00$        \\
                  & $(0.00)$      & $(0.00)$      & $(0.00)$      \\
Service*Balance HIGH  & $0.06$        & $0.08$        & $-0.11$       \\
                  & $(0.16)$      & $(0.16)$      & $(0.18)$      \\
Service*Balance VHIGH  & $0.40^{*}$    & $0.42^{*}$    & $0.33$        \\
                  & $(0.20)$      & $(0.20)$      & $(0.21)$      \\
Civic Duty              & $0.16$        & $0.13$        & $0.21$        \\
                  & $(0.19)$      & $(0.19)$      & $(0.21)$      \\
Civic Duty*Balance MOD   & $0.54^{**}$   & $0.58^{***}$  & $0.58^{**}$   \\
                  & $(0.17)$      & $(0.17)$      & $(0.18)$      \\
Civic Duty*Balance HIGH   & $-0.08$       & $-0.07$       & $-0.19$       \\
                  & $(0.17)$      & $(0.17)$      & $(0.18)$      \\
Civic Duty*Balance VHIGH   & $-0.49^{**}$  & $-0.45^{*}$   & $-0.61^{**}$  \\
                  & $(0.17)$      & $(0.17)$      & $(0.19)$      \\
Threat            & $-0.05$       & $-0.01$       & $-0.13$       \\
                  & $(0.26)$      & $(0.26)$      & $(0.29)$      \\
Threat*Balance MOD & $0.07$        & $0.10$        & $0.11$        \\
                  & $(0.16)$      & $(0.16)$      & $(0.17)$      \\
Threat*Balance HIGH & $-0.50^{**}$  & $-0.49^{**}$  & $-0.64^{***}$ \\
                  & $(0.17)$      & $(0.18)$      & $(0.19)$      \\
Threat*Balance VHIGH & $-0.04$       & $-0.03$       & $-0.20$       \\
                  & $(0.16)$      & $(0.16)$      & $(0.17)$      \\
\hline
Log Likelihood    & -2010.55      & -1948.32      & -1639.28      \\
Num. obs.         & 4927          & 4749          & 3888          \\
\end{longtable}
\end{center}


Table \ref{YY} shows the indicator variables for taxes owed by
quartile are significantly negative - that is, the more a taxpayer
owes the less likely he is to pay their taxes.  The Civic Duty
treatment helps to moderate this growing negative effect of tax debt,
but only for those who owe a moderate amount of taxes.  For taxpayers
with a very large tax debt, the Civic Duty letter discourages payment.
Exactly the opposite responses are observed for those who receive the
Public Service letter.  Those with low or moderate levels of taxes
owed react negatively or not at all to the service letter, while those
with high levels of taxes owed are more likely to make a contribution
when they receive the Public Service letter.  The Threat letter never
helps tax payment and significantly discourages payment by those with
high levels of taxes owed.  One can speculate as to why motives for
payment are tied to the levels of taxes owed - civic duty is ``price
elastic" and free riding falls with larger property holdings and
greater payments - but the important conclusion here is that treatment
strategies need to be targeted strategies.

\begin{table}[htbp]
\caption{Marginal  Predictions -- Ever Paid}  \label{ZZ} 
\bigskip
\begin{center}
\begin{tabular}{| l | c | c | c | c |}
  \hline
 & LOW & MOD & HIGH & VHIGH \\ 
  \hline
Control & 23.40 & 16.10 & 9.80 & 8.00 \\ 
  Service & 18.50 & 12.70 & 12.10 & 11.40 \\ 
  Civic Duty & 26.40 & 15.20 & 14.40 & 8.20 \\ 
  Threat & 22.40 & 12.20 & 13.40 & 7.10 \\ 
   \hline
\multicolumn{5}{l}{NOTE: This table reports the marginal effects} \\
\multicolumn{5}{l}{from the Logit Model with interactions that uses} \\
\multicolumn{5}{l}{``ever paid" as outcome.} \\
\end{tabular}
\end{center}
\end{table}


Table \ref{ZZ} shows the marginal predictions for the probability that
properties in each treatment group and for each quartile of taxes owed
will make some payment (``ever paid").  The values here represent the
predicted probability of payment, computed for the ``average taxpayer"
as represented by the sample average level of all indicator control
variables and the median values of the continuous control variables.
Here we observe the final impacts of the treatments as they apply to
taxpayers with different levels of taxes owed.  The Civic Duty letter
increases the chance of payment over the control letter for taxpayers
with low debt by about 3 percentage points and for taxpayers with
relatively high payments by as much 4 percentage points.  The Public
Service letter is most effective for taxpayers with very high levels
of taxes owed.

Next we examine whether our treatment strategies might also impact the
larger matter: When do taxpayers pay their full amount of taxes owed?
The ever-paid outcome does not differentiate between taxpayers that
made full payment and those who made only a partial contribution.

Table \ref{VV} presents the results for the Logit specification for
the outcome ``paid in full" by the end of our 30 day payment period.
Again the analysis is separated into that for the full sample and the
two subsamples.  The results are similar to those for ``ever paid."
The Threat letter is never effective.  The Public Service letter
discourages full payment while the Civic Duty letter encourages full
payment.

\begin{table}[htbp]
\caption{Logistic Regressions -- Paid in Full}\label{VV}
\begin{center}
\begin{tabular}{| l c | c | c | }
\hline
               & Full Sample & Non-Commercial & Sole Owner \\
\hline
Intercept      & $-2.23^{***}$ & $-2.22^{***}$ & $-2.29^{***}$ \\
               & $(0.10)$      & $(0.10)$      & $(0.12)$      \\
Service          & $-0.42^{**}$  & $-0.44^{**}$  & $-0.29$       \\
               & $(0.13)$      & $(0.14)$      & $(0.15)$      \\
Civic Duty           & $0.24$        & $0.24$        & $0.41^{**}$   \\
               & $(0.14)$      & $(0.14)$      & $(0.16)$      \\
Threat         & $-0.21$       & $-0.18$       & $-0.04$       \\
               & $(0.19)$      & $(0.20)$      & $(0.21)$      \\
\hline
Log Likelihood & -1435.15      & -1395.06      & -1175.05      \\
Num. obs.      & 4927          & 4749          & 3888          \\
\hline
\multicolumn{4}{l}{$^{***}p<0.001$, $^{**}p<0.01$, $^*p<0.05$.} \\
\multicolumn{4}{l}{NOTE: This table reports the parameter estimates from the} \\
\multicolumn{4}{l}{basic Logit Model that uses ``paid in full" as outcome.}
\end{tabular}
\end{center}
\end{table}


Table \ref{WW} presents the results that allow for the influence of
taxes owed - MOD, HIGH, and VHIGH - on ``paid in full."  Owing more
taxes reduces the likelihood of paying in full and the negative effect
increases with the level of taxes owed.  These effects are even larger
than those in the ``ever paid" analysis suggesting that many tardy
taxpayers in the higher quartiles of taxes owed make only partial
payments when the respond (if at all) to the control and treatment
letters.  We continue to see the negative impact of the Public Service
letter on full payment, but again, as for ever paid, the strong
negative effect disappears for those with the greatest tax debts.  The
letter that has the greatest positive impact on encouraging full tax
payment is the Civic Duty letter, and this is particularly so for
those with low and moderate levels of taxes owed.


\begin{center}
\begin{longtable}{| l | c |  c| c|}
\caption{Logistic Regressions -- Paid in Full} \label{WW} \\
\hline 
 & Full Sample & Non-Commercial & Sole Owner \\
\hline 
\endfirsthead
\multicolumn{4}{c}%
{{ \tablename\ \thetable{} -- continued from previous page}} \\
\hline
& Full Sample & Non-Commercial & Sole Owner \\
\hline 
\endhead
\hline \multicolumn{4}{|l|}{{Continued on next page}} \\ \hline
\endfoot
\hline 
\multicolumn{4}{l}{$^{***}p<0.001$, $^{**}p<0.01$, $^*p<0.05$. Control coefficients omitted.} \\
\multicolumn{4}{l}{NOTE: This table reports the parameter estimates from the} \\
\multicolumn{4}{l}{Logit Model with interactions that uses ``paid in full" as outcome.}
\endlastfoot
Balance MOD        & $-1.28^{***}$ & $-1.42^{***}$ & $-1.35^{***}$ \\
                  & $(0.27)$      & $(0.28)$      & $(0.31)$      \\
Balance HIGH        & $-2.32^{***}$ & $-2.18^{***}$ & $-3.18^{***}$ \\
                  & $(0.39)$      & $(0.37)$      & $(0.61)$      \\
Balance VHIGH        & $-3.27^{***}$ & $-2.85^{***}$ & $-3.83^{***}$ \\
                  & $(0.74)$      & $(0.61)$      & $(1.03)$      \\
Service             & $-0.45^{*}$   & $-0.49^{*}$   & $-0.49^{*}$   \\
                  & $(0.19)$      & $(0.20)$      & $(0.22)$      \\
Service*Balance MOD  & $0.00$        & $0.00$        & $0.00$        \\
                  & $(0.00)$      & $(0.00)$      & $(0.00)$      \\
Service*Balance HIGH  & $-0.27$       & $-0.23$       & $-0.54^{*}$   \\
                  & $(0.23)$      & $(0.24)$      & $(0.26)$      \\
Service*Balance VHIGH  & $0.61^{*}$    & $0.65^{*}$    & $0.49$        \\
                  & $(0.26)$      & $(0.26)$      & $(0.27)$      \\
Civic Duty  & $0.25$        & $0.22$        & $0.29$        \\
                  & $(0.19)$      & $(0.19)$      & $(0.22)$      \\
Civic Duty*Balance MOD   & $1.01^{*}$    & $0.99^{*}$    & $1.06^{*}$    \\
                  & $(0.42)$      & $(0.42)$      & $(0.46)$      \\
Civic Duty*Balance HIGH   & $-0.18$       & $-0.13$       & $-0.33$       \\
                  & $(0.25)$      & $(0.25)$      & $(0.26)$      \\
Civic Duty*Balance VHIGH   & $-0.39$       & $-0.32$       & $-0.54^{*}$   \\
                  & $(0.23)$      & $(0.23)$      & $(0.25)$      \\
Threat            & $-0.07$       & $-0.03$       & $-0.05$       \\
                  & $(0.27)$      & $(0.27)$      & $(0.29)$      \\
Threat*Balance MOD & $0.21$        & $0.25$        & $0.13$        \\
                  & $(0.22)$      & $(0.22)$      & $(0.24)$      \\
Threat*Balance HIGH & $-0.64^{**}$  & $-0.65^{**}$  & $-0.80^{**}$  \\
                  & $(0.25)$      & $(0.25)$      & $(0.27)$      \\
Threat*Balance VHIGH & $0.20$        & $0.22$        & $-0.03$       \\
                  & $(0.22)$      & $(0.22)$      & $(0.24)$      \\
\hline
Log Likelihood    & -1150.17      & -1120.45      & -919.68       \\
Num. obs.         & 4927          & 4749          & 3888          \\
\end{longtable}
\end{center}


These impacts are seen most clearly in Table \ref{TT} (constructed as
was Table \ref{ZZ}) which presents the predicted probabilities for
payment in full for the control letter and each of the three treatment
letters, disaggregated by the level of taxes owed.  The Civic Duty
letter has strongest effect on the decision by tardy taxpayers to meet
their full tax obligations.


\begin{table}[htbp]
\caption{Marginal Predictions -- Paid in Full}  \label{TT}
\bigskip
\begin{center}
\begin{tabular}{| l | c | c | c | c |}
  \hline
 & LOW & MOD & HIGH & VHIGH \\ 
  \hline
Control & 19.90 & 6.40 & 2.40 & 0.90 \\ 
  Service & 13.60 & 4.00 & 2.90 & 1.20 \\ 
  Civic Duty & 24.10 & 6.90 & 3.10 & 1.80 \\ 
  Threat & 18.80 & 3.50 & 4.20 & 0.90 \\ 
   \hline
\multicolumn{5}{l}{NOTE: This table reports the marginal effects} \\
\multicolumn{5}{l}{from the Logit Model with interactions that uses} \\
\multicolumn{5}{l}{``paid in full" as outcome.} \\
\end{tabular}
\end{center}
\end{table}

Our study reveals there is heterogeneity in response to different
treatments.  A preferred overall strategy might take advantage of
these differential responses of taxpayers to the treatment
letters. More research is clearly needed to assess the efficiency of
targeted reminding strategies.


\section{Conclusions}

This field experiment evaluated three alternative notification
strategies intended to increase property tax compliance.  We have
implemented our experiment in collaboration with Philadelphia's
Department of Revenue (DoR).  We feel this initial study of property
tax compliance in Philadelphia has value for at least three reasons.
First, it is the first study that systematically examines alternative
tax compliance strategies for taxation in a large city.  Second, the
study of property tax compliance for which there is a known tax
liability has allowed us to focus directly on motives for paying
taxes.  Third, great care was given to separately specify, identify,
and directly compare the three common motives for tax payment that
play a prominent role in the tax compliance literature.

Our findings provide tentative support that appeals both to public
services provided and to a citizen's sense of civic duty can improve
tax compliance. These findings are consistent with other recent tax compliance
experiments (Fellner, Sausgruber, and Traxler 2013). Providing social information about
tax compliance provides a modest increase in collection (Wenzel and
Taylor 2004; Wenzel 2005; Hallsworth et al. 2014).

In contrast to several papers that show the benefits of audit threats
(Kleven et al. 2011; Slemrod, Blumenthal, and Christian 2001), we find
no support for use of a threats over that of DoR's usual reminder of
taxes owed.  It would be useful to examine the benefits of an
intervention with a more specific threat, such as randomly assigning
tax delinquents to publicity in a local newspaper or website.
Alternatively, one could add more local specificity to the
notification letter. For example, one could provide information about
local sheriff sales or foreclosures. More research is clearly needed
to determine effective notification letters that reinforce the
likelihood of fines and penalties.


There are limitations to our study, of course.  Strictly speaking, our
conclusions apply only to Philadelphia taxpayers, and among those
citizens, only those who are tardy in paying their taxes. Second, our
sample of taxpayers is small, only 4900 in total.  And finally, while
our focus on property tax compliance has the advantage of allowing us
to more cleanly identify motives for tax payments, Philadelphia and
other cities raise significant revenues from wage taxes, income and
profits taxes, sales taxes, and fees.  Payment compliance for cities
for these other revenue instruments deserves careful analysis too.
All said, however, we feel our work here is an encouraging first step
towards introducing the new methodologies of tax compliance into the
practice of city government finances.

\newpage

\section*{References}

Aaron, Henry (1975), Who Pays the Property Tax? Washington, DC:
Brookings Institution. \\
\\
Ariel, Barak (2012), ``Deterrence and moral persuasion effects on
corporate tax compliance: Findings from a randomized controlled
trial." Criminology, 50 (1), 27-69. \\
\\
Allingham, Michael G., and Agnar Sandmo (1972) ``Income Tax Evasion: A
Theoretical Analysis." Journal of Public Economics, 1: 323-38. \\
\\
Alm, James, Gary H. McClelland, and William D. Schulze (1992), ``Why Do People
Pay Taxes?" Journal of Public Economics 48: 21-38. \\
\\
Alm, James (1999), ``Tax compliance and administration." In: Hildreth,
W. Bartley and James A. Richardson (eds.) Handbook on Taxation. New
York, USA, Marcel Dekker, Inc., pp. 741-768. \\ 
\\
Andreoni, James, Erard, Brian and Jonathan Feinstein (1998), ``Tax
compliance." Journal of Economic Literature, 36, 818-860. \\ 
\\
Becker, Gary S. (1968), ``Crime and Punishment: An Economic Approach."
Journal of Political Economy 76: 169-217.\\
\\
Benabou, Roland and Jean Tirole (2011), ``Laws and Norms," NBER
Working Paper, No. 17579. \\
\\
Bernheim, B. Douglas (1994), ``A Theory of Conformity." Journal of
Political Economy, 102, 5, 841-877. \\
\\
Besley, Timothy, Anders Jensen, and Torsten Persson (2014), ``Norm,
Enforcement, and Tax Evasion," London School of Economics.  \\ 
\\
Blumenthal, Marsha, Christian, Charles and Joel Slemrod (2001), ``Do
normative appeals affect tax compliance? Evidence from a controlled
experiment in Minnesota." National Tax Journal, 54 (1), 125 -
138. \\ 
\\ 
Cowell, Frank A. and James P. F. Gordon (1988), `` Unwillingness to
pay tax: tax evasion and public provision."  Journal of Public
Economics, 36, 305-321.\\ 
\\ 
Feddersen, Timothy and Alvaro Sandroni (2006), ``A Theory of
Participation in Elections," American Economic Review, Vol. 96
(September), 1271-1282.  \\ 
\\ 
Fehr, Ernst and Simon Gachter (1998), ``Reciprocity and economics: The
economic implications of homo reciprocans." European Economic Review
42 (3-5), 845-59. \\ 
\\ 
Fellner, Gerlinde, Rupert Sausgruber, and Christian Traxler (2013),
``Testing Enforcement Strategies in the Field: Threat, Moral Appeal
and Social Information." Journal of the European Economic Association
11, 3, 634-60.\\ 
\\ 
Frey, Bruno S., and Lars P. Feld (2002), ``Deterrence and Morale in
Taxation: An Empirical Analysis."  CESifo Working Paper no. 760,
August 2002. \\ 
\\ 
Hallsworth, Michael., John List, Robert Metcalfe and Ivo Vlaev (2014),
``The Behavioralist as Tax Collector," Using Natural Field Experiments
to Enhance Tax Compliance." NBER Working Paper 20007. \\ 
\\ 
Hamilton, Bruce (1975), ``Zoning and Property Taxation in a System of
Local Governments," Urban Studies, Vol. 12 (June),
205-211. \\ 
\\ 
Harrison, Glenn W. and John A. List (2004), ``Field Experiments."
Journal of Economic Literature, 42 (4), 1009-1055.\\ 
\\ 
Keen, Michael and Ben Lockwood (2010), ``The Value-Added Tax: Its
Causes and Consequences," Journal of Development Economics, Vol. 92
(July), 138-151. \\ 
\\ 
Kleven, Henrik J., Knudsen, Martin B., Kreiner, Claus T., Pedersen,
Soren and Emmanuel Saez (2011), ``Unwilling or Unable to Cheat?
Evidence From a Tax Audit Experiment in Denmark."  Econometrica, 79
(3), 651-692. \\ 
\\ 
Mieszkowski, Peter (1972), ``The Property Tax: an Excise or a Profits
Tax?" Journal of Public Economics, Vol. 1 (April),
73-96. \\ 
\\ 
Organization for Economic Cooperation and Development 2011, Tax
Administration in OECD and Selected Non-OECD Countries: Comparative
Information Series (2010), Center for Tax Policy and Administration.
\\ 
\\ 
Perez-Truglia, Ricardo and Ugo Troiano (2015), ``Tax Debt Enforcement:
Theory and Evidence from a Field Experiment in the United States,"
University of Michigan.  \\ 
\\ 
Pew Charitable Trust (2013), ``Delinquent Property Tax in
Philadelphia." Technical Report. \\ 
\\ 
Pomeranz, Dina (2013), ``No taxation without information: Deterrence
and self-enforcement in the Value Added Tax."  Harvard Business School
Working Paper. \\ 
\\ 
Posner, Eric (2000), ``Law and Social Norms: the Case of Tax
Compliance," Virginia Law Review, Vol. 86, (No. 8)
1781-1819. \\ 
\\ 
Reckers, Philip M. J., Sanders, Debra L. and Stephen J. Roark (1994),
``The influence of ethical attitudes on taxpayer compliance." National
Tax Journal, 47 (4), 825-836. \\ 
\\ 
Sherman, Lawrence (1993), ``Defiance, deterrence, and irrelevance: A
theory of the criminal sanction."  Journal of Research in Crime and
Delinquency, 30, 445-473. \\ 
\\ 
Slemrod, Joel (2007), ``Cheating ourselves: The economics of tax
evasion." Journal of Economic Perspectives, 21 (1),
25-48. \\ 
\\ 
Slemrod, Joel, Marsha Blumenthal, and Charles Christian (2001),
``Taxpayer Response to an Increased Probability of Audit: Evidence
from a Controlled Experiment in Minnesota." Journal of Public
Economics 79, 3, 455-83.\\ 
\\ 
Torgler, Benno (2002), ``Moral-suasion: An alternative tax policy
strategy?  Evidence from a controlled field experiment in
Switzerland." Economics of Governance 5 (3), 235-253. \\ 
\\ 
Torgler, Benno (2012), ``A field experiment on moral-suasion and tax
compliance focusing on under-declaration and over-deduction." QUT
School of Economics and Finance Working Paper no. 285. \\ 
\\ 
Wenzel, Michael (2005), ``Misperceptions of social Norms about Tax
Compliance: From Theory to Intervention." Journal of Economic
Psychology 26, 6, 862-83\\ 
\\ 
Wenzel, Michael and Natalie Taylor (2004), ``An experimental
evaluation of tax-reporting schedules: a case of evidence-based tax
administration." Journal of Public Economics, 88 (12), 2785-2799.

\bigskip

\bigskip

\section*{Appendix A: Figures 1 through 5}
 \newpage
 
\begin{figure}[htpb]
\begin{center}
\caption{Standard Due Letter}
\bigskip
\includegraphics[width=6in]{PastDueLetter.pdf}
\end{center}
\end{figure}
\newpage
\begin{figure}[htpb]
\begin{center}
\caption{Treatment 1: Deterrence}
\bigskip
\includegraphics[width=6in]{flyer_options_141104_treat1.pdf}
\end{center}
\end{figure}
\newpage
\begin{figure}[htpb]
\begin{center}
\caption{Treatment 2: Service Appeal}
\bigskip
\includegraphics[width=6in]{flyer_options_141104_treat2.pdf}
\end{center}
\end{figure}
\newpage
\begin{figure}[htpb]
\begin{center}
\caption{Treatment 3: Civic Duty}
\bigskip
\includegraphics[width=6in]{flyer_options_141104_treat3.pdf}
\end{center}
\end{figure}
\newpage
\begin{figure}[htpb]
\begin{center}
\caption{Control}
\bigskip
\includegraphics[width=6in]{flyer_options_141104_treat4.pdf}
\end{center}
\end{figure}

\end{document}


